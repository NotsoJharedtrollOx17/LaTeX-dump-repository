% Preámbulo
\documentclass[stu, 12pt, letterpaper, donotrepeattitle, floatsintext, natbib]{apa7}
\usepackage[utf8]{inputenc}
\usepackage{comment}
\usepackage{marvosym}
\usepackage{graphicx}
\usepackage{float}
\usepackage[normalem]{ulem}
\usepackage[spanish]{babel} 
\usepackage{lastpage} %para le formato que quiere la profe QUITAR SI QUIERES OG APA7
\usepackage{ragged2e} %para le formato que quiere la profe QUITAR SI QUIERES OG APA7
\usepackage{indentfirst} %para le formato que quiere la profe QUITAR SI QUIERES OG APA7
\DeclareCaptionLabelSeparator*{spaced}{\\[2ex]}
\captionsetup[table]{textfont=it,format=plain,justification=justified,
  singlelinecheck=false,labelsep=spaced,skip=1pt}

\selectlanguage{spanish}
\useunder{\uline}{\ul}{}
\newcommand{\myparagraph}[1]{\paragraph{#1}\mbox{}\\}

\rfoot{Página \thepage \hspace{1pt} de \pageref{LastPage}}%QUITAR SI QUIERES OG APA7 
\rhead{} %QUITAR SI QUIERES OG APA7
\setcounter{secnumdepth}{3} %permite enumerar las secciones QUITAR SI QUIERES OG APA7
\setlength{\parindent}{1.27cm} %sangria forzada QUITAR SI QUIERES OG APA7

% Portada
\thispagestyle{empty}
\title{\Large La Lucha de Camboya por el Mekong}
\author{Abraham Jhared Flores Azcona} % (autores separados, consultar al docente)
% Manera oficial de colocar los autores:
%\author{Autor(a) I, Autor(a) II, Autor(a) III, Autor(a) X}
\affiliation{Instituto Tecnológico de Tijuana}
\course{ACD-0908SC5C Desarrollo Sustentable}
\professor{M.C. Trinidad Castro Villa}
\duedate{30 de noviembre de 2021}

\renewcommand\labelitemi{$\bullet$}

\newcommand*\chem[1]{\ensuremath{\mathrm{#1}}}

\begin{document}
\maketitle


% Índices
\pagenumbering{arabic}
    % Contenido
\renewcommand\contentsname{Contenido}
\tableofcontents

% Cuerpo 
    %NOTA: PARA CITAR ESTILO "Merts (2003)" usar \cite{<nombre_cita_bib>}
    %                        "(Metz, 1978)" usar \citep{<nombre_cita_bib>}
\newpage
\section{La Lucha de Camboya por el Mekong}
\subsection{Resumen}
A \begin{justifying}
  grandes rasgos, el documental \citep{dw-documentary-2021} explica la situación socioeconómica de los habitantes aledaños
del rio Mekong de Camboya, donde actualmente han llenado dicho río con arena para poder generar 
más suelo para el desarrollo urbano del Mekong. Para los pescadores y agricultores, las acciones de desarrollo
urbano los han perjudicado por años debido a que la pesca termina siendo futíl y los
suelos terminan siendo infértiles para la siembra; aunado a lo anterior tenemos el desplazo de dichas
personas de sus hogares donde ciertos individuos aceptaron la relocalización de las viviendas por parte del gobierno mientras que otros
la negaron por principios morales.\par
\end{justifying}
Para \begin{justifying}
  los desarrolladores urbanos, el interés principal recae en la antes mencionada urbanización que atrae
la inversión extranjera al país, generando un ciclo de retroalimentación financiero y economicante benéfico al país, sin embargo
termina desplazando a las personas de escasos recursos. Un detalle relevante es que uno de los principales inversionistas financieros
es China con proyectos de infraestructura como el de la presa mostrada en el video, que les permite ser uno de los principales distribuidores
de energía en Asia.\par
\end{justifying}
\vspace{\baselineskip}
\subsection{Tangentes}
\subsubsection{Financimiento Chino para Proyectos de Infraestructura Internacionales}
Una \begin{justifying}
  de las maneras en la cual el gobierno chino ha procurado mostrar su poderio económico
  por medio del financiamiento de proyectos de infraestructura alrededor del mundo, principalmente en Africa. Con dicha
  opción más proactiva, los gobiernos del mundo prefieren mantener alianzas estratégicas con China por su apertura de inversión, comparado a
  las instituciones financieras establecidas como el Banco Mundial a pesar de que muchos analístas económicos financieros advierten de que
  el trato incluye la creación de bases militares en los países donde estén desarrollando los proyectos, así como el endeudamiento de los entes \cite{simmons-2021}.\par
\end{justifying}
\vspace{\baselineskip}
\subsubsection{Desarrollo de la Infraestructura Hidrológica en el Mekong}
Como \begin{justifying}
  se menciona brevemente en el video, los planes de infraestructura hidrológica consideran el tratamiento del río respecto a las aguas negras
  de la ciudad, que claman que dicha infrestructura puede gestionar de mejor manera los desechos. Sin embargo, acorde a \cite{nhess-2021-65} %citar al de 
   el desarrollo debe ser muy resistente a los cambios naturales de la región, y por ende el desarrollo del proyecto debe tomar en cuenta que
  el márgen de alteraciones que provocaría dicho proyecto debe considerarse hasta el año 2040, donde proyectan que las
  inundaciones van a ser muy severas comparado a lo actual.\par
\end{justifying}
\newpage
% Referencias
\setcounter{secnumdepth}{0} %permite enumerar las secciones QUITAR SI QUIERES OG APA7
\renewcommand\refname{\textbf{Referencias}}
\bibliography{referencias} %el archivo 'referencias.bib' debe estar dentro del mismo folder donde se encuentra el archivo .tex para citar las referencias deseadas

\end{document}