% Preámbulo
\documentclass[stu, 12pt, letterpaper, donotrepeattitle, floatsintext, natbib]{apa7}
\usepackage[utf8]{inputenc}
\usepackage{comment}
\usepackage{marvosym}
\usepackage{graphicx}
\usepackage{float}
\usepackage[normalem]{ulem}
\usepackage[spanish]{babel} 
\usepackage{lastpage} %para le formato que quiere la profe QUITAR SI QUIERES OG APA7
\usepackage{ragged2e} %para le formato que quiere la profe QUITAR SI QUIERES OG APA7
\usepackage{indentfirst} %para le formato que quiere la profe QUITAR SI QUIERES OG APA7
\DeclareCaptionLabelSeparator*{spaced}{\\[2ex]}
\captionsetup[table]{textfont=it,format=plain,justification=justified,
  singlelinecheck=false,labelsep=spaced,skip=1pt}

\selectlanguage{spanish}
\useunder{\uline}{\ul}{}
\newcommand{\myparagraph}[1]{\paragraph{#1}\mbox{}\\}

\rfoot{Página \thepage \hspace{1pt} de \pageref{LastPage}}%QUITAR SI QUIERES OG APA7 
\rhead{} %QUITAR SI QUIERES OG APA7
\setcounter{secnumdepth}{3} %permite enumerar las secciones QUITAR SI QUIERES OG APA7
\setlength{\parindent}{1.27cm} %sangria forzada QUITAR SI QUIERES OG APA7

% Portada
\thispagestyle{empty}
\title{\Large Huella Ecológica}
\author{Abraham Jhared Flores Azcona} % (autores separados, consultar al docente)
% Manera oficial de colocar los autores:
%\author{Autor(a) I, Autor(a) II, Autor(a) III, Autor(a) X}
\affiliation{Instituto Tecnológico de Tijuana}
\course{ACD-0908SC5C Desarrollo Sustentable}
\professor{M.C. Trinidad Castro Villa}
\duedate{22 de noviembre de 2021}

\renewcommand\labelitemi{$\bullet$}

\newcommand*\chem[1]{\ensuremath{\mathrm{#1}}}

\begin{document}
\maketitle


% Índices
\pagenumbering{arabic}
    % Contenido
\renewcommand\contentsname{Contenido}
\tableofcontents
\renewcommand{\listtablename}{Tablas}
\listoftables

% Cuerpo 
    %NOTA: PARA CITAR ESTILO "Merts (2003)" usar \cite{<nombre_cita_bib>}
    %                        "(Metz, 1978)" usar \citep{<nombre_cita_bib>}
\newpage
\section{Huella Ecológica}
\subsection{Concepto}
Acorde a \cite{secretaria-de-medio-ambiente-y-recursos-naturales-2017} %citar al gobmx
\begin{justifying}
  es un indicador para conocer el grado de impacto de la sociedad sobre el ambiente. También agrega que
  es una herramienta para determinar cuánto espacio terrestre y marino se necesita para producir todos los recursos
  y bienes que se consumen, así como la superficcie para absorber los desechos que se generan, usando la tecnología
  actual.\par
\end{justifying}
\vspace{\baselineskip}
\subsection{Mi Huella Ecológica}
\begin{table}[H]
  \centering
\caption{Desglose de huella ecológica personal}
\begin{tabular}{lllll}
\hline
\multicolumn{1}{c}{\textbf{17-nov}}                            & \multicolumn{1}{c}{\textbf{18-nov}}                            & \multicolumn{1}{c}{\textbf{19-nov}}                            & \multicolumn{1}{c}{\textbf{20-nov}}                            & \multicolumn{1}{c}{\textbf{21-nov}}                            \\ \hline
\begin{tabular}[c]{@{}l@{}}Papel \\ higienico\end{tabular}     & \begin{tabular}[c]{@{}l@{}}Papel\\ higienico\end{tabular}      & \begin{tabular}[c]{@{}l@{}}Papel\\ higienico\end{tabular}      & \begin{tabular}[c]{@{}l@{}}Papel \\ higienico\end{tabular}     & \begin{tabular}[c]{@{}l@{}}Papel\\ desechable\end{tabular}     \\
\begin{tabular}[c]{@{}l@{}}Platos\\ desechables\end{tabular}   & \begin{tabular}[c]{@{}l@{}}Platos \\ desechables\end{tabular}  & \begin{tabular}[c]{@{}l@{}}Platos \\ desechables\end{tabular}  & \begin{tabular}[c]{@{}l@{}}Platos \\ desechables\end{tabular}  & \begin{tabular}[c]{@{}l@{}}Botella \\ de plástico\end{tabular} \\
\begin{tabular}[c]{@{}l@{}}Vasos \\ desechables\end{tabular}   & \begin{tabular}[c]{@{}l@{}}Vasos \\ desechables\end{tabular}   & \begin{tabular}[c]{@{}l@{}}Vasos \\ desechables\end{tabular}   & \begin{tabular}[c]{@{}l@{}}Vasos \\ desechables\end{tabular}   &                                                                \\
\begin{tabular}[c]{@{}l@{}}Botella \\ de plástico\end{tabular} & \begin{tabular}[c]{@{}l@{}}Botella\\ de plástico\end{tabular}  & \begin{tabular}[c]{@{}l@{}}Botella \\ de plástico\end{tabular} & \begin{tabular}[c]{@{}l@{}}Botella \\ de plástico\end{tabular} &                                                                \\
\begin{tabular}[c]{@{}l@{}}Papel \\ higiénico\end{tabular}     & \begin{tabular}[c]{@{}l@{}}Papel\\ higiénico\end{tabular}      & \begin{tabular}[c]{@{}l@{}}Papel \\ higiénico\end{tabular}     & \begin{tabular}[c]{@{}l@{}}Botella \\ de plástico\end{tabular} &                                                                \\
\begin{tabular}[c]{@{}l@{}}Botella \\ de plástico\end{tabular} & \begin{tabular}[c]{@{}l@{}}Botella \\ de plástico\end{tabular} & \begin{tabular}[c]{@{}l@{}}Botella \\ de plástico\end{tabular} &                                                                &                                                                \\ \hline
\end{tabular}
\bigskip
  \\\small\textit{Nota}. Autoría propia. %citar al tmta.
\end{table}
Debido \begin{justifying}
  a que las rutinas de mi hogar son frecuentes, se puede concluír que hacemos mucho desperdício debido al uso de 
desechables. Por una parte, la cuestión de la higiéne en el papel higiénico ons da una grán excepción por los posibles riesgos de salubridad mietras
que el resto de los artículos se pueden reducir en grán manera e inclusíve utilizasr dichos desechos en otras aplicaciones tales como obras de arte visual.\par
\end{justifying} 

\newpage
% Referencias
\setcounter{secnumdepth}{0} %permite enumerar las secciones QUITAR SI QUIERES OG APA7
\renewcommand\refname{\textbf{Referencias}}
\bibliography{referencias} %el archivo 'referencias.bib' debe estar dentro del mismo folder donde se encuentra el archivo .tex para citar las referencias deseadas

\end{document}