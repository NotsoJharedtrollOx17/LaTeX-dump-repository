% Preámbulo
\documentclass[stu, 12pt, letterpaper, donotrepeattitle, floatsintext, natbib]{apa7}
\usepackage[utf8]{inputenc}
\usepackage{comment}
\usepackage{marvosym}
\usepackage{graphicx}
\usepackage{float}
\usepackage[normalem]{ulem}
\usepackage[spanish]{babel} 
\usepackage{lastpage} %para le formato que quiere la profe QUITAR SI QUIERES OG APA7
\usepackage{ragged2e} %para le formato que quiere la profe QUITAR SI QUIERES OG APA7
\usepackage{indentfirst} %para le formato que quiere la profe QUITAR SI QUIERES OG APA7
\DeclareCaptionLabelSeparator*{spaced}{\\[2ex]}
\captionsetup[table]{textfont=it,format=plain,justification=justified,
  singlelinecheck=false,labelsep=spaced,skip=1pt}

\selectlanguage{spanish}
\useunder{\uline}{\ul}{}
\newcommand{\myparagraph}[1]{\paragraph{#1}\mbox{}\\}

\rfoot{Página \thepage \hspace{1pt} de \pageref{LastPage}}%QUITAR SI QUIERES OG APA7 
\rhead{} %QUITAR SI QUIERES OG APA7
\setcounter{secnumdepth}{3} %permite enumerar las secciones QUITAR SI QUIERES OG APA7
\setlength{\parindent}{1.27cm} %sangria forzada QUITAR SI QUIERES OG APA7

% Portada
\thispagestyle{empty}
\title{\Large Contaminación Ambiental}
\author{Abraham Jhared Flores Azcona} % (autores separados, consultar al docente)
% Manera oficial de colocar los autores:
%\author{Autor(a) I, Autor(a) II, Autor(a) III, Autor(a) X}
\affiliation{Instituto Tecnológico de Tijuana}
\course{ACD-0908SC5C Desarrollo Sustentable}
\professor{M.C. Trinidad Castro Villa}
\duedate{23 de noviembre de 2021}

\renewcommand\labelitemi{$\bullet$}

\newcommand*\chem[1]{\ensuremath{\mathrm{#1}}}

\begin{document}
\maketitle


% Índices
\pagenumbering{arabic}
    % Contenido
\renewcommand\contentsname{Contenido}
\tableofcontents

% Cuerpo 
    %NOTA: PARA CITAR ESTILO "Merts (2003)" usar \cite{<nombre_cita_bib>}
    %                        "(Metz, 1978)" usar \citep{<nombre_cita_bib>}
\newpage
\section{Contaminación Ambiental}
\subsection{Concepto}
Para \cite{romero-2021} %citar a BBVA
\begin{justifying}
  es la presencia de componentes nocivos, bien sean de naturaleza biológica, química o de otra clase, en el medioambiente,
  de modo que supongan un perjuicio para los seres vivos que habitan un espacio, incluyendo, por supuesto a los seres humanos.
  Generalmente la cotaminación ambietal tiene su origen en alguna actividad humana.\par
\end{justifying}
Para \cite{unknown-author-no-dateA} %citar a los de lineaverde
\begin{justifying}
  es la presencia de componentes nocivos en el medio ambiente, que suponen un perjuicio para los seres vivos que lo habitan, incluyendo
  a los seres humanos. Generalmente está origiada por derivados de la actividad humana, como la emisión a la atmosfera de gases de
  efecto invernadero o la explotación desmedida de los recusos naturales.\par
\end{justifying}
\vspace{\baselineskip}
\subsection{Contaminación Ambiental en Agua}
\subsubsection{Concepto}
Para \cite{unknown-author-2020} %citar a los de iberdrola
\begin{justifying}
  Como la contaminación que hace sufrir cambios a la composición del agua hasta quedar inservible. Por lo anterior
  dicha agua no se puede beber ni destinar a actividades esenciales como la agricultura.\par
\end{justifying}
\vspace{\baselineskip}
\subsubsection{Ejemplo}
Un\begin{justifying}
  ejemplo de este tipo de contaminación es aquella de los derrames de los combustibles ya que el transporte y almacenamiento de petróleo y derivados
  dan lugar a filtraciones que pueden llegar a las fuenntes de agua.\par
  \end{justifying}
\vspace{\baselineskip}
\subsection{Contaminación Ambiental en el Aire}
\subsubsection{Concepto}
Acorde a \cite{unknown-author-no-dateB} %citar a medline plus
\begin{justifying}
  es aquella donde el aire tiene una mezcla de partículas sólidas y gases en el aire. Debido a que algunos cotasminantes
  son tóxicos, su inhalación puede aumentar las posibilidades de tener problemas de salud.\par
\end{justifying}
\vspace{\baselineskip}
\subsubsection{Ejemplo}
Un \begin{justifying}
  ejemplo ya muy trillado es aquel de las emisiones de gases invernadero producidas producidas por los automoviles de combustión fósil, ya que
al quemar dicha gasolina para generar las explosiones que mueven al motor de la máquina, el excedente se expulsa por el escape de éste. Dichas emisiones
se agravian si el automóvil está en malas condiciones de mantenimiento, cmbinando otros químicos que permiten el buen funcionamiento del carro.\par
\end{justifying}
\vspace{\baselineskip}
\subsection{Contaminación Ambiental en la Tierra}
\subsubsection{Concepto}
Para \cite{unknown-author-2019} %citar al reciclado
\begin{justifying}
  es la alteración del suelo por la presencia de sustancias químicas producidas por el hombre. Curiosamente, un suelo contaminado
  también podrá contaminar el aire y el agua, expandiendo el rango de afección.\par
\end{justifying}
\vspace{\baselineskip}
\subsubsection{Ejemplo}
Un ejemplo de la contaminación de los suelos es una muy simple, la deforestación y erosión de los suelos; al no tener
vegetación, la tierra se vuelve seca y estéril. Esto tambien genera un ran riesgo de suelos muy inestables para construir, ya que las raíces
de dicha vegetación permiten la comptación y firmeza de los suelos.\par
\newpage
% Referencias
\setcounter{secnumdepth}{0} %permite enumerar las secciones QUITAR SI QUIERES OG APA7
\renewcommand\refname{\textbf{Referencias}}
\bibliography{referencias} %el archivo 'referencias.bib' debe estar dentro del mismo folder donde se encuentra el archivo .tex para citar las referencias deseadas

\end{document}