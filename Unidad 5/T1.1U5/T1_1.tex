% Preámbulo
\documentclass[stu, 12pt, letterpaper, donotrepeattitle, floatsintext, natbib]{apa7}
\usepackage[utf8]{inputenc}
\usepackage{comment}
\usepackage{marvosym}
\usepackage{graphicx}
\usepackage{float}
\usepackage[normalem]{ulem}
\usepackage[spanish]{babel} 
\usepackage{lastpage} %para le formato que quiere la profe QUITAR SI QUIERES OG APA7
\usepackage{ragged2e} %para le formato que quiere la profe QUITAR SI QUIERES OG APA7
\usepackage{indentfirst} %para le formato que quiere la profe QUITAR SI QUIERES OG APA7

\selectlanguage{spanish}
\useunder{\uline}{\ul}{}
\newcommand{\myparagraph}[1]{\paragraph{#1}\mbox{}\\}

\rfoot{Página \thepage \hspace{1pt} de \pageref{LastPage}}%QUITAR SI QUIERES OG APA7 
\rhead{} %QUITAR SI QUIERES OG APA7
\setcounter{secnumdepth}{3} %permite enumerar las secciones QUITAR SI QUIERES OG APA7
\setlength{\parindent}{1.27cm} %sangria forzada QUITAR SI QUIERES OG APA7

% Portada
\thispagestyle{empty}
\title{\Large Censo de Población}
\author{Abraham Jhared Flores Azcona} % (autores separados, consultar al docente)
% Manera oficial de colocar los autores:
%\author{Autor(a) I, Autor(a) II, Autor(a) III, Autor(a) X}
\affiliation{Instituto Tecnológico de Tijuana}
\course{ACD-0908SC5C Desarrollo Sustentable}
\professor{M.C. Trinidad Castro Villa}
\duedate{24 de noviembre de 2021}

\renewcommand\labelitemi{$\bullet$}

\newcommand*\chem[1]{\ensuremath{\mathrm{#1}}}

\begin{document}
\maketitle


% Índices
\pagenumbering{arabic}
    % Contenido
\renewcommand\contentsname{Contenido}
\tableofcontents

% Cuerpo 
    %NOTA: PARA CITAR ESTILO "Merts (2003)" usar \cite{<nombre_cita_bib>}
    %                        "(Metz, 1978)" usar \citep{<nombre_cita_bib>}
\newpage
\section{Censo de Población y Vivienda del INEGI AÑO 2020}
Acorde \begin{justifying}
  al propio instituto \citep{geografia-2020}%citar al INEGI
  dicho censo se realizó del 2 al 27 de marzo del 2020 con el afán de obtener información sobre estas, contar a la población que vive en México
  e indagar sobre sus principales características demográficas, socioeconómicas y culturales.\par
\end{justifying}
Específicamente \begin{justifying}
  con el censo de ese año, se lleva a cabo el décimo cuarto censo de la población en la historia moderna de los cencos en México,
  los cuales se realizan desde 1895. Para el resto de la redacción se toma como referencia la publicación ejecutiva del mismo censo \citep{inegi-2020}. %citar la versión ejecutiva
  \par
\end{justifying}
\vspace{\baselineskip}
\subsection{Datos recabados}
\subsubsection{Innovaciones Tecnológicas}
Destacan \begin{justifying}
  el levantamiento de las encuestas con dispositivos móviles y el uso de encriptación y cifrado para la información recabada. En el auge de la pandemia,
  muchas de las preocupaciones se tornaron al ámbito digital debido a la cutre infraestructura digital \citep{moore-2021}, por lo que el implementar protocolos de seguridad digital
  permite a los ciudadanos sentirse más tranquilos con lo provisto en dichas encuestas.\par
\end{justifying}
\vspace{\baselineskip}
\subsubsection{Resultados}
En \begin{justifying}
  lo competente a los resultados tenemos los típicos datos:
  \begin{itemize}
    \item La población total del territoria mexicano es de 126014024.
    \item El promedio anual de la tasa de crecimiento poblacional es del 1.2 desde 2010 hasta el 2020.
    \item La cantidad de hombres dentro del país es de 61473390 mientras que la cantidad de mujeres es de 64540634.
    \item La edad mediana de la población es de 29 años.
    \item El estado más poblado es el Estado de México.
    \item La município más poblado es Tijuana, BC.
    \item La tasa de participación económica es del 62\%.
  \end{itemize}
\end{justifying}
\newpage
% Referencias
\setcounter{secnumdepth}{0} %permite enumerar las secciones QUITAR SI QUIERES OG APA7
\renewcommand\refname{\textbf{Referencias}}
\bibliography{referencias} %el archivo 'referencias.bib' debe estar dentro del mismo folder donde se encuentra el archivo .tex para citar las referencias deseadas

\end{document}