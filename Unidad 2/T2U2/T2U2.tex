\documentclass[letterpaper, 12pt]{article}
\usepackage[letterpaper, top=2.5cm, bottom=2.5cm, left=3cm, right=3cm]{geometry} %margenes
\usepackage[utf8]{inputenc} %manejo de caracteres especiales
\usepackage[spanish]{babel} %manejo de encabezados de inglés a español
\usepackage{fancyhdr} %formato de los encabezados de página
\usepackage{ragged2e} %alineado real justficado
\usepackage{graphicx} %manejo de imagenes
\usepackage{amsmath} %manejo de notación matemática
\usepackage{mathtools} %manejo de notación matemática
\usepackage{blindtext} %texto de relleno
\usepackage{cancel} %permite la simbolización de cancelación de terminos
\usepackage{enumitem}[shortlabels] %listas con letras
\usepackage{amssymb} %manejo de simbología matematica
\usepackage{float}

\pagestyle{fancy}
\fancyhf{}
\rfoot{}

\begin{document}
\thispagestyle{fancy}
\lhead{\textbf{Tarea 2, U2}}
\rhead{\textbf{14 de abril de 2021}}
\section*{Reducción de orden}
\subsection*{Determinar \(y_2\) dado lo siguiente:}
\justify
{\large
\(y_1=\sin 7x \,\text{ es una solución a }\, y^{\prime\prime}+49y=0\)\\\newline
{\large \textbf{ • Solución}\\
Se recuerda que \(y_2=u(x)y_1\) y que:
\[u(x)=\int \frac{\text{exp}\left(-\int p(x)\, dx\right)}{y_1^2}\, dx\]
Por lo tanto:
\begin{equation*}
    \begin{aligned}
u(x)=\int \frac{\text{exp}\left(-\int 0\, dx\right)}{\sin^2 (7x)}\, dx=& 
\int \frac{e^c}{\sin^2 (7x)}\, dx= e^c\!\int \frac{dx}{\sin^2 (7x)} =\\[5pt]
e^c\!\int \csc^2 (7x)\, dx = \underbrace{\frac{e^c}{7}\!\int \csc^2 (u)\, du}_{u=7x\, \rightarrow\, du=7\, dx}=&
-\frac{e^c}{7} \cot u = -\frac{e^c}{7} \cot (7x) \therefore \\[5pt]
\therefore\, y^2 =& -\frac{e^c}{7}\cot (7x)\sin (7x)\\[5pt]
                 =& -\frac{e^c}{7}\left(\frac{\cos (7x)}{\sin (7x)}\sin (7x)\right)\\[5pt]
             y^2 =& -\frac{e^c}{7}\cos (7x)\: \textbf{(1.1)}\\[5pt]
    \end{aligned}
\end{equation*}}
Donde la ecuación \textbf{(1.1)} es la respuesta a lo solicitado.
\end{document}