\documentclass[stu, 12pt, letterpaper, donotrepeattitle, floatsintext, natbib]{apa7}
\usepackage[utf8]{inputenc}
%\usepackage{fontspec} %paquete para usar la fuente Arial 12
\usepackage{comment}
\usepackage{marvosym}
\usepackage{graphicx}
\usepackage{float}
\usepackage[normalem]{ulem}
\usepackage[spanish]{babel} 
\usepackage{lastpage} %para le formato que quiere la profe QUITAR SI QUIERES OG APA7
\usepackage{ragged2e} %para le formato que quiere la profe QUITAR SI QUIERES OG APA7
\usepackage{indentfirst} %para le formato que quiere la profe QUITAR SI QUIERES OG APA7

\setcounter{secnumdepth}{0} %permite enumerar las secciones QUITAR SI QUIERES OG APA7

%comando para ajustar la fuente Arial en todo el documento
%\setmainfont{Arial} %COMPILAR DOC CON XeLateX DOS VECES

\DeclareCaptionLabelSeparator*{spaced}{\\[2ex]}
\captionsetup[table]{textfont=it,format=plain,justification=justified,
  singlelinecheck=false,labelsep=spaced,skip=1pt}

\selectlanguage{spanish}

\useunder{\uline}{\ul}{}
\newcommand{\myparagraph}[1]{\paragraph{#1}\mbox{}\\}

%\rfoot{Página \thepage \hspace{1pt} de \pageref{LastPage}}%QUITAR SI QUIERES OG APA7 
\rhead{} %QUITAR SI QUIERES OG APA7
%\setcounter{secnumdepth}{3} %permite enumerar las secciones QUITAR SI QUIERES OG APA7
\setlength{\parindent}{1.27cm} %sangria forzada QUITAR SI QUIERES OG APA7

\renewcommand\labelitemi{$\bullet$}

\newcommand*\chem[1]{\ensuremath{\mathrm{#1}}}

\begin{document}
    %PORTADA
    \begin{titlepage}
        \begin{figure}[ht]
            \centering
            \includegraphics[width=15cm]{logosITT.png}
        \end{figure}
        \centering
        {\Large Tecnológico Nacional de México\\Instituto Tecnológico de Tijuana\par}
        \vspace{1cm}
        {\Large SCD-1011SC6C Ingeniería de Software\par}
        \vspace{1cm}
        {\Large Unidad 2\par}
        \vspace{2cm}
        {\Large\bfseries Actividad 1\par}
        \vspace{2cm}
        {\large Dra. Martha Elena Pulido\par}
        \vfill
            {\large Abraham Jhared Flores Azcona, 19211640\par}
        \vfill
        {\large 3 de marzo de 2021}
    \end{titlepage}

% Índices
\pagenumbering{arabic}
    % Contenido
\renewcommand\contentsname{Contenido}
\tableofcontents

% Cuerpo 
    %NOTA: PARA CITAR ESTILO "Merts (2003)" usar \cite{<nombre_cita_bib>}
    %                        "(Metz, 1978)" usar \citep{<nombre_cita_bib>}
\newpage
\section{Introducción}
Como parte importante de la ingeniería de software, se deben plantear de manera consisa y precisa las partes que van
a conformar un sistema de software. En esta breve redacción, resumimos los puntos importantes del modelo de diseño para futura referencia.
\vspace{\baselineskip}
\section{Modelos de diseño}
Según \cite{unknown-author-no-date} %citar IBM 
    \begin{justifying}
        este se basa sobre el modelo de análisis que describe, en mayor detalle,
            la estructura del sistema y cómoo será implementado el sistema. Las clases
            identificadas en el análisis son refinadas para incluir construcciones de implementación.\par
            Sus principios son:
            \begin{itemize}
                \item El diseño debe ser trazado hacia el modelo de análisis.
                \item Siempre se debe considerar la arquitectura del sistema a construir.
                \item Enfocarse en el diseño de los datos: su diseño debe recabar la manera en cómo los objetos de datos están reaizados dentro del diseño.
                \item Las interfaces de usuario deben considerar primeramente al usuario.
                \item Los componentes deben estár ligeramente acoplados.
                \item Las interfaces internas como de usuario deben ser diseñadas.
            \end{itemize}\par
    \end{justifying}    
    \vspace{\baselineskip}
\subsection{Tipos}
A \begin{justifying}
    grandes rasgos se divide en cuatro tipos: \citep{tawde-2021} \emph{diseño de datos, diseño de la arquitectura, diseño de las interfaces de usuario y el diseño de los niveles
    de componentes}.\par  
\end{justifying}
\subsubsection{Diseño de datos}
Representa \begin{justifying}
    los objetos de datos y sus relaciones en un diagrama entidad relación que muestra la información requeridad por cada entidad así como
    la relación entre ellos mismos.\par
\end{justifying}
\subsubsection{Diseño de la arquitectura}
Define \begin{justifying}
    la relación entre los elementos estructurales mayores del software. Se trata de descomponer el sistema
    en sus componentes que interactuan entre sí.\par
\end{justifying}
\subsubsection{Diseño las interfaces de usuario}
Representa \begin{justifying}
    cómo el software se va a comunicar con el usuario; interacciones con controles o despliegues del producto.\par
\end{justifying}
\subsubsection{Diseño de los niveles de componentes}
Transforma \begin{justifying}
    los elementos estructurales de la arquitectura de software en una descripción procedual de los componentes de software. Es una manera perfecta
    para compartir una gran cantidad de datos.\par
\end{justifying}
\vspace{\baselineskip}
\section{Conclusión}
En \begin{justifying}
    este diseño nos damos cuenta que se debe de tomar con seriedad para evitar tener una mala interpretación del sistema a desarrollar y, por ende, hacerlo mucho más tedioso
    de lo que és realmente. Siempre se recomienda revisar dichos puntos para mantener éstandares de calidad.\par
\end{justifying}

\setcounter{secnumdepth}{0} %permite enumerar las secciones QUITAR SI QUIERES OG APA7
\renewcommand\refname{\textbf{Referencias}}
\bibliography{referencias}
\end{document}