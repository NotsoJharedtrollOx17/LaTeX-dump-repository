\documentclass[letterpaper, 12pt]{article}
\usepackage[letterpaper, top=2.5cm, bottom=2.5cm, left=3cm, right=3cm]{geometry} %margenes
\usepackage[utf8]{inputenc} %manejo de caracteres especiales
\usepackage[spanish]{babel} %manejo de encabezados de inglés a español
\usepackage{fancyhdr} %formato de los encabezados de página
\usepackage{ragged2e} %alineado real justficado
\usepackage{graphicx} %manejo de imagenes
\usepackage{amsmath} %manejo de notación matemática
\usepackage{mathtools} %manejo de notación matemática
\usepackage{blindtext} %texto de relleno
\usepackage{cancel} %permite la simbolización de cancelación de terminos
\usepackage{amssymb} %manejo de simbología matematica
\usepackage{float} %centrado de imágenes
\usepackage{polynom} %formato de polinómios

\pagestyle{fancy}
\fancyhf{}
\rfoot{}

\begin{document}
\thispagestyle{fancy}
\lhead{\textbf{Tarea 4, U2}}
\rhead{\textbf{21 de abril de 2021}}
\section*{ED de orden superior}
\subsection*{Resolver la siguiente ecuación diferencial}
\justify
{\large \centering
\(y^{\prime\prime\prime}-5y^{\prime\prime}+17y^{\prime}-13y=0\)\\
}
\justify
%solución 1
{\large \textbf{• Solución:}
\begin{equation*}
    \begin{aligned}
        y^{\prime\prime\prime}-5y^{\prime\prime}+17y^{\prime}-13y=0&\rightarrow m^3-5m^2+17m-13=0; \\[5pt]
        \text{Factor}(13)= \pm 1,\, \pm 13 \therefore &\:\frac{m^3-5m^2+17m-13}{x-a}=\\[5pt]
        \frac{m^3-5m^2+17m-13}{x-1}\:&=\polyhornerscheme[x=1] {m^3-5m^2+17m-13}
    \end{aligned}
\end{equation*}}
\end{document}