\documentclass[letterpaper, 12pt]{article}
\usepackage[letterpaper, top=2.5cm, bottom=2.5cm, left=3cm, right=3cm]{geometry} %margenes
\usepackage[utf8]{inputenc} %manejo de caracteres especiales
\usepackage[spanish]{babel} %manejo de encabezados de inglés a español
\usepackage{fancyhdr} %formato de los encabezados de página
\usepackage{ragged2e} %alineado real justficado
\usepackage{graphicx} %manejo de imagenes
\usepackage{amsmath} %manejo de notación matemática
\usepackage{mathtools} %manejo de notación matemática
\usepackage{blindtext} %texto de relleno
\usepackage{cancel} %permite la simbolización de cancelación de terminos
\usepackage{enumitem}[shortlabels] %listas con letras
\usepackage{amssymb} %manejo de simbolog►1a matematica

\pagestyle{fancy}
\fancyhf{}
\rfoot{\thepage}

\begin{document}
\setcounter{page}{1}
\thispagestyle{fancy}
\lhead{\textbf{Tarea 1, U2}}
\rhead{\textbf{13/10/2020}}
\section*{Curvas planas, ecuaciones paramétricas y coordenadas polares}
\subsection*{Obtener la \emph{derivada}, \emph{tangente vertical y tangente horizontal de las sig. curvas paramétricas:}}
\[1.\, x(t)=t^3-t, \, y(t)=t^2\]
\[2.\, x(t)=t-1, \, y(t)=t^3-3t^2\]
\subsection*{Procedimiento}
\subsubsection*{Curva 1}
\justify
Derivada: 
\[\frac{dy}{dx}=\frac{\frac{dy}{dt}}{\frac{dx}{dt}}\therefore \frac{dx}{dt}=3t^2-1,\, \frac{dy}{dt}=2t\]
Tangente vertical: 
\[\tan_{\text{vertical}}\rightarrow\frac{dx}{dt}=0 \therefore \begin{matrix}
    \frac{dx}{dt}&=&3t^2-1\\
    0&=&3t^2-1\\
    1&=&3t^2\\
    \frac{1}{3}&=&t^2\\
    \pm\frac{1}{\sqrt{3}}&=&t
\end{matrix}\]
Tangente horizontal:
\[\tan_{\text{horizontal}}\rightarrow\frac{dy}{dt}=0 \therefore \begin{matrix}
    \frac{dy}{dt}&=&2t\\
    0&=&2t\\
    0&=&t
\end{matrix}\]
\subsubsection*{Curva 2}
\justify
Derivada:
\[\frac{dy}{dx}=\frac{\frac{dy}{dt}}{\frac{dx}{dt}}\therefore \frac{dx}{dt}=1,\, \frac{dy}{dt}=3t^2-6t\]
Tangente vertical:
\[\tan_{\text{vertical}}\rightarrow\frac{dx}{dt}=0 \therefore \begin{matrix}
    \frac{dx}{dt}&=&1\\
    0&=&1\\
    \therefore &\tan_{\text{vertical}} \text{ no existe}
\end{matrix}\]
Tangente horizontal:
\[\tan_{\text{horizontal}}\rightarrow\frac{dy}{dt}=0 \therefore \begin{matrix}
    \frac{dy}{dt}&=&3t^2-6t\\
    0&=&3t^2-6t\\
    0&=&t^2-2t\\
    2t&=&t^2\\
    2&=&t
\end{matrix}\]
\end{document}