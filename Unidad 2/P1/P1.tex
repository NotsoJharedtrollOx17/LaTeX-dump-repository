% Preámbulo
\documentclass[stu, 12pt, letterpaper, donotrepeattitle, floatsintext, natbib]{apa7}
\usepackage[utf8]{inputenc}
\usepackage{comment}
\usepackage{marvosym}
\usepackage{graphicx}
\usepackage{float}
\usepackage[normalem]{ulem}
\usepackage[spanish]{babel} 
\usepackage{lastpage} %para le formato que quiere la profe QUITAR SI QUIERES OG APA7
\usepackage{ragged2e} %para le formato que quiere la profe QUITAR SI QUIERES OG APA7
\usepackage{indentfirst} %para le formato que quiere la profe QUITAR SI QUIERES OG APA7

\selectlanguage{spanish}
\useunder{\uline}{\ul}{}
\newcommand{\myparagraph}[1]{\paragraph{#1}\mbox{}\\}

\rfoot{Página \thepage \hspace{1pt} de \pageref{LastPage}}%QUITAR SI QUIERES OG APA7 
\rhead{} %QUITAR SI QUIERES OG APA7
\setcounter{secnumdepth}{3} %permite enumerar las secciones QUITAR SI QUIERES OG APA7
\setlength{\parindent}{1.27cm} %sangria forzada QUITAR SI QUIERES OG APA7

% Portada
\thispagestyle{empty}
\title{\Large Legislación ambiental}
\author{Abraham Jhared Flores Azcona} % (autores separados, consultar al docente)
% Manera oficial de colocar los autores:
%\author{Autor(a) I, Autor(a) II, Autor(a) III, Autor(a) X}
\affiliation{Instituto Tecnológico de Tijuana}
\course{ACD-0908SC5C Desarrollo Sustentable}
\professor{M.C. Trinidad Castro Villa}
\duedate{17 de septiembre de 2021}

\begin{document}
\maketitle


% Índices
\pagenumbering{arabic}
    % Contenido
\renewcommand\contentsname{Contenido}
\tableofcontents

% Cuerpo 
    %NOTA: PARA CITAR ESTILO "Merts (2003)" usar \cite{<nombre_cita_bib>}
    %                        "(Metz, 1978)" usar \citep{<nombre_cita_bib>}
\newpage
\section*{Introducción}
\addcontentsline{toc}{section}{Introducción}
Acorde \begin{justifying}
    al \cite{sistema-de-informacion-legislativa-no-date}, la ley es un precepto o conjunto de estos, dictados por la autoridad, mediante el cuál se manda o prohibe algo acordado por los
    órganos legislativos competentes, dentro del procedimiento legislativo prescrito, entendiendo que dichos órganos son la expresión de la
    voluntad popular representada por el Parlamento o Poder Legislativo. Especificamente, la cualidad más relevante de las leyes es que deben
    ser \emph{imperativas}; refiere a que dicha ley se sobrepone a la voluntad de los sujetos cuya conducta encauza, independientemente que la
    voluntad de estos sea contraria a la ley.\par 
\end{justifying}
Aunado \begin{justifying}
  a esto, las leyes se basan en \emph{un contrato social}
que es un acuerdo entre los gobernantes y gobernados por un beneficio en común \citep{unknown-author-2021}, %citar la enciclopedia
sin embargo, esto genera el dilema de aquellos que no siguen la ley y, por ende, ¿porqué seguirla nosotros? A pesar de ello, las leyes
son \emph{perfectibles} y son la mejor opción para regir una sociedad civil con un enfoque de Desarrollo Sustentable, que con la ley elegida, sus características
y su relación con la materia, serán útiles como un futuro lineamiento para crear o modificar nuestras soluciones y regulaciones sustentables acordes.\par
\end{justifying}
\vspace{\baselineskip}
\section{Ley Seleccionada}
La \begin{justifying}
    selección de ley es aquella de la \emph{Ley del Comercio Exterior} porque los aspectos socioeconómicos generalmente no son profundizados con el debido enfoque en los temas del
    Desarrollo Sustentable, aparte de que dicha legislación por si misma es relevante estudiar para entender los lineamientos del comercio mexicano.\par
\end{justifying}
\vspace{\baselineskip}
\subsection{Descripción}
Acorde \begin{justifying}
    al primer artículo \citep{ley-comercio} de
    la Ley de Comercio Exterior, ésta fue decretada con el fín de regular y promover el comercio exterior así como favorecer la competitividad de la economía del país, propiciar el uso eficiente
    de los recursos, integrar la economía mexicana con la internacional, y sobre todo defender a esta economía de prácticas desleales por el bién de la población.\par
\end{justifying}
Mas allá, \begin{justify}
    establece las formas de arbitraje para resolver las problematicas de los costos preferenciales para productos de origen extranjero que dejen económicamente obsoletos a los producidos por la nación, así
    como quienes supervisan dicho arbitraje, que se considera como prácticas desleales, que exepciones se pueden aplicar, como presentar solicitudes para la revisión de este arbitraje, etc.\par
\end{justify}
\vspace{\baselineskip}
\subsection{Articulos Relevantes}
\subsubsection{Artículo 4o.}
Explica \begin{justifying}
    las facultades del Presidente de México en relación con el comercio exterior. Este puede modificar aranceles previa publicación en el Diario Oficial de la Federación, decretar los productos que pueden ingresar
    al territorio nacional así como su tránsito dentro del territorio. Es relevante para comprender de mejor manera las fracciones indicadas del Articulo 15\par
\end{justifying}
\vspace{\baselineskip}
\subsubsection{Artículo 5o. Fracción III-V}
Describe \begin{justifying}
    las facultades de la Secretaría Economía para el enforzamiento de medidas regulatorias, y emisión de permisos para la exportación, importación, circulació, y tránsito de mercancías.\par
\end{justifying}
\vspace{\baselineskip}
\subsubsection{Artículo 15 Fracción III-V}
Menciona \begin{justifying}
    los casos para las restricciones de mercancia en cuestión de aranceles, tránsito e importación/exportación; específicamente a aquellos que estén vetados por la Carta Magna, cuando se trate de preservar la flora y fauna
    en peligro de extinción o por fine de conservación y por el afán de conservar los bienes histórico, artístico o arqueológicos de la nación.\par
\end{justifying}
\vspace{\baselineskip}
\section{Vínculo con el Desarrollo Sustentable}
Como \begin{justifying}
    se mencionan en los artículos destacados y principalmente en las Fracciones III a V del Artículo 15, su relevancia como Ley en relación del Desarrollo Sustentable es clara. El poder tener una Ley expresa en términos económicos
    que especifíque las posibles restricciones de mercancia con fines de preservación del patrimonio natural y cultural de la nación, así como el poder dar un incetivo relativamente preferencial a la producción méxicana permite 
    un aumento la calidad de productos, generando un mayor estímulo socioeconómico benéfico a la nación y puede proteger lo que se tiene para tener mejor ambiente de desarrollo.\par
\end{justifying}
\vspace{\baselineskip}
\section*{Conclusión}
\addcontentsline{toc}{section}{Conclusión}
Como \begin{justifying}
   se ha expresado a lo largo de la redacción, el poder analizar las cuestiones jurídicas y legales de las distintas leyes aplicables en el territorio mexicano, especialmente aquellas leyes con relación al Desarrollo Sustentable nos permite
   tener en cuenta el panorama jurídico para tener un bien actuar en nuestros esfuerzos para nuevas legislaciones, nuevos actos y sobre todo, nuevas perspectivas a cosiderar en nuestras estrategias para integrar plenamente un Desarrollo Sustentable en
   territorio mexicano.\par
\end{justifying}

\newpage
% Referencias
\setcounter{secnumdepth}{0} %permite enumerar las secciones QUITAR SI QUIERES OG APA7
\renewcommand\refname{\textbf{Referencias}}
\bibliography{referencias}

\end{document}