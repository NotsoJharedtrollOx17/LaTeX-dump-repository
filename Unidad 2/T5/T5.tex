% Preámbulo
\documentclass[stu, 12pt, letterpaper, donotrepeattitle, floatsintext, natbib]{apa7}
\usepackage[utf8]{inputenc}
\usepackage{comment}
\usepackage{marvosym}
\usepackage{graphicx}
\usepackage{float}
\usepackage[normalem]{ulem}
\usepackage[spanish]{babel} 
\usepackage{lastpage} %para le formato que quiere la profe QUITAR SI QUIERES OG APA7
\usepackage{ragged2e} %para le formato que quiere la profe QUITAR SI QUIERES OG APA7
\usepackage{indentfirst} %para le formato que quiere la profe QUITAR SI QUIERES OG APA7
\usepackage{multirow,booktabs,setspace,caption} %formato de figuras APA
\DeclareCaptionLabelSeparator*{spaced}{\\[2ex]}
\captionsetup[figure]{textfont=it,format=plain,justification=justified,
  singlelinecheck=false,labelsep=spaced,skip=0pt}

\selectlanguage{spanish}
\useunder{\uline}{\ul}{}
\newcommand{\myparagraph}[1]{\paragraph{#1}\mbox{}\\}

\rfoot{Página \thepage \hspace{1pt} de \pageref{LastPage}}%QUITAR SI QUIERES OG APA7 
\rhead{} %QUITAR SI QUIERES OG APA7
\setcounter{secnumdepth}{3} %permite enumerar las secciones QUITAR SI QUIERES OG APA7
\setlength{\parindent}{1.27cm} %sangria forzada QUITAR SI QUIERES OG APA7

% Portada
\thispagestyle{empty}
\title{\Large Servicios Ambientales}
\author{Abraham Jhared Flores Azcona} % (autores separados, consultar al docente)
% Manera oficial de colocar los autores:
%\author{Autor(a) I, Autor(a) II, Autor(a) III, Autor(a) X}
\affiliation{Instituto Tecnológico de Tijuana}
\course{ACD-0908SC5C Desarrollo Sustentable}
\professor{M.C. Trinidad Castro Villa}
\duedate{27 de septiembre de 2021}

\renewcommand\labelitemi{$\bullet$}

\newcommand*\chem[1]{\ensuremath{\mathrm{#1}}}

\begin{document}
\maketitle


% Índices
\pagenumbering{arabic}
    % Contenido
\renewcommand\contentsname{Contenido}
\tableofcontents
\renewcommand{\listfigurename}{Figuras}
\listoffigures

% Cuerpo 
    %NOTA: PARA CITAR ESTILO "Merts (2003)" usar \cite{<nombre_cita_bib>}
    %                        "(Metz, 1978)" usar \citep{<nombre_cita_bib>}
\newpage
\section*{Introducción}
\addcontentsline{toc}{section}{Introducción}
Uno \begin{justifying}
    de los aspectos vitales para la subsitencia de cualquier organismo vivo es el saber el qué, cómo, cuándo, dónde obtener
    recursos para poder sobrevivir, sin embargo no conocemos con certeza el por qué es necesario. Conociendo los mecanismos del ecosistema
    nos permite tener una mejor idea y principalmente aquellos mecanismos con el nombre de \emph{Servicios Ambientales}, los cuales se describen
    en esta redacción.\par
\end{justifying}
\vspace{\baselineskip}
\section{Servicios Ambientales}
\subsection{Concepto}
Acorde \begin{justifying}       
    a \cite{conabio-no-date}
    son procesos ecológicos de los ecosistemas que aportan una gama importante de servicios gratuitos
    de los cuales los humanos dependen de ellos. Eston incluyen: mantenimiento de la calidad gaseaosa de la atmósfera;
    mejoramiento de la calidad del agua; control de los ciclos hidrológicos que reducen probabilidades de inundaciones y sequías;
    protección de las zonas costeras, etc.\par
\end{justifying}
Para \begin{justifying}
    la \cite{unece-no-date}
    estos servicios simplemente son los beneficios explicitos e implícitos que los ecosistemas aportan a los humanos; mencionan que este significado fue popularizado por
    la Evaluación del Ecosistema del Milenio (Millennium Ecosystem Assessment por sus siglas en inglés). Destacan de manera breve que estos servicios
    proveen la base para muchos productos, servicios y bienestar social; su valuación económica cubre las fuentes monetarias y no-monetarias de valor.\par
\end{justifying}
La \begin{justifying}
    \cite{national-wildlife-federation-no-date}
    destaca que los servicios ambientales también se conocen como \emph{Servicios del Ecosistema}. Desarrolla que
    el ecosistema nos provee cuatro tipos de servicios, listados a continuación:
    \begin{itemize}
        \item \emph{Provisión:} Beneficio o beneficios del ecosistema para la gente el cual se pueden extraer de la naturaleza.
        \item \emph{Regulación:} Beneficio o beneficios provisto por el ecosistema encargado de moderar fenómenos naturales.
        \item \emph{De Cultura:} Beneficio o beneficios intangibles que contribuye al desarrollo y avance cultural de la gente.
        \item \emph{De Soporte:} Son aquellos servicios fundamentales que permiten sostener a todos los organismos y elementos de los ecosistemas.
    \end{itemize}\par
\end{justifying}        
\vspace{\baselineskip}
\subsection{Importancia}
Para \begin{justifying}
    lo competente en la materia, el principal beneficio de los servicios ambientales recae en que nos permiten poder vivir en armonía con la naturaleza y el entorno. 
    Como lo mencionan \cite{hartel-2014},
    muchas sociedades rurales de entornos culturales tradicionales están caracterizados por un sistema bien desarrollado de conocimiento ecológico para evaular
    la calidad de los bienes y servicios. También destacan que estas sociedades han desarrollado una serie de reglas, normas y conductas colectivas que aseguran
    que los recursos críticos sean compartidos por todos los miembros de la comunidad. El poder recabar estas integraciones de culturas rurales con arraigo en el ecosistema
    nos permite tener una plantilla de acción para replicar en los centros urbanos para asegurar un mejor equilibrio con el ecosistema.\par
\end{justifying}
\vspace{\baselineskip}
\subsection{Ejemplos}
Retomando \begin{justifying}
    los 4 puntos destacados por la \cite{national-wildlife-federation-no-date}
    podemos describir ejemplos claros sobre estos servicios.
    \begin{itemize}
        \item \emph{Provisión:} Cómida, agua potable, madera, carbón vegetal, gás natural, petroleo, plantas medicinales, etc.
        \item \emph{Regulación:} Limpieza del aire y filtrado de agua, polinización, solidificación de suelos que permiten la prevención de la erosión, etc.
        \item \emph{De Cultura:} La construcción de conocimiento y la propagación de ideas, nacimiento de la creatividad por las interacciones con la naturaleza, recreación, etc.
        \item \emph{De Soporte:} Fotosíntesis, ciclado de nutrientes, creación de suelos, el ciclo del agua, etc.
    \end{itemize}\par  
\end{justifying}
\vspace{\baselineskip}
\section*{Conclusión}
\addcontentsline{toc}{section}{Conclusión}
El \begin{justifying}
    conocer de manera general los servicios provistos por el ecosistema permite considerar en nuestras acciones de subsistencia biólogica, el poder usar
    a la naturaleza con mesura. El poder escalar este acercamiento para los servicios básicos urbanos aceleraría de gran manera la integración del desarrollo sustentable
    el cuál culturas rurales nos muestran que es posible.\par
\end{justifying}
\newpage
% Referencias
\setcounter{secnumdepth}{0} %permite enumerar las secciones QUITAR SI QUIERES OG APA7
\renewcommand\refname{\textbf{Referencias}}
\bibliography{referencias}

\end{document}