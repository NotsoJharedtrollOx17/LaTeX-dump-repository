\documentclass[stu, 12pt, letterpaper, donotrepeattitle, floatsintext, natbib]{apa7}
\usepackage[utf8]{inputenc}
%\usepackage{fontspec} %paquete para usar la fuente Arial 12
%\usepackage{comment}
\usepackage{marvosym}
\usepackage{graphicx}
\usepackage{float}
\usepackage{amsmath}
\usepackage{mathtools}
\usepackage[normalem]{ulem}
\usepackage[spanish]{babel}
%\usepackage{listings} 
\usepackage{lastpage} %para le formato que quiere la profe QUITAR SI QUIERES OG APA7
\usepackage{ragged2e} %para le formato que quiere la profe QUITAR SI QUIERES OG APA7
\usepackage{indentfirst} %para le formato que quiere la profe QUITAR SI QUIERES OG APA7

%comando para ajustar la fuente Arial en todo el documento
%\setmainfont{Arial} %COMPILAR DOC CON XeLateX DOS VECES

%\DeclareCaptionLabelSeparator*{spaced}{\\[2ex]}
%\captionsetup[table]{textfont=it,format=plain,justification=justified,
%  singlelinecheck=false,labelsep=spaced,skip=1pt}

\selectlanguage{spanish}

\useunder{\uline}{\ul}{}
\newcommand{\myparagraph}[1]{\paragraph{#1}\mbox{}\\}

%\rfoot{Página \thepage \hspace{1pt} de \pageref{LastPage}}%QUITAR SI QUIERES OG APA7 
\rhead{} %QUITAR SI QUIERES OG APA7
\setcounter{secnumdepth}{3} %permite enumerar las secciones QUITAR SI QUIERES OG APA7
\setlength{\parindent}{1.27cm} %sangria forzada QUITAR SI QUIERES OG APA7

\renewcommand\labelitemi{\(\bullet\)}

\newcommand*\chem[1]{\ensuremath{\mathrm{#1}}}

\begin{document}
    %PORTADA
    \begin{titlepage}
        \begin{figure}[ht]
            \centering
            \includegraphics[width=15cm]{logosITT.png}
        \end{figure}
        \centering
        {\Large Tecnológico Nacional de México\\Instituto Tecnológico de Tijuana\par}
        \vspace{1cm}
        {\Large SCD-1015SC6C Lenguajes y Automatas I\par}
        \vspace{1cm}
        {\Large Unidad 2\par}
        \vspace{1.5cm}
        {\Large\bfseries Ejercicios de Clase\par}
        \vspace{2cm}
        {\large Lic. Gloria Leticia Morales Rios\par}
        \vfill
            {\large Abraham Jhared Flores Azcona, 19211640\par}
        \vfill
        {\large 4 de abril de 2022}
    \end{titlepage}

% Índices
\pagenumbering{arabic}
    % Contenido
% \renewcommand\contentsname{Contenido}
% \tableofcontents

% Cuerpo 
    %NOTA: PARA CITAR ESTILO "Merts (2003)" usar \cite{<nombre_cita_bib>}
    %                        "(Metz, 1978)" usar \citep{<nombre_cita_bib>}
\newpage
\section*{Introducción}
\begin{justifying}
El manejo de las expresiones regulares generalmente tiende a ser muy abstracto sin previa práctica analítica de dichas operaciones.
En esta breve redacción mostramos el desarrollo de los ejercicios solicitados respecto a \emph{regex} para ilustrar el álgebra
que rige a dichas expresiones.\par
\end{justifying}
\vspace{\baselineskip}
\section*{Ejercicios de clase}
\subsection*{Ejercicio 1}
\begin{justifying}
Sea el alfabeto = \(\{a, b, c\}\), resolver \((a+b)c^*\).
\end{justifying}
\begin{equation*}
    \begin{align}
        a & = & \{a\}\\
        b & = & \{b\}\\
        c & = & \{c\}\\
        r & = & \{a+b\} = \{a, b\}\\
        j & = & c^* = \{\epsilon, c, cc, ccc, \dots\}\\
        (a+b)c^* & = & \{\epsilon, a, b, ac, bc, \dots\}
    \end{align}
\end{equation*}
\vspace{\baselineskip}
\subsection*{Ejercicio 2}
\begin{justifying}
Resolver la expresión regular \((00)^{*}+\left(1(11)^{*}\right)\) definida en \(\Sigma=\{0, 1\}\).\par
\vspace{\baselineskip}
Permitamos que \(A=\left(00\right)^*\), \(B=\left(11\right)^{*}\) y \(C=1B\) para resolver dichas \emph{regex}
modularmente.
\end{justifying}
\begin{equation*}
    \setlength{\jot}{10pt}
    \begin{align}
        \begin{split}
            A ={}& (00)^{*} {}=\{\varepsilon, 00, 0000, \dots\} \\
        \end{split}\\
        \begin{split}
            B ={}& \left(11\right)^{*} {}= \{\varepsilon, 11, 1111, \dots\} \\
        \end{split}\\
        \begin{split}
            C ={}& 1B {}= 1\{\varepsilon, 11, 1111, \dots\} \\
              ={}& \{1, 111, 11111, \dots\}\\
        \end{split}
    \end{align}
\end{equation*}\par
\begin{justifying}
Finalmente realizamos la operación \(A+C\) para obtener el resultado del \emph{regex} solicitado.
\end{justifying}
\[A + C = \{\epsilon, 00, 0000, \dots\} \cup \{1, 111, 11111, \dots\} = \{\epsilon, 00, 0000, \dots, 1, 111, 11111, \dots\}\]
\newpage
\subsection*{Ejercicio 3}
\begin{justifying}
Resolver \((a^{*}c)+(ab^{*})\).\par
\vspace{\baselineskip}
Similar al ejercicio anterior, permitamos que \(r=\left(a^{*}c\right)\) y que \(s=\left(ab^{*}\right)\) para resolver modularmente.
\end{justifying}
\begin{equation*}
    \setlength{\jot}{10pt}
    \begin{align*}
        \begin{split}
            r ={}& \left(a^{*}c\right) {}= \{\varepsilon, a, aa, \dots\}c\\
              ={}& \{c, ac, aac, \dots\}
        \end{split}\\
        \begin{split}
            s ={}& \left(ab^{*}\right) {}= a\{\varepsilon, b, bb ,\dots\}\\
              ={}& \{a, ab, abb, \dots\}
        \end{split}
    \end{align*}
\end{equation*}\par
\begin{justifying}
Por lo que el resultado de la \emph{regex} solicitada se obtiene por \(r+s\).
\end{justifying}
\[s+r = \{c, ac, aac, \dots\}\cup\{a, ab, abb, \dots\}=\{a, c, ab, ac, aac,abb,\dots\}\]
\vspace{\baselineskip}
\section*{Conclusión}
\begin{justifying}
El realizar los ejercicios analíticos de las expresiones regulares permite familiarizarnos con sus mecanismos
que nos permiten aplicarlas de mewjor manera en nuestras aplicaciones programáticas.\par
\end{justifying}
\end{document}