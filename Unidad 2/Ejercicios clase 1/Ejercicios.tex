\documentclass[stu, 12pt, letterpaper, donotrepeattitle, floatsintext, natbib]{apa7}
\usepackage[utf8]{inputenc}
%\usepackage{fontspec} %paquete para usar la fuente Arial 12
%\usepackage{comment}
\usepackage{marvosym}
\usepackage{graphicx}
\usepackage{float}
\usepackage{amsmath}
\usepackage{mathtools}
\usepackage[normalem]{ulem}
\usepackage[spanish]{babel}
%\usepackage{listings} 
\usepackage{lastpage} %para le formato que quiere la profe QUITAR SI QUIERES OG APA7
\usepackage{ragged2e} %para le formato que quiere la profe QUITAR SI QUIERES OG APA7
\usepackage{indentfirst} %para le formato que quiere la profe QUITAR SI QUIERES OG APA7

%comando para ajustar la fuente Arial en todo el documento
%\setmainfont{Arial} %COMPILAR DOC CON XeLateX DOS VECES

%\DeclareCaptionLabelSeparator*{spaced}{\\[2ex]}
%\captionsetup[table]{textfont=it,format=plain,justification=justified,
%  singlelinecheck=false,labelsep=spaced,skip=1pt}

\selectlanguage{spanish}

\useunder{\uline}{\ul}{}
\newcommand{\myparagraph}[1]{\paragraph{#1}\mbox{}\\}

%\rfoot{Página \thepage \hspace{1pt} de \pageref{LastPage}}%QUITAR SI QUIERES OG APA7 
\rhead{} %QUITAR SI QUIERES OG APA7
\setcounter{secnumdepth}{3} %permite enumerar las secciones QUITAR SI QUIERES OG APA7
\setlength{\parindent}{1.27cm} %sangria forzada QUITAR SI QUIERES OG APA7

\renewcommand\labelitemi{\(\bullet\)}

\newcommand*\chem[1]{\ensuremath{\mathrm{#1}}}

\begin{document}
    %PORTADA
    \begin{titlepage}
        \begin{figure}[ht]
            \centering
            \includegraphics[width=15cm]{logosITT.png}
        \end{figure}
        \centering
        {\Large Tecnológico Nacional de México\\Instituto Tecnológico de Tijuana\par}
        \vspace{1cm}
        {\Large SCD-1015SC6C Lenguajes y Automatas I\par}
        \vspace{1cm}
        {\Large Unidad 2\par}
        \vspace{1.5cm}
        {\Large\bfseries Ejercicios de Clase\par}
        \vspace{2cm}
        {\large Lic. Gloria Leticia Morales Rios\par}
        \vfill
            {\large Abraham Jhared Flores Azcona, 19211640\par}
        \vfill
        {\large 4 de abril de 2022}
    \end{titlepage}

% Índices
\pagenumbering{arabic}
    % Contenido
% \renewcommand\contentsname{Contenido}
% \tableofcontents

% Cuerpo 
    %NOTA: PARA CITAR ESTILO "Merts (2003)" usar \cite{<nombre_cita_bib>}
    %                        "(Metz, 1978)" usar \citep{<nombre_cita_bib>}
\newpage
\section*{Resumen}
\begin{justifying}
El video adjunto explica a grandes rasgos las propiedades formales de las expresiones regulares, las cuales utilizamos
para esta actividad. Con dicha explicacion inicial, se procede a ejemplificar con soluciones a los ejercicios\par
\end{justifying}
\vspace{\baselineskip}
\section*{Ejercicio propuesto}
\begin{justifying}
Indica \emph{Verdadero} o \emph{Falso}, jutificando:\par
\begin{enumerate}
    \item \((a\cdot b)^{*}=a^{*}\cdot b^{*}\).
    \item \((a+b)^{*}=a^{*}+b^{*}\).
    \item \((a+b)^{*}=\{a, b\}^{*}\).
    \item \(\left(a^{*}\cdot b^{*}\right)^{*}=(a+b)^{*}\).
\end{enumerate}
\subsection*{1)}
\begin{justifying}
    Permitamos que \(A=(a\cdot b)^{*}\) y que \(B=a^{*}\cdot b^{*}\) para comprobar de manera mas cómoda si \(A=B\).
    Empezando por \(A\)
\end{justifying}
\begin{equation*}
    \begin{align*}
        A = {}& (a\cdot b)^{*} = (ab)^{*} = \{\lambda, ab, abab, \dots\}. \\
        B = {}& a^{*}\cdot b^{*} = \{\lambda, a , aa, \dots\} \cdot  \{\lambda, b , bb, \dots\} \\
          = {}& \{\lambda, a, aa, aab, aabb, b, bb, \dots\}.
    \end{align*}
\end{equation*}
\begin{justifying}
    Debido a que \(a,b \notin A\), \(A=B\) es FALSO.\par
\end{justifying}
\vspace{\baselineskip}
\subsection*{2)}
\begin{justifying}
    Permitamos que \(A=(a+b)^{*}\) y que \(B=a^{*}+b^{*}\). Empezando por \(A\)
\end{justifying}
\begin{equation*}
    \begin{align*}
        A = {}& (a+b)^{*} = \{a, b\}^{*} = \{\lambda, a, b, aa, ab, ba, bb, aaa, \dots\}. \\
        B = {}& a^{*}+b^{*} = \{\lambda, a , aa, \dots\} + \{\lambda, b , bb, \dots\} \\
          = {}& \{\lambda, a, aa, b, bb, \dots\}.
    \end{align*}
\end{equation*}
\begin{justifying}
    Debido a que \(a^{*}\) nunca contendrá elementos de \(b^{*}\) y viceversa, \(A=B\) es FALSO.\par
\end{justifying}
\vspace{\baselineskip}
\subsection*{3)}
\begin{justifying}
    Retomando \(A\) de 2), tenemos que \((a+b)^{*} = \{a, b\}^{*}\) lo cual es derivado de la definición de \(+\)
    el cual indica que \(L_1+L_2=\{L_1, L_2\}\), por lo que 3) es VERDADERO.\par
\end{justifying}
\vspace{\baselineskip}
\subsection*{4)}
    Retomando \(B\) de 1), tenemos que \(a^{*}\cdot\, b^{*} = \{\lambda, a, aa, aab, aabb, b, bb, \dots\}\) 
    por lo que \(\left(a^{*}\cdot b^{*}\right)^{*} = \{\lambda, a, aa, aab, aabb, b, bb, \dots\}^{*}\), y 
    retomando 3), se puede notar que ambos conjuntos son iguales, por lo que 4) es VERDADERO.\par
\vspace{\baselineskip}
\end{document}