\documentclass[stu, 12pt, letterpaper, donotrepeattitle, floatsintext, natbib]{apa7}
\usepackage[utf8]{inputenc}
%\usepackage{fontspec} %paquete para usar la fuente Arial 12
\usepackage{comment}
\usepackage{marvosym}
\usepackage{graphicx}
\usepackage{float}
\usepackage[normalem]{ulem}
\usepackage[spanish]{babel} 
\usepackage{lastpage} %para le formato que quiere la profe QUITAR SI QUIERES OG APA7
\usepackage{ragged2e} %para le formato que quiere la profe QUITAR SI QUIERES OG APA7
\usepackage{indentfirst} %para le formato que quiere la profe QUITAR SI QUIERES OG APA7

\setcounter{secnumdepth}{0} %permite enumerar las secciones QUITAR SI QUIERES OG APA7

%comando para ajustar la fuente Arial en todo el documento
%\setmainfont{Arial} %COMPILAR DOC CON XeLateX DOS VECES

%\DeclareCaptionLabelSeparator*{spaced}{\\[2ex]}
%\captionsetup[table]{textfont=it,format=plain,justification=justified,
% singlelinecheck=false,labelsep=spaced,skip=1pt}

\selectlanguage{spanish}

%\useunder{\uline}{\ul}{}
%\newcommand{\myparagraph}[1]{\paragraph{#1}\mbox{}\\}

%\rfoot{Página \thepage \hspace{1pt} de \pageref{LastPage}}%QUITAR SI QUIERES OG APA7 
\rhead{} %QUITAR SI QUIERES OG APA7
\setcounter{secnumdepth}{3} %permite enumerar las secciones QUITAR SI QUIERES OG APA7
\setlength{\parindent}{1.27cm} %sangria forzada QUITAR SI QUIERES OG APA7

\renewcommand\labelitemi{$\bullet$}

\newcommand*\chem[1]{\ensuremath{\mathrm{#1}}}

\begin{document}
    %PORTADA
    \begin{titlepage}
        \begin{figure}[ht]
            \centering
            \includegraphics[width=15cm]{logosITT.png}
        \end{figure}
        \centering
        {\Large Tecnológico Nacional de México\\Instituto Tecnológico de Tijuana\par}
        \vspace{1cm}
        {\Large SCD-1011SC6C Ingeniería de Software\par}
        \vspace{1cm}
        {\Large Unidad 2\par}
        \vspace{2cm}
        {\Large\bfseries Caso práctico 3\par}
        \vspace{2cm}
        {\large Dra. Martha Elena Pulido\par}
        \vfill
            {\large Abraham Jhared Flores Azcona, 19211640\par}
        \vfill
        {\large 16 de marzo de 2022}
    \end{titlepage}

% Índices
\pagenumbering{arabic}
    % Contenido
\renewcommand\contentsname{Contenido}
\tableofcontents

% Cuerpo 
    %NOTA: PARA CITAR ESTILO "Merts (2003)" usar \cite{<nombre_cita_bib>}
    %                        "(Metz, 1978)" usar \citep{<nombre_cita_bib>}
\newpage
\section{Entradas, procesamiento y salidas del sistema de rastreo de paquetes de UPS}
El \begin{justifying}
   proceso completo consiste en que el cliente pida lo que requiere, a su paquete se le asigna
   un código de barras donde se le adjunta la información relevante, se transporta y al momento
   del envio se puede rastrear para que finalmente el cliente pueda proporcionar una firma digital 
   que confirme la llegada y recepción de su envio.\par
\end{justifying}
\vspace{\baselineskip}
\section{Tecnologías empleadas}
La \begin{justifying}
    herramientas utilizadas son el \emph{DIAD, Document Exchange e Inventory Express}. El primero es simplemente un dispositivo que
    captura la información de entrega y mantiene dicha información que se puede monitorear por medio de internet.
    El segundo se utiliza para intercambiar documentos en la red y la tercer herramienta agiliza el control de los inventarios 
    asignando fechas específicas y almacenando paquetes en tiempos muy cortos.\par
\end{justifying}
\vspace{\baselineskip}
\section{Relación de las tecnologías con la estrategia de negocios de UPS}
Sus \begin{justifying}
    tecnologías, específicamente las explicadas en el escrito adjunto muestran que UPS quiere posicionarse
    como una empresa de logística (específicamente de paquetería) que le provee a sus clientes comodidad y servicios baratos.\par
\end{justifying}
\vspace{\baselineskip}
\section{¿Qué sucedería si no se contara con esas tecnologías?}
Para \begin{justifying}
    ellos, sería un caso catastrófico porque nunca pudiesen haberse posicionado a tiempo,
    específicamente con las innovaciones relacionadas en la red. Debido a que dichos avances también tuvieron
    que ver con golpes de suerte de apropiación tecnológica, puede haber una posibilidad de igual manera serían la misma
    paquetería que conocemos, pero con servicios más cutres.\par
\end{justifying}
\end{document}