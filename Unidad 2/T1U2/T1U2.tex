\documentclass[letterpaper, 12pt]{article}
\usepackage[letterpaper, top=2.5cm, bottom=2.5cm, left=3cm, right=3cm]{geometry} %margenes
\usepackage[utf8]{inputenc} %manejo de caracteres especiales
\usepackage[spanish]{babel} %manejo de encabezados de inglés a español
\usepackage{fancyhdr} %formato de los encabezados de página
\usepackage{ragged2e} %alineado real justficado
\usepackage{graphicx} %manejo de imagenes
\usepackage{amsmath} %manejo de notación matemática
\usepackage{mathtools} %manejo de notación matemática
\usepackage{blindtext} %texto de relleno
\usepackage{cancel} %permite la simbolización de cancelación de terminos
\usepackage{enumitem}[shortlabels] %listas con letras
\usepackage{amssymb} %manejo de simbología matematica
\usepackage{float}

\pagestyle{fancy}
\fancyhf{}
\rfoot{}

\arraycolsep=5pt %espaciado entre columnas de un arreglo

\begin{document}
\thispagestyle{fancy}
\lhead{\textbf{Tarea 1, U2}}
\rhead{\textbf{13 de abril de 2021}}
\section*{Independencia y Dependencia Lineal}
\subsection*{Determinar \(W(x)\) para saber la independencia ó dependencia lineal de los siguientes conjuntos de funciones:}
\justify
{\large
\(\{x^a,\, x^b\}\: \textbf{(1.1)}\)\\\newline
\(\{1-x,\, x^2-2,\, x^2+3x\}\: \textbf{(1.2)}\)\\\newline
{\large \textbf{ • Solución para (1.1):}\\
Para simplificar la redacción se permite que \textbf{(1.1)} sea \(\theta\).
\begin{equation*}
    \begin{aligned}
W(\theta)(x)=\begin{array}{|cc|}
    x^a & x^b \\
    ax^{a-1} & bx^{b-1}
\end{array}= \left(x^a\cdot bx^{b-1}\right)-\left(ax^{a-1}\cdot x^b\right) =& \\[5pt]
=bx^{a+b-1}-ax^{a+b-1}=(b-a)x^{a+b-1}\therefore&\\[5pt]
\therefore\, \theta \text{ tiene dependencia lineal si } a=b.&
    \end{aligned}
\end{equation*}}
{\large\justify
\textbf{• Solución para (1.2):}\\
Para simplificar la redacción se permite que \textbf{(1.2)} sea \(\theta\).
\begin{equation*}
    \begin{aligned}
W(\theta)(x)=\begin{array}{|ccc|}
    1-x & x^2-2 & x^2+3x\\
    -1 & 2x & 2x+3\\
    0 & 2 & 2
\end{array}=0((x^2-2)(2x+3)-(2x)(x^2+3x))&\\[5pt]
-2((1-x)(2x+3))-(-1)(x^2+3x))+2((1-x)(2x)-(-1)(x^2-2))=&\\[5pt]
=-2(2x+3-2x^2-3x+x^2+3x)+2(2x-2x^2+x^2-2)=& \\[5pt]
=4x^2-4x-6-2x^2+4x-4=2x^2-10\, \therefore& \\[5pt]
\therefore \theta \text{ tiene independencia lineal.}&
    \end{aligned}
\end{equation*}
\end{document}