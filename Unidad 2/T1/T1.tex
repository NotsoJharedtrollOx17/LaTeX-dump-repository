% Preámbulo
\documentclass[stu, 12pt, letterpaper, donotrepeattitle, floatsintext, natbib]{apa7}
\usepackage[utf8]{inputenc}
\usepackage{comment}
\usepackage{marvosym}
\usepackage{graphicx}
\usepackage{float}
\usepackage[normalem]{ulem}
\usepackage[spanish]{babel} 
\usepackage{lastpage} %para le formato que quiere la profe QUITAR SI QUIERES OG APA7
\usepackage{ragged2e} %para le formato que quiere la profe QUITAR SI QUIERES OG APA7
\usepackage{indentfirst} %para le formato que quiere la profe QUITAR SI QUIERES OG APA7

\selectlanguage{spanish}
\useunder{\uline}{\ul}{}
\newcommand{\myparagraph}[1]{\paragraph{#1}\mbox{}\\}

\rfoot{Página \thepage \hspace{1pt} de \pageref{LastPage}}%QUITAR SI QUIERES OG APA7 
\rhead{} %QUITAR SI QUIERES OG APA7
\setcounter{secnumdepth}{3} %permite enumerar las secciones QUITAR SI QUIERES OG APA7
\setlength{\parindent}{1.27cm} %sangria forzada QUITAR SI QUIERES OG APA7

% Portada
\thispagestyle{empty}
\title{\Large Ecosistema}
\author{Abraham Jhared Flores Azcona} % (autores separados, consultar al docente)
% Manera oficial de colocar los autores:
%\author{Autor(a) I, Autor(a) II, Autor(a) III, Autor(a) X}
\affiliation{Instituto Tecnológico de Tijuana}
\course{ACD-0908SC5C Desarrollo Sustentable}
\professor{M.C. Trinidad Castro Villa}
\duedate{14 de septiembre de 2021}

\renewcommand\labelitemi{$\bullet$}

\begin{document}
\maketitle


% Índices
\pagenumbering{arabic}
    % Contenido
\renewcommand\contentsname{Contenido}
\tableofcontents

% Cuerpo 
    %NOTA: PARA CITAR ESTILO "Merts (2003)" usar \cite{<nombre_cita_bib>}
    %                        "(Metz, 1978)" usar \citep{<nombre_cita_bib>}
\newpage
\section*{Introducción}
\addcontentsline{toc}{section}{Introducción}
Uno \begin{justifying}
    de los conceptos que están relacionados al Desarrollo Sustentable es aquel del Ecosistema; si dicha palabra la separamos en ``Eco'' y en ``Sistema'' podemos definir que 
    \emph{Eco} es un prefíjo que refiere al planeta Tierra como nuestra casa y todo lo relacionado en términos biológicos \citep{unknown-author-no-date},
     mientras que \emph{Sistema} es un conjunto de elementos organizados que interactuan entre sí, que funcionan como un todo, y que tienen un determinado objetivo \citep{alburquerque-2018}. 
     Esto nos permite inferir que vulgarmente el ecosistema es un conjunto de elementos organizados en el planeta Tierra, sin embargo ¿cómo se define exactamente un ecosistema y para qué es necesario? En esta
     breve investigación se exponen tres conceptos del tema en cuestión, así como los elementos relevantes y su importancia para precisar la palabra como una futura referencia dentro y fuera de la materia.\par
\end{justifying}
\vspace{\baselineskip}
\section{Ecosistema}
\subsection{Concepto}
Acorde \begin{justifying}
    a \cite{rintoul-no-date}
      es una comunidad de organismos vivos y sus interacciones con el entorno no-vivo. También especifican que existen tres categorias generales basadas en su entorno general:
      agua, oceano y terrestre. A grandes rasgos, la relación y acciones de los organismos vivos dentro de su entorno inmediato.\par
\end{justifying}
Para \begin{justifying}
    \cite{khan-academy-no-date}
    los ecosistemas son las comunidades de organismos que viven juntos en combinacióm con sus entorno físico. También mencionan que los conceptos de comunidad y ecosistema van de la mano.\par
\end{justifying}
Finalmente, \begin{justifying}
    para \cite{conabio-no-date}
    el ecosistema es el conjunto de especias de un área determinada que interactúan entre ellas y con su ambiente abiótico; mediante procesos como la depredación, el parasitísmo, la competencia
    y la simbiosis, y con sus ambiente al desintegrarse y volver a ser parte del ciclo de energía y nutrientes. Se agrega que las especias del ecosistema, incluyendo bacterias, hongos, plantas y animales
    dependen unas de otras y que las relaciones entre las especies y su medio, resultan en el flujo de materia y energía del ecosistema.\par
\end{justifying}
Como \begin{justifying}
    se remarca en el desarrollo de los tres conceptos, dicha triada coincide en que un ecosistema es una relación interdependiente de los organismos vivos con su entorno inmediato no-vivo.\par
\end{justifying}
\vspace{\baselineskip}
\subsection{Elementos que lo Integran}
Teniendo \begin{justifying}
claro un concepto general, ahora podemos describir los elementos que conforman al ecosistema. Es de vital importancia destacar que los ecosistemas son complejos \citep{maayan-2017} con muchas partes interactuando entre sí, por lo que
diseccionar los roles de estos elementos interactivos puede ser un reto, aparte que son expuestos rutinariamente a cambios ó perturbanciones del entorno que afectan sus composiciones \citep{rintoul-no-date}, por lo que el alcance de los elementos descritos
es una generalización.\par
\end{justifying}
\vspace{\baselineskip}
\subsubsection{Bióticos}
Estos \begin{justifying}
    factores incluyen las interacciones entre los organismos, como las enfermedades, caza, parasitísmo, y competencia entre distintas especies o dentro de solo una especia; también se consideran a los organismos mismos \citep{spanner-no-date}. Estos son:
    \begin{itemize}
        \item \emph{Productores:} Son los organismos que convierten factores abióticos en comida. Generalmente, dicho alimento es producido por medio de la fotosíntesis. Estos organismos son conocidos como autótropos.
        \item \emph{Consumidores:} Son aquellos animales que no producen su propia comida. En vez de ello, consumen a los productores o a otros consumidores para obtener energía de alimento. Son también conocidos como heterotrópos.
        \item \emph{Descomponedores:} Son los organismos que pueden desmoronar materia orgánica de plantas y animales fallecidos en componentes inorgánicos necesarios para la vida. Conocidos como saprótopos.
    \end{itemize}
\end{justifying}
\vspace{\baselineskip}
\subsubsection{Abióticos}
Contrario \begin{justifying}
    al elemento anterior; incluye los factores físico-químicos. Estos factores influyen en otros factores abióticos, principalmente la variedad y abundancia de la vida en un ecosistéma. Sin estos factores
    los factores bióticos no podrían alimentarse, crecer y reproducirse \citep{spanner-no-date}. Algunos de los factores abióticos significativos son los siguientes:\par
    \begin{itemize}
        \item \emph{Luz solar:} La fuente energética más grande. Provee a los autótropos la energía para la fotosíntesis y afecta la temperatura.
        \item \emph{Oxígeno: }Permite a los organismos respirar y liberar energía de sus alimentos.
        \item \emph{Temperatura:} Dependiendo de los rangos de temperatura, esto dictamina el cómo sobrevivír a los distintos organismos. También afecta el metabolismo del organismo.
        \item \emph{Viento:} Puede mover factores abióticos, dispersar semillas y expandir incendos, así como el poder afectar la temperatura, etc.
        \item \emph{Agua:} Dependiendo del ecosistema, este puede ser de mayor o menor abundancia; en los que dicho recurso es escaso, los organismos desarrollan hábitos y comportamientos que les permiten sobrevivir cultivando y almacenando agua de manera eficiente.
        \item \emph{Corrientes oceánicas:} Revuelven en el movimiento del agua, que facilita el movimiento de factores bióticos y abióticos, así como el afectar la temperatura del aguar y el clíma.
        \item \emph{Nutrientes:} Son los elementos químicos que los organismos requieren para comer y crecer.
    \end{itemize}       
\end{justifying}
\vspace{\baselineskip}
\subsection{Importancia}
Mas \begin{justifying}
    allá de la importancia inmediata como concepto de la materia, el ecosístema (como se ha descrito a lo largo de esta redacción) es el ente
    intermediario por el cual todos los factores bióticos y abióticos interactuan entre sí; esto nos incluye a los seres humanos. Lo anterior es de alta relevancia ya que
    cualquier factor que altere el equilibrio de este sistema dinámico puede desencadenar otros sucesos \citep{rintoul-no-date} que se pueden explicar de manera conjunta como
    un sistema \citep{alburquerque-2018} y así poder apreciar de una manera más detalla la influencia de los factores que intervienen para tomarlos en cuenta en el estudio del Desarrollo Sustentable
    y posiblemente, soluciones a dichos desbalances.\par
\end{justifying}
\vspace{\baselineskip}
\section*{Conclusión}
\addcontentsline{toc}{section}{Conclusión}
Un \begin{justifying}
    aspecto de suma importancia para el estudio del Desarrollo Sustentable es aquél del ecosistema. Poder contrastar sus distintas definiciones, elementos y su respectiva importancia
   permite dar una gran pauta para tener en perspectiva la monumentalidad \citep{solar-sands-2021} de dichos sistemas para poder estudiarlos y preservalos con la debida escala de influencia en nuestra vida diaria así como en lo competente
   a la materia.\par 
\end{justifying}

\newpage
% Referencias
\setcounter{secnumdepth}{0} %permite enumerar las secciones QUITAR SI QUIERES OG APA7
\renewcommand\refname{\textbf{Referencias}}
\bibliography{referencias}

\end{document}