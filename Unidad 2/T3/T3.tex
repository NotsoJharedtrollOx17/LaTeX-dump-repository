% Preámbulo
\documentclass[stu, 12pt, letterpaper, donotrepeattitle, floatsintext, natbib]{apa7}
\usepackage[utf8]{inputenc}
\usepackage{comment}
\usepackage{marvosym}
\usepackage{graphicx}
\usepackage{float}
\usepackage[normalem]{ulem}
\usepackage[spanish]{babel} 
\usepackage{lastpage} %para le formato que quiere la profe QUITAR SI QUIERES OG APA7
\usepackage{ragged2e} %para le formato que quiere la profe QUITAR SI QUIERES OG APA7
\usepackage{indentfirst} %para le formato que quiere la profe QUITAR SI QUIERES OG APA7

\selectlanguage{spanish}
\useunder{\uline}{\ul}{}
\newcommand{\myparagraph}[1]{\paragraph{#1}\mbox{}\\}

\rfoot{Página \thepage \hspace{1pt} de \pageref{LastPage}}%QUITAR SI QUIERES OG APA7 
\rhead{} %QUITAR SI QUIERES OG APA7
\setcounter{secnumdepth}{3} %permite enumerar las secciones QUITAR SI QUIERES OG APA7
\setlength{\parindent}{1.27cm} %sangria forzada QUITAR SI QUIERES OG APA7

% Portada
\thispagestyle{empty}
\title{\Large Biodiversidad}
\author{Abraham Jhared Flores Azcona} % (autores separados, consultar al docente)
% Manera oficial de colocar los autores:
%\author{Autor(a) I, Autor(a) II, Autor(a) III, Autor(a) X}
\affiliation{Instituto Tecnológico de Tijuana}
\course{ACD-0908SC5C Desarrollo Sustentable}
\professor{M.C. Trinidad Castro Villa}
\duedate{21 de septiembre de 2021}

\renewcommand\labelitemi{$\bullet$}

\begin{document}
\maketitle


% Índices
\pagenumbering{arabic}
    % Contenido
\renewcommand\contentsname{Contenido}
\tableofcontents

% Cuerpo 
    %NOTA: PARA CITAR ESTILO "Merts (2003)" usar \cite{<nombre_cita_bib>}
    %                        "(Metz, 1978)" usar \citep{<nombre_cita_bib>}
\newpage
\section*{Introducción}
\addcontentsline{toc}{section}{Introducción}
Dentro \begin{justifying}
    del estudio de los ecosistemas, uno de los elementos más importantes son los individuos que habitan en ella, los que se conocen como los organismos.
    Estos junto a las interacciones con el entorno permiten estudiarlos como un ente que trabaja para un fín especifico. Sin embargo, las generalizaciones
    terminan depreciando la abundancia de variedades de los organismos como de los mismos escosistemas; esto es el rubro del que se encarga la biodiversidad la cual
    se expone en esta breve redacción así como los aspectos socio-económicos de esta.\par
\end{justifying}
\section{Biodiversidad}
\subsection{Concepto}
Dentro \begin{justifying}
    del artículo en linea de \cite{hancock-no-date} para el World Wildlife Fund (El Fondo Mundial para la Vida Silvestre), 
    la biodiversidad son todos los tipos distintos de vida que uno se pueda encontrar en un area; la variedad de animales,
    plantas, fungi, e inclusve los microorganismos como las bacterias que forman parte de nuestro mundo natural. Destacana que 
    la biodiversidad soporta todo lo necesario para sobrevivir: comida, agua limpia, medicina y refugio debido a que los ecosistemas
    albergan a los organismos que conforman la biodiversidad.\par
\end{justifying}
Otros \begin{justifying}
    conceptos de la palabra definen que la biodiversidad es la variedad de vidan en el planeta Tierra en todos los niveles,
    desde los genes hasta los ecosistemas, y puede abarcar los procesos evolutivos, ecológicos y culturales que sostienen a la vida.
    Se considera que los humanos y sus diversidades culturales también son parte de la biodiversidad ya que los humanos y sus conocimientos,
    usos y creencias influyen y son influidas por los ecosistemas haciendo que toda la biodiversidad y los enlaces culturales de los lugares en donde
    vivimos son importantes a nuestro bienestar y por ende, permiten mantener un planeta diverso y saludable \citep{american-museum-of-natural-history-no-date}.\par%citar al de amnh
\end{justifying}
Para \begin{justifying}
    \cite{benn-2010} dentro de una publicación de la UNESCO, %citar a la unesco
    la biodiversidad es la variedad de vieda en la Tierra; incluye a todos los organismos, especies y poblaciones; las variaciones genéticas
    en ellas; y sus ensambles complejos de comunidades y ecosistemas. También menciona que se tienen tres niveles de biodiversidad, los cuales se describen
    a continuación:
    \begin{itemize}
        \item \emph{Diversidad Genética:} Son todos los genes contenidos en las especies vivas, incluyendo plantas individuales, animales, fungi, y microorganismos.
        \item \emph{Diversidad de especies:} Son todas las distintas especies, así como las diferencias dentro de ellas y entre otras.
        \item \emph{Diversidad de ecosistemas:} Son todos los hábitats diferentes, comunidades biológicas y procesos ecológicos, así como las variaciones entre ecosistemas individuales.
    \end{itemize}\par
\end{justifying}
\vspace{\baselineskip}
\subsection{Aspecto Ecológico}
El \begin{justifying}
    lo respectivo a la ecología, el tener en cuenta la biodiversidad en nuestro análisis de las relaciones
    de los organismos vivos y su entorno físico (especificamente del ecosistema) enriquece las consideraciones de estudio para 
    mantener un planeta diverso y saludable, como bien lo menciona \cite{american-museum-of-natural-history-no-date}.\par %citar al de amnh
\end{justifying}
\vspace{\baselineskip}
\subsection{Aspecto Económico}
Teniendo \begin{justifying}
    en cuenta la biodiversidad, se deriva el concepto de la \emph{Bioeconomia} la cual se define como la producción
    de recursos biológicamente renovables y sus conversiones en comida, alimento, productos bio-basados y bioenergía. Específicamente,
    la biodiversidad en la Bioeconomía recae en que las variaciones de los organismos dentro de un ecosistema terminan dictaminando las posibles
    restricciones considerablesa dentro de los objetivos de la Bioeconomía: asegurar una seguridad de alimento; administrar la sustentabilidad de recursos naturales
    limitados y en escacez; reducir la dependencia en recursos no-renovables; mitigar y adaptarse a los efectos del cambio climatico; y crear
    trabajos y mantener la competitividad de las sociedades \citep{szekacs-2017}.\par %citar al de springer 
\end{justifying}
\vspace{\baselineskip}
\subsection{Aspecto Científico}
Una \begin{justifying}
    de las principales consideraciones en los estudios cientificos es que se requiere de cantidades sustanciales de información para estimas mediciones básicas. Esto es debido a que
    para tener datos fidedignos de una variedad inmensa de especies, es muy necesario considerar la localización de la recolección de datos, la fecha de recolección y el nombre del organismo
    recolectado u observado. También es necesario tener en cuenta la información de la taxonomía de la especie y las relaciones evolutivas entre especies \citep{mickevich-1999}.\par %citar al de mary
\end{justifying}
\vspace{\baselineskip}
\section*{Conclusión}
\addcontentsline{toc}{section}{Conclusión}
Para \begin{justifying}
    comprender de mejor manera los mecanismos del ecosistema, es importante tener en cuenta la biodiversidad para poder comprender de mejor manera los fines del Desarrollo Sustentable
    y mediar los aspectos descritos para mejores estudios con estos enfoques.\par
\end{justifying}

\newpage
% Referencias
\setcounter{secnumdepth}{0} %permite enumerar las secciones QUITAR SI QUIERES OG APA7
\renewcommand\refname{\textbf{Referencias}}
\bibliography{referencias}

\end{document}