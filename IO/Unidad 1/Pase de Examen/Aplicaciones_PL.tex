\documentclass[letterpaper, 12pt]{article}
\usepackage[letterpaper, top=2.5cm, bottom=2.5cm, left=3cm, right=3cm]{geometry} %margenes
\usepackage[utf8]{inputenc} %manejo de caracteres especiales
\usepackage[spanish]{babel} %manejo de encabezados de inglés a español
\usepackage{fancyhdr} %formato de los encabezados de página
\usepackage{ragged2e} %alineado real justficado
\usepackage{graphicx} %manejo de imagenes
\usepackage{amsmath} %manejo de notación matemática
\usepackage{mathtools} %manejo de notación matemática
\usepackage{blindtext} %texto de relleno
\usepackage{amssymb}
\usepackage[backend=biber]{biblatex}\addbibresource{bibliografia.bib}

\pagestyle{fancy}
\fancyhf{}
\rfoot{\thepage}

\nocite{*}

\begin{document}
    
    %PORTADA
    \begin{titlepage}
        \begin{figure}[ht]
            \centering
            \includegraphics[width=15cm]{logosITT.png}
        \end{figure}
        \centering
        {\scshape\LARGE Tecnológico Nacional de México\\Instituto Tecnológico de Tijuana\par}
        \vspace{1cm}
        {\scshape\Large Investigación de Operaciones\par}
        \vspace{1cm}
        {\scshape\Large Unidad 1\par}
        \vspace{1.5cm}
        {\huge\bfseries Aplicaciones Diversas de la Programación Lineal\par}
        \vspace{2cm}
        {\Large\itshape C. Abraham Jhared Flores Azcona\\19211640\par}
        \vfill
        Profesora: \par
        Ing. Igreyne Aracely Ruiz Romero
        
        \vfill

        {\large 14 de octubre del 2020}
    \end{titlepage}

    \newpage
    \thispagestyle{empty}
    \tableofcontents

    \newpage
    \lhead{\textbf{Aplicaciones de la Programación Lineal}}
    \section{Introducción}
    Como se ha visto a lo largo de la unidad, la Investigación de Operaciones es una materia bastante util en distintos ambitos donde se requiera la optimización
    como un objetivo principal. Por ende, se explicarán tres aplicaciones relevantes de los temas de Programación Lineal. 
    \section{Aplicaciones}
        \subsection{Optimización de Portafolio}
        \justify
        Para el mundo de las finanzas e inversiones, la optimización del portafolio es crucial para obtener la mayor ganancia posible con el menor riesgo. Dependiendo de los
        instrumentos de inversión disponibles varian las ganancias y los riesgos. \\ \newline
        En este caso, para el instrumento financiero de las acciones, hay una proporción favorable de ganancia y de riesgo y, por ende, muchos inversionistas y/o analistas financieros
        se ven beneficiados por la aplicación mas intuitiva de la Programación Lineal ya que consideran los siguientes puntos:
        \begin{itemize}
            \item Se conoce el instrumento de inversión y su forma de posible ganancia y pérdida.
            \item En cada inversión se enfoca su posible ganancia.
            \item Por ser inversiones en acciones, tambien se considera el riesgo.
        \end{itemize}
        Por lo tanto, el razonamiento de la Programación Lineal perimite obtener la \emph{ganancia esperada} del portafolio.
        \subsection{Optimización de Transporte}
        \justify
        Si se puede decir, es uno de los campos donde se requiere la ayuda de la Programación Lineal. Como la logística es uno de los aspectos mas cruciales para una cadena de suministros, el optimizar
        los costos relacionados a dicha cadena permite darle al area administrativa un mejor panorama para que tomen la mejor desición que consideren pertinente. 
        \\ \newline
        Un modelo sencillo de Programación Lineal para esta situación expuesta es la siguiente:
        \begin{itemize}
            \item Un conjunto de puntos de suministro donde los productos son enviados.
            \item Un conjunto de destinos de demanda a los cuales el producto es enviado.
            \item El costo variable del producto producido en el punto de suministro.
        \end{itemize} 
        \subsection{Aprendizaje Supervisado (Aprendizaje de Máquina)}
        \justify
        En general, la gran mayoria de las aplicaciones de la Programación Lineal han ganado un reciente interés por la ganancia monetaria y de popularidad de las aplicaciones de las Ciencias Computacionales.
        \\ \newline
        En este tipo de Aprendizaje de Máquina:
        \begin{itemize}
            \item El sistema a mejorar conoce de antemano los patrones y dichos patrones están bién definidos basandose en datos e información previos.
            \item Se entrena al sistema para encajar con un modelo matemático de una función de los datos de entrada que predice valores de una prueba de datos desconocida.
        \end{itemize}
    \section{Conclusión}
    \justify
    Como se apreció, las aplicaciones varian desde las finanzas hasta lo más prometedor como el Aprendizaje de Máquina, mostrando la importancia y el potencial de la Programación Lineal como una herramienta crucial para
    lograr distintos objetivos.

    \newpage
    \lhead{}
    \addcontentsline{toc}{section}{Referencias}
    \printbibliography

\end{document}