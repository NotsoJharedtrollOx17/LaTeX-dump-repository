\documentclass[letterpaper, 12pt]{article}
\usepackage[letterpaper, top=2.5cm, bottom=2.5cm, left=3cm, right=3cm]{geometry} %margenes
\usepackage[utf8]{inputenc} %manejo de caracteres especiales
\usepackage[spanish]{babel} %manejo de encabezados de inglés a español
\usepackage{fancyhdr} %formato de los encabezados de página
\usepackage{ragged2e} %alineado real justficado
\usepackage{graphicx} %manejo de imagenes
\usepackage{amsmath} %manejo de notación matemática
\usepackage{mathtools} %manejo de notación matemática
\usepackage{blindtext} %texto de relleno
\usepackage{amssymb} %manejo de simbología variada
\usepackage[titles]{tocloft} %manejo de elementos para el índice
\usepackage{float} %manejo de centrado para figuras
\usepackage{hyperref} %manejo d hipervínculos
\usepackage[backend=biber]{biblatex}\addbibresource{referencias.bib}%bibliografía

\hypersetup{
    colorlinks=true,      
    urlcolor=blue,
    linkcolor=blue
}

\pagestyle{fancy}
\fancyhf{}
\rfoot{\thepage}

\nocite{*}

\begin{document}
    
    %PORTADA
    \begin{titlepage}
        \begin{figure}[ht]
            \centering
            \includegraphics[width=15cm]{logosITT.png}
        \end{figure}
        \centering
        {\scshape\LARGE Tecnológico Nacional de México\\Instituto Tecnológico de Tijuana\par}
        \vspace{1cm}
        {\scshape\Large Investigación de Operaciones\par}
        \vspace{1cm}
        {\scshape\Large Unidad 5\par}
        \vspace{1.5cm}
        {\huge\bfseries Big Data\par}
        \vspace{2cm}
        {\Large\itshape C. Abraham Jhared Flores Azcona \, 19211640\par}
        \vfill
        Profesora: \par
        Ing. Igreyne Aracely Ruiz Romero
        
        \vfill

        {\large 7 de diciembre del 2020}
    \end{titlepage}

    \newpage
    \lhead{\textbf{Big Data}}
    \rhead{\textbf{7 de diciembre del 2020}}
    \section*{¿Qué es?}
    \justify
    El \emph{Big Data}, traducido al español como Dato Grande es un término que engloba un gran volumen de datos -estructurados y no-estructurados- que inundan a los
    negocios día con día. No importa la cantidad de datos, sino lo que dichas organizaciones realizan con la misma ya que esta puede ser analizado para tener perspectivas que
    dirijan una mejor toma de desiciones y los movimientos empresariales estratégicos.
    \section*{¿Cuándo surge?}
    \justify
    El arte de accesar y almacenar grandes cantidades de información para los análisis ha estado a lo largo del tiempo. El término coloquial de \emph{Big Data} se le atribuye al 
    analísta industrial Doug Laney quien definió a la misma en los inicios del año 2000 con las \emph{``Tres V's''}:
    \begin{itemize}
        \item \textbf{Volúmen: }Las organizaciones recolectan información de distintas fuentes, incluyendo transacciones, dispositivos inteligentes, equipamiento industrial, videos, redes sociales y demás.
        En el pasado, almacenar dichos datos sería un problema. Pero plataformas de almacenamiento barato han facilitado dicha tarea.
        \item \textbf{Velocidad: }Con el crecimiento del Internet de las Cosas, los flujos de datos hacia los negocios con velocidades sin precedentes deben ser manejados en una manera temporalizada. Los sensores,
        los identificadores RFID y demás están conduciendo la necesidad de lidiar con dichos torrentes de información en un tiempo (casi) real.
        \item \textbf{Variedad: }Los datos vienen en todos los tipos de formato - desde datos numericos estructurados de una base de datos tradicional hasta documentos de texto no-estructurados, correos electrónicos,
        videos, audios, etc.
    \end{itemize}
    \section*{Importancia}
    \justify
    Como ya se mencionó antes, la importancia del \emph{Big Data} no recae en cuanta información tienes, sino que haces con ella. Junto a los análisis respectivos podemos destacar que permite:
    \begin{itemize}
        \item Reducir costos.
        \item Reducir tiempos.
        \item Desarrollo de nuevos productos y ofertas optimizadas.
        \item Toma inteligente e informada de desiciones.
        \item Determinar la raíz de los fallos, problemas y defectos en un tiempo (casi) real.
        \item Generar cupones en el punto de venta basado en los hábitos de consumo del cliente.
        \item Recalcular portafolios de riesgo en minutos.
        \item Detectar actividad fraudulenta antes de que afecte a la organización.
    \end{itemize}
    \section*{Pre-requisitos}
    \justify
    Como toda disciplina, se requiere de ciertas disciplinas previas para comprender de mejor manera los contenidos del tema. A continuación se lista los conocimientos necesarios para ello:
    \begin{itemize}
        \item \emph{Estadística:} Como se sabe, estaremos haciendo bastantes análisis de datos y, por ende, conocer las herramientas estadísticas a nuestra disposición permite un mejor avance en la disciplina. Se puede decir que es el pre-requisito más importante.
        \item \emph{Programación:} En la era de la automatización, el saber programar permite realizar códigos a la medida para los análisis muy robustos por la cantidad de datos a depurar. Dependiendo del lenguaje de programación que se conozca dependeran las herramientas disponibles.
        \item \emph{Optimización:} Como manejamos un grán volúmen de datos, es necesario saber como curar tanto los códigos como los datos para que estos puedan ser introducidos, procesados y finalmente manipulados acorde a lo esperado.
        \item \emph{Algebra Lineal:} Al estar manejando cantidades inmensas de datos, la gran mayoria se pueden almacenar en estructuras similares a las vistas en la materia.
    \end{itemize}

    \newpage
    \printbibliography[title={Referencias}]
\end{document}