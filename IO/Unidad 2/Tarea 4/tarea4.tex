\documentclass[letterpaper, 12pt]{article}
\usepackage[letterpaper, top=2.5cm, bottom=2.5cm, left=3cm, right=3cm]{geometry} %margenes
\usepackage[utf8]{inputenc} %manejo de caracteres especiales
\usepackage[spanish]{babel} %manejo de encabezados de inglés a español
\usepackage{fancyhdr} %formato de los encabezados de página
\usepackage{ragged2e} %alineado real justficado
\usepackage{graphicx} %manejo de imagenes
\usepackage{amsmath} %manejo de notación matemática
\usepackage{mathtools} %manejo de notación matemática
\usepackage{blindtext} %texto de relleno
\usepackage{amssymb} %manejo de simbolog►1a matematica


\pagestyle{fancy}
\fancyhf{}
\rfoot{\thepage}

\begin{document}
    
    %PORTADA
    \begin{titlepage}
        \begin{figure}[ht]
            \centering
            \includegraphics[width=15cm]{logosITT.png}
        \end{figure}
        \centering
        {\scshape\LARGE Tecnológico Nacional de México\\Instituto Tecnológico de Tijuana\par}
        \vspace{1cm}
        {\scshape\Large Investigación de Operaciones\par}
        \vspace{1cm}
        {\scshape\Large Unidad 1\par}
        \vspace{1.5cm}
        {\huge\bfseries Tarea 4\par}
        \vspace{2cm}
        {\Large\itshape C. Abraham Jhared Flores Azcona\\19211640\par}
        \vfill
        Profesora: \par
        Ing. Igreyne Aracely Ruiz Romero
        
        \vfill

        {\large 12 de octubre del 2020}
    \end{titlepage}

    \setcounter{page}{1}
    \thispagestyle{fancy}
    \lhead{\textbf{Tarea 4, U1}}
    \rhead{\textbf{12 de octubre del 2020}}
    Resover los problemos con el Método \emph{Simplex}
    \section{Problema 1}
    \subsection{Modelo}
    \[\begin{matrix}
        F.O\!:&\text{Max}(z)=&50x_1+80x_2&&\\
        S.a\!:&&x_1+x_2&\leq&1000\\
        &&x_1&\geq&250\\
        &&x_2&\geq&250\\
        &&x_1-2x_2&\geq&0
    \end{matrix}\]
    \subsection{Aplicación del método}
    \section{Problema 2}
    \subsection{Modelo}
    \[\begin{matrix}
        F.O\!:&\text{Max}(z)=&0.40x_1+1.40x_2&&\\
        S.a\!:&&0.6x_1+0.15x_2&\leq&8100\\
        &&0.45x_1+0.3x_2&\leq&3000\\
        &&x_1,x_2&\geq&0
    \end{matrix}\]
    \subsection{Aplicación del método}
    \[\begin{matrix}
        z&=&-0.40x_1-1.40x_2&=&0\\

    \end{matrix}\]
\end{document}