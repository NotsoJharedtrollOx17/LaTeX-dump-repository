\documentclass[letterpaper, 12pt]{article}

\usepackage[letterpaper, top=2.5cm, bottom=2.5cm, left=3cm, right=3cm]{geometry} %margenes
\usepackage[backend=biber]{biblatex}\addbibresource{Bibliografia.bib} %manejo de bibliografía (BORRAR SI NO ES NECESARIO)
\usepackage[utf8]{inputenc} %manejo de caracteres especiales
\usepackage[spanish]{babel} %manejo de encabezados de inglés a español
\usepackage{fancyhdr} %formato de numeros de pagina
\usepackage{ragged2e} %alineado real justficado
\usepackage{graphicx} %manejo de imagenes

\renewcommand{\headrulewidth}{0pt}
\renewcommand{\footrulewidth}{0pt}

\pagestyle{fancy}
\fancyhf{}        
\rfoot{\thepage}

\nocite{*}

\begin{document}
    \thispagestyle{empty}
    \begin{figure}[ht]
        \centering
        \includegraphics[width=15cm]{logosITT.png}
    \end{figure}
    \begin{LARGE}
        \centering
        \textbf{Tecnológico Nacional de México}
        \\
            \bigbreak
        \textbf{Instituto Tecnológico de Tijuana}
        \\
            \bigbreak
            \bigbreak
            \bigbreak
        \textbf{Materia:}
        \\
            \bigbreak
        Programación Orientada a Objetos
        \\
            \bigbreak 
            \bigbreak
            \bigbreak
        \textbf{Unidad:} 6. Flujos y Archivos
        \\
            \bigbreak
            \bigbreak
            \bigbreak 
        \textbf{Recopilación de Investigaciones}
        \\
            \bigbreak
            \bigbreak
            \bigbreak
        \textbf{Maestro:} M.C. Luz Elena Cortez Galvan
        \\
            \bigbreak
            \bigbreak
            \bigbreak 
        \textbf{Alúmno:} Abraham Jhared Flores Azcona
        \\
            \bigbreak
            \bigbreak
            \bigbreak
        \textbf{Núm. Control:} 19211640
        \\
            \bigbreak
            \bigbreak
            \bigbreak
        \textbf{Fecha de entrega:} Desconocido
        \\ 
    \end{LARGE}
    \thispagestyle{empty}

    \newpage
        \setcounter{page}{1}
        \begin{justify}
        \section*{\huge{Nota al lector}}
        Este documento fue redactando usando el lenguaje de marcado \LaTeX{} el cual permite 
        realizar ediciones muy exigentes al documento sin tanto problema por el acomodo. 
        Aparte, éste permite el uso de expresiones matemáticas complejas.
        La finalidad de esta compilación es la de hacer una revision total de los trabajos y solamente agregar
        los temas mas relevantes que se vieron en la materia asi como ciertos consejos al momento de estar escribiendo el código
        de los ejemplos.
        \end{justify}
    
    \newpage
        \tableofcontents

    \newpage
        \justify
        \begin{huge}
            \bigbreak
            \bigbreak
            \bigbreak
            \section*{\huge{Unidad 1.}}
            \subsection*{\huge{Introducción al Paradigma de la \\Programación Orientada \\a Objetos (POO)}}
        \end{huge}
    
    \newpage
        \begin{justify}
            \section{Elementos del modelo de objetos}
            \subsection{Clase}
            Es un plano para crear objetos provisto de valores iniciales para el estado (variables miembro o atributos) e 
            implementaciones de comportamiento (funciones miembro o métodos).
            Una clase proporciona una definición para instancias creadas dinámicamente de la clase, también conocidas como objetos. 
            Las clases admiten herencia y polimorfismo, mecanismos por los que las clases derivadas pueden extender y especializar clases base. 
            Los miembros de una clase son miembros estáticos o miembros de instancia. Los miembros estáticos pertenecen a clases y los miembros de instancia 
            pertenecen a objetos (instancias de clases).
            Este permite la abstracción de datos de una manera util.
            \\
            \newline
            A continuación se proporciona una visión general de los tipos de miembros mas importantes que puede contener una clase:
            \begin{itemize}
                \item \emph{Campos:} Variables de la clase. 
                \item \emph{Métodos:} Cálculos y acciones que pueden realizarse en dicha clase.
                \item \emph{Constructores:} Acciones que permiten la instancia (creación) de la clase.
                \item \emph{Destructores} Acciones que permiten la destrucción de la clase.
                \item \emph{Operadores:} Conversiones y operadores de expresión admitidos por la clase.
                \item \emph{Propiedades:} Acciones asociadas a la lectura y escritura de la clase.
                \item \emph{etc.}
            \end{itemize}
            \textbf{\\Sintáxis:}
            \begin{verbatim}
//<acceso> class <nombre de la clase>
//{
    //Campos:
    //<acceso> <tipo de variable> <identificador>;

    //Métodos:
    //<acceso> <tipo de método> <nombre>(){}

    //Constructor:
    //public <nombre de la clase>(//campos)
    //{//asignación de campos}

    //Destructor:
    //~<nombre de la clase>(){}
//}

//Ejemplo:
public class Carro
{
    public int year;
    public string model;

    public void Datos()
    {
        Console.WriteLine("Datos del vehículo:");
        Console.WriteLine("Modelo: {0}", model);
        Console.WriteLine("Año: {0}", year);
    }

    public Carro(int year, string model)
    {
        this.year=year;
        this.model=model;
    }

    ~Carro(){}
}
            \end{verbatim}
        \subsection{Objeto}
        Es un componente auto-contenido el cual consiste en métodos y propiedades para 
        hacer útil a un tipo de dato en particular. El objeto determina el comportamiento de 
        una clase y este puede ser una variable o una función. Es una unidad básica de la programación 
        orientada a objetos y representa entidades de la vida real. 
        \\
        \newline
        Un objeto consiste en lo siguiente:
        \begin{itemize}
            \item \emph{Estado:} Refleja las propiedades del objeto. Es representado por los atributos (valores) del objeto.
            \item \emph{Comportamiento:} Refleja la respuesta del objeto con otros objetos. Es representado por los métodos (funciones) del objeto.
            \item \emph{Identidad:} Le da al objeto un nombre específico y esto permite que interactúe con otros objetos.
        \end{itemize}
        Cabe aclarar que al momento de crear un objeto en base a una clase, se dice que estamos \emph{“instanciando una clase”}.\newline
        \textbf{\\Sintáxis:}
            \begin{verbatim}
//<nombre de la clase> <nombre para el objeto> = 
new <nombre de la clase>(<parametros (si son necesarios)>);

//Ejemplo:
public void Main(string[] args)
{
    int year = 2001;
    string model = Model S;
    Carro Tesla = new Carro(year,model);
}
            \end{verbatim}
        \subsection{Abstracción}
        Es el proceso de ocultación de datos e implementaciones del mundo exterior. Es usado para crear un límite entre la aplicación y los programas del cliente.
        Un objeto se le puede proveer acceso a ciertas funciones y propiedades requeridas hacia otros programas. Hay dos típos:
        \begin{itemize}
            \item \emph{De datos:} los datos del objeto no son visibles hacia el mundo exterior. Si es necesario, el acceso es provisto por unos métodos.
            \item \emph{De procesos:} no se necesita proveer detalles sobre todas las funciones de un objeto, por lo tanto, las ocultamos en la operación del usuario.
        \end{itemize}
        \subsection{Modularidad}
        Es la propiedad que permite subdividir una aplicación en partes más pequeñas (llamadas \emph{módulos}), cada una de las cuales debe se tan independiente
        como sea posible de la aplicación en sí y de las partes restantes. Se recomienda escribir el código fuente de un objeto puede ser escrito así para darle
        mantenimiento, independientemente del código fuente de otros objetos. Así mismo, un objeto puede ser transferido alrededor del sistema sin alterar su estado y conducta.
        \subsection{Herencia}
        Es un mecanismo que permite derivar una clase de otra para una jerarquia de clases que comparten un mismo conjunto de atributos y métodos. Se puede considerar
        la capacidad de crear nuevas clases a partir de otras ya definidas. Existen dos tipos de clases en la herencia las cuales son las siguientes:
        \begin{itemize}
            \item \emph{Base:} aquellas cuyos miembros se heredan a sus clases derivadas.
            \item \emph{Derivada:} aquellas que reciben los miembros de su clase base. A esta se le pueden definir atributos y/o métodos nuevos propios. En ella se le puede clasificar 
            la herencia en los dos tipos siguientes:
            \begin{itemize}
                \item \emph{Simple:} se encuentra generalmente en lenguajes modernos como C\# y Java. Es simple cuando la clase derivada solo tiene una clase base.
                \item \emph{Múltiple:} signifíca que una clase derivada es heredada de dos o más clases base. Puede utilizarse para agrupar atributos y métodos desde varias clases de una sola.
            \end{itemize}
        \end{itemize}
        \textbf{\\Sintáxis:}
            \begin{verbatim}
<Acceso> class <Clase derivada>:<Clase base>
{
    //cuerpo de la clase
    //Constructor de la clase derivada\
    public <Clase derivada>(//declaración de parámetros que recibe):
    base(//nombre de parámetros heredados)
    {//asignación de parámetros}
}

//Ejemplo:
public class Camión:Carro
{
    int Cantidad_remolques;
    public Camión(int Cantidad_remolques, int year, string model):base
    (year,model)
    {
        //asignación
    }
}
            \end{verbatim}
        \subsection{Encapsulación}
        Conocido de mejor manera como los tipos de acceso. En la encapsulación, los datos no son accesados directamente; son accesados entre las otras funciones presentes dentro de la clase. 
        En otras palabras, los atributos de la clase se mantienen como captadores públicos y privados, y los métodos de establecimiento son provistos para manipularlos.
        Sirven para limitar el acceso a distintos elementos de una clase, para evitar discrepancias en la ejecución de un programa.
        Los modificadores de acceso mas comúnes son los siguientes:
        \begin{itemize}
            \item \emph{public: }el tipo o miembro puede ser accedido por cualquier código dentro del ensamble u otro ensamble que lo referencie. 
            \item \emph{private: }el tipo o miembro solo puede ser accesado por el código que esté dentro de la misma clase o estructura.
            \item \emph{protected: }el tipo o miembro solo puede ser accesado por el código de la clase o por una clase derivada.
        \end{itemize}
        \subsection{Polimorfísmo}
        Es la habilidad de un lenguaje de programación de procesar objetos de manera diferente dependiendo del tipo de datos o clase. Específicamente, es la habilidad de redefinir métodos 
        para clases derivadas. También se define como la propiedad que permite enviar mensajes sintácticamente iguales a objetos de distintos tipos. 
        Permite procesar objetos de distintas maneras según lo que más convenga para el programa.
        Existen distintos tipos de polimorfísmo:
        \begin{itemize}
            \item \emph{Sobrecarga: }representa diferentes maneras de realizar una misma acción.
            \item \emph{Herencia: }una subclase hereda los elementos de una clase base.
            \item \emph{Sobreescritura: }una subclase re-define los elementos heredados de una clase base.
        \end{itemize}
        \end{justify}

    \newpage
        \justify
        \begin{huge}
            \bigbreak
            \bigbreak
            \bigbreak
            \section*{\huge{Unidad 2.}}
            \subsection*{\huge{Clases y Objetos}}
        \end{huge}

    \newpage
    \begin{justify}
        \section{Clases y Objetos}
        \subsection{Métodos}
        Es un conjunto de instrucciones que realizan una tarea específica y bien definida. Se invocan a través de su nombre asignado en un proceso,
        todas las veces que se necesite. También es un algoritmo asociado a un objeto cuya ejecución se desencadena tras la recepción de un mensaje (llamada al método).
        Lo que los diferencia a las funciones es que los métodos no tienen un valor de retorno específico (por lo cual se escribe \emph{void}) y no aceptan
        la palabra reservada \emph{return.} Cabe destecar que la asociación entre la clase y el método se conoce como \emph{binding.}
        
        \textbf{\\Sintáxis:}
            \begin{verbatim}
//<acceso> <tipo retorno> <nombre>(<parámetros>)
{
    //cuerpo del método
}

//Ejemplo:
public void Mensaje()
{
    Console.WriteLine("Desplegué el mensaje");
}
            \end{verbatim}
        \subsection{Constructor}
        Es un método de una clase o estructura que inicializa un objeto de ese tipo. Un constructor tiene el mismo nombre que el de la clase, y puede
        ser usado para establecer los valores de los miembros de un objeto, a los defecto o a los valores definidos por el usuario. Cabe destacar lo siguiente:
        \begin{itemize}
            \item El constructor tiene el mismo nombre de la clase.
            \item Este no tiene un tipo de valor de retorno.
            \item Si no se especifica un constructor, C\# genera uno para nosotros.
            \item Se puede dotar de mas de un constructor a la clase.
            \item Debe ser público.
        \end{itemize}
        \textbf{\\Sintaxis:}
            \begin{verbatim}
//public <nombre de la clase>(//parámetros)
{
    //asignación de parametros;
}

//Ejemplo:
class Persona
{
    int edad;
    string nombre;

    //Constructor de la clase
    public Persona(int edad, string nombre)
    {
        this.edad = edad;
        this.nombre = nombre;
    }
}
            \end{verbatim}
        \subsection{Destructor}
        Homologo del constructor. Es un método especial llamado automáticamente durante la destrucción de un objeto. Sus funciones principales son las de
        cerrar conexiones de archivos o bases de datos y liberar recursos de la red, candados, etc. Cabe destacar que son escritos con el nombre de clase precedidos
        por el símbolo "~". Cabe destacar que para que las instrucciones escritas dentro del destructor sean visibles, es necesario ejecutar el código con las sig. teclas:
        CTRL + F5. Sus propiedades son las siguientes:
        \begin{itemize}
            \item Invocación automática sin llamada explícita en el código.
            \item No se puede heredar y por ende, aplicar algún tipo de polimorfísmo.
            \item El orden de ejecución va desde la clase con mayor a menor derivación.
            \item Es usado en clases pero no en estructuras.
            \item Se usa para liberar recursos costosos que el objeto mantiene.
        \end{itemize}
        \textbf{\\Sintáxis:}
            \begin{verbatim}
//~<nombre de la clase>()
{
    //Instrucciones a ejecutar durante su llamada
}

//Ejemplo:
class Mensaje
{
    //Destructor
    ~Mensaje(){}
}
            \end{verbatim}
        \subsection{Sobrecarga de métodos}
        Es una manera que permite a múltiples métodos tener el mismo nombre y contener parámetros distintos. Esto permite procesar objetos de
        distintas maneras según lo que mas convenga para el programa.
        Cabe recalcar los siguientes puntos:
        \begin{itemize}
            \item El compilador (al ejecutar el método) selecciona el que coincide con el número y tipo de dato de los parámetros que se le envían.
            \item Pueden existir dos o más métodos con el mismo nombre, siempre y cuando tengan distintos parámetros.
            \item Todos los métodos se pueden sobrecargar en una clase a excepción del destructor.
        \end{itemize}
        \textbf{\\Ejemplo:}
            \begin{verbatim}
class Sumas de enteros
{
    int respuesta;

    //sobrecarga 1
    public int Suma(int x1, int x2)
    {
        respuesta=x1+x2;
        return respuesta;
    }

    //sobrecarga 2
    public int Suma(int x1, int x2, int x3)
    {
        respuesta=x1+x2+x3;
        return respuesta;
    }
}
            \end{verbatim}
        \subsection{Sobrecarga de operadores}
        Es la habilidad de definir el significado de uno o varios operadores en el contexto de una determinada clase de objetos. Nos ofrece la oportunidad
        de crear nuestras propias operaciones entre tipos sin ningun problema. Estos se definen de manera similar a una función. Deben ser estáticos y se utliza la palabra reservada \emph{operator} delante del operador a sobrecargar.
        Generalmente la cantidad de argumentos que un operador acepta son dos pero existen ciertos operadores que utilizan un solo operando. Se debe definir cada operación
        a soportar, incluyendo a cada uno de los parámetros de la operación y su respectivo orden. Cabe recalcar que:
        \begin{itemize}
            \item \emph{Operadores unarios: }representan las operacionesque reciben un solo operando.
            \item \emph{Operadores binarios: }representan operaciones que reciben dos operandos.
        \end{itemize}
        \textbf{\\Sintáxis:}
            \begin{verbatim}
//public <valor de retorno> <nombre de clase> operator <símbolo del operador>
(//argumentos que recibe)
    => new <nombre de clase> (//operaciones)

//Ejemplo:

class Fraccion
{
    int numerador, denominador;

    public Fraccion(int num, int den)
    {
        if(den==0)
            throw new ArgumentException("El denominador no puede ser cero!", 
            nameof(den));
        numerador=num;
        denominador=den;
    }

    //sobrecarga
    public static Fraccion operator +(Fraccion a, Fraccion b)
        => new Fraccion(a.numerador*b.denominador + b.numerador*a.denominador, 
        a.denominador*b.denominador)
}
            \end{verbatim}
    \end{justify}

    \newpage
        \justify
        \begin{huge}
            \bigbreak
            \bigbreak
            \bigbreak
            \section*{\huge{Unidad 3.}}
            \subsection*{\huge{Herencia}}
        \end{huge}

    \newpage
    \begin{justify}
        \section{Herencia}
        \subsection{Reutilización de miembros heredados}
        Es una manera que permite reusar el código existente por medio de alguna forma de interfaz provista por llamadas a una subrutína, a un objeto, o a una clase.
        La reutilización de software reduce el tiempo de desarrollo de los programas, por ello una parte/totalidad del código de un programa se puede reemplazar en la construcción
        de otro programa.
        \textbf{\\Sintáxis:}
        \begin{itemize}
            \item Para declarar un nuevo atribúto con el mismo identificador que el heredado, quedando el atribúto oculto con "new":
                \begin{verbatim}
public class Punto { public int x,y; }

public class Punto3D:Punto { new public int x,y; public int z; }
                \end{verbatim}
            \item Para declarar un nuevo método con misma cabecera que el de la clase. Se supone sobreescritura con "new":
                \begin{verbatim}
public class A { public void X()
{ Console.WriteLine("A.X"); } }

public class B:A { new public void X()
{ Console.WriteLine("B.X"); } }
                \end{verbatim}
            \item Declarar un nuevo método/clase:
                \begin{verbatim}
public class A { public void X()
{ Console.WriteLine("A.X"); } }

public class B:A { public void Y()
{   X();
    Console.WriteLine("B.X"); } }
                \end{verbatim}
        \end{itemize}
        \subsection{Constructores y destructores en clase base y derivada}
        \begin{itemize}
            \item \emph{Constructores: }al instanciar objetos de clases derivadas se inicia una cadena de invocaciones a constructores en las cuales
            el constructor de la clase derivada (antes de realizar sus tareas propias) invoca al constructor de la clase base.
            En resumen:
            \begin{itemize}
                \item \emph{1ro. se ejecuta el constructor de la clase base.}
                \item \emph{2do. se ejecuta el constructor de la clase derivada.}
            \end{itemize}
            \item \emph{Destructores: }la cadena de invocación a destructores se ejecuta en orden contrario a su homólogo, esto es, primero se ejecuta
            el destructor de la clase base.
            En otras palabras:
            \begin{itemize}
                \item \emph{1ro. se ejecuta el destructor de la clase derivada.}
                \item \emph{2do. se ejecuta el destructor de la clase base.}
            \end{itemize}
        \end{itemize}
        Debido a esto, el constructor crea un objeto extra de la clase base por cada objeto creado de la misma clase o sus derivads en la ejecucción del programa para que el
        objeto "extra" sea quien pueda heredar sus elementos a los derivados.
        \subsection{Redefinición de métodos en clases derivadas}
        Esta técnica se realiza cuando una subclase provée una implementación específica de un método provisto por la clase base y este método tiene el mismo nombre,
        tipo de dato y parámetros. Para ello, se acostumbra a realizar lo siguiente:
        \begin{itemize}
            \item Los métodos de la clase base que usan la palabra reservada \emph{virtual} reciben el nombre de \emph{métodos virtuales.}
            \item Los métodos de la clase derivada que usan la palabra reservada \emph{override} reciben el nombre de \emph{métodos de reemplazo.}
        \end{itemize}
        Cabe destacar que \emph{el modificador virtual no se puede utilizar con los siguientes modificadores: static, abstract, override}\\
        \textbf{\\Ejemplo:}
            \begin{verbatim}
class A
{
    public virtual void X()
    { Console.WriteLine("Método de la clase A"); }
}

class B:A
{
    public override void X()
    { Console.WriteLine("Método X de la clase B"); }
}
            \end{verbatim}
    \end{justify}

    \newpage
        \justify
        \begin{huge}
            \bigbreak
            \bigbreak
            \bigbreak
            \section*{\huge{Unidad 4.}}
            \subsection*{\huge{Polimorfísmo}}
        \end{huge}

    \newpage
    \begin{justify}
        \section{Polimorfísmo}
        \subsection{Clases abstractas}
        Es una definición de plantillas de métodos y variables de una clase que contiene uno o más métodos abstractos. Clases u objetos pueden ser abstraídos,
        lo que signifíca que son resumidos en características que son relevantes en la operación del programa. 
        Declarar una clase como abstracta signifíca que no puede ser instanciada directamente, lo que signifíca que un objeto no puede ser creado de él. Esto protege
        al código de ser usado de manera inapropiada. Estas requieren subclases que definan aún mas los atributos necesarios para instanciarlos individualmente.     
        Las características principales son las siguientes:
        \begin{itemize}
            \item Estas \emph{pueden contener o no métodos abstractos.}
            \item \emph{Si una clase contiene (al menos) un método abstracto, la clase debe ser declarada como abstracta.}
            \item Si una clase es declarada como abstracta, \emph{esta no puede ser instanciada.}
        \end{itemize}
        Para su creación solo necesitamos agregar la palabra reservada \emph{abstract} para definir una clase como abstracta. Su implementación se aplica cuando una
        clase hereda características de la clase abstracta.
        \newline
        \textbf{\\Ejemplo:}
            \begin{verbatim}
//clase base
public abstract class Figuras
{
    public double radio, altura, lado, volumen;

    abstract public double calcularVolumen();
    abstract public void desplegarVolumen();

    ~Figuras()
    {
        Console.WriteLine("Memoria Objeto Clase Base Figuras Liberada");
    }
}

//clase derivada
public class Esfera:Figuras
{
    public Esfera(double radio)
    {
        this.radio = radio;
        lado = 0;
        altura = 0;
    }

    //implementación de los métodos abstractos
    public override double calcularVolumen()
    {
        return voulmen = (4/3) * Math.PI
        * Math.Pow(radio, 3);
    }

    public override void desplegarVolumen()
    {
        Console.WriteLine("Esfera");
        Console.WriteLine("Radio: {0}m",radio);
        Console.WriteLine("Volúmen: {0}m^3",volumen);
    }
}
            \end{verbatim}
        \subsection{Interfaces}
        Es completamente "una clase abstracta", la cual solo puede contener métodos y propiedades abstractas. 
        Sus características principales son las siguientes:
        \begin{itemize}
            \item Similar a las clases abstractas, \emph{no se pueden instanciar objetos con interfaces.}
            \item \emph{Sus métodos no tienen cuerpo,} porque éste es provisto por la clase que los implementa.
            \item Al implementar una interfaz, \emph{debemos sobreescribir todos los métodos de la misma.}
            \item \emph{Pueden contener propiedades y métodos, pero no campos/variables.}
            \item \emph{Las propiedades y métodos (por defecto) son abstractos y públicos.}
        \end{itemize}
        Para crear una interfaz, necesitamos usar la palabra reservada \emph{interface} seguida del nombre deseado para la misma. Dentro de los corchetes
        definimos méetodos sin cuerpo agregando solamente \emph{void }debido a que por defecto, los métodos son abstractos y públicos. En lo que respecta
        a su implementación, debemos crear una clase y los métodos serán aquellos que herede la interfaz, similar a la implementación de clases abstractas.
        \newline
        \textbf{\\Ejemplo:}
            \begin{verbatim}
public interface Figuras
{
    /Método
    double calcularArea();
    void desplegarArea();
}

public class Triangulo:Figuras
{
    //Campos
    public float base, altura;
    public double area;

    //Constructor
    public Triangulo(float base, floar altura)
    {
        this.base = base;
        this.altura = altura;
    }

    //Implementación de métodos
    public double calcularArea()
    {
        return this.area= (base * altura)/2;
    }

    public double desplegarArea()
    {
        Console.WriteLine("Triángulo");
        Console.WriteLine("Base: {0}", this.base);
        Console.WriteLine("Altura: {0}", this.altura);
        Console.WriteLine("Area: {0}", this.area);
    }
}
            \end{verbatim}
            \subsection{Librerías en C\#}
            Es un conjunto de campos y métodos que pueden ser llamados por una aplicación. En C\#, cualquier librería enfocada en .NET Standard 2.0 permite que cualquier
            aplicación enfocada en implementaciones .NET Standard use la librería. Las librerías pueden tener extensiones diferentes, las más comunes son: \emph{.lib, .bpl, .a, .dll, .h}
            entre otras. Nos permite crear nuestros programas de una manera más modular y reutilizable facilitando un desarrollo del mismo bastante complejo en pocas lineas de código.\newline
            \\
            Las librerías no son únicamente archivos externos creados por otros, también \emph{es posible crear e implementar nuestras propias librerias y utilizarlas en nuestros programas.}

            \textbf{\\Para crear una librería en C\#:}
            \begin{enumerate}
                \item En la ventana de inicio de Visual Studio, elegir \textbf{Create a new proyect.}
                \item Seleccionar \textbf{Class Library (.NET Framework).}
                \item En el cuadro \textbf{Proyect name} escribir el nombre para la librería (será el \emph{namespace}).
                \item En la clase \textbf{Class1} que se crea por defecto, modificar al nombre deseado.
                \item Escribir el código necesario a la clase: campos, constructor, destructor, etc.
                \item Añadir las clases necesarias.
                \item Compilar \textbf{(Build)} la librería.
                \item Verificar en la carpeta \textbf{bin}, subcarpeta \textbf{Debug} del proyecto, se encuentre el archivo .dll
            \end{enumerate}
            \textbf{Para agregar una librería en un proyecto de C\#:}
            \begin{enumerate}
                \item Seleccionar \textbf{File, New project.}
                \item Elegir \textbf{ConsoleApp (.NET Framework).}
                \item En \textbf{References} del \textbf{Solution Explorer} dar click en el botón derecho del mouse.
                \item Presionar \textbf{Add Reference} en el cuadro de diálogo.
                \item Buscar el archivo .dll en la carpeta de la librería, agregar y presionar OK.
                \item Escribir el comando \textbf{using nombrelibrería;}
                \item Añadir el código deseado al programa.
                \item Grabar y ejecutarlo.
            \end{enumerate}
            \textbf{Ejemplo de la creación de la librería:}
                \begin{verbatim}
namespace VOLUMENES
{
    public class Esfera
    {
        \\campos
        float radio;

        \\constructor
        public Esfera (float radio)
        {
            this.radio = radio;
        }

        \\métodos
        public double calcularVolumen()
        {
            return (4/3) * Math.PI * Math.Pow(radio , 3);
        }

        public void desplegarVolumen()
        {
            Console.WriteLine("Esfera");
            Console.WriteLine("Radio: {0}", radio);
            Console.WriteLine("Volúmen: {0}", calcularVolumen());
        }

        \\destructor
        ~Esfera(){}
    }
}
                \end{verbatim}
            \textbf{Ejemplo de la implementación de la librería en un proyecto:}
                \begin{verbatim}
using VOLUMENES;

class Programación
{
    static void Main(string[] args)
    {
        //creación del objeto por parte de la librería VOLUMENES
        Esfera E1 = new Esfera(1.3f);

        //ejecución de métodos y despliegue de resultados
        Console.WriteLine("RESULTADOS OBTENIDOS");
        E1.desplegarVolumen();
    }
}
                \end{verbatim}
    \end{justify}

    \newpage
        \justify
        \begin{huge}
            \bigbreak
            \bigbreak
            \bigbreak
            \section*{\huge{Unidad 5.}}
            \subsection*{\huge{Excepciones}}
        \end{huge}    

    \newpage
    \begin{justify}
        \section{Excepciones}
        \subsection{Definición}
        Son problemas que surgen durante la ejecucción de un programa. Una excepción de C\# es la respuesta ante dicha circunstancia que surge.
        Un ejemplo de ello es el intentar la división entre cero. Dichas excepciones están compuestas por los siguientes bloques/palabras reservadas:
        \begin{itemize}
            \item \emph{try:} Este bloque identifíca un bloque de código para el cual se le identificarán las excepciones. Es seguido por uno o más bloques \emph{catch.}
            \item \emph{catch: }Por su traducción literal al español, este bloque atrapa la excepción con el manejador de excepciones en el lugar deseado del programa.
            \item \emph{finally: }Este bloque es usado para ejecutar un bloque de código independientemente de que se haya atrapado una excepción o no.
            \item \emph{throw: }Arroja una excepción cuando un problema aparece.
        \end{itemize}
        \textbf{\\Sintáxis:}
            \begin{verbatim}
try { //código que causa la excepción }

catch( <Nombre para la excepción> <identificador> ) 
{ //código que maneja la excepción <identificador> 
  //pueden existir cualesquiera de estos bloques si se considera 
  necesario }

finally { //código a ejecutar despues de haberse ejecutado 
el bloque "try" }
            \end{verbatim}
        \subsection{Tipos de excepciones:}
        Estas son representadas por clases. La clase de excepciones en C\# esta directa o indirectamente derivada de la clase \emph{System.Excepction.}
        \newline
        \\
        \emph{System.ApplicationException }soporta excepciones generadas por programas de aplicación. Por lo tanto las excepciones de
        finidas por el programador deben derivarse de esta clase.
        \newline
        \\
        \emph{System.SystemException }es la clase base para todas las excepciones predefinidas por el sistema. Sus excepciones son las siguientes:
        \begin{itemize}
            \item \emph{System.IOException:} maneja errores I/O.
            \item \emph{System.IndexOutOfRangeException:} maneja errores generados cuando un método refere un índice fuera de rango de un arreglo.
            \item \emph{System.ArrayMismatchException: }maneja errores generados cuando se encuentra una discrepancia con el tipo de dato manejado en un arreglo.            \item \emph{System.DivideByZeroException: }maneja errores generados por dividir un dividendo con cero.
            \item \emph{System.InvalidCastException:} maneja errores generados al momento de invocar algún método.
            \item \emph{System.OutofMemoryException: }maneja errores generados de memoria libre insuficiente.
            \item \emph{System.StackOverflowException:}maneja errores generados por los stacks llenos que ya no pueden ser sobreescritos.
        \end{itemize}
        \textbf{\\Propiedades:}
        \begin{itemize}
            \item \emph{HelpLink: }puede mantener una liga a un archivo de ayuda que provee información extensiva sobre la causa de la excepción.
            \item \emph{InnerException: }puede ser usada para crear y preservar una serie de excepciones durante el manejo de excepciones.
            \item \emph{Message: }provee detalles acerca de la causa de la excepción.
            \item \emph{Source: }obtiene o fija el nombre del objeto o aplicación que causa el error.
            \item \emph{StackTrace: }contiene el trazo del stack que puede ser usado para determinar dónde ocurrió el error en el código.
        \end{itemize}

        \subsection{Propagación de excepciones}
        Esta se realiza mediante los bloques \emph{try, catch y finally} mencionados anteriormente. El código que se desea evaluar para encontrar las excpeciones se escribe dentro del bloque \emph{try.}
        Una vez iniciada, la excepción se propaga hasta que se encuentra y se atrapa con el bloque \emph{catch.} Aquellas no detectadas se controlan mediante mensajes preestablecidos por el sistema. El
        bloque \emph{finally} se ejecuta después de haber evaluado todo lo anterior.

        \subsection{Gestión de excepciones}
        Una manera de generar explícitamente las excepciones donde sea necesario es con la palabra reservada \emph{throw.} Su sintáxis es la siguiente: \emph{throw [exp];} donde \emph{[exc]} es una instancia
        de una clase derivada de \emph{System.Exception.}
        \newline
        \\
        Estas deben de lanzarse si una o más de las siguientes condiciones son verdaderas:
        \begin{itemize}
            \item El método no puede completar su funcionalidad definida.
            \item Se realiza una llamada inapropiada a un objeto, en función del estado del objeto.
            \item Cuando un argumento a un método causa una excepción.
        \end{itemize}

        \subsection{Creación y manejo de excepciones en C\#}
        Para ello debemos de realizar lo siguiente:
        \begin{enumerate}
            \item Crear una clase con el nombre de la excepción la cual hereda propiedades de la clase \emph{ApplicationException.} Su sintáxis es la siguiente:
                \begin{verbatim}                    
class <Nombre de la excepción>:ApplicationException{}                \end{verbatim}
            \item Crear el constructor e indicarle en los campos que heredo el mensaje deseado a arrojar por nuestra excepción. Ejemplo:
                \begin{verbatim}
class NegativosException:ApplicationException
{
    public NegativosException():
            base("NO SE PUEDE CALCULAR EL SENO DE UN 
            ANGULO NEGATIVO!"){}
} \end{verbatim}
        \end{enumerate}
        \textbf{Programa ejemplo:}
                \begin{verbatim}
using System;
using System.Collection.Generic;
using System.Linq;
using System.Text;
using System.Threading.Tasks;

namespace EJEMPLOEXCEPCION
{
    class Program
    {
        static void Main(string[] args)
        {
            //variables auxiliares
            int base, altura, area;

            //manejo de exepciones
            try
            {
                //Captura de datos
                Console.Write("Base: ");
                base = Int16.Parse(Console.ReadLine());
                Console.Write("Altura: ");
                altura = Int16.Parse(Console.ReadLine());

                //Cálculo del área
                area = base * altura / 2;

                //Despliegue de resultados
                Console.WriteLine("RESULTADOS");
                Console.WriteLine("Area del triángulo: {0}", area);
                Console.Write("\nPresione cualquier 
                tecla para continuar: ");
                Console.ReadKey();
            }
            catch (Format Exception fe)
            {
                Console.WriteLine("\nMensaje del  error: {0}", 
                fe.Message);
                Console.WriteLine("\nRuta del error: {0}", 
                fe.StackTrace);
                Console.Write("\nPresione cualquier tecla 
                para continuar: ");
                Console.ReadKey();
            }
            finally {}
        }
    }
}
                \end{verbatim}
    \end{justify}

    \newpage
    \justify
    \begin{huge}
        \bigbreak
        \bigbreak
        \bigbreak
        \section*{\huge{Unidad 6.}}
        \subsection*{\huge{Flujos y Archivos}}
    \end{huge}     

    \newpage
    \begin{justify}
        \section{Flujos y Archivos}
        \subsection{Archivos}
        Se le conoce como archivo o fichero a un conjunto organizado de unidades de información (\emph{bits}) almacenados en un dispositivo. Se les 
        denomina de esa manera debido a la metáfora de representar archivos tradicionales de oficina, escritos en papel. Estos poseen una identificación
        única, la cual puede ser modificada o asignada a voluntad del usuario o programador y una extensión que determina el tipo de archivo y las funciones que cumple.
        En C\# se utiliza la siguiente sentencia para manejar archivos: \emph{using System.IO;}
        \newline
        \\
        Aunque C\# ve a los archivos como un flujo de bytes, es conveniente concebirlos como un conjunto de registros que poseen una marca de fín de archivo. Estos flujos 
        se organizan en registros, los registros en campos, los campos en bytes, y los bytes en bits.
        \newline
        \\
        Dependiendo del tipo de datos que se manejen en sus registros, los archivos se clasifican en \emph{archivos de texto y archivos binarios.}
        \begin{itemize}
            \item \emph{Archivos binarios: }Es un archivo que contiene información de cualquier tipo codificada en binario para el propósito de almacenamiento y procesamiento en ordenadores.
            Para su formato, se piensa a estos archivos como una secuencia de bytes. Muchos de estos contienen partes que pueden ser interpretadas como texto. Se usan cuando se requiere efectuar operaciones
            con datos de alguno de los tipos primitivos del lenguaje.
            \item \emph{Archivos de texto: }Conocidos de mejor manero como \emph{archivos de texto plano.} Son aquellos que están compuestos únicamente por texto sin formato, sólo caracteres. Estos caracteres
            se pueden codificar de distintos modos dependiendo de la lengua usada. Algunos sistemas de codificación usados para los mismos son: ASCII, ISO-8859-1, Unicode, etc. \emph{Su manejo se puede llevar a cabo
            por medio de los flujos de bytes y de caracteres.}
        \end{itemize}
        \subsection{Flujos}
        En el contexto de informática y computación un flujo signifíca la entrada, procesamiento y salida de datos en un mismo flujo. Por ende podemos decir que los flujos nos permiten el manejo de archivos de texto.
        \newline
        \\
        Los flujos se clasifican en \emph{flujos de bytes y flujos de caracteres:}
        \begin{itemize}
            \item \emph{De bytes:} Al escribir un valor almacenado en un objeto a un flujo binario, este va a leer exactamente lo que fue almacenado en el objeto cuando fue escrito.
            \item \emph{De caracteres:} Con este flujo, se le puede escribir al archivo una o más líneas de texto que pueden ser escritas a un despliegue. Cuando se escribe a un flujo de texto, el programa escribe una nueva línea
            al final de cada línea de texto.
        \end{itemize}
        \subsection{Operaciones con archivos de texto}
        Debido a la existencia de dos flujos distintos para los archivos de texto, las operaciones varían.
        \newline
        \\
        Si se desea escribir o leer datos de tipo \emph{byte}, se declara el flujo correspondiente de la clase \emph{FileStream} con los siguientes constructores:
            \begin{verbatim}
FileStream(string <Nombre>, FileMode <Modo>)

FileStream(string <Nombre>, FileMode <Modo>, FileAccess <Acceso>) \end{verbatim}
        Donde:
        \begin{itemize}
        \item \emph{Nombre: }es el nombre del archivo que influye su ruta de ubicación. Ejemplo:
            \begin{verbatim}
C:\\POO\\TRABAJOSPOO\\archivo.txt \end{verbatim}
        \item \emph{Modo: }es un valor del tipo enumerado \emph{FileMode;} este puede tomar uno de los siguientes valores:
                \begin{itemize}
                    \item \emph{CreateNew:} Crea un nuevo archivo. Si este existe, lanzará una excepción de tipo \emph{IOException.}
                    \item \emph{Create:} Crea un nuevo archivo. Si este existe, se sobreescribe.
                    \item \emph{Open:} Abre un archivo existente.
                    \item \emph{OpenOrCreate:} Abre un archivo existente, si existe; en caso contrario, se crea un nuevo archivo.
                    \item \emph{Truncate: }Abre in archivo existente y lo trunca a cero bytes de longitud.
                    \item \emph{Append: }Abre un archivo para agregar datos al final. Si el archivo no existe, lo crea.
                \end{itemize}
        \item \emph{Acceso: }es un valor del tipo enumerado \emph{FileAccess;} puede tomar uno de los siguientes valores:
                \begin{itemize}
                    \item \emph{Read: }Permite leer un archivo.
                    \item \emph{ReadWrite: }Permite leer o escribir en el archivo.
                    \item \emph{Write: }Permite escribir en el archivo.
                \end{itemize}
        \end{itemize}
        Para la escritura se utilíza \emph{StreamWriter} y sus constructores son los siguientes:
        \begin{itemize}
            \item \emph{StreamWriter(string Nombre):} Abre un nuevo flujo para escribir en un archivo especificado por nombre.
            \item \emph{StreamWriter(Stream Flujo):} Utiliza un flujo existente para escribir.
        \end{itemize}
        Sus métodos más destacables son:
        \begin{itemize}
            \item \emph{Write().}
            \item \emph{WriteLine().}
            \item \emph{Flush().}
        \end{itemize}
        Para la lectura se utiliza \emph{StreamReader} y sus constructores son los siguientes:
        \begin{itemize}
            \item \emph{StreamReader(string Nombre):} Abre un nuevo flujo para leer de un archivo especificado por nombre.
            \item \emph{StreamReader(Stream Flujo):} Utiliza un flujo existente para leer.
        \end{itemize}
        Sus métodos más destacables son:
        \begin{itemize}
            \item \emph{Read().}
            \item \emph{ReadLine().}
            \item \emph{Peek().}
            \item \emph{DiscardBufferData().}
        \end{itemize}
        \subsection{Operaciones en Archivos Binarios}
        Para la escritura se utiliza \emph{BinaryWriter} y su constructor es: \emph{BinaryWriter(Stream Flujo).} Como se puede apreciar, este constructor requiere un flujo de la clase
        \emph{Stream} o sus derivadas. Un ejemplo es:
        \begin{verbatim}
FileStream fe = new FileStream("datos.dat", FileMode.Create, FileAccess.Write);

BinaryWriter bw = new BinaryWriter(fe);
        \end{verbatim}
        A continuación se muestran algunos de los principales métodos y propiedades de la clase \emph{BinaryWriter:}
        \begin{itemize}
            \item \emph{Write(dato):} Escribe el valor deseado del tipo indicado. Los tipos de datos que acepta son:
            \begin{itemize}
                \item \emph{byte.}
                \item \emph{byte[].}
                \item \emph{char.}
                \item \emph{char[].}
                \item \emph{short.}
                \item \emph{int.}
                \item \emph{long.}
                \item \emph{decimal.}
                \item \emph{float.}
                \item \emph{double.}
                \item \emph{string.}
            \end{itemize}
            \item \emph{BaseStream:} Obtiene el flujo base.
            \item \emph{Close:} Cierra el flujo y libera los recursos adquiridos.
            \item \emph{Flush:} Limpia el buffer asociado con el flujo.
            \item \emph{Seek:} Establece el apuntador de Lectura/Escritura en el flujo.
        \end{itemize}
        Para la lectura se utiliza \emph{BinaryReader} y su constructor es: \emph{BinaryReader(Stream Flujo).} Como se puede apreciar, este constructor requiere un flujo de la clase 
        \emph{Stream} o sus derivadas. Un ejemplo es:
        \begin{verbatim}
FileStream fs = new FileStream("data.dat", FileMode.Open, FileAccess.Read);

BinaryReader br = new BinaryReader(fs);
        \end{verbatim}
        A continuación se muestran algunos de los principales métodos y propiedades de la clase \emph{BinaryReader:}
        \begin{itemize}
            \item \emph{Read(dato):} Escribe el valor deseado del tipo indicado. Los tipos de datos que acepta son:
            \begin{itemize}
                \item \emph{byte.}
                \item \emph{byte[].}
                \item \emph{char.}
                \item \emph{char[].}
                \item \emph{short.}
                \item \emph{int.}
                \item \emph{long.}
                \item \emph{decimal.}
                \item \emph{float.}
                \item \emph{double.}
                \item \emph{string.}
            \end{itemize}
            \item \emph{BaseStream:} Obtiene el flujo base.
            \item \emph{Close:} Cierra el flujo y libera los recursos adquiridos.
            \item \emph{Flush:} Limpia el buffer asociado con el flujo.
            \item \emph{PeekChar:} Obtiene el siguiente carácter, sin extraerlo.
        \end{itemize}
        \subsection{Ejemplo}
        \begin{verbatim}
using System;
using System.Collections.Generic;
using System.Linq;
using System.Text;
using System.Threading.Tasks;
using System.IO; //librería para el manejo de archivos

namespace EJEMPLO_ARCHIVO_texto
{
    class Program 
    {
        //clase para archivos
        public class TextoBytes
        {
            //campos
            FileStream fe = null //flujo de entrada-lectura 
            y salida-escritura
            char[] cBuffer = new char[81];  //arreglo de caracteres 
            byte[] bBuffer = new byte[81]; //arreglo de bytes
            int nbytes, car = 0;

            //métodos
            public void CrearArchivo(string archivo)
            {
                try
                {
                    //crea el flujo para el archivo
                    fe = new FileStream(archivo, FileMode.Create,
                    FileAccess.Write);

                    Console.WriteLine("Escriba el texto a almacenar,
                    al terminar presione ENTER:");

                    while ((car = Console.ReadLine()) != '\r' 
                    && (nbytes < bBuffer.Length))
                    {
                        //Convierte en byte el carácter leído
                        bBuffer[nbytes] = (byte)car;
                        nbytes++;
                    }
                    Console.Read();

                    //Escribe la línea de texto en el archivo
                    fe.Write(bBuffer, 0, nbytes);
                    Console.WriteLine("El archivo ha sido creado\n");
                }
                catch (IOException es)
                {
                    Console.WriteLine("Mensaje del Error: {0}", 
                    es.Message);
                    Console.WriteLine("Ruta del Error: {0}", 
                    es.StackTrace);
                }
                finally
                {
                    //Cierra el flujo de escritura
                    if (fe != null) fe.Close();
                    Console.Write("Presione ENTER para continuar:");
                    Console.ReadKey();
                }
            }

            public void LeerArchivo(string archivo)
            {
                try
                {
                    //Crea un flujo desde el archivo texto.txt
                    fe = new FileStream(archivo, FileMode.Open,
                    FileAccess.Read);

                    //Lee una línea de texto del archivo
                    nbytes = fe.Read(bBuffer, 0, 81);

                    //Crea un objeto string con el texto leído
                    Array.Copy(bBuffer, cBuffer, bBuffer.Length);
                    String str = new String(cBuffer, 0, nbytes);

                    //Muestra el texto leídp
                    Console.WriteLine("\nCONTENIDO DEL ARCHIVO");
                    Console.WriteLine(str);
                }
                catch (IOException le)
                {
                    Console.WriteLine("Mensaje del Error: {0}", 
                    le.Message);
                    Console.WriteLine("Ruta del Error: {0}", 
                    le.StackTrace);
                }
                finally
                {
                    //Cierra el flujo de lectura
                    if (fe != null) fe.Close();
                    Console.Write("Presione ENTER para continuar:");
                    Console.ReadKey();
                }
            }
        }

        static void Main(string[] args)
        {
            //Objeto para Archivos
            TextoBytes tb = new TextoBytes();

            //Variable auxiliar
            string Arc;

            //Captura del archivo
            Console.Write("Nombre del archivo a Crear y Leer: ");
            Arc = Console.ReadLine();

            //Creación del archivo
            Console.Clear();
            tb.CrearArchivo(Arc);

            //Lectura del archivo
            Console.Write("\n");
            tb.LeerArchivo();
        }
    }
}
        \end{verbatim}
    \end{justify}

    \newpage
        \printbibliography[heading=bibintoc]    
\end{document}