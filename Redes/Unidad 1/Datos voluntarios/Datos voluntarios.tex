\documentclass[stu, 12pt, letterpaper, donotrepeattitle, floatsintext, natbib]{apa7}
\usepackage[utf8]{inputenc}
%\usepackage{fontspec} %paquete para usar la fuente Arial 12
\usepackage{comment}
\usepackage{marvosym}
\usepackage{graphicx}
\usepackage{float}
\usepackage[normalem]{ulem}
\usepackage[spanish]{babel} 
\usepackage{lastpage} %para le formato que quiere la profe QUITAR SI QUIERES OG APA7
\usepackage{ragged2e} %para le formato que quiere la profe QUITAR SI QUIERES OG APA7
\usepackage{indentfirst} %para le formato que quiere la profe QUITAR SI QUIERES OG APA7

\setcounter{secnumdepth}{0} %permite enumerar las secciones QUITAR SI QUIERES OG APA7

%comando para ajustar la fuente Arial en todo el documento
%\setmainfont{Arial} %COMPILAR DOC CON XeLateX DOS VECES

\DeclareCaptionLabelSeparator*{spaced}{\\[2ex]}
\captionsetup[table]{textfont=it,format=plain,justification=justified,
  singlelinecheck=false,labelsep=spaced,skip=1pt}

\selectlanguage{spanish}

\useunder{\uline}{\ul}{}
\newcommand{\myparagraph}[1]{\paragraph{#1}\mbox{}\\}

%\rfoot{Página \thepage \hspace{1pt} de \pageref{LastPage}}%QUITAR SI QUIERES OG APA7 
\rhead{} %QUITAR SI QUIERES OG APA7
\setcounter{secnumdepth}{3} %permite enumerar las secciones QUITAR SI QUIERES OG APA7
\setlength{\parindent}{1.27cm} %sangria forzada QUITAR SI QUIERES OG APA7

\renewcommand\labelitemi{$\bullet$}

\newcommand*\chem[1]{\ensuremath{\mathrm{#1}}}

\begin{document}
    %PORTADA
    \begin{titlepage}
        \begin{figure}[ht]
            \centering
            \includegraphics[width=15cm]{logosITT.png}
        \end{figure}
        \centering
        {\Large\scshape Tecnológico Nacional de México\\Instituto Tecnológico de Tijuana\par}
        \vspace{1cm}
        {\Large SCD-1021SC6C Redes de computadoras\par}
        \vspace{1cm}
        {\Large Unidad 1\par}
        \vspace{2cm}
        {\Large\bfseries Datos voluntarios, observados e inferidos\par}
        \vspace{2cm}
        {\large Lic. Sergio Contreras Hernandez\par}
        \vfill
            {\large Abraham Jhared Flores Azcona, 19211640\par}
        \vfill
        {\large 11 de febrero de 2021}
    \end{titlepage}

% Índices
\pagenumbering{arabic}
    % Contenido
\renewcommand\contentsname{Contenido}
%\tableofcontents

% Cuerpo 
    %NOTA: PARA CITAR ESTILO "Merts (2003)" usar \cite{<nombre_cita_bib>}
    %                        "(Metz, 1978)" usar \citep{<nombre_cita_bib>}
\newpage
\section*{Resumen de los videos adjuntos}
En \begin{justifying}
    esencia los videos adjuntos describen las distintas maneras en las cuales nuestros datos digitales le permiten a Google facilitarnos
    muchos de los servicios y comodidades de UX que tomamos por hecho. A grandes rasgos, el proceso consiste en tres pasos:
    \emph{Uso de datos individuales, Uso de datos aledaños y Resultados}.\par
\end{justifying}
\begin{justifying}
    \begin{enumerate}
        \item \emph{Uso de datos individuales:} Google recolecta los datos del individuo con la problemática para discernír su situación actual acorde al servicio.
        \item \emph{Uso de datos aledaños:} Google recolecta los datos de otros usuarios que se han presentado con la problemática actual.
        \item \emph{Resultados:} Google determina acorde a los puntos anteriores las mejores alternativas/recomendaciones para el usuario y se las presenta.
    \end{enumerate}
\end{justifying}

\renewcommand\refname{\textbf{Referencias}}
%\bibliography{referencias} %el archivo 'referencias.bib' debe estar dentro del mismo folder donde se encuentra el archivo .tex para citar las referencias deseadas

\end{document}