\documentclass[stu, 12pt, letterpaper, donotrepeattitle, floatsintext, natbib]{apa7}
\usepackage[utf8]{inputenc}
%\usepackage{fontspec} %paquete para usar la fuente Arial 12
\usepackage{comment}
\usepackage{marvosym}
\usepackage{graphicx}
\usepackage{float}
\usepackage[normalem]{ulem}
\usepackage[spanish]{babel} 
\usepackage{lastpage} %para le formato que quiere la profe QUITAR SI QUIERES OG APA7
\usepackage{ragged2e} %para le formato que quiere la profe QUITAR SI QUIERES OG APA7
\usepackage{indentfirst} %para le formato que quiere la profe QUITAR SI QUIERES OG APA7
\usepackage{amsmath,multienum,booktabs}
\renewcommand{\regularlisti}{\setcounter{multienumi}{0}%
  \renewcommand{\labelenumi}%
  {\addtocounter{multienumi}{1}\alph{multienumi}}}

\setcounter{secnumdepth}{0} %permite enumerar las secciones QUITAR SI QUIERES OG APA7

%comando para ajustar la fuente Arial en todo el documento
%\setmainfont{Arial} %COMPILAR DOC CON XeLateX DOS VECES

\DeclareCaptionLabelSeparator*{spaced}{\\[2ex]}
\captionsetup[table]{textfont=it,format=plain,justification=justified,
  singlelinecheck=false,labelsep=spaced,skip=1pt}

\selectlanguage{spanish}

\useunder{\uline}{\ul}{}
\newcommand{\myparagraph}[1]{\paragraph{#1}\mbox{}\\}

%\rfoot{Página \thepage \hspace{1pt} de \pageref{LastPage}}%QUITAR SI QUIERES OG APA7 
\rhead{} %QUITAR SI QUIERES OG APA7
\setcounter{secnumdepth}{3} %permite enumerar las secciones QUITAR SI QUIERES OG APA7
%\setlength{\parindent}{1.27cm} %sangria forzada QUITAR SI QUIERES OG APA7

\renewcommand\labelitemi{$\bullet$}

\newcommand*\chem[1]{\ensuremath{\mathrm{#1}}}

\begin{document}
    %PORTADA
    \begin{titlepage}
        \begin{figure}[ht]
            \centering
            \includegraphics[width=15cm]{logosITT.png}
        \end{figure}
        \centering
        {\Large\scshape Tecnológico Nacional de México\\Instituto Tecnológico de Tijuana\par}
        \vspace{1cm}
        {\Large SCD-1021SC6C Redes de computadoras\par}
        \vspace{1cm}
        {\Large Unidad 2\par}
        \vspace{2cm}
        {\Large\bfseries Direccionamiento IP\par}
        \vspace{2cm}
        {\large Lic. Sergio Contreras Hernandez\par}
        \vfill
            {\large Abraham Jhared Flores Azcona, 19211640\par}
        \vfill
        {\large 8 de abril de 2021}
    \end{titlepage}

% Índices
\pagenumbering{arabic}
    % Contenido
\renewcommand\contentsname{Contenido}
%\tableofcontents

% Cuerpo 
    %NOTA: PARA CITAR ESTILO "Merts (2003)" usar \cite{<nombre_cita_bib>}
    %                        "(Metz, 1978)" usar \citep{<nombre_cita_bib>}
\newpage
\section*{Tabla 1}
\subsection*{Datos}
\begin{table}[]
    \centering
    \begin{tabular}{|l|l|l|l|l|}
    \hline
    \begin{tabular}[c]{@{}l@{}}Dirección del host\end{tabular}              & 10       & 14       & 1        & 124      \\ \hline
    \begin{tabular}[c]{@{}l@{}}Máscara  de subred\end{tabular}              & 255      & 255      & 240      & 0        \\ \hline
    \begin{tabular}[c]{@{}l@{}}Dirección del  host \\en binario\end{tabular}  & 00001010 & 00001110 & 00000001 & 01111100 \\ \hline
    \begin{tabular}[c]{@{}l@{}}Máscara de subred \\en binario\end{tabular} & 11111111 & 11111111 & 11110000 & 00000000 \\ \hline
    \begin{tabular}[c]{@{}l@{}}Dirección de\\ red en decimal\end{tabular}     &     10     &    14      &    0      &    0      \\ \hline
    \begin{tabular}[c]{@{}l@{}}Dirección de\\ red en binario\end{tabular}     &   00001010       &  00001110        &  00000000        &  00000000        \\ \hline
    \end{tabular}
    \end{table}
\subsection*{Procedimiento}
Debido \begin{justifying}
    a que estas direcciones usualmente se componen de 4 bytes (4 dígitos de 8 bits) podemos partír dichos valores de la siuiente manera:
    \[ \begin{matrix}
        \textbf{H} & = & H_1&.&H_2&.&H_3&.&H_4 \\
        \textbf{S} & = & S_1&.&S_2&.&S_3&.&S_4 \\
        \textbf{R} & = & R_1&.&R_2&.&R_3&.&R_4 \\
        \end{matrix}  \]\par
\end{justifying}
donde \begin{justifying}
    \textbf{H} representa todos los dígitos de la dirección del host y $H_1$ hasta $H_4$ son los bytes que lo componen.
Similarmente tenemos dicha convención para \textbf{S} y \textbf{R} que representa la máscara de subred y la dirección de red respectimiente;
junto a las convenciones respectivas de $S_1$ hasta $S_4$ y de $R_1$ hasta $R_4$.\par
    \end{justifying}
    \vspace{\baselineskip}
La fórmula general para obtener \textbf{R} es la siguiente:
\[\textbf{R}=\textbf{H}\land\textbf{S}\, .\]\par 
Por \begin{justifying}
    las convenciones de notación planteadas, podemos aplicar dicha operación a cada grupo de bytes para facilitar el cálculo y 
la presentación en la redacción como una operación aritmética vertical:
\[
\begin{array}{lr}
   & \textbf{H}  \\
\land  & \textbf{S} \\
\cmidrule(lr){2-2}
  & \textbf{R}  \\
\end{array}
\]
\end{justifying}\par
\vspace{\baselineskip}
Ahora podemos calcular sin mucho problema los bytes de \textbf{R} que son $R_1, R_2, R_3$ y $R_4$:
\[
    \begin{matrix}
        \begin{array}{lr}
           R_1:   & \\
               & 00001010  \\ %Host
            \land  & 11111111 \\ %Subnet
            \cmidrule(lr){2-2}
              & 00001010  \\ %Red
            \end{array}\,\,
        & 
        \begin{array}{lr}
            R_2:   & \\
               & 00001110  \\
            \land  & 11111111 \\
            \cmidrule(lr){2-2}
              & 00001110  \\
            \end{array}\,\,
        & 
        \begin{array}{lr}
            R_3:   & \\
               & 00000001  \\
            \land  & 11110000 \\
            \cmidrule(lr){2-2}
              & 00000000  \\
            \end{array}\,\,
        & 
        \begin{array}{lr}
            R_4:   & \\
               & 01111110  \\
            \land  & 00000000 \\
            \cmidrule(lr){2-2}
              & 00000000  \\
            \end{array}
        \end{matrix}
\]
por lo que $\textbf{R}_2=R_1$.$R_2$.$R_3$.$R_4=00001010$.$00001110$.$00000000$.$00000000$.\par
\vspace{\baselineskip}
Para calcular $\textbf{R}_{10}$ debemos aplicar la siguiente fórmula:
\[b_n2^{n-1}+b_{n-1}2^{n-2}+\dots+b_12^0\]
donde $b_n$ es el dígito binario de la $n$-ésima posición. Modularizamos los bytes de $\textbf{R}_2$ para
calcularos individualmente y después juntarlos sin problema.
\[\begin{matrix}
    R_1: & 0\times2^7+0\times2^6+0\times2^5+0\times2^4+1\times2^3+0\times2^2+1\times2^1+0\times2^0=8+2=10_{10} \\
    R_2: & 0\times2^7+0\times2^6+0\times2^5+0\times2^4+1\times2^3+1\times2^2+1\times2^1+0\times2^0=8+4+2=14_{10} \\
    R_3: & 0\times2^7+0\times2^6+0\times2^5+0\times2^4+0\times2^3+0\times2^2+0\times2^1+0\times2^0=0_{10}\\
    R_4: & 0\times2^7+0\times2^6+0\times2^5+0\times2^4+0\times2^3+0\times2^2+0\times2^1+0\times2^0=0_{10}\\     
\end{matrix}\]\par
por lo que $\textbf{R}_{10}=R_1$.$R_2$.$R_3$.$R_4=10$.14.0.0.
\vspace{\baselineskip}
\section*{Tabla 2}
\subsection*{Datos}
\begin{table}[]
    \centering
    \begin{tabular}{|l|l|l|l|l|}
        \hline
    \begin{tabular}[c]{@{}l@{}}Dirección del host\end{tabular}              & 172      & 184      & 169      & 75       \\ \hline
    \begin{tabular}[c]{@{}l@{}}Máscara de subred\end{tabular}              & 255      & 255      & 0        & 0        \\ \hline
    \begin{tabular}[c]{@{}l@{}}Dirección del  host \\en binario\end{tabular}  & 10101100 & 10111000 & 10101001 & 01001011 \\ \hline
    \begin{tabular}[c]{@{}l@{}}Máscara de subred \\en binario\end{tabular} & 11111111 & 11111111 & 00000000 & 00000000 \\ \hline
    \begin{tabular}[c]{@{}l@{}}Dirección de red \\en decimal\end{tabular}     &      172    &   184       &    0      &   0       \\ \hline
    \begin{tabular}[c]{@{}l@{}}Dirección de red \\en binario\end{tabular}     &      10101100    &     10111000     &     00000000     &    00000000      \\ \hline
    \end{tabular}
    \end{table}
\subsection*{Procedimiento}
Usaremos la misma notación de la Tabla 1 para encontrar \textbf{S} en ésta tabla. Calculamos los bytes de \textbf{R}:
\[
    \begin{matrix}
        \begin{array}{lr}
           R_1:   & \\
               & 10101100  \\ %Host
            \land  & 11111111 \\ %Subnet
            \cmidrule(lr){2-2}
              & 10101100  \\ %Red
            \end{array}\,\,
        & 
        \begin{array}{lr}
            R_2:   & \\
               & 10111000  \\
            \land  & 11111111 \\
            \cmidrule(lr){2-2}
              & 10111000  \\
            \end{array}\,\,
        & 
        \begin{array}{lr}
            R_3:   & \\
               & 10101001  \\
            \land  & 00000000 \\
            \cmidrule(lr){2-2}
              & 00000000  \\
            \end{array}\,\,
        & 
        \begin{array}{lr}
            R_4:   & \\
               & 01001011  \\
            \land  & 00000000 \\
            \cmidrule(lr){2-2}
              & 00000000  \\
            \end{array}
        \end{matrix}
\]
por lo que $\textbf{R}_2=R_1$.$R_2$.$R_3.R_4=10101100$.$10111000$.$00000000$.$00000000$.\par
\vspace{\baselineskip}
Ahora procedemos a calcular $\textbf{R}_{10}$ de manera similar a la Tabla 1:
\[\begin{matrix}
    R_1: & 1\times2^7+0\times2^6+1\times2^5+0\times2^4+1\times2^3+1\times2^2+0\times2^1+0\times2^0=8+2=\\
        & 128+32+8+4=172_{10}\\
    R_2: & 1\times2^7+0\times2^6+1\times2^5+1\times2^4+1\times2^3+0\times2^2+0\times2^1+0\times2^0=\\
        & 128+32+16+8=184_{10}\\
    R_3: & 0\times2^7+0\times2^6+0\times2^5+0\times2^4+0\times2^3+0\times2^2+0\times2^1+0\times2^0=0_{10}\\
    R_4: & 0\times2^7+0\times2^6+0\times2^5+0\times2^4+0\times2^3+0\times2^2+0\times2^1+0\times2^0=0_{10}\\     
\end{matrix}\]\par
por lo que $\textbf{R}_{10}=R_1$.$R_2$.$R_3$.$R_4=172$.184.0.0.
\end{document}