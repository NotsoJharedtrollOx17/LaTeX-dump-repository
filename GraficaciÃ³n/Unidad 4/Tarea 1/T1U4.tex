% Preámbulo
\documentclass[stu, 12pt, letterpaper, donotrepeattitle, floatsintext, natbib]{apa7}
\usepackage[utf8]{inputenc}
\usepackage{comment}
\usepackage{marvosym}
\usepackage{graphicx}
\usepackage{float}
\usepackage{amsmath}
\usepackage[normalem]{ulem}
\usepackage[spanish]{babel} 
\usepackage{indentfirst} %para le formato que quiere la profe QUITAR SI QUIERES OG APA7
\usepackage{ragged2e} %para le formato que quiere la profe QUITAR SI QUIERES OG APA7
\usepackage{indentfirst} %para le formato que quiere la profe QUITAR SI QUIERES OG APA7
\usepackage{multirow,booktabs,setspace,caption} %formato de figuras APA
\DeclareCaptionLabelSeparator*{spaced}{\\[2ex]}
\DeclareCaptionLabelSeparator*{spaced}{\\[2ex]}
\captionsetup[figure]{textfont=it,format=plain,justification=justified,
  singlelinecheck=false,labelsep=spaced,skip=0pt}

\selectlanguage{spanish}
\useunder{\uline}{\ul}{}
\newcommand{\myparagraph}[1]{\paragraph{#1}\mbox{}\\}

% Portada
%\thispagestyle{empty}
\title{\Large Tarea 1 Unidad 4: Técnicas para relleno, iluminación y sombreado}
\author{Abraham Jhared Flores Azcona} % (autores separados, consultar al docente)
% Manera oficial de colocar los autores:
%\author{Autor(a) I, Autor(a) II, Autor(a) III, Autor(a) X}
\affiliation{Instituto Tecnológico de Tijuana}
\course{SCC-1010SC5C: Graficación}
\professor{Dra. Martha Elena Pulido}
\duedate{19 de septiembre de 2021}

\begin{document}
    % Índices
    \pagenumbering{arabic}
    \maketitle

    % Contenido
    \renewcommand\contentsname{Contenido}
    \tableofcontents

    % Cuerpo 
    %NOTA: PARA CITAR ESTILO "Merts (2003)" usar \cite{<nombre_cita_bib>}
    %    
    \newpage
    \section{Técnicas}
    \subsection{De relleno}
    A \begin{justifying}
    grandes rasgos, el relleno depende (por razones de dimensión) si éste es para
    objetos 2D ó 3D. En general el relleno tridimensional se clasifica en métodos geométricos
    y en volumétricos donde el primero se basa en un meshing de triángulos donde los huecos son parchados
    al lidiar con los triángulos directamente; y el segundo funciona con el rellenado en un espacio volumétrico \citep{inproceedings}.\par %citar a los egípcios
  \end{justifying}
    En lo competente a los objetos 2D se hacen específicamente por polígonos; de ahí se parte a otros estílos.
    Los de scan-line trazan líneas de color fila por fila; la línea de barrido básicamente es una manera heurística
    de encontrar puntos en base a quienes están dentro de las líneas; la inundación consiste en un punto inicial
    y pinta con idcho color del punto el resto de la imágen \citep{unknown-author-no-dateA}.\par %citar a lupita
    \vspace{\baselineskip}
    \subsection{De iluminación}
    Compete \begin{justifying}
      al cómo calcular la intensidad de la luz al reflejarse aen un punto dado de una superficie. La primer técnica es aquella de
      la iluminación ambiental es donde su fuente de luz es indirecta; la reflexión difusa ocurre en superficies que son ásperas ó granulares
      donde dicho brillo depende del ángulo hecho por la fuente de luz y la superficie; y finalmente la reflexión especular que ocurre en superficies
      brillosas o de vídrio que refleja de vuelta dicha luz \citep{geeksforgeeks-2020}.\par %el geeks for geeks
    \end{justifying}
    \vspace{\baselineskip}
    \subsection{De sombreado}
    En \begin{justifying}
      general, exísten tres métodos principales: sombreado plano, sombreado Gouraud y sombreado Phong.
      El primero consiste en definír un sólo color para un objeto por un filtro bínario; Gouraud consiste en usar
      la norma de un vector donde su color se calcula en el shader vectorial; finalmente la técnica Phong consiste
      en calcular los colores por fragmentos, tomando en cuenta la norma del vector interpolado \citep{unknown-author-no-dateB}.\par %citar al de computer graphics
    \end{justifying}
    \vspace{\baselineskip}
    
    \newpage   
    % Referencias
    \renewcommand\refname{\textbf{Referencias}}
    \bibliography{referencias}
    
\end{document}