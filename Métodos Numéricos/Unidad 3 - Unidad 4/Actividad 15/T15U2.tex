\documentclass[letterpaper, 12pt]{article}
\usepackage[letterpaper, top=2.5cm, bottom=2.5cm, left=3cm, right=3cm]{geometry} %margenes
\usepackage[utf8]{inputenc} %manejo de caracteres especiales
\usepackage[spanish]{babel} %manejo de encabezados de inglés a español
\usepackage{fancyhdr} %formato de los encabezados de página
\usepackage{ragged2e} %alineado real justficado
\usepackage{graphicx} %manejo de imagenes
\usepackage{amsmath} %manejo de notación matemática
\usepackage{mathtools} %manejo de notación matemática
\usepackage{blindtext} %texto de relleno
\usepackage{amssymb} %manejo de simbología matematica

\pagestyle{fancy}
\fancyhf{}
\rfoot{}

\begin{document}
\thispagestyle{fancy}
\lhead{\textbf{Nombre: Abraham Jhared Flores Azcona\\\#: 19211640}}
\rhead{\textbf{Actividad 15\\Integración por Simpson 1/3}}
\subsection*{Elabora un resúmen de lo explicado en los dos videos adjuntos a la actividad de Classroom con procedimientos y fórmulas}
\justify
Recordando que:
{\large \[\int_a^b f(x)\, dx\]}
\justify
Si se desea calcular dicha integral con el método de Simpson (al menos por un polinómio de tres términos) se tiene que tener en cuenta lo siguiente:
\\\newline
Dado \(\int_{1.2}^{1.6} \sin \left(x^2\right)\, dx\), Simpson sugiere que aproximemos la curva a integrar con un polinómio formado (para el caso de 3 términos)
por tres valores:
{\large \[g(x)=-3.8701x^2+9.7311x-5.1129\]}
Donde dicha triada es relacionada a la evaulación siguiente:
{\large \[\begin{matrix}
    x & f(x) \\
    1.2 & 0.9914\\
    1.4 & 0.9252\\
    1.6 & 0.5493
\end{matrix}\]}
\justify
\(g(x)\) se integra dentro del mismo intervalo de la integral original.
\\\newline
De manera más general, la evaulación de la tabla anterior es:
{\large\[\begin{matrix}
    x & f(x) \\
    a & f(a) \\
    {a+b\over 2} & f\left({a+b\over 2}\right)\\
    b & f(b)
\end{matrix}\]}
\justify
Y su polinómio se calcula con el método de Lagrange, el cual se generaliza como:
{\large \[f(x)={(x-m)(x-b)\over (a-m)(a-b)}f(a)+{(x-a)(x-b)\over (m-a)(m-b)}f(m)+{(x-a)(x-m)\over (b-a)(b-m)}f(b)\]}
\justify
Donde \(m={a+b\over 2}\).
\\\newline
Desarrollando, \(\int_a^b f(x)\, dx\) finalmente se calcula por el método del trapecio, lo que nos queda:
{\large \[\int_a^b f(x)\, dx=\left({b-a\over 6}\right)\left(f(a)+4f\left({a+b\over 2}\right)+f(b)\right)\]}
\end{document}