\documentclass[letterpaper, 12pt]{article}
\usepackage[letterpaper, top=2.5cm, bottom=2.5cm, left=3cm, right=3cm]{geometry} %margenes
\usepackage[utf8]{inputenc} %manejo de caracteres especiales
\usepackage[spanish]{babel} %manejo de encabezados de inglés a español
\usepackage{fancyhdr} %formato de los encabezados de página
\usepackage{ragged2e} %alineado real justficado
\usepackage{graphicx} %manejo de imagenes
\usepackage{amsmath} %manejo de notación matemática
\usepackage{mathtools} %manejo de notación matemática
\usepackage{blindtext} %texto de relleno
\usepackage{amssymb} %manejo de simbología matematica
\usepackage{tikz} %figuras 

\pagestyle{fancy}
\fancyhf{}
\rfoot{}

\begin{document}
\thispagestyle{fancy}
\lhead{\textbf{Nombre: Abraham Jhared Flores Azcona\\\#: 19211640}}
\rhead{Actividad 14\\Integración por trapecio múltiple}
\subsection*{Elabora un resumen de lo explicado en los dos videos adjuntos a la actividad de Classroom con procedimientos y fórmulas}
\justify
Como bien se sabe, una de las interpretaciones de la integral definida es como el área debajo de la gráfica de una función donde una de sus definiciones
es la de la sumatoria de Riemman:
{\large \[\int_a^b f(x)\, dx =\lim_{n \rightarrow \infty}\sum_{i=1}^{n}f(c_i)\Delta x,\, c_i=a+i\Delta x,\, \Delta x = {b-a\over n}\]}
\justify
Si se decide eliminar el límite, se obtiene la variante finíta de la integral, lo cual es una aproximación a dicha integral y se puede considerar como
un método numérico:
{\large \[\int_a^b f(x)\, dx \approx \sum_{i=1}^{n}f(c_i)\Delta x,\, c_i=a+i\Delta x,\, \Delta x = {b-a\over n}\]}
\justify
Notese que en ambas expresiones, \(f(c_i)\) y \(\Delta x\) cumplen la función de ser la base y la altura respectivas de un rectangulo ya que:
\\
\begin{center}
    \begin{tikzpicture}
        \draw (0,0)  -| (2,4) 
    node[pos=0.25,below] {$\beta=\Delta x$} 
    node[pos=0.75,right] {$h=f(c_i)$}
    -| (0,0); 
    \node[text width=5cm, anchor=west, right] at (4,2)
        {{\large \[A_{rectangulo}=\beta \times h\]}};
    \end{tikzpicture}
\end{center}
Aplicando el mismo principio de área, ahora recordemos la fórmula del área de un trapecio:
{\large \[A_{trapecio}=\frac{1}{2}(B+\beta)\, h\]}
\justify
Donde \(B\) es la base mayor, \(\beta\) es la base menor y \(h\) representa la altura. Si se reescribe \(A_{trapecio}\) con los términos similares a la sumatoria de Riemman:
{\large \[A_{trapecio}=\frac{1}{2}(f(a)+f(b))\Delta x=\frac{1}{2}(f(a)+f(b))\left({b-a\over n}\right)\]}
\justify
La expresión anterior solo es valida si \(n=1\) ya que literalmente estamos calculando el area de un solo trapecio y la idea de la integral (al menos por Riemann) es una sumatoria del area
de dichas figuras. Notese lo siguiente cuando se desea evaluar \(\int_2^6 f(x)\, dx\):
{\large \[A_{trapecio}=\frac{1}{2}(6-2)(f(2)+f(6)),\, n=1;\]}
{\large \[A_{trapecio}=\frac{1}{2}\left({6-2 \over 2}\right)(f(2)+f(4))+\frac{1}{2}\left({6-2 \over 2}\right)(f(4)+f(6)),\, n=2;\]}
{\large \begin{equation*}
    \begin{aligned}
        A_{trapecio}&=\frac{1}{2}\left({6-2 \over 4}\right)(f(2)+f(3))+\frac{1}{2}\left({6-2 \over 4}\right)(f(3)+f(4))\\[5pt]
        +\frac{1}{2}&\left({6-2 \over 4}\right)(f(4)+f(5))+\frac{1}{2}\left({6-2 \over 4}\right)(f(5)+f(6)),\, n=4;\\[5pt]
    \end{aligned}
\end{equation*}}
\justify
Dependiendo de \(n\), se determinan las subdivisiones de incrementos, y por ende los calculos para la función, simplificando:
{\large \[A_{trapecio}=\frac{1}{2}(6-2)(f(2)+f(6)),\, n=1;\]}
{\large \[A_{trapecio}=\frac{1}{2}\left({6-2 \over 2}\right)\left(f(2)+f(4)+f(4)+f(6)\right),\, n=2;\]}
{\large \begin{equation*}
    \begin{aligned}
        A_{trapecio}&=\frac{1}{2}\left({6-2 \over 4}\right)(f(2)+f(3)+f(3)+f(4)+f(4)+f(5)+\\[5pt]
        f(5)+&f(6)),\, n=4;\\[5pt]
    \end{aligned}
\end{equation*}}
\justify
Se factoriza \(\frac{1}{2}\left({b-a\over n}\right)\) de la expresión, lo que nos da una pista. Prosiguiendo:
{\large \[A_{trapecio}=\frac{1}{2}(6-2)(f(2)+f(6)),\, n=1;\]}
{\large \[A_{trapecio}=\frac{1}{2}\left({6-2 \over 2}\right)\left(f(2)+2f(4)+f(6)\right),\, n=2;\]}
{\large \begin{equation*}
    \begin{aligned}
        A_{trapecio}&=\frac{1}{2}\left({6-2 \over 4}\right)\left(f(2)+2(f(3)+f(4)+f(5))+f(6)\right),\, n=4;\\[5pt]
    \end{aligned}
\end{equation*}}
\justify
\(f(a)\) y \(f(b)\) siempre quedan en los extremos laterales de la expresión y los demás terminos se duplican, el incremento de estos es determinado por \(c_i\), empleado en la sumatoria de Riemman.
Con la intuición planteada, la integral por el método del trapecio termina siendo:
{\large \[\frac{1}{2}\Delta x\left(f(a)+2\sum_{i=1}^{n-1}f(c_i)+f(b)\right),\, c_i=a+i\Delta x,\, \Delta x = {b-a\over n}\]}
\end{document} 