\documentclass[letterpaper, 12pt]{article}
\usepackage[letterpaper, top=2.5cm, bottom=2.5cm, left=3cm, right=3cm]{geometry} %margenes
\usepackage[utf8]{inputenc} %manejo de caracteres especiales
\usepackage[spanish]{babel} %manejo de títulos automáticos a español
\usepackage{fancyhdr} %formato de los encabezados de página
\usepackage{ragged2e} %alineado real justficado
\usepackage{graphicx} %manejo de imagenes
\usepackage{amsmath} %manejo de notación matemática
\usepackage{mathtools} %manejo de notación matemática
\usepackage{blindtext} %texto de relleno
\usepackage{amssymb} %manejo de simbología matematica
\usepackage{float} % centrado de imagenes

\pagestyle{fancy}
\fancyhf{}
\rfoot{}

\begin{document}
\thispagestyle{fancy}
\lhead{\textbf{Nombre: Abraham Jhared Flores Azcona\\\#: 19211640}}
\rhead{Actividad 20\\Interpolación polinomial por sistemas de ecuaciones}
\subsection*{Elaborar un resumen del video adjunto a la actividad de Classroom}
\justify
Para realizar la Interpolación polinomial de primer grado (la recta) se necesitan al menos dos puntos
cartesianos \(P_1\) y \(P_2\) y la expresión polinomial de la recta en forma de ordenada al origen:
{\large \[f(x)=a_0+a_1x\]}
\justify
Donde \(a_0\) y \(a_1\) son constantes.
\subsubsection*{Ejemplo:}
\justify
Teniendo los puntos \((2,1)\) y \((6,4)\), se desea obtener la interpolación lineal de estos puntos.
\\\newline
Recordando que \((x,y)=(x,f(x))\), procedemos a sustituir los valores en la expresión polinomial antes expuesta:
\\\newline
• Para (2,1):
{\large\begin{equation*}
    \begin{aligned}
        f(x)&=a_0+a_1x\\
        1&=a_0+a_1(2)\\
        1&=a_0+2a_1
    \end{aligned}
\end{equation*}}
\justify
• Para (6,4):
{\large\begin{equation*}
    \begin{aligned}
        f(x)&=a_0+a_1x\\
        4&=a_0+a_1(6)\\
        4&=a_0+6a_1
    \end{aligned}
\end{equation*}}
\justify
Por lo que se forma el sistema de ecuaciones:
{\large\begin{equation*}
    \begin{aligned}
        a_0+2a_1=1\\
        a_0+6a_1=4
    \end{aligned}
\end{equation*}}
\justify
Que se procede a resolver para \(a_0\) y \(a_1\). Usando un emulador de una calculadora CASIO, nos arroja que 
\(a_0=-\frac{1}{2}\) y que \(a_1=-\frac{3}{4}\) por lo que la interpolación lineal arroja la siguiente ecuación:
{\large\[f(x)=-\frac{1}{2}+\frac{3}{4}x\]}

\newpage
\justify
Un proceso similar ocurre para la interpolación cuadrática por sistemas de ecuaciones. La condición es que se haya provisto
de tres puntos cartesianos \(P_1,P_2\) y \(P_3\) que nos arrojaría la siguiente expresión polinomial de segundo grado:
{\large\[f(x)=a_0+a_1x+a_2x^2\]}
\justify
Donde \(a_0,a_1\) y \(a_2\) son constantes.
\subsubsection*{Ejemplo}
\justify
Teniendo los puntos \((0,2)\), \((4,6)\) y \((8,3)\), se desea obtener la interpolación cuadrática de estos.
\\\newline
Procedemos a sustitur los valores en la expresión polinomial cuadrática antes expuesta:
\\\newline
• Para (0,2):
{\large\begin{equation*}
    \begin{aligned}
        f(x)&=a_0+a_1x+a_2x^2\\
        2&=a_0+a_1(0)+a_2(0)^2\\
        2&=a_0+0a_1+0a_2
    \end{aligned}
\end{equation*}}
• Para (4,6):
{\large\begin{equation*}
    \begin{aligned}
        f(x)&=a_0+a_1x+a_2x^2\\
        6&=a_0+a_1(4)+a_2(4)^2\\
        6&=a_0+4a_1+16a_2
    \end{aligned}
\end{equation*}}
• Para (8,3):
{\large\begin{equation*}
    \begin{aligned}
        f(x)&=a_0+a_1x+a_2x^2\\
        3&=a_0+a_1(8)+a_2(8)^2\\
        3&=a_0+8a_1+64a_2
    \end{aligned}
\end{equation*}}
Por lo que se forma el sistema de ecuaciones:
{\large\begin{equation*}
    \begin{aligned}
        a_0+0a_1+0a_2=2\\
        a_0+4a_1+16a_2=6\\
        a_0+8a_1+64a_2=3
    \end{aligned}
\end{equation*}}
\justify
Que se procede a resolver para \(a_0,a_1\) y \(a_2\). Usando un emulador de una calculadora CASIO, nos arroja que
\(a_0=2,a_1=\frac{15}{8}\) y que \(a_2=-\frac{7}{32}\) por lo que la interpolación cuadrática arroja la siguiente ecuación:
{\large \[f(x)=2+\frac{15}{8}x-\frac{7}{32}x^2\]}

\newpage
\justify
Similarmente, el proceso es parecido para interpolaciones cúbicas. Se necesitan cuatro puntos cartesioanos que arrojarían
la siguiente expresión:
{\large \[f(x)=a_0+a_1x+a_2x^2+a_3x^3\]}
Donde \(a_0,a_1,a_2\) y \(a_3\) son constantes.
\subsubsection*{Ejemplo}
\justify
Teniendo los puntos \((0,2), (3,0), (5,4)\) y \((10,0)\), se desea obtener la interpolación cúbica de estos de estos puntos.
\\\newline
Procedemos a sustitur los valores en la expresión polinomial cúbica antes expuesta:
\\\newline
• Para (0,2):
{\large\begin{equation*}
    \begin{aligned}
        f(x)&=a_0+a_1x+a_2x^2+a_3x^3\\
        2&=a_0+a_1(0)+a_2(0)^2+a_3(0)^3\\
        2&=a_0+0a_1+0a_2+0a_3
    \end{aligned}
\end{equation*}}
• Para (3,0):
{\large\begin{equation*}
    \begin{aligned}
        f(x)&=a_0+a_1x+a_2x^2+a_3x^3\\
        0&=a_0+a_1(3)+a_2(3)^2+a_3(3)^3\\
        0&=a_0+3a_1+9a_2+27a_3
    \end{aligned}
\end{equation*}}
• Para (5,4):
{\large\begin{equation*}
    \begin{aligned}
        f(x)&=a_0+a_1x+a_2x^2+a_3x^3\\
        4&=a_0+a_1(5)+a_2(5)^2+a_3(5)^3\\
        4&=a_0+5a_1+25a_2+125a_3
    \end{aligned}
\end{equation*}}
• Para (10,0):
{\large\begin{equation*}
    \begin{aligned}
        f(x)&=a_0+a_1x+a_2x^2+a_3x^3\\
        0&=a_0+a_1(10)+a_2(10)^2+a_3(10)^3\\
        0&=a_0+10a_1+100a_2+1000a_3
    \end{aligned}
\end{equation*}}
Por lo que se forma el siguiente sistema de ecuaciones:
{\large\begin{equation*}
    \begin{aligned}
        a_0+0a_1+0a_2+0a_3=2\\
        a_0+3a_1+9a_2+27a_3=0\\
        a_0+5a_1+25a_2+125a_3=4\\
        a_0+10a_1+100a_2+1000a_3=0
    \end{aligned}
\end{equation*}}
\justify
Que se procede a resolver para \(a_0,a_1,a_2\) y \(a_3\). Usando un emulador de una calculadora CASIO, nos arroja que
\(a_0=2,a_1=-\frac{11}{3},a_2=\frac{32}{25}\) y que \(a_3=-\frac{7}{75}\) por lo que la interpolación cúbica arroja la siguiente ecuación:
{\large\[f(x)=2-\frac{11}{3}x+\frac{32}{25}x^2-\frac{7}{75}x^3\]}
\end{document}