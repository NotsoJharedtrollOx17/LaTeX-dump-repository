\documentclass[letterpaper, 12pt]{article}
\usepackage[letterpaper, top=2.5cm, bottom=2.5cm, left=3cm, right=3cm]{geometry} %margenes
\usepackage[utf8]{inputenc} %manejo de caracteres especiales
\usepackage[spanish]{babel} %manejo de encabezados de inglés a español
\usepackage{fancyhdr} %formato de los encabezados de página
\usepackage{ragged2e} %alineado real justficado
\usepackage{graphicx} %manejo de imagenes
\usepackage{amsmath} %manejo de notación matemática
\usepackage{mathtools} %manejo de notación matemática
\usepackage{blindtext} %texto de relleno
\usepackage{amssymb} %manejo de simbología matematica
\usepackage{float} % centrado de imagenes

\pagestyle{fancy}
\fancyhf{}
\rfoot{}

\begin{document}
\thispagestyle{fancy}
\lhead{\textbf{Nombre: Abraham Jhared Flores Azcona\\\#: 19211640}}
\rhead{Actividad 17\\Lagrange parte I}
\subsection*{Elaborar un resumen del video adjunto a la actividad de Classroom}
\justify
El polinomio de Lagrange requiere de al menos tres puntos provistos, que en este caso son \(\left(1,10\right)\), 
\(\left(2,15\right)\) y \(\left(3,30\right)\). El bloque base para esta interpolación es el siguiente:
{\large\[{(x-1)(x-2)(x-3) \over (\:\:\:-1)(\:\:\:-2)(\:\:\:-3)}\]}
Con ello, la nueva expresión queda de la siguiente manera:
{\large\begin{equation*}
    \begin{aligned}
        f(x)=\:&\underbrace{{(x-1)(x-2)(x-3)\over(1-1)(1-2)(1-3)}\cdot 10}_{\left(1,10\right)}+
\underbrace{{(x-1)(x-2)(x-3)\over(2-1)(2-2)(2-3)}\cdot 15}_{\left(2,15\right)}+\\[5pt]
&\underbrace{{(x-1)(x-2)(x-3)\over(3-1)(3-2)(3-3)}\cdot 30}_{\left(3,30\right)}\\[5pt]
    \end{aligned}
\end{equation*}}
Borramos algunos términos:
{\large\begin{equation*}
    \begin{aligned}
        f(x)=\:&\underbrace{{(x-2)(x-3)\over(1-2)(1-3)}\cdot 10}_{\left(1,10\right)}+
\underbrace{{(x-1)(x-3)\over(2-1)(2-3)}\cdot 15}_{\left(2,15\right)}+
\underbrace{{(x-1)(x-2)\over(3-1)(3-2)}\cdot 30}_{\left(3,30\right)}\\[5pt]
    \end{aligned}
\end{equation*}}
Y reducimos las expresiones para tener la interpolación por Langrange:
{\large\begin{equation*}
    \begin{aligned}
        f(x)=\:&\underbrace{{(x-2)(x-3)\over(-1)(-2)}\cdot 10}_{\left(1,10\right)}+
\underbrace{{(x-1)(x-3)\over(1)(-1)}\cdot 15}_{\left(2,15\right)}+
\underbrace{{(x-1)(x-2)\over(2)(1)}\cdot 30}_{\left(3,30\right)}\\[5pt]
    \end{aligned}
\end{equation*}}
\justify
Con ello, procedemos a comprobar el polinomio de Lagrange con los valores de \(x\) de las coordenadas provistas.
\\\newline
\textbf{• \(\left(1,10\right)\):}
{\large\begin{equation*}
    \begin{aligned}
        f(1)=\:&{(1-2)(1-3)\over(-1)(-2)}\cdot 10+
{(1-1)(1-3)\over(1)(-1)}\cdot 15+
{(1-1)(1-2)\over(2)(1)}\cdot 30=\\[5pt]
&{(-1)(-2)\over(-1)(-2)}\cdot 10=10
    \end{aligned}
\end{equation*}}
\textbf{• \(\left(2,15\right)\):}
{\large\begin{equation*}
    \begin{aligned}
        f(2)=\:&{(2-2)(2-3)\over(-1)(-2)}\cdot 10+
{(2-1)(2-3)\over(1)(-1)}\cdot 15+
{(2-1)(2-2)\over(2)(1)}\cdot 30=\\[5pt]
&{(1)(-1)\over(1)(-1)}\cdot 15=15
    \end{aligned}
\end{equation*}}
\textbf{• \(\left(3,30\right)\):}
{\large\begin{equation*}
    \begin{aligned}
        f(3)=\:&{(3-2)(3-3)\over(-1)(-2)}\cdot 10+
{(3-1)(3-3)\over(1)(-1)}\cdot 15+
{(3-1)(3-2)\over(2)(1)}\cdot 30=\\[5pt]
&{(2)(1)\over(2)(1)}\cdot 30=30\\[5pt]
    \end{aligned}
\end{equation*}}
\justify
Lo que confirma que nuestra interpolación por Langrange es correcta para los tres puntos provistos.
\end{document}