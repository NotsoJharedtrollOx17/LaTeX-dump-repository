% Preámbulo
\documentclass[stu, 12pt, letterpaper, donotrepeattitle, floatsintext, natbib]{apa7}
\usepackage[utf8]{inputenc}
\usepackage{comment}
\usepackage{marvosym}
\usepackage{graphicx}
\usepackage{float}
\usepackage{amsmath}
\usepackage[normalem]{ulem}
\usepackage[spanish]{babel} 
\usepackage{indentfirst} %para le formato que quiere la profe QUITAR SI QUIERES OG APA7
\usepackage{ragged2e} %para le formato que quiere la profe QUITAR SI QUIERES OG APA7
\usepackage{indentfirst} %para le formato que quiere la profe QUITAR SI QUIERES OG APA7

\selectlanguage{spanish}
\useunder{\uline}{\ul}{}
\newcommand{\myparagraph}[1]{\paragraph{#1}\mbox{}\\}

% Portada
%\thispagestyle{empty}
\title{\Large Tarea 3 Unidad 5: Aspectos Matemáticos de las Principales Técnicas de Animación 3D}
\author{Abraham Jhared Flores Azcona} % (autores separados, consultar al docente)
% Manera oficial de colocar los autores:
%\author{Autor(a) I, Autor(a) II, Autor(a) III, Autor(a) X}
\affiliation{Instituto Tecnológico de Tijuana}
\course{SCC-1010SC5C: Graficación}
\professor{Dra. Martha Elena Pulido}
\duedate{11 de noviembre de 2021}

\begin{document}
    % Índices
    \pagenumbering{arabic}
    \maketitle

    % Cuerpo 
    %NOTA: PARA CITAR ESTILO "Merts (2003)" usar \cite{<nombre_cita_bib>}
    %    
    \newpage
    \section{Cel-Shaded}
    Es \begin{justifying}
      un tipo de renderizaci no fotorealísta diseñada para hacer que los gráficos por computadora parezcan dibujados a mano.
      Es comúnmente usado para imitar el estilo de los cómics o dibujos animados a mano. \citep{blanchard-2015}\par %citar al blanchard
    \end{justifying}
    Comienza \begin{justifying}
      con un modelo 3D típico y se diferencia de la renderización convencional en su uso de iluminación no fotorrealistica, en la forma de sombrear
      y dado a que no requiere una gran cantidad de calculos dado a su diseño minimalista.\par
    \end{justifying}
    \vspace{\baselineskip}
    \section{Morph}
    Es un \begin{justifying}
      metodo donde se tiene una posicion neutral, y ciertas deformidades objetivo o \emph{morphs}. Este es muy bueno para animaciones
      que no requieren una estruc tura esqueletica.\par \citep{h-2018}%citar al HD
    \end{justifying}
    El cambio \begin{justifying}
      de estado puede ocurrir cambiando el color de piel en función al valor del pixel final exclusivamente. O donde el pixel cambia de posición
      y ocurre una direccionalidad o movimiento de la imagen, donde ahora el valor depende de un recorrido en las imágenes.\par
    \end{justifying}
    \vspace{\baselineskip}
    \section{Skeletal}
    Es un \begin{justifying}
      estilo de animacion donde dos partes individuales estan coordenadas: la primera es la piela y la segunda es un conjunto de huesos o el esqueleto, el cual se
      usa para conducir los comandos para la animacion.\par \citep{techopedia-2015}%citar a techopedia
    \end{justifying}
    Al usar un esqueleto, éste puede moverse mediante el uso de coordenadas. Para movimientos muy pelirosos para acores de movimiento, existen simulaciones computaizadas que
    automaticamente calculen la física del movimiento y la resistencia de los esqueletos.\par
    \vspace{\baselineskip}
    \section{Interpolated Motion Capture}
    Esta \begin{justifying}
      tecnica se utiliza para grabar movimiento en personas. Esta informacion de acciones y movimiento, es utilizada para poder aplicarla como animacion a personajes digitales.
      Los proceso de motion capture mas convencionales sirven para capturar movimientos o expresiones de los usuarios. Los datos almacenados por la captura de sewnsoresa son enviados
      a un modelo 3D, que se ejecuto el actor o la persona que fue grabada con los sensores. \citep{unknown-author-1963}\par %citear a D.
    \end{justifying}
    Los \begin{justifying}
      sensores de movimiento mandan coordenadas \(x, y\) y \(z\) al computador. Este realiza una serie de cálculos para que el modelo realice los movimientos capturados por el traje.\par
    \end{justifying}
    \vspace{\baselineskip}
    \section{Crowds}
    Este es \begin{justifying}
      un sistema diseñado para simular el comportamiento de tumultos reales. Esta anima a delegados; objetos auxiliares que actuan como representantes
      y a estos se le proveen lineamientos para comportarsde, y la simulacion de la horda calcula el movimiento. \citep{autodesk-2018}\par %citar al autodesk
    \end{justifying}
    \vspace{\baselineskip}
    
    \newpage   
    % Referencias
    \renewcommand\refname{\textbf{Referencias}}
    \bibliography{referencias}
    
\end{document}