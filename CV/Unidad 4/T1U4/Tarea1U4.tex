\documentclass[letterpaper, 12pt]{article}
\usepackage[letterpaper, top=2.5cm, bottom=2.5cm, left=3cm, right=3cm]{geometry} %margenes
\usepackage[utf8]{inputenc} %manejo de caracteres especiales
\usepackage[spanish]{babel} %manejo de encabezados de inglés a español
\usepackage{fancyhdr} %formato de los encabezados de página
\usepackage{ragged2e} %alineado real justficado
\usepackage{graphicx} %manejo de imagenes
\usepackage{amsmath} %manejo de notación matemática
\usepackage{mathtools} %manejo de notación matemática
\usepackage{blindtext} %texto de relleno
\usepackage{cancel} %permite la simbolización de cancelación de terminos
\usepackage{enumitem}[shortlabels] %listas con letras
\usepackage{amssymb} %manejo de simbolog►1a matematica

\pagestyle{fancy}
\fancyhf{}
\rfoot{\thepage}

\begin{document}
\setcounter{page}{1}
\thispagestyle{fancy}
\lhead{\textbf{Tarea 1, U4}}
\rhead{\textbf{1 de diciembre del 2020}}
\section*{Funciones reales de varias variables}
\subsection*{Encontrar el dominio de las siguientes funciones}
\[f(x,y)=\sqrt{xy}\]
\[f(x,y)=\ln\,(x-y^2)\]
\subsection*{Resolución}
\justify
1. \(f(x,y)=\sqrt{xy}\):
\[\begin{matrix}
    \sqrt{xy}&\geq&0\\
    xy&\geq&0\\
    x&\geq&0\\
    y&\geq&0
\end{matrix}\]
\[\therefore \mathbb{D}: \{(x,y)|\, x \geq 0 \lor y \geq 0 \}\]
2. \(f(x,y)=\ln\, (x-y^2)\):
\[\begin{matrix}
    x-y^2&\geq&0\\
    y^2-x&\leq&0\\
    y^2&\leq&x\\
    y&\leq&\pm \sqrt{x} \therefore\\
    x&\geq&0
\end{matrix}\]
\[\therefore \mathbb{D}: \{(x,y)|\, x \geq 0 \land y \leq \sqrt{x}\}\]
\end{document}