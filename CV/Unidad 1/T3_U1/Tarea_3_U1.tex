\documentclass[letterpaper, 12pt]{article}
\usepackage[letterpaper, top=2.5cm, bottom=2.5cm, left=3cm, right=3cm]{geometry} %margenes
\usepackage[utf8]{inputenc} %manejo de caracteres especiales
\usepackage[spanish]{babel} %manejo de encabezados de inglés a español
\usepackage{fancyhdr} %formato de los encabezados de página
\usepackage{ragged2e} %alineado real justficado
\usepackage{graphicx} %manejo de imagenes
\usepackage{amsmath} %manejo de notación matemática
\usepackage{mathtools} %manejo de notación matemática
\usepackage{blindtext} %texto de relleno
\usepackage{cancel} %permite la simbolización de cancelación de terminos
\usepackage{enumitem}[shortlabels] %listas con letras
\usepackage{amssymb} %manejo de simbolog►1a matematica

\pagestyle{fancy}
\fancyhf{}
\rfoot{\thepage}

\begin{document}

\setcounter{page}{1}
\thispagestyle{fancy}
\lhead{\textbf{Tarea 3, U1}}
\rhead{\textbf{21/09/2020}}
\section{Vectores en el espacio}
\subsection*{Encuentra los 6 vertices restantes del paraplepido}
\subsubsection*{-Bosquejo representativo:}
\centering
\includegraphics[width=10cm]{representación.png}
\justify
\subsection*{Cálculos:}
\justify
\(A=(-1,6,7),\,B=(3,3,4)\). Debido a ser un paraplepido, podemos interpretar que 
\[P=\{A,B,V_1,V_2,...,V_6|A,B,V_{1,...,6}=(x,y,z)\}\]
Ya que conocemos al menos dos puntos de la figura, podemos denominar a \(A=(x_A,y_A,z_A)\) y \(B=(x_B,y_B,z_B)\). Debido a que los vertices están en relación
a estos dos puntos conocidos, podemos inferir que con \(A\) y \(B\) hay una cierta manera de acomodar sus valores para que nos den los demás vertices, esto se debe a que la figura descrita
es una con aristas regulares, y por ende existe una cierta simetría entre sus vertices. 
\\ \newline
Tomando \(V_1\) como referencia, de manera geométrica se puede notar que:
\begin{itemize}
    \item Está a la misma altura de \(A\).
    \item Está arriba de \(B\).
\end{itemize}
Lo que expresado de manera mas formal es:
\begin{itemize}
    \item \(z_{V_1}=z_A\).
    \item \(x_{V_1},y_{V_1}=x_B,y_B\).
\end{itemize}
Lo que confirma nuestra suposición de relación. Notese tambien que para los vertices restantes, sus coordenadas pueden ser una combinación de las coordenadas de \(A\) y \(B\) por lo que podemos
decir que existe un patrón de estructura binomial en todas las coordenadas de la figura, dados dos puntos del mismo. Por simplicidad, se procede a considerar que:
\(A\rightarrow 0\) y \(B\rightarrow 1\), así que:
\begin{center}
    \begin{tabular}{ c c c }
    \(A\) & \((x_A,y_A,z_A)\) & 000\\
          &                   & 001\\
          &                   & 010\\
          &                   & 011\\
          &                   & 100\\
          &                   & 101\\
          &                   & 110\\
    \(B\) & \((x_B,y_B,z_B)\) & 111
    \end{tabular}
\end{center}
Como se puede apreciar, la dicha relación se puede expresar en binario desde el valor decimal de cero hasta siete donde el valor inicial (\(000\)) corresponde a \(A\), el valor final (\(111\)) a \(B\)
y los demas puntos correponden a lo que está en medio de \(A\) y \(B\) por lo que:
\begin{center}
    \begin{tabular}{ c c c }
        \(V_1\)& \((x_A,y_A,z_B)\) & 001\\
        \(V_2\)& \((x_A,y_B,z_A)\) & 010\\
        \(V_3\)& \((x_A,y_B,z_B)\) & 011\\
        \(V_4\)& \((x_B,y_A,z_A)\) & 100\\
        \(V_5\)& \((x_B,y_A,z_B)\) & 101\\
        \(V_6\)&  \((x_B,y_B,z_A)\) & 110\\
    \end{tabular}
\end{center}
Realizamos las sustituciones correspondientes:
\begin{center}
    \begin{tabular}{ c c c }
        \(V_1\)& \(=(x_A,y_A,z_B)=\) & \((-1,6,4)\)\\
        \(V_2\)& \(=(x_A,y_B,z_A)=\) & \((-1,3,7)\)\\
        \(V_3\)& \(=(x_A,y_B,z_B)=\) & \((-1,3,4)\)\\
        \(V_4\)& \(=(x_B,y_A,z_A)=\) & \((3,6,7)\)\\
        \(V_5\)& \(=(x_B,y_A,z_B)=\) & \((3,6,4)\)\\
        \(V_6\)&  \(=(x_B,y_B,z_A)=\) & \((3,3,7)\)\\
    \end{tabular}
\end{center}
Dando por concluido la tarea.
\end{document}