\documentclass[letterpaper, 12pt]{article}
\usepackage[letterpaper, top=2.5cm, bottom=2.5cm, left=3cm, right=3cm]{geometry} %margenes
\usepackage[utf8]{inputenc} %manejo de caracteres especiales
\usepackage[spanish]{babel} %manejo de encabezados de inglés a español
\usepackage{fancyhdr} %formato de los encabezados de página
\usepackage{ragged2e} %alineado real justficado
\usepackage{graphicx} %manejo de imagenes
\usepackage{amsmath} %manejo de notación matemática
\usepackage{mathtools} %manejo de notación matemática
\usepackage{blindtext} %texto de relleno
\usepackage{cancel} %permite la simbolización de cancelación de terminos
\usepackage{enumitem}[shortlabels] %listas con letras
\usepackage{amssymb} %manejo de simbolog►1a matematica

\pagestyle{fancy}
\fancyhf{}
\rfoot{\thepage}

\begin{document}
    \setcounter{page}{1}
    \thispagestyle{fancy}
    \lhead{\textbf{Tarea 8, U1}}
    \rhead{\textbf{29/09/2020}}
    \section{Vectores en el espacio}
    \subsection*{Encuentre la recta dado los planos \(5x-4y-9z=8\) y \(x+4y+3z=4\); Encuentre el punto de intersección de 
    \(<\!x,y,z\!>\,=\,<\!1+2t,2-t,-3t\!>,\,2x-3y+2z=-7\)}
    \subsubsection*{Calculos:}
    \justify
    \emph{Para la recta dado los planos:}\\ \newline
    \(\begin{matrix}
        5x-4y-9z=8\\
        x+4y+3z=4
    \end{matrix},\,z=t\,\)
    \(\begin{matrix}
        5x-4y=8+9t\\
        x+4y=4-3t\\
        \Sigma: 6x=12+6t \\
        \therefore x=2+t
    \end{matrix}\rightarrow\begin{matrix}
        x+4y=4-3t\\
        4y=4-3t-x\\
        y=1-\frac{3}{4}t-\frac{1}{4}x\\
        y=1-\frac{3}{4}t-\frac{1}{4}(2+t)
    \end{matrix}\)\\ \newline
    \(\therefore y=\frac{1}{2}-t \therefore\,<\!x,y,z\!>\,=\,<\!2+t,\frac{1}{2}-t,t\!>\)
    \\ \newline
    \emph{Para el punto de intersección:}\\ \newline
    \(<\!x,y,z\!>\,=\,<\!1+2t,2-t,-3t\!>\,=\,<\!x_0,y_0,z_0\!>\,\therefore \text{Eq}_{\text{plano}}(<\!x_0,y_0,z_0\!>)\)
    \\ \newline
    \(\rightarrow 2x_0-3y_0+2z_0=-7 \rightarrow 2(1+2t)-3(2-t)+2(-3t)=-7\rightarrow 2+4t-6+3t-6t=-7\)
    \\ \newline
    \(t-4=-7 \rightarrow t=-3 \therefore P_{\text{intersección}}=\,<\!1+2(-3),2-(-3),-3(-3)\!>\,=\,<\!-5,5,9\!>\)
\end{document}