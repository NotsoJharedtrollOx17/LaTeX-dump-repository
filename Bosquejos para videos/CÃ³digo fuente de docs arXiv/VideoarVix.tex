\documentclass[letterpaper, 12pt]{article}
\usepackage[letterpaper, top=1.7cm, bottom=1.7cm, left=1.7cm, right=1.7cm]{geometry} %margenes
\usepackage[utf8]{inputenc} %manejo de caracteres especiales
\usepackage[T1]{fontenc} %manejo de otras fuentes de letra
\usepackage[spanish]{babel} %manejo de encabezados de inglés a español
\usepackage{ragged2e} %alineado real justficado
\usepackage{graphicx} %manejo de imagenes
\usepackage{amsmath} %manejo de notación matemática
\usepackage{mathtools} %manejo de notación matemática
\usepackage{blindtext} %texto de relleno
\usepackage{cancel} %permite la simbolización de cancelación de terminos
\usepackage{amssymb} %manejo de simbolog►1a matematica
\usepackage{float} %mejor centrado
\usepackage{hyperref}

%parametros del título
\title{Bosquejo para \textbf{Tutorial:} 
    {\fontfamily{qag}\selectfont
        \emph{How to download the source .tex file of an arVix paper on Windows 10}
    }
}
\author{Abraham Jhared Flores Azcona a.k.a \emph{NotsoJharedtroll23}}
\date{\today}

\pagestyle{empty}

\begin{document}

\maketitle
\thispagestyle{empty}
\section*{Workflow}
Respeta lo más que puedas lo planteado en esta sección para que no batalles.
    \begin{enumerate}
        \item \textbf{Plantear} las escenas a usar para el video.
        \item \textbf{Denotar} que expresiones se van a usar para redactarlas de antemano en \LaTeX.
        \item \textbf{Redactar} lo que vas a decir para la explicación
        \item \textbf{Probar} cada escena con sus repectivas narraciones para detectar puntos de mejora o posibles errores visuales.
        \item \textbf{Grabar e incluir} la narración del video final en el código. Si se puede, agregar música de fondo.
        \item \textbf{Preguntar} en foros y a tus asesorados sus opiniones respecto al contenido para refinar el producto final.
        \item \textbf{Publicar.}
    \end{enumerate}\par
\section*{Escenas}
\subsection*{{\fontfamily{qag}\selectfont 1. Briefing}}
We give a brief review of prerequisites for the correct download of the .tex source file.\par
Order of review:
\begin{enumerate}
    \item List of prerequisites: winRAR and the arXiv webpage.
    \item Example of the final result.
\end{enumerate}
\subsection*{{\fontfamily{qag}\selectfont 2. Tutorial}}
%Primero muestra que página vamos a consultar, luego la opción si estamos buscando dicha investigaci
Show which page we're going to use, then the option when we are searching the paper based on its DOI or URL.\par
The steps should be the following:
\begin{enumerate}
    \item Go to arvix.com
    \item Write on the page's \emph{``search bar''} the DOI or URL of the paper you need.
    \item When the webpage of the respective paper if open, go to the rightmost part of it and look for the \emph{``Download''} section and click on \emph{``Other formats''}.
    \item When \emph{``Other formats''} opens, look for the \emph{``Source''} section and click on the \emph{``Download code''} link.
    \item A window will pop-up; choose the PC folder of your liking and rename the file if you want and then click on the \emph{``Save''} button of the aforementioned window.
    \item When the dowload has finished, go to the \emph{``File Explorer''} and search the folder in which the download was supposed to be saved.
    \item When the dowloaded file was found, right-click on it an look for the \emph{``Open file with...''} option of the pop-up menu and point it with the cursor.
    \item Another menu will display on the right of said option; on it, click the \emph{``Choose another app''} option.
    \item On the other window that appeared, look for the compressed file manager that you have available; in this case, it will be winRAR; if it doesn't appear, click on \emph{``More apps''}; it it still does not appear, click on \emph{``Look for another app in this PC''} and when the other window appears, write the name of the file manager that you have and then click on \emph{``Open''} and on said folder, click on the .exe file of the manager.
    \item The file will covert itself to a compressed version of itself with the icon of the compressed file manager you have; click on it.
    \item When the page of the compressed file manager appears, click on \emph{``Extract in''} and choosee the folder in which the extracted files will reside.
    \item Finally, look for the aforementioned folder and click on it, look for the .tex file and click on it, depending on which programs you have installed, it will open acoordingly to the preffered program.
\end{enumerate}\par
\subsection*{{\fontfamily{qag}\selectfont 3. Acknowledgments}}
Thank the audience for sticking to the end. Acknowledge that this method can fail by many other circumstances outside of the scope of the  brief tutorial.
\end{document}