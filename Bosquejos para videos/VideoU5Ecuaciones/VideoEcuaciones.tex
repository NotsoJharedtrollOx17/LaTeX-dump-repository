\documentclass[letterpaper, 12pt]{article}
\usepackage[letterpaper, top=1.7cm, bottom=1.7cm, left=1.7cm, right=1.7cm]{geometry} %margenes
\usepackage[utf8]{inputenc} %manejo de caracteres especiales
\usepackage[T1]{fontenc} %manejo de otras fuentes de letra
\usepackage[spanish]{babel} %manejo de encabezados de inglés a español
\usepackage{ragged2e} %alineado real justficado
\usepackage{graphicx} %manejo de imagenes
\usepackage{amsmath} %manejo de notación matemática
\usepackage{mathtools} %manejo de notación matemática
\usepackage{blindtext} %texto de relleno
\usepackage{cancel} %permite la simbolización de cancelación de terminos
\usepackage{amssymb} %manejo de simbolog►1a matematica
\usepackage{float} %mejor centrado
\usepackage{hyperref}

%parametros del título
\title{Bosquejo para \textbf{Video de aplicación:} 
    {\fontfamily{qag}\selectfont
        \emph{Atractores caóticos para generar arte}
    }
}
\author{Equipo \#10}
\date{\today}

\pagestyle{empty}

\begin{document}

\maketitle
\thispagestyle{empty}
\section*{Workflow}
\justify
Respeta lo más que puedas lo planteado en esta sección para que no batalles. Copia y pega lo escrito por aquí en forma de comentario en el {\fontfamily{cmtt}\selectfont<código>.py} de la animación.
    \begin{enumerate}
        \item \textbf{Plantear} las escenas a usar para el video.
        \item \textbf{Denotar} que expresiones se van a usar para redactarlas de antemano en \LaTeX.
        \item \textbf{Redactar} el código para el video; este se va a llamar {\fontfamily{cmtt}\selectfont videoEcuaciones.py}.
        \item \textbf{Probar} cada escena con sus repectivos bloques de código para detectar puntos de mejora o posibles errores visuales.
        \item \textbf{Grabar e incluir} la narración del video final en el código. Si se puede, agregar música de fondo.
        \item \textbf{Preguntar} en foros y a tus asesorados sus opiniones respecto al contenido para refinar el producto final.
        \item \textbf{Publicar.}
    \end{enumerate}

\section*{Escenas}
\subsection*{{\fontfamily{qag}\selectfont Portada}} \justify
Se muestran los datos típicos para un trabajo universitario.
\subsection*{{\fontfamily{qag}\selectfont 1. Planteamiento}} \justify
Se hace un repaso muy coloquial de los temas vistos hasta este punto de la materia, incluyendo expresiones matemáticas y reprentaciones gráficas de las mismas.
De ahí plantear los formalismos para la Ecuación por Diferencias, que es el punto clave para entender de una mejor manera (al menos por la parte rigorosamente matemática) los atractores caóticos
de la segunda escena.
\\\newline
Orden de repaso:
\begin{enumerate}
    \item Notaciones de la derivada: \(\frac{dy}{dx}=f^\prime(x)=y^\prime\).
    \item Ejemplo de una Ecuación Diferencial muy básica: \(\frac{dy}{dx}=y\).
    \item Representación gráfica de las ecuaciones características.
    \item Dominio de las ecuaciones Diferencial: \(t\in\mathbb{R}\).
    \item Ecuaciones por Diferencias, notación y comparación con las Ecuaciones Diferenciales.
    \item Representación gráfica de las Ecuaciones por Diferencias.
\end{enumerate}
\subsection*{{\fontfamily{qag}\selectfont 2. Atractores/Teoría del Caos}} \justify
Se explica a un nivel coloquial mostrando las expresiones, lo que es un atractor caótico acorde a la escena anterior. Mostrar ejemplos de atractores, y definir cual usar para el resto del video.
\subsection*{{\fontfamily{qag}\selectfont 3. Arte Generativo}} \justify
Se define lo que es y se muestran ejemplos de ello. Mas que nada, es explicación, NADA DE EXPRESIONES MATEMATICAS Y/O CODIGO EN ESTA PARTE.
\subsection*{{\fontfamily{qag}\selectfont 4. Librería de javascript a usar}} \justify
Se explica brevemente la herramienta de código a usar para el arte generativo.
\subsection*{{\fontfamily{qag}\selectfont 5. Idea de aplicación}} \justify
Con lo anterior, se plantea el poder usar un atractor caótico como generador de valores para alimentar
los distintos parámetros de nuestra idea del código del arte generativo. Incluir una explicación breve del código a usar.
\subsection*{{\fontfamily{qag}\selectfont 6. Ejemplos}} \justify
Se muestran los resultados y ejemplos de las distintas obras como ``producto final''.
\end{document}