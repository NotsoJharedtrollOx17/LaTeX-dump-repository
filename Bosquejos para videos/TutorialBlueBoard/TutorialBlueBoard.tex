\documentclass[letterpaper, 12pt]{article}
\usepackage[letterpaper, top=1.7cm, bottom=1.7cm, left=1.7cm, right=1.7cm]{geometry} %margenes
\usepackage[utf8]{inputenc} %manejo de caracteres especiales
\usepackage[T1]{fontenc} %manejo de otras fuentes de letra
\usepackage[spanish]{babel} %manejo de encabezados de inglés a español
\usepackage{ragged2e} %alineado real justficado
\usepackage{graphicx} %manejo de imagenes
\usepackage{amsmath} %manejo de notación matemática
\usepackage{mathtools} %manejo de notación matemática
\usepackage{blindtext} %texto de relleno
\usepackage{cancel} %permite la simbolización de cancelación de terminos
\usepackage{amssymb} %manejo de simbolog►1a matematica
\usepackage{float} %mejor centrado
\usepackage{hyperref}

%parametros del título
\title{Bosquejo para \textbf{Tutorial:} 
    {\fontfamily{qag}\selectfont
        \emph{How to use an iRigBlueBoard as a MIDI macro controller}
    }
}
\author{C. Abraham Jhared Flores Azcona a.k.a: \emph{NotsoJharedtroll23}}
\date{\today}

\pagestyle{empty}

\begin{document}

\maketitle
\thispagestyle{empty}
\section*{Workflow}
\justify
Respeta lo más que puedas lo planteado en esta sección para que no batalles. Copia y pega lo escrito por aquí en forma de comentario en el {\fontfamily{cmtt}\selectfont<código>.py} de la animación.
    \begin{enumerate}
        \item \textbf{Plantear} las escenas a usar para el video.
        \item \textbf{Denotar} que expresiones se van a usar para redactarlas de antemano en \LaTeX.
        \item \textbf{Redactar} el código para el video; este se va a llamar {\fontfamily{cmtt}\selectfont iRigBlueBoard.py}.
        \item \textbf{Probar} cada escena con sus repectivos bloques de código para detectar puntos de mejora o posibles errores visuales.
        \item \textbf{Grabar e incluir} la narración del video final en el código. Si se puede, agregar música de fondo.
        \item \textbf{Preguntar} en foros y a tus asesorados sus opiniones respecto al contenido para refinar el producto final.
        \item \textbf{Publicar.}
    \end{enumerate}

\section*{Escenas}
\subsection*{{\fontfamily{qag}\selectfont 1. Motivation}} \justify
Here you'll explain why did you create the tutorial. Explain where did you got the idea and elaborate further in to it. BE PRECISE.
\subsection*{{\fontfamily{qag}\selectfont 2. Pre-requisites}} \justify
Remark the things you need beforehand for achieving the objective:
\begin{itemize}
    \item iRigBlueBoard (70 DLLS New, 30 DLLS Used).
    \item A computer with W7 for the firmware updater (Less than 100 DLLS).
    \item A computer with W10 (250 DLLS).
    \item 4x AAA bateries (10 DLLS).
    \item A microUSB cable (Less than 5 DLLS).
    \item A Bluetooth dongle capable of Low Energy for MIDI connetions via Bluetooth.
    \item loop-MIDI program.
    \item MIDIberry program found on the Windows Store.
    \item iRigBlueBoard 1.0 Firmware updater.
    \item MIDIMacros program.
\end{itemize}
\subsection*{{\fontfamily{qag}\selectfont 3. Precautions}} \justify
Advice about the problems that you had when doing the whole process including:
\begin{itemize}
    \item Bluetooth LE pairing protocol and possible dongle.
    \item Blueboard firmware updater not working on W10 computers.
    \item The possible bricking of the BlueBoard by not following the instructions during the firmware update.
    \item The process does not guarantee the proper link between the computer and the pedalboard.
\end{itemize}
SHOW THE REASONS WHY.
\subsection*{{\fontfamily{qag}\selectfont 4. Steps}} \justify
Here you'll explain to the viewer the process:
\begin{enumerate}
    \item iRigBlueBoard firmware update.
    \item loop-MIDI setup.
    \item MIDIberry setup.
    \item Bluetooth pairing on the W10 computer and BlueBoard.
    \item MIDIMacros setup.
\end{enumerate}
\subsection*{{\fontfamily{qag}\selectfont 5. References}} \justify
Explain where you got the info and the process.
\begin{itemize}
    \item \url{https://www.ikmultimedia.com/firmware/blueboard/index_usb.php}
    \item \url{https://zubersoft.com/mobilesheets/forum/showthread.php?tid=6389} 
    \item \url{https://cgi.ikmultimedia.com/ikforum/viewtopic.php?f=9&t=24780} 
\end{itemize}
\end{document}