\documentclass[letterpaper, 12pt]{article}
\usepackage[letterpaper, top=1.7cm, bottom=1.7cm, left=1.7cm, right=1.7cm]{geometry} %margenes
\usepackage[utf8]{inputenc} %manejo de caracteres especiales
\usepackage[T1]{fontenc} %manejo de otras fuentes de letra
\usepackage[spanish]{babel} %manejo de encabezados de inglés a español
\usepackage{ragged2e} %alineado real justficado
\usepackage{graphicx} %manejo de imagenes
\usepackage{amsmath} %manejo de notación matemática
\usepackage{mathtools} %manejo de notación matemática
\usepackage{blindtext} %texto de relleno
\usepackage{cancel} %permite la simbolización de cancelación de terminos
\usepackage{amssymb} %manejo de simbolog►1a matematica
\usepackage{float} %mejor centrado

%parametros del título
\title{\textbf{Guión:} 
    {\fontfamily{qag}\selectfont
        \emph{A dive into this coding iceberg... }
    }
}
\author{NotsoJharedtroll23}
\date{\today}

\pagestyle{empty}

\begin{document}

\maketitle
\thispagestyle{empty}
\section*{Workflow}
\justify
Respeta lo más que puedas lo planteado en esta sección para que no batalles.
    \begin{enumerate}
        \item \textbf{Plantear} las escenas a usar para el video.
        \item \textbf{Denotar} que ideas explicar en el iceberg.
        \item \textbf{Redactar} los códigos o tomár videos de Youtube citandolos para evitar reclamos de copyright.
        \item \textbf{Probar} cada escena con sus repectivos bloques de código para detectar puntos de mejora o posibles errores visuales.
        \item \textbf{Grabar e incluir} la narración del video final en el video. Si se puede, agregar música de fondo.
        \item \textbf{Preguntar} en foros y a tus compas sus opiniones respecto al contenido para refinar el producto final.
        \item \textbf{Publicar.}
    \end{enumerate}

\section*{Escenas}
\subsection*{{\fontfamily{qag}\selectfont Introducción}} \justify
Se explica que rollo con el pollo.\\\textbf{Guión (ENG):}\\\newline
This is going to be another iceberg video here on Youtube... notheless I'm humbled that you clicked on it so thank you dear viewer. Apologies before hand if my english pronunciation butchers the language, it's not my mother toungue, so there goes that... 
As a shameless plug, if you're interested to check out my clown ass on my little stage, go ahead and watch 
the cringecomp playlist of my channel, and go check out my sister's channel too, she makes better videos than I but anyway... You know why you clicked the video, you know the drill, if not, let's briefly summarize what the iceberg meme template is all about:
The top of the iceberg represents topics that are considered ``common knowledge'' and from there, it descends to the different layers analogous to lesser know facts of said topic.\\
It is important to point out from here that this iceberg instead of getting way creepier, gets way interesting... and somewhat nerdy so beware, nerdy stuff up ahead. Without further a do, let's dive into this coding language iceberg meme.
\subsection*{{\fontfamily{qag}\selectfont 1. The Sky}} \justify
Level 1: The Sky.
\\\newline
\textbf{HTML:} The \emph{HyperText Markup Language} a.k.a HTML is a languange used for developing wep-pages; the \emph{Hypertext} refers to links and the \emph{markup} part refers to other contents of the page such as text, images, audio, etc. It's main purpouse is to layer the
structure of the content of a page rather than its own behavior or it's layout (which can be altered with Javascript and CSS respectively, and we're going to explain those in a further bit). The syntax of HTML is really easy to understand and really straight foward too (at least for anyone that understands English). 
It's latest version (until the date of recording) is HTML5 that let us plug videos on to webpages.
\\\newline
\textbf{JavaScript:} According to the english Wikipedia page, JS (a.k.a. JavaScript) is a programming language that adds the behavior/back-end of a web-page, as we've mentioned before. Major web browsers such as Chrome, Opera, etc. have a dedicated JavaScript engine to execute code written in a .js file on the user's device.To add proper back-end to an HTML file, you'll need to 
add a <script></script> block with a reference (the file name and directory) to the .js file you're working on. Even tho JavaScript has ``Java'' in it's name, those we're developed by different individuals: James Gosling developed Java while Brendan Eich developed JavaScript, both in 1995. Interesting features of the language are that you can
write down or ommit semicolons on the code you're writing, and that value equality and type equality needs to be written as follows: ===. One really cool library is the one of p5.js (a.k.a. Processing) that aids on the redaction of code that results in generative artwork, just as the one showing up right now. 
\\\newline
\textbf{CSS:} By it's abreviation the \emph{Cascading Style Sheets} (a.k.a CSS) is used for the presentation of a document written in a markup language such as HTML, that we've explained at the beggining of the video. Is designed to enable the separation of presentation and content, including layout, colors and fonts and hence improve content accessibility. The file extension of a CSS file
is .css. Apart of HTML, CSS is supported by XHTML, plain XML, SVG and XUL. This is somewhat brief, but that's CSS in a nutshell, so let's move on.
\\\newline
\textbf{PHP:} By the PHP manual, PHP is the recursive acronym for \emph{Hypertext Preprocessor} which is a widely used open source general-purpouse scripting language that is especially suited for web development and can be embedded into HTML. PHP scripts have a .php extension and the language syntax is similar to C. Common uses of the language are to peform system functions, handle forms and hence retrieve data from those,
add, delete and modify elements within your database, access and set cookies, encrypt data, etc. Many developers have a consensus of having a hatred to this language because of several problems with the desing of it, and just because to have a hatred on it (the reasons are still up to debate on many web forums).
\\\newline
\textbf{Python:} Oh Python, oooooooh Python... argueably one of the, if not the most popular programming language in the world. By the ``Executive Summary'' of the Python.org webpage: Python is an interpreted, object-oriented, high-level programming language with dynamic binding, make it very attractive for Rapid Application Development, as well as for use as a scripting or glue language to connect existing component together.
Python's simple, easy to learn syntax emphasizes readability and therefore reduces the cost of program maintenance. Python supports modules and packages, which encourages program modularity and code reuse. The Python interpreter and the extensive standard library are available in source of binary form without charge for all major platforms, and can be freely distributed. TL;DW: it's easy to learn and really versatile and (somewhat) fool-proof. 

\subsection*{{\fontfamily{qag}\selectfont 2. The Tip}} \justify

\subsection*{{\fontfamily{qag}\selectfont 3. Below the Surface}} \justify

\subsection*{{\fontfamily{qag}\selectfont 4. Dark Waters}} \justify

\subsection*{{\fontfamily{qag}\selectfont 5. The Depths}} \justify

\subsection*{{\fontfamily{qag}\selectfont 6. Going Deeper}} \justify

\subsection*{{\fontfamily{qag}\selectfont 7. The Abyss}} \justify

\subsection*{{\fontfamily{qag}\selectfont 8. Mariana's Trench}} \justify

\end{document}