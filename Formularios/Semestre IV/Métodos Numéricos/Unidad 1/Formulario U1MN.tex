\documentclass[letterpaper, 12pt]{article}
\usepackage[letterpaper, top=2.5cm, bottom=2.5cm, left=3cm, right=3cm]{geometry} %margenes
\usepackage[utf8]{inputenc} %manejo de caracteres especiales
\usepackage[spanish]{babel} %manejo de encabezados de inglés a español
\usepackage{fancyhdr} %formato de los encabezados de página
\usepackage{ragged2e} %alineado real justficado
\usepackage{graphicx} %manejo de imagenes
\usepackage{amsmath} %manejo de notación matemática
\usepackage{mathtools} %manejo de notación matemática
\usepackage{blindtext} %texto de relleno
\usepackage{cancel} %permite la simbolización de cancelación de terminos
\usepackage{enumitem}[shortlabels] %listas con letras
\usepackage{amssymb} %manejo de simbolog►1a matematica
\usepackage{hyperref} %manejo de enlaces

\hypersetup{
  colorlinks   = true, %Colours links instead of ugly boxes
  urlcolor     = blue, %Colour for external hyperlinks
  linkcolor    = blue, %Colour of internal links
  citecolor   = red %Colour of citations
}

\pagestyle{fancy}
\fancyhf{}
\rfoot{\thepage}

\begin{document}
    
    %PORTADA
    \begin{titlepage}
        \begin{figure}[ht]
            \centering
            \includegraphics[width=15cm]{logosITT.png}
        \end{figure}
        \centering
        {\scshape\LARGE Tecnológico Nacional de México\\Instituto Tecnológico de Tijuana\par}
        \vspace{1cm}
        {\scshape\Large Métodos Numéricos\par}
        \vspace{1cm}
        {\scshape\Large IV Semestre\par}
        \vspace{1.5cm}
        {\huge\bfseries Formulario (U1)\par}
        \vspace{2cm}
        {\Large\itshape C. Abraham Jhared Flores Azcona\\19211640\par}
        \vfill
        Profesor: Ing. Tonallí Cuauhtemoc Galícia Lopez\par
        
        \vfill

        {\large 18 de marzo de 2021}
    \end{titlepage}

    \newpage
    \begin{justify}
        \setcounter{page}{1}
        \thispagestyle{fancy}
        \lhead{\textbf{Unidad 1}}
        \section{Raíces de ecuaciones}
        \justify
        En todos se muestra la fórmula matemática o algorítmo y su vaga representación en Excel.
        \subsection*{Teoría}
        \justify
        LA MISMA TEORIA DE CALCULO DIFERENCIAL E INTEGRAL, SOLO APROXIMAMOS NUMERICAMENTE!!!
        \subsection*{Bisección}
        \justify
        \textbf{• Fórmula para \(x_r\):}
        {\large \[x_r=\frac{x_i+x_u}{2}\]}
        \justify
        \textbf{• Algorítmo:}
{\large\begin{verbatim}
xr=(xi+xu)/2;

if f(xi)*f(xr)>0 then
    xi=xr;
else
    xu=xr;
\end{verbatim}}
        \justify
        \textbf{• En Excel:}
{\large\begin{verbatim}
Para la nueva celda xi: 
=SI(<celda_f(xi)>*<celda_f(xr)>0,<celda_xr_anterior>,
<celda_xi_anterior>)

Para la nueva celda xu:
=SI(<celda_f(xi)>*<celda_f(xr)>0,<celda_xu_anterior>,
<celda_xr_anterior>)

Para obtener xr:
=(<celda_xi>+<celda_xu>)/2
\end{verbatim}}
        \subsection*{Falsa posición}
        \justify
        \textbf{• Fórmula para \(x_r\):}
        {\large \[x_r=x_i-f(x_i)\left(\frac{x_u-x_i}{f(x_u)-f(x_i)}\right)\]}
        \justify
        \textbf{• Algorítmo:}
{\large \begin{verbatim}
xr = xi - f(xi)((xu-xi)/(f(xu)-f(xi)));

if f(xi)*f(xr)>0 then
    xi=xr;
else
    xu=xr;
\end{verbatim}}
        \justify
        \textbf{• En Excel:}
{\large\begin{verbatim}
Para la nueva celda xi: 
=SI(<celda_f(xi)>*<celda_f(xr)>0,<celda_xr_anterior>,
<celda_xi_anterior>)
    
Para la nueva celda xu:
=SI(<celda_f(xi)>*<celda_f(xrs)>0,<celda_xu_anterior>,
<celda_xr_anterior>)

Para obtener xr:
=<celda_xi>-<celda_f(xi)>*((<celda_xu>-<celda_xi>)
/(<celda_f(xu)>-<celda_f(xi)>))
\end{verbatim}}
        \subsection*{Newton-Raphson}
        \justify
        \textbf{• Fórmula para \(x_{i+1}\):}
        {\large \[x_{i+1}=x_i-\frac{f(x_i)}{f^{\prime}(x_i)}\]}
        \justify
        \textbf{• Fórmula para Excel:}
{\large\begin{verbatim}
Para la nueva celda xi:
=<celda_xi_anterior>-(<celda_f(xi)_anterior>/
<celda_f'(xi)_anterior>)
\end{verbatim}}
        \end{justify}
\end{document}