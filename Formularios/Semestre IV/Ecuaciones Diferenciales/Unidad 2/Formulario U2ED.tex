\documentclass[letterpaper, 12pt]{article}
\usepackage[letterpaper, top=2.5cm, bottom=2.5cm, left=3cm, right=3cm]{geometry} %margenes
\usepackage[utf8]{inputenc} %manejo de caracteres especiales
\usepackage[spanish]{babel} %manejo de encabezados de inglés a español
\usepackage{fancyhdr} %formato de los encabezados de página
\usepackage{ragged2e} %alineado real justficado
\usepackage{graphicx} %manejo de imagenes
\usepackage{amsmath} %manejo de notación matemática
\usepackage{mathtools} %manejo de notación matemática
\usepackage{blindtext} %texto de relleno
\usepackage{cancel} %permite la simbolización de cancelación de terminos
\usepackage{enumitem}[shortlabels] %listas con letras
\usepackage{amssymb} %manejo de simbolog►1a matematica
\usepackage{hyperref} %manejo de enlaces

\pagestyle{fancy}
\fancyhf{}
\rfoot{}

\begin{document}
    
    %PORTADA
    \begin{titlepage}
        \begin{figure}[ht]
            \centering
            \includegraphics[width=15cm]{logosITT.png}
        \end{figure}
        \centering
        {\scshape\LARGE Tecnológico Nacional de México\\Instituto Tecnológico de Tijuana\par}
        \vspace{1cm}
        {\scshape\Large Ecuaciones Diferenciales\par}
        \vspace{1cm}
        {\scshape\Large IV Semestre\par}
        \vspace{1.5cm}
        {\huge\bfseries Formulario (U2)\par}
        \vspace{2cm}
        {\Large\itshape C. Abraham Jhared Flores Azcona\\19211640\par}
        \vfill
        Profesora: Ing. Mariana Huizar Tejada\par
        
        \vfill

        {\large 7 de marzo de 2021}
    \end{titlepage}

    \newpage
    \begin{justify}
        \thispagestyle{fancy}
        \lhead{\textbf{Unidad 2}}
        \section{Ecuaciones Diferenciales Lineales de orden superior}
        \justify
        En esta unidad es repasar otras habilidades matemáticas mas allá que derivar o diferenciar.
        \subsection{Problema del valor inicial}
        \justify
        Es cuando se nos da \(n\) condiciones de la forma:
        {\large \[y(x_0)=y_0,\, y^{\prime}(x_0)=y_1,\, y^{\prime\prime}(x_0)=y_2,\, \dots, \, y^{(n-1)}(x_0)=y_{n-1}\]}
        \subsection{Problema del valor en la frontera}
        \justify
        Es cuando se nos da \(n\) condiciones de la forma:
        {\large \[y(x_0)=y_0,\, y(x_1)=y_1,\, y(x_2)=y_2,\, \dots,\, y(x_{n-1})=y_{n-1}\]}
        \justify
        Tanto para el valor inicial como en la frontera, se resuelven para encontrar los valores de las constantes de la solución
        general para obtener la solución particular a la ecuación diferencial.
        \subsection{Dependencia e independencia lineal}
        \justify
        Dado \(\theta=\{f_1,f_2,\dots, f_n\}\), si se quiere saber la independencia lineal de \(\theta\) se 
        calcula el Wronskiano correspondiente, tal que: 
        {\large\begin{equation*}
            \begin{aligned}
                W(\theta)=\begin{array}{@{\mkern3mu}|cccc|@{\mkern3mu}}
                    f_1 & f_2 & \dots & f_n\\
                    f^{\prime}_1 & f^{\prime}_2 & \dots & f^{\prime}_n\\
                    f^{\prime\prime}_1 & f^{\prime\prime}_2 & \dots & f^{\prime\prime}_n\\
                    \vdots & \vdots & \vdots & \vdots\\
                    f^{(n-1)}_1 & f^{(n-1)}_2 & \dots & f^{(n-1)}_n
                \end{array}\neq 0
            \end{aligned}
        \end{equation*}}
        De lo contrario, \(\theta\) es linealmente dependiente.
        \subsection{Reducción de orden}
        \justify
        Para obtener la solución general \(y=c_1y_1+c_2y_2\) de la ecuación diferencial \(y^{\prime\prime}+p(x)y^{\prime}+q(x)y=0\) dado \(y_1\) entonces ocupamos calcular \(y_2\):
        {\large\[y_2=u(x)\, y_1,\, u(x)=\int \frac{\text{exp}\left(-\int p(x)\, dx\right)}{y^2_1}\, dx\]}
        \subsection{Con coeficientes constantes}
        \justify
        Son aquellas ecuaciones las cuales tienen la forma:
        {\large\[a_ny^{(n)}+a_{n-1}y^{(n-1)}+\dots+a_2y^{\prime\prime}+a_1y^{\prime}+a_0y=0\]}
        \justify
        Por fines de la explicación, se continua con una ecuación diferencial de segundo orden y que \(y_1=e^{mx}\):
        {\large\[a_2y^{\prime\prime}+a_1y^{\prime}+a_0y=0\rightarrow a_2m^2+a_1m+a_0=0\]}
        \justify
        Resolvemos la ecuación en términos de \(m\) para obtener sus raices y determinamos la solución general:
        \newline\\
        {\large\textbf{• Caso 1: \(m\) con reales diferentes}\[\{e^{m_1x},\, e^{m_2x}\}\rightarrow y=c_1e^{m_1x}+c_2e^{m_2x}\]}
        {\large\textbf{• Caso 2: \(m\) con reales repetidos}\[\{e^{mx},\, xe^{mx}\}\rightarrow y=c_1e^{mx}+c_2xe^{mx}\]}
        {\large\textbf{• Caso 3: \(m\) complejos donde \(m=a\pm bi\)}\[\{e^{ax}\cos(bx),\, e^{ax}\sin(bx)\}\rightarrow y=c_1e^{ax}\cos(bx)+c_2e^{ax}\sin(bx)\]}
        \subsubsection{De orden superior}
        \justify
        Si la ecuación es de un orden mayor al de 2do. grado, se puede reducir usando la división sintética hasta reducirla al segundo ordén.
        Los casos anteriores aplican, pero las constantes y cantidades de términos en la solución general puede variar (puede haber 3, 4, etc.)
        \subsection{Ecuaciones diferenciales lineales no homogéneas}
        \justify
        Son aquellas las cuales tienen la forma:
        {\large\[a_ny^{(n)}+a_{n-1}y^{(n-1)}+\dots+a_2y^{\prime\prime}+a_1y^{\prime}+a_0y=g(x)\]}
        Donde \(g(x)\neq 0\). La solución para dicha ecuación es:
        {\large\[y=y_c+y_p\]}
        Donde \(y_c\) es la solución complementaria y \(y_p\) es la solución particular de \(g(x)\).
        \\\newline
        Notese lo siguiente:
        {\large\[\arraycolsep=10pt\def\arraystretch{1.8}\begin{array}{ccc}
            g(x) & m & y_p\\
            1 & 0 & A\\
            2x+4 & 0,0 & Ax+B\\
            3e^x & 1 & Ae^x\\
            5xe^{4x} & 4,4 & Ae^{4x} + Bxe^{4x}\\
            4x^2e^{3x} & 3,3,3 & Ae^{3x} + Bxe^{3x} + Cx^2e^{3x}\\
            -8\cos(2x) & \pm 2i & A\cos(2x) + B\sin(2x)
        \end{array}\]}
        \justify
        Lo que se desea mostrar es que \(y_p\) varia según las raices (\(m\)) de \(g(x)\), y los coeficientes de dicha solución se calculan de la siguiente manera:
        \begin{enumerate}
            \item Calcular \(y_c\) por los métodos anteriores.
            \item Revisar \(g(x)\) y determinar las raices correspondientes y la ``plantilla'' de \(y_p\).
            \item Derivar \(y_p\) hasta el mayor órden de la Ecuación Diferencial dada y sustituar las derivadas en dicha ecuación.
            \item Reducir la expresión e igualar términos semejantes para obtener los coeficientes de \(y_p\).
            \item Sustituir los coeficientes obtenidos en \(y_p\).
            \item \(y=y_c+y_p\).
        \end{enumerate}
        \subsubsection{Caso especial: raices de \(g(x)\) ya encontradas para \(y_c\)}
        \justify
        Si \(g(x)\) contiene una de las raices para \(y_c\), se escala su solución por factor de \(x\).
        \\\newline
        Ejemplo:
        \\
        Se sabe que para una ecuación diferencial lineal no homogénea \(g(x)=-4e^{4x}\) y que \(y_c=c_1e^{4x}+c_2e^{6x}\). Determina su posible \(y_p\):
        {\large\[g(x)=-4e^{4x}\rightarrow y_p=A\left(x\cdot e^{4x}\right)\]}
        Y se resuelve como se indica anteriormente.
        \subsection{Variación de parámetros}
        \justify
        Para resolver una Ecuación Diferencial Lineal No Homogenea de 2do. orden por este método, se tiene en cuenta lo siguiente:
        {\large\[y=y_c+\underbrace{v_1y_1+v_2y_2}_{y_p}\]}
        Donde \(y_c=c_1y_1+c_2y_2\) (que se calcula como se ha visto antes) y
        {\large\[v_1=-\int {g(x)y_2 \over W(y_1,y_2)}\, dx,\: v_2=\int {g(x)y_1 \over W(y_1,y_2)}\, dx\]}
    \end{justify}
\end{document}