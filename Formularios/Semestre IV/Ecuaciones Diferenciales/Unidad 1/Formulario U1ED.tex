\documentclass[letterpaper, 12pt]{article}
\usepackage[letterpaper, top=2.5cm, bottom=2.5cm, left=3cm, right=3cm]{geometry} %margenes
\usepackage[utf8]{inputenc} %manejo de caracteres especiales
\usepackage[spanish]{babel} %manejo de encabezados de inglés a español
\usepackage{fancyhdr} %formato de los encabezados de página
\usepackage{ragged2e} %alineado real justficado
\usepackage{graphicx} %manejo de imagenes
\usepackage{amsmath} %manejo de notación matemática
\usepackage{mathtools} %manejo de notación matemática
\usepackage{blindtext} %texto de relleno
\usepackage{cancel} %permite la simbolización de cancelación de terminos
\usepackage{enumitem}[shortlabels] %listas con letras
\usepackage{amssymb} %manejo de simbolog►1a matematica
\usepackage{hyperref} %manejo de enlaces

\hypersetup{
  colorlinks   = true, %Colours links instead of ugly boxes
  urlcolor     = blue, %Colour for external hyperlinks
  linkcolor    = blue, %Colour of internal links
  citecolor   = red %Colour of citations
}

\pagestyle{fancy}
\fancyhf{}
\rfoot{}

\begin{document}
    
    %PORTADA
    \begin{titlepage}
        \begin{figure}[ht]
            \centering
            \includegraphics[width=15cm]{logosITT.png}
        \end{figure}
        \centering
        {\scshape\LARGE Tecnológico Nacional de México\\Instituto Tecnológico de Tijuana\par}
        \vspace{1cm}
        {\scshape\Large Ecuaciones Diferenciales\par}
        \vspace{1cm}
        {\scshape\Large IV Semestre\par}
        \vspace{1.5cm}
        {\huge\bfseries Formulario (U1)\par}
        \vspace{2cm}
        {\Large\itshape C. Abraham Jhared Flores Azcona\\19211640\par}
        \vfill
        Profesora: Ing. Mariana Huizar Tejada\par
        
        \vfill

        {\large 24 de marzo de 2021}
    \end{titlepage}

    \newpage
    \begin{justify}
        \thispagestyle{fancy}
        \lhead{\textbf{Unidad 1}}
        \rhead{\textbf{Ecuaciones Diferenciales}}
        \section*{Ecuaciones Diferenciales ordinarias de primer orden}
        \justify
        Para aprenderse los tipos, se recomienda el acrónimo en inglés llamado \textbf{S H I E L D} (o \emph{escúdo} en español):
        \begin{itemize}
            \item \textbf{S:} Separables (\emph{separable}).
            \item \textbf{H:} Homogéneas (\emph{homogeneous}).
            \item \textbf{I:} Factor Integrante (\emph{integrating factor}).
            \item \textbf{E:} Exactas (\emph{exact}).
            \item \textbf{L:} Lineales (\emph{linear}).
            \item \textbf{D:} Integración directa (\emph{direct integration}).
        \end{itemize}
        \subsection*{Separables}
        {\large\begin{equation*}
            \begin{aligned}
                \frac{dy}{dy}&=f(x)g(y)\\[5pt]
                \int \frac{dy}{g(y)}&=\int f(x)\, dx\\[5pt]
                G(y)&=F(x)+c\\[5pt]
                c&=F(x)-G(y)
            \end{aligned}
        \end{equation*}}
        \subsection*{Homogéneas}
        \justify
        Son de la forma: \(M(x,y)\, dx+N(x,y)\, dy=0\) y que \(M(x,y)\) y \(N(x,y)\) sean homogéneas.
        \\\newline
        \textbf{• Para \(u=\frac{y}{x}\):}
        {\large\begin{equation*}
            \begin{aligned}
                \frac{dy}{dx}&=-\frac{M(x,y)}{N(x,y)}\\[5pt]
                \frac{dy}{dx}&=-\frac{x^nM(u)}{x^nN(u)}\\[5pt]
                \frac{dy}{dx}&=G(u)\\[5pt]
                \int \frac{dx}{x}&=\int \frac{du}{G(u)-u}
            \end{aligned}
        \end{equation*}}
        \justify
        \textbf{• Para \(v=\frac{x}{y}\):}
        {\large\begin{equation*}
            \begin{aligned}
                \frac{dx}{dy}&=-\frac{N(x,y)}{M(x,y)}\\[5pt]
                \frac{dx}{dy}&=-\frac{y^nN(x,y)}{y^nM(x,y)}\\[5pt]
                \frac{dx}{dy}&=H(v)\\[5pt]
                \int \frac{dy}{y}&=\int \frac{dv}{H(v)-v}
            \end{aligned}
        \end{equation*}}
        \subsection*{Exactas}
        \justify
        Dada \(M(x,y)\, dx+N(x,y)\, dy=0\), esta expresión es exacta si 
        \[\frac{\partial M}{\partial y}=\frac{\partial N}{\partial x}\]
        \textbf{• Para \(M(x,y)\):}
        {\large\begin{equation*}
            \begin{aligned}
                \frac{\partial f}{\partial x}&=M(x,y)\\[5pt]
                f&=\int M(x,y)\, \partial x+c(y)\\[5pt]
                \frac{\partial}{\partial y}\int M(x,y)\, \partial x+c^{\prime}(y)&=N(x,y)\\[5pt]
                \int c^{\prime}(y)&=\int N(x,y)-\frac{\partial}{\partial y}\int M(x,y)\, \partial x\\[5pt]
                f&=\int M(x,y)\, \partial x+\int c^{\prime}(y)
            \end{aligned}
        \end{equation*}}
        \justify
        \textbf{• Para \(N(x,y)\):}
        {\large\begin{equation*}
            \begin{aligned}
                \frac{\partial f}{\partial y}&=N(x,y)\\[5pt]
                f&=\int N(x,y)\, \partial y+c(x)\\[5pt]
                \frac{\partial}{\partial x}\int N(x,y)\, \partial y+c^{\prime}(x)&=M(x,y)\\[5pt]
                \int c^{\prime}(x)&=\int N(x,y) -\frac{\partial}{\partial x}\int N(x,y)\, \partial y\\[5pt]
                f&=\int N(x,y)\, \partial y+\int c^{\prime}(x) 
            \end{aligned}
        \end{equation*}}
        \subsection*{Factor Integrante}
        \justify
        Si una ED de la forma \(M(x,y)\, dx+N(x,y)\, dy=0\) no es exacta, se puede hacer exacta usando el factor integrante:
        \\\newline
        \textbf{• En términos de \(x\):}
        {\large\begin{equation*}
            \begin{aligned}
                \mu(x)&=e^{\,\int \frac{M_y-N_x}{N}\, dx}\\[5pt] 
                \text{ para que }&\mu(x)\left(M(x,y)\, dx+N(x,y)\, dy\right)=0 \textbf{ sea exacta.}\\[5pt]
            \end{aligned}
        \end{equation*}}
        Donde \(M_y=\frac{\partial}{\partial y}M(x,y)\), \(N_x=\frac{\partial}{\partial x}N(x,y)\) y \(N=N(x,y)\).
        \\\newline
        \textbf{• En términos de \(y\):}
        {\large\begin{equation*}
            \begin{aligned}
                \mu(y)&=e^{\,\int \frac{N_x-M_y}{M}\, dy}\\[5pt]
                \text{ para que }&\mu(y)\left(M(x,y)\, dx+N(x,y)\, dy\right)=0 \textbf{ sea exacta.}\\[5pt]
            \end{aligned}
        \end{equation*}}
        Donde \(N_x=\frac{\partial}{\partial x}N(x,y)\), \(M_y=\frac{\partial}{\partial x}M(x,y)\) y \(M=M(x,y)\).
        \\\newline
        Al balancear la ED con el factor integrante, dicha ecuación se resuelve como una ED Exacta.    
        \subsection*{Lineales}
        \justify
        Se acostumbra a trabajar con las lineales de 1er. orden.
        {\large\begin{equation*}
            \begin{aligned}
                a_1(x)y^{\prime}+a_0(x)y&=g(x)\\[5pt]
                \frac{a_1(x)}{a_1(x)}y^{\prime}+\frac{a_0(x)}{a_1(x)}y&=\frac{g(x)}{a_1(x)}\\[5pt]
                y^{\prime}+p(x)y = f(x)
            \end{aligned}
        \end{equation*}}
        Si \(f(x)=0\):
        {\large\begin{equation*}
            \begin{aligned}
                y^{\prime}+p(x)\, y=0\:\textbf{es separable}
            \end{aligned}
        \end{equation*}}
        De lo contrario:
        {\large\begin{equation*}
            \begin{aligned}
                y^{\prime}+p(x)\, y&=f(x)\\[5pt]
                \frac{dy}{dx}+p(x)\, y&=f(x)\\[5pt]
                \mu(x)&=\text{exp}\left(\int p(x)\, dx\right)\\[5pt]
                \mu(x)\frac{dy}{dx}+\mu(x)\, p(x)\, y&=\mu(x)f(x)\\[5pt]
                D_x\,\mu(x)\, y&=\mu(x)f(x)\\[5pt]
                \int D\, \mu(x)\, y&=\int \mu(x)f(x)\, dx\\[5pt]
                \mu(x)\, y&=\int \mu(x)f(x)\, dx
            \end{aligned}
        \end{equation*}}
        \subsection*{De Bernoulli}
        \justify
        Son una variante de las Lineales. Son de la forma:
        {\large\[\frac{dy}{dx}+p(x)\, y=f(x)\, y^n,\: \{\forall n\,|\, n\neq 0 \land n\neq 1\}\]}
        Se resuelven de la siguiente manera:
        {\large \begin{equation*}
            \begin{aligned}
                \frac{dy}{dx}+p(x)\, y&=f(x)\, y^n\\[5pt]
                \frac{dy}{dx}&=f(x)\, y^n-p(x)\, y\\[5pt]
                \frac{du}{dx}&=\frac{d}{dx}\, u\\[5pt]
                &=\frac{d}{dx}\, y^{1-n}\\[5pt]
                &=(1-n)\, y^{-n}\frac{dy}{dx}\\[5pt]
                &=(1-n)\, y^{-n}\left(f(x)\, y^n-p(x)\, y\right)\\[5pt]
                &=(1-n)f(x)-(1-n)\, p(x)\, y^{1-n}\\[5pt]
                \frac{du}{dx}&=(1-n)f(x)-(1-n)\, p(x)\, u\\[5pt]
                \frac{du}{dx}+(1-n)\, p(x)\, u&=(1-n)f(x)\\[5pt]
            \end{aligned}
        \end{equation*}}
        \justify
        Donde \(\frac{du}{dx}+(1-n)\, p(x)\, u=(1-n)f(x)\) es una ecuación diferencial lineal y se resuelve como tal.
    \end{justify}
\end{document}