\documentclass[letterpaper, 12pt]{article}
\usepackage[letterpaper, top=2.5cm, bottom=2.5cm, left=3cm, right=3cm]{geometry} %margenes
\usepackage[utf8]{inputenc} %manejo de caracteres especiales
\usepackage[spanish]{babel} %manejo de encabezados de inglés a español
\usepackage{fancyhdr} %formato de los encabezados de página
\usepackage{ragged2e} %alineado real justficado
\usepackage{graphicx} %manejo de imagenes
\usepackage{amsmath} %manejo de notación matemática
\usepackage{mathtools} %manejo de notación matemática
\usepackage{blindtext} %texto de relleno
\usepackage{cancel} %permite la simbolización de cancelación de terminos
\usepackage{amssymb} %manejo de simbolog►1a matematica
\usepackage{hyperref} %manejo de enlaces
\usepackage[scr]{rsfso} %simbolo de la "l" de la transformada de laplace
\newcommand{\Laplace}{\mathscr{L}}

\pagestyle{fancy}
\fancyhf{}
\rfoot{}

\begin{document}
    
    %PORTADA
    \begin{titlepage}
        \begin{figure}[ht]
            \centering
            \includegraphics[width=15cm]{logosITT.png}
        \end{figure}
        \centering
        {\scshape\LARGE Tecnológico Nacional de México\\Instituto Tecnológico de Tijuana\par}
        \vspace{1cm}
        {\scshape\Large Ecuaciones Diferenciales\par}
        \vspace{1cm}
        {\scshape\Large IV Semestre\par}
        \vspace{1.5cm}
        {\huge\bfseries Formulario (U3)\par}
        \vspace{2cm}
        {\Large\itshape C. Abraham Jhared Flores Azcona\\19211640\par}
        \vfill
        Profesora: Ing. Mariana Huizar Tejada\par
        
        \vfill

        {\large 7 de junio de 2021}
    \end{titlepage}

    \newpage
    \begin{justify}
        \thispagestyle{fancy}
        \lhead{\textbf{Unidad 3}}
        \section{Transformada de Laplace}
        \justify
        El formulario de las Transformadas está hasta el final del documento.
        \subsection{1er. Teorema de Traslación}
        \justify
        \textbf{Para la transformada normal:}
        {\large\[\Laplace\{e^{at}f(t)\}=F(s)|_{s-a}=F(s-a)\]}
        \justify
        \textbf{Para la transformada inversa:}
        {\large\[\Laplace^{-1}\{F(s-a)\}=\Laplace^{-1}\{F(s)|_{s-a}\}=\Laplace^{-1}\{F(s)\}\, e^{at}=f(t)\, e^{at}\]}
        \subsection{2do. Teorema de Traslación}
        \justify
        \textbf{Para la transformada normal:}
        {\large\[\Laplace\{f(t-a)\mathscr{U}(t-a)\}=e^{-as}\Laplace\{f(t)\}=e^{-as}F(s)\]}
        \justify
        \textbf{Para la transformada inversa:}
        {\large\[\Laplace^{-1}\{e^{-as}F(s)\}=\Laplace^{-1}\{F(s)\}|_{t-a}\mathscr{U}(t-2)=f(t-a)\mathscr{U}(t-2)\]}
        \section{Fracciones parciales}
        \justify
        La gran mayoria de las veces para resolver ED por Laplace y/o para obtener la transformada inversa de alguna expresión es necesario la descomposición
        por fracciones parciales. Para abreviar, en todos los casos se asume que \(P(x)\) y \(Q(x)\) son polinomios, que \(P(x)\) es de menor grado que \(Q(x)\) y
        que \(P(x),Q(x)\neq 0\).
        \\\newline
        \textbf{SE DEBE DE TENER EN CUENTA QUE \(Q(x)\) PUEDE TENER DISTINTAS COMBINACIONES DE LOS CASOS EXPUESTOS POR LO QUE SE DEBE DE PRESTAR ATENCION A LOS MISMOS.}
        \subsection{Caso 1}
        \justify
        Se interpreta que \(Q(x)\) no tiene factores cuadráticos que se repiten. Se determinan los coeficientes \(A\) y \(B\).
        {\large \[\frac{P(x)}{Q(x)}=\frac{P(x)}{(a_1x+b_1)(a_2x+b_2)}=\frac{A}{(a_1x+b_1)}+\frac{B}{(a_2x+b_2)}, a_1\neq a_2, b_1\neq b_2\]}
        \subsection{Caso 2}
        \justify
        Se entiende que \(c_1,c_2,\dots,c_n\) son constantes a determinar.
        {\large \[\frac{P(x)}{Q(x)}=\frac{P(x)}{(ax+b)^n}=\frac{c_1}{ax+b}+\frac{c_2}{(ax+b)^2}+\dots+\frac{c_n}{(ax+b)^n}\]}
        \subsection{Caso 3}        
        \justify
        Se interpreta que \(Q(x)\) no tiene factores cuadráticos irreducibles que no se repiten. Se determinan los coeficientes \(A\) y \(B\).
        {\large \[\frac{P(x)}{Q(x)}=\frac{P(x)}{ax^2+bx+c}=\frac{Ax+B}{ax^2+bx+c}\]}

    \end{justify}
\end{document}