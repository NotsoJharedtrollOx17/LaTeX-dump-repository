\documentclass[letterpaper, 12pt]{article}
\usepackage[letterpaper, top=2.5cm, bottom=2.5cm, left=3cm, right=3cm]{geometry} %margenes
\usepackage[utf8]{inputenc} %manejo de caracteres especiales
\usepackage[spanish]{babel} %manejo de encabezados de inglés a español
\usepackage{fancyhdr} %formato de los encabezados de página
\usepackage{ragged2e} %alineado real justficado
\usepackage{graphicx} %manejo de imagenes
\usepackage{amsmath} %manejo de notación matemática
\usepackage{mathtools} %manejo de notación matemática
\usepackage{blindtext} %texto de relleno
\usepackage{cancel} %permite la simbolización de cancelación de terminos
\usepackage{enumitem}[shortlabels] %listas con letras
\usepackage{amssymb} %manejo de simbolog►1a matematica
\usepackage{algorithm2e} %manejo de algoritmos

\pagestyle{fancy}
\fancyhf{}
\rfoot{\thepage}

\begin{document}
    
    %PORTADA
    \begin{titlepage}
        \begin{figure}[ht]
            \centering
            \includegraphics[width=15cm]{logosITT.png}
        \end{figure}
        \centering
        {\scshape\LARGE Tecnológico Nacional de México\\Instituto Tecnológico de Tijuana\par}
        \vspace{1cm}
        {\scshape\Large Cálculo Vectorial\par}
        \vspace{1.5cm}
        {\huge\bfseries Formulario (U1)\par}
        \vspace{2cm}
        {\Large\itshape C. Abraham Jhared Flores Azcona\\19211640\par}
        \vfill
        Profesora: \par
        Ing. Mariana Huizar Tejada
        
        \vfill

        {\large 07/oct/2020}
    \end{titlepage}

    \newpage
    \thispagestyle{empty}
    \tableofcontents

    \newpage
    \setcounter{page}{1}
    \thispagestyle{fancy}
    \lhead{\textbf{Unidad 1}}
    \rhead{\textbf{07/09/2020}}
    \section{Vectores}
    \begin{itemize}
        \item En \(\mathbb{R}^2\):\[\vec{v}=\,<\!v_x,v_y\!> \text{ ó } <\!v_1,v_2\!> \text{ ó } v_1\hat{\imath}+v_2\hat{\jmath}\]
        \item En \(\mathbb{R}^3\): \[\vec{v}=\,<\!v_x,v_y,v_z\!> \text{ ó } <\!v_1,v_2,v_3\!> \text{ ó } v_1\hat{\imath}+v_2\hat{\jmath}+v_3\hat{k}\]
    \end{itemize}
    Si:\[-\vec{v}\therefore-\vec{v}=(-1)\vec{v}\]
    \subsection{Escalares}
    Son constantes numericas representadas por minusculas (\(a=12,b=5\),etc.) Alargan o encogen a un vector:
    \[k\vec{v}=\,<\!kv_1,kv_2,kv_3\!>\]
    \subsection{Magnitudes}
    Vulgarmente conocidos como la longitud de un vector. Se representa de la sig. manera:
    \[\lVert\vec{v}\rVert=\sqrt{v_1^2+v_2^2+v_3^2}\]
    \subsection{Vectores unitarios}
    Son vectores de referencia, usados generalmente para indicar la dirección de los ejes
    \[\hat{\imath}=\,<\!1,0,0\!>\,\hat{\jmath}=\,<\!0,1,0\!>\,\hat{k}=\,<\!0,0,1\!>\,\]
    Para obtener el vector unitario de un vector cualquiera, se realiza lo siguiente:
    \[\lambda\vec{v}=\frac{\vec{v}}{\lVert\vec{v}\rVert}\]
    \subsection{Suma}
    \[\vec{u}+\vec{v}=\,<\!u_1+v_1,u_2+v_2,u_3+v_3\!>\]
    \subsection{Resta}
    \[\vec{u}-\vec{v}=\,<\!u_1-v_1,u_2-v_2,u_3-v_3\!>\]
    \subsection{Producto punto}
    Genera un escalar:
    \[\vec{u}\cdot\vec{v}=u_1v_1+u_2v_2+u_3v_3\]
    Algunas propiedades son las siguientes:
    \[\vec{u}\cdot\vec{v}=\lVert\vec{u}\rVert^2\]
    \[\vec{u}\cdot\vec{v}=\lVert\vec{u}\rVert\lVert\vec{v}\rVert\cos\theta\]
    Donde:
    \[\vec{u}\cdot\vec{v}>0\leftrightarrow\theta \text{ es agudo}\]
    \[\vec{u}\cdot\vec{v}<0\leftrightarrow\theta \text{ es obtuso}\]
    \[\vec{u}\perp\vec{v}\leftrightarrow\vec{u}\cdot\vec{v}=0\]
    \subsection{Producto cruz}
    Genera un vector perpendicular a los vectores que se usaron en la operación:
    \[\vec{u}\times\vec{v}=\begin{pmatrix}
    \hat{\imath} &\hat{\jmath}& \hat{k}\\
    u_x&u_y&u_z\\
    v_x&v_y&v_z
    \end{pmatrix}=\hat{\imath}\begin{pmatrix}
    u_y&u_z\\
    v_y&v_z
    \end{pmatrix}-\hat{\jmath}\begin{pmatrix}
    u_x&u_z\\
    v_x&v_z
    \end{pmatrix}+\hat{k}\begin{pmatrix}
    u_x&u_y\\
    v_x&v_y
    \end{pmatrix}\]
    Donde:
    \[\vec{u}\times\vec{v}=0\rightarrow\vec{u}\parallel\vec{v}\]
    \subsection{Proyección}
    Dados \(\vec{u},\,\vec{v}\) donde \(\lVert\vec{v}\rVert>\lVert\vec{u}\rVert\), si se desea saber la Proyección de \(\vec{u}\) sobre \(\vec{v}\) se debe de hacer lo siguiente:
        \[\text{comp}_{\vec{v}}\vec{u}=\lVert\vec{u}\rVert\cos\theta\left(\lambda\vec{v}\right)=\frac{\vec{u}\cdot\vec{v}}{\lVert\vec{v}\rVert}\leftarrow\text{escalar}\]
        \[\text{proy}_{\vec{v}}\vec{u}=\text{comp}_{\vec{v}}\vec{u}\left(\lambda\vec{v}\right)=\frac{\vec{u}\cdot\vec{v}}{\lVert\vec{v}\rVert^2}\left(\vec{v}\right)\leftarrow\text{vector}\]
    \subsection{Cosenos directores}
    Dado \(\vec{u}\), para obtener \(\cos\alpha,\,\cos\beta,\,\cos\gamma\):
    \[\cos\alpha\,\text{(ángulo respecto al eje }x)=\frac{\vec{u}\cdot \hat{\imath}}{\lVert\vec{u}\rVert\lVert\hat{i}\rVert}\]
    \[\cos\beta\,\text{(ángulo respecto al eje }y)=\frac{\vec{u}\cdot \hat{\jmath}}{\lVert\vec{u}\rVert\lVert\hat{j}\rVert}\]
    \[\cos\gamma\,\text{(ángulo respecto al eje }z)=\frac{\vec{u}\cdot \hat{k}}{\lVert\vec{u}\rVert\lVert\hat{k}\rVert}\]
    Notese que la referencia respecto a cada eje es debido al uso de los vectores unitarios (\(\hat{\imath},\,\hat{\jmath},\,\hat{k}\)). 
    \section{Rectas en \(\mathbb{R}^3\)}
    Se necesitan al menos dos puntos y el vector direccional que pasa por los dos puntos. Otra opción puede ser un punto (\(\vec{r_0}\)) y el vector direccional (\(\vec{v}\)).
    \[<\!x,y,z\!>\,=\,<\!x_0+at,y_0+bt,z_0+ct\!>,\, \vec{v}=\,<\!a,b,c\!>,\, \vec{r_0}=\,<\!x_0,y_0,z_0\!>\]
    \subsection{Ecuaciones simetricas}
    Dados:
    \[R_1=\,<\!x_0+at,y_0+bt,z_0+ct\!>\]
    \[R_2=\,<\!x_1+ds,y_1+es,z_1+fs\!>\]
    Donde:
    \[\vec{v}_1=\,<\!a,b,c\!>\]
    \[\vec{v}_2=\,<\!d,e,f\!>\]
    Pasos:\\\newline
    \begin{algorithm}[H]
        \SetAlgoLined
        \KwData{Dos ecuaciones de rectas}
        \KwResult{Los valores de intersección de ambas rectas}
        INICIO\;
        Obtener \(\vec{v}_1\cdot\vec{v}_2\)\;
        \eIf{\(\vec{v}_1\cdot\vec{v}_2=0\)}{
        \(\vec{v}_2=k\vec{v}_1\)\;
        }{
            Realizar un sistema de ecuaciones de la forma:\\
            \(\begin{matrix}
                x_{R_1}&=&x_{R_2}\\
                y_{R_1}&=&y_{R_2}\\
                z_{R_1}&=&z_{R_2}
            \end{matrix}\rightarrow\begin{matrix}
                x_0+at&=&x_1+ds\\
                y_0+bt&=&y_1+es\\
                z_0+ct&=&z_1+fs\
            \end{matrix}\)
            \\Resolver el sistema para \(t\) y \(s\)\;
            Comprobar los valores anteriores\;
            Evaluar dichos valores en cualquiera de las ecuaciones de recta\;
            Escribir el punto de intersección\;
        }
        FIN\;
    \end{algorithm}
    \subsection{Planos}
    \[\vec{n}=a\hat{\imath}+b\hat{\jmath}+c\hat{k},\,\vec{u}=\,<\!x-x_0,y-y_0,z-z_0\!> \text{donde }\vec{u}\cdot\vec{n}=0\]
    \[\therefore \text{Ec}_{\text{plano}}=a(x-x_0)+b(y-y_0)+c(z-z_0)=0\]
    \[ax+by+cz+d=0\]
\end{document}