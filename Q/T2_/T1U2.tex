\documentclass[letterpaper, 12pt]{article}
\usepackage[letterpaper, top=2.5cm, bottom=2.5cm, left=3cm, right=3cm]{geometry} %margenes
\usepackage[utf8]{inputenc} %manejo de caracteres especiales
\usepackage[spanish]{babel} %manejo de encabezados de inglés a español
\usepackage{fancyhdr} %formato de los encabezados de página
\usepackage{ragged2e} %alineado real justficado
\usepackage{graphicx} %manejo de imagenes
\usepackage{amsmath} %manejo de notación matemática
\usepackage{mathtools} %manejo de notación matemática
\usepackage{blindtext} %texto de relleno
\usepackage{cancel} %permite la simbolización de cancelación de terminos
\usepackage{enumitem}[shortlabels] %listas con letras
\usepackage{amssymb} %manejo de simbolog►1a matematica
\usepackage{mhchem}
\usepackage[backend=biber]{biblatex}\addbibresource{referencias.bib}

\pagestyle{fancy}
\fancyhf{}
\rfoot{\thepage}

\nocite{*}

\begin{document}
\begin{titlepage}
    \begin{figure}[ht]
        \centering
        \includegraphics[width=15cm]{logosITT.png}
    \end{figure}
    \centering
    {\scshape\LARGE Tecnológico Nacional de México\\Instituto Tecnológico de Tijuana\par}
    \vspace{1cm}
    {\scshape\Large Química\par}
    \vspace{1cm}
    {\scshape\Large Unidad 2\par}
    \vspace{1.5cm}
    {\huge\bfseries Efectos a la salúd de los Clorofluorocarbonos (CFC), Arsénico (\ce{As}) y Plomo (\ce{Pb})\par}
    \vspace{2cm}
    {\Large\itshape C. Abraham Jhared Flores Azcona\\19211640\par}
    \vfill
    Profesor: \par
    Dr. Luis Ernesto Solís Delgado
    
    \vfill

    {\large 3 de noviembre del 2020}
\end{titlepage}

    \newpage
    \thispagestyle{empty}
    \tableofcontents
    
    \newpage
        \begin{justify}
    \rhead{\textbf{3 de noviembre del 2020}}
    \section{Clorofluorocarbonos (CFC)}
    \subsection{¿Qué son?}
    Como su nombre técnico implica, son derivados de los hidrocarburos saturados y originados por los intercambios de los átomos de \ce{H} con los átomos de \ce{F} y \ce{Ce}
    , siendo estos los principales elementos.
    \\ \newline
    Estos son una familia de gases que se emplean en diversas aplicaciones, principalmente en la industria dde refrigeración y de propelentes de aerosoles.  Están presentes en aislantes térmicos.
    Tienen una gran persistencia en la atmósfera, de 50 a más o menos 200 años. Eventualmente alcanzan la estratosfera, donde se disocian por acción de la radiación ultravioleta, liberando el Clorofluorocarbonos
    y dando comienzo al proceso de la capa de ozono.
    \subsection{Efectos a la salúd:}
    \begin{itemize}
        \item Problemas dermatológicos causados por la debilitación de la capa de ozono.
        \item Intoxicación y descoordinación motríz.
        \item Pérdida de la conciencia.
        \item Lesiones hepáticas y renales.
    \end{itemize}
    \section{Arsénico (\ce{As})}
    \subsection{¿Qué es?}
    Es un elemento natural que se encuentra en la tierra y entre los minerales en niveles altos. Los componentes del arsénico se usan para preservar madera, como plaguicidas y en ciertas industrias. Este forma parte del aire, el agua y la tierra a través
    del polvo que se lleva el viento. También puede penetrar en el agua debido a los desbordamientos. 
    \subsection{Efectos a la salúd:}
    \begin{itemize}
        \item Muy tóxico en su forma inorgánica.
        \item Puede causar cáncer.
        \item Lesiones cutáneas.
        \item Problemas de desarrollo.
        \item Enfermedades cardiovasculares.
        \item Diabetes.
    \end{itemize}
    \section{Plomo (\ce{Pb})}
    \subsection{¿Qué es?}
    Es un elemento químico tóxico presente de forma natural en la corteza terrestre. De manera pura no es muy útil o perjudicial. Su riesgo recae en los distintos compuestos químicos utilizados en la industria.
    Este se usa en baterías, pgmentos de pintura, vidrios y cables eléctricos de revestimiento. Su gran masa le permite ser usado como escudo para materiales radiactivos.\\ \newline
    El efecto nocivo es causado por la exposición y la acumulación gradual del elemento en el cuerpo.
    \subsection{Efectos a la salúd:}
    \begin{itemize}
        \item Problemas al sistema nervioso, renal, oseo y gastrointestinal.
        \item Propiciador de la anemia, hipertensión, disfunción renal.
        \item Inmunotoxicidad y toxicidad reproductiva.
        \item En mujeres embarazadas, propicia el aborto natural, muerte fetal, parto prematuro y bajo peso, y malformaciones leves en el feto.
    \end{itemize}
        \end{justify}

        \newpage
        \addcontentsline{toc}{section}{Referencias}
        \printbibliography
\end{document}