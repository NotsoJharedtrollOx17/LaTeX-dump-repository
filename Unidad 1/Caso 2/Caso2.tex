\documentclass[stu, 12pt, letterpaper, donotrepeattitle, floatsintext, natbib]{apa7}
\usepackage[utf8]{inputenc}
%\usepackage{fontspec} %paquete para usar la fuente Arial 12
\usepackage{comment}
\usepackage{marvosym}
\usepackage{graphicx}
\usepackage{float}
\usepackage[normalem]{ulem}
\usepackage[spanish]{babel} 
\usepackage{lastpage} %para le formato que quiere la profe QUITAR SI QUIERES OG APA7
\usepackage{ragged2e} %para le formato que quiere la profe QUITAR SI QUIERES OG APA7
\usepackage{indentfirst} %para le formato que quiere la profe QUITAR SI QUIERES OG APA7

\setcounter{secnumdepth}{0} %permite enumerar las secciones QUITAR SI QUIERES OG APA7

%comando para ajustar la fuente Arial en todo el documento
%\setmainfont{Arial} %COMPILAR DOC CON XeLateX DOS VECES

\DeclareCaptionLabelSeparator*{spaced}{\\[2ex]}
\captionsetup[table]{textfont=it,format=plain,justification=justified,
  singlelinecheck=false,labelsep=spaced,skip=1pt}

\selectlanguage{spanish}

\useunder{\uline}{\ul}{}
\newcommand{\myparagraph}[1]{\paragraph{#1}\mbox{}\\}

%\rfoot{Página \thepage \hspace{1pt} de \pageref{LastPage}}%QUITAR SI QUIERES OG APA7 
\rhead{} %QUITAR SI QUIERES OG APA7
\setcounter{secnumdepth}{3} %permite enumerar las secciones QUITAR SI QUIERES OG APA7
\setlength{\parindent}{1.27cm} %sangria forzada QUITAR SI QUIERES OG APA7

\renewcommand\labelitemi{$\bullet$}

\newcommand*\chem[1]{\ensuremath{\mathrm{#1}}}

\begin{document}
    %PORTADA
    \begin{titlepage}
        \begin{figure}[ht]
            \centering
            \includegraphics[width=15cm]{logosITT.png}
        \end{figure}
        \centering
        {\Large Tecnológico Nacional de México\\Instituto Tecnológico de Tijuana\par}
        \vspace{1cm}
        {\Large SCD-1011SC6C Ingeniería de Software\par}
        \vspace{1cm}
        {\Large Unidad 1\par}
        \vspace{2cm}
        {\Large\bfseries Caso práctico 2\par}
        \vspace{2cm}
        {\large Dra. Martha Elena Pulido\par}
        \vfill
            {\large Abraham Jhared Flores Azcona, 19211640\par}
        \vfill
        {\large 1 de marzo de 2021}
    \end{titlepage}

% Índices
\pagenumbering{arabic}
    % Contenido
\renewcommand\contentsname{Contenido}
\tableofcontents

% Cuerpo 
    %NOTA: PARA CITAR ESTILO "Merts (2003)" usar \cite{<nombre_cita_bib>}
    %                        "(Metz, 1978)" usar \citep{<nombre_cita_bib>}
\newpage
\section{Segmentos-Objetivo de IKEA}
El \begin{justifying}
    grupo de clientes ideal fue primeramente los desplzados de la 2da. Guerra Mundial, por la necesidad de reponer los muebles que posiblemente
vieron destrozados por la guerra, por lo que serían clientes con una alta necesidad económica. Hoy en día la filosofía del negocio sigue siendo
similar, solo que se agregaron más detalles para hacer la compra del mueble una experiencia en sí para toda la familia.\par
\end{justifying}
\vspace{\baselineskip}
\section{Posicionamiento de marca}
La \begin{justifying}
    marca aprovechó una deficiencia del mercado de muebles por la finalización de la 2da. Guerra Mundial. Esto resultó
en hacer una experiencia de lujo para los clientes, aparte de precios bajos, productos contemporaneos y una imágen de lo que
Suecia ofrece ante el mundo.\par
\end{justifying}
\vspace{\baselineskip}
\section{Manejo de los precios bajos}
La \begin{justifying}
    primera opción muy coherente fue la de contratar mano de obra barata, fabricación de los muebles sin ensamblar
con maderas asequibles, locales instalados en lugares muy baratos de bienes raíces y la experiencia de que el usuario debe de ensamblar su producto y sobre todo, relaciones estrategias
con sus proveedores de material para asegurar precios accesibles en éste mercado.\par
\end{justifying}
\vspace{\baselineskip}
\section{Coherencia de la mezcla de marketing con la estrategia de marketing}
Para \begin{justifying}
    los tiempos de la pandemia, el producir un catálogo impreso resulta muy ineficiente debido a que de tods modos sus anucios se puede exparsir
pr medios digitales y permiten hacer efectos de bola de nieve. Por otro punto, el imprimir catálogos les permite
acceder con mercados más renuantes de tecnología, que prefieren un experiencia más clásica y más minusciosa.\par
\end{justifying}
\end{document}