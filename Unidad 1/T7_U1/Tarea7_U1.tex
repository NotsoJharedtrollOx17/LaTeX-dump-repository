\documentclass[letterpaper, 12pt]{article}
\usepackage[letterpaper, top=2.5cm, bottom=2.5cm, left=3cm, right=3cm]{geometry} %margenes
\usepackage[utf8]{inputenc} %manejo de caracteres especiales
\usepackage[spanish]{babel} %manejo de encabezados de inglés a español
\usepackage{fancyhdr} %formato de los encabezados de página
\usepackage{ragged2e} %alineado real justficado
\usepackage{graphicx} %manejo de imagenes
\usepackage{amsmath} %manejo de notación matemática
\usepackage{mathtools} %manejo de notación matemática
\usepackage{blindtext} %texto de relleno
\usepackage{cancel} %permite la simbolización de cancelación de terminos
\usepackage{enumitem}[shortlabels] %listas con letras
\usepackage{amssymb} %manejo de simbolog►1a matematica

\pagestyle{fancy}
\fancyhf{}
\rfoot{\thepage}

\begin{document}
    
\setcounter{page}{1}
\thispagestyle{fancy}
\lhead{\textbf{Tarea 7, U1}}
\rhead{\textbf{28/09/2020}}
\section{Vectores en el espacio}
\subsection*{Obtener la ecuación vectorial de la recta:}
\begin{enumerate}
    \item \((1,8,-2),\,\vec{v}=-7\hat \imath-8\hat \jmath\)
    \item \((1,1,-1)\text{ y }(-4,1,-1)\)
    \item \(<\!x,y,z\!>\,=\,<\!2+5t,-1+\frac{1}{3}t,9-2t\!>\) (Encontrar el punto con el que cruza el plano \(yz\)).
\end{enumerate}
\subsection{Calculos:}
Formula: 
\[<\!x,y,z\!>\,=\,<\!x_0+at,y_0+bt,z_0+ct\!>,\,\vec{v}=\,<\!a,b,c\!>,\, \vec{r_0}=\,<\!x_0,y_0,z_0\!>\]
1.\\
\(\vec{r_0}=\,<\!1,8,-2\!>, \vec{v}=\,<\!-7,-8,0\!> \therefore <\!x,y,z\!>\,=\,<\!1-7t,8-8t,-2\!>\)\\ \newline
2.\\
\(\vec{r_0}=\,<\!1,1,-1\!>, \vec{r_1}=\,<\!-4,1,-1\!>, \vec{v}=\vec{r_0}-\vec{r_1} \text{ ó } \vec{v}=\vec{r_1}-\vec{r_0};\)\\ \newline
\(\vec{v}=\vec{r_0}-\vec{r_1}=\,<\!1+4,1-1,-1+1\!>\,=\,<\!5,0,0\!>\therefore <\!x,y,z\!>\,=\,<\!1+5t,1,-1\!>\)\\ \newline
3.\\
Para el tercer reactivo se debe de mantener la intuición de si tal ecuación debe de cruzar cierto plano (en este caso el plano \(yz\)) en \(\mathbb{R}^3\)
se asume que el valor para el eje no presente (en este caso el eje \(x\)) es cero. Por lo tanto:\\
\[<\!x,y,z\!>\,=\,<\!2+5t,-1+\frac{1}{3}t,9-2t\!> \rightarrow x=2+5t\]
Desarrollando la ecuación de \(x\) en \(z=0\):\\
\[\begin{matrix}
    x&=&2+5t\\
    0&=&2+5t\\
    -2&=&5t\\
    -\frac{2}{5}&=&t
\end{matrix}\]
Con este valor de \(t\), simplemente se procede a sustitur dicho valor en el resto de la ecuación
\[<\!x,y,z\!>\,=\,<\!0,-1+\frac{1}{3}\left(-\frac{2}{5}\right)\!,9-2\left(-\frac{2}{5}\right)\!>\]
\[<\!x,y,z\!>\,=\,<\!0,-\frac{4}{3},\frac{49}{5}\!>\]
Dando por concluido lo indicado.
\end{document}