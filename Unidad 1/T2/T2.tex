% Preámbulo
\documentclass[stu, 12pt, letterpaper, donotrepeattitle, floatsintext, natbib]{apa7}
\usepackage[utf8]{inputenc}
\usepackage{comment}
\usepackage{marvosym}
\usepackage{graphicx}
\usepackage{float}
\usepackage[normalem]{ulem}
\usepackage[spanish]{babel} 
\usepackage{lastpage} %para le formato que quiere la profe QUITAR SI QUIERES OG APA7
\usepackage{ragged2e} %para le formato que quiere la profe QUITAR SI QUIERES OG APA7
\usepackage{indentfirst} %para le formato que quiere la profe QUITAR SI QUIERES OG APA7

\selectlanguage{spanish}
\useunder{\uline}{\ul}{}
\newcommand{\myparagraph}[1]{\paragraph{#1}\mbox{}\\}

\rfoot{Página \thepage \hspace{1pt} de \pageref{LastPage}}%QUITAR SI QUIERES OG APA7 
\rhead{} %QUITAR SI QUIERES OG APA7
\setcounter{secnumdepth}{3} %permite enumerar las secciones QUITAR SI QUIERES OG APA7
\setlength{\parindent}{1.27cm} %sangria forzada QUITAR SI QUIERES OG APA7

% Portada
\thispagestyle{empty}
\title{\Large Economía, sociedad y naturaleza mediante}
\author{Abraham Jhared Flores Azcona} % (autores separados, consultar al docente)
% Manera oficial de colocar los autores:
%\author{Autor(a) I, Autor(a) II, Autor(a) III, Autor(a) X}
\affiliation{Instituto Tecnológico de Tijuana}
\course{ACD-0908SC5C Desarrollo Sustentable}
\professor{M.C. Trinidad Castro Villa}
\duedate{30 de agosto de 2021}

\begin{document}
\maketitle


% Índices
\pagenumbering{arabic}
    % Contenido
\renewcommand\contentsname{Contenido}
\tableofcontents

% Cuerpo 
    %NOTA: PARA CITAR ESTILO "Merts (2003)" usar \cite{<nombre_cita_bib>}
    %                        "(Metz, 1978)" usar \citep{<nombre_cita_bib>}
\newpage
\section{Introducción}
En \begin{justifying}
    esta redacción se exponen los distintos significados de la economía, sociedad, naturaleza mediante y de los puntos desarrollados en el Informe Bruntland. A travez del tiempo
las palabras siempre cambian de significado por distintas razones, en el mundo contemporaneo es de vital imoprtancia revisar dichos conceptos para tener bases sólidas para entender 
la escala del cambio que puede causar el Desarrollo Sustentable.\par
\end{justifying}
\vspace{\baselineskip}
\section{Economía}
Generalmente \begin{justifying}asociado al comercio y mundo de negocios. De manera similar, los conceptos de la misma se relacionan con alguno de los aspectos anteriores.
Acorde a \cite{kenton-no-date}
en la página web de Investopedia define a la palabra como el conjunto grande de la inter-relacionada producción, consumo, e intercambio de actividades que apoyan en la determinación
de cómo se alocan los recursos escasos. Incluye también que la producción, consumo y la distribución de los bienes y servicios son usados para satisfacer las 
necesidades de aquellos que viven y operan dentro de la economía, a lo cual se le refiere como un sistema económico.\par
\vspace{\baselineskip}
Otros conceptos refieren a esta palabra como una disciplina académica; estrictamente una disciplina de las ciencias sociales. Para \cite{alburquerque-2018}, en lo que respecta de los estudios
de economía de nivel universitario, se le considera la Teoría Económica convencial en la cual su concepto es el estudio de cómo la sociedad lleva a cabo las actividades
orientadas a la atención de las necesidades de la población a través y distribución de los bienes y servicios generados para ello en un enfoque abstracto y ahistórico, a partir
de supuestos de funcionamiento aplicables para cualquier contexto.\par
\vspace{\baselineskip}
También se le acostumbra a tratar como un sistema a la economía. \cite{alburquerque-2018} elabora
porque los elementos de estudio de la economía interactuan entre sí con un objetivo determinado, satisfacer
sus necesidades. Formalmente, un sistema económico está constituido por estructuras de producción, distribución y consumo de bienes y servicios donde el contexto social, cultural 
y político influyen tanto en el sistema como en el análisis del mismo.\par
\vspace{\baselineskip}
\noindent En consenso, la economía trata el cómo la producción, consumo y distribución de bienes y servicios afecta a los elementos que participan en ella.\par\end{justifying}
\vspace{\baselineskip}

\section{Sociedad}
Aquello \begin{justifying}en donde todo ser humano interactua queriendo o no. El poder definir dicho concepto es vital para el estudio de las ciencias sociales.
Acorde a \cite{thompson-2017}
la primer distinción del concepto recae en una realidad independiente de los individuos que participan en ella. El concepto entonces se define como las relaciones
e instituciones dentro de una gran comunidad de personas la cual no se puede reducir a una colección simple o la agregación de los individuos.\par
\vspace{\baselineskip}
Para \cite{westreicher-2021}
la sociedad es un conjunto de individuos que conviven en un mismo territorio bajo un determinado esquema de organicación, compartiendo lazos económicos, políticos y culturales. Un aspecto muy relevante que 
él destaca dentro de las características de la sociedad es que los integrantes tienen un sentido de indentidad y pertenencia a un mismo colectivo y que siempre evolucionan, lo que nos dice que solo por ser personas no necesariamente nos 
hace una sociedad a través de nuestra existencia.\par
\vspace{\baselineskip}
También se considera que la sociedad es un organismo \citep{spencer-2004}.
El crecimiento social se prolonga habitualmente hasta cierto punto en que colapsan que por la definición anterior, suena congruente
inferir que (a pesar de no ser un organismo biológicamente vivo) éste también sucita las fases de la vida de un organismo vivo.\par
\vspace{\baselineskip}
En resumén, la sociedad es un conjunto de individuos que comparten algo en común, y que sus mecanísmos lo hacen tan complejo como un organísmo biológicamente vivo.\par\end{justifying}
\vspace{\baselineskip}
\section{El Informe Brundtland}
También \begin{justifying}conocido como Nuestro Futuro Común (Our Common Future). Acorde a lo descrito en el Prefacio del Presidente del mismo informe (\citeyear{world-commission-on-environment-and-development-1987}), el objetivo principal es el de persuadir a las naciones en la necesidad de regresar al multilateralísmo agregando que uno de los desafíos primordiales
es el encontrar rutas de desarrollo sustentable ya que el proveer del ímpetu para una búsqueda renovada de soluciones multilaterales y un sistema económico internacional reestructurado; esto
cortando las divisiones de la soberanía nacional, las estrategias límitadas por la ganancia económica, y de las disciplinas separadas de la ciencia.\par
\vspace{\baselineskip}
Algunos aspectos importantes que se mencionan en el informe son los siguientes:
\begin{itemize}
    \item En general, la expectativa de vida va en aumento, la literacia de los adultos está escalando; la cantidad de infantes empezando la escuela va en aumento y la producción de alimentos aumenta.
    \item Al mismo tiempo, los fallos en la gestión del entorno humano ha escalado en igual proporción la gente iliterada, la cantidad de hogares sin agua potable o seguras en general.
    \item El incremento de población se proyecta que ocurra con un 90\% de incremento en países pobres y el 90\% de dicho crecimiento en ciudades a reventar.
    \item La actividad económica se ha multiplicado creando una economía mundial de 13 millones de dólares. Mucho de ése crecimiento económico toma materias prímas.
    \item El alcanzar a saciar las necesidades básicas requiere no solo de una nueva era de crecimiento económico para las naciones, sino que exista una igualdad económica entre los estratos sociales.
    \item El Desarrollo Sustentable solo puede ser perseguido si el tamaño de la población y el crecimiento estén en armonía con el potencial productivo cambiante del ecosistema.
    \item La respuesta gubernamental para el Desarrollo Sustentable ha sido creando instituciones de relleno que terminan culpando a la mala administración guvernamental; eventualmente por la abundancia de dichas instituciones, se crea una fachada de buena obra para el ciudadano desinformado.
\end{itemize}\par\end{justifying}
\vspace{\baselineskip}
\section{Naturaleza}
Sin \begin{justifying}lugar a dudas, la primer palabra que se asocia a temas de Desarrollo Sustentable, y si no es así, se relaciona con el colór verde que generalmente simboliza vida.
En la relación naturaleza-sociedad, la naturaleza es un objeto de uso, apropiación y explotación para el ser humano y para la sociedad; más que nada, un objeto con fines económicos acorde
a lo expuesto en la definición de Economía de éste escrito. A travéz de la historia de las civilizaciones, la relación indicada era recíproca como en las sociedades nómadas
conformadas por tribus recolectoras y cazadoras porque dependían de las dinámicas ambientales por lo que sostenían una conexión entre el orden natural y el orden bienestar \citep{sarmiento-2017}.\par 
\vspace{\baselineskip}
Hablando en un contexto estrictamente semántico y etimológico, la palabra es muy difícil de definir de manera unificada desde La Antigüedad. Como lo describe
\cite{ducarme-2020}
desde la Antigua Grecia la palabra la cuál terminó siendo traducida a ``naturaleza'' es \emph{phusis}, la cuál es basada en en verbo para ``creciendo, produciendo''. Aristóteles
en su libro \emph{Física} define a la naturaleza como la esencia de las cosas, de qué están hechas y que encomiendan su destino. En la Antigua Roma, Cicerón intrudoce una oposición como ejemplo de
qué es la naturaleza y qué es cultura, la primera siendo un estado inicial vacio de influencia humana, y la segunda correspondiendo a una apropiación por las sociedades humanas dado que los Romanos
consideraban la vida en las ciudades como lugares de suciedad y de pecado, mientras que el campo era para ``la buena vida''.\par
\vspace{\baselineskip}
Continuando con los autores anteriores, su definición en las sociedades cristianas era algo que Diós provió al Hombre para que éste lo poblase, bajo la gracia de Diós. De una manera mucho más
contemporanea, ni siquiera los biólogos saben como definirle, plantena que es necesario protegerla y conservarla, pero no la han podido definir de manera concreta.\par
\vspace{\baselineskip}
En general, la naturaleza se coincide que se debe proteger ya que es una herramienta, y como toda herramienta, si no se cuida y se mantiene, deja de cumplir su función de manera eficente procurando
un punto de equilibrio.\par\end{justifying}
\vspace{\baselineskip}
\section{Mediante}
De \justifying primera mano, se puede inferir que tiene algo que ver con la palabra "medio", alguien que es un tercero imparcial (hablando de aspectos jurídicos). Esta palabra proviene de la palabra ``mediar'' la cual 
(en el contexto de la materia) se define como el actuar entre dos o más partes para ponerlas de acuerdo en un pleito o negocio \citep{-asale-2020}. 
Otros significados un tanto breves mantienen que el mediar es interceder o rogar por alguien, también como el participar o intervenir en algo, coincidiendo en cierto modo con la inferencia escrita.\par
\vspace{\baselineskip}
Con ámbas definiciones desarrolladas en distintas perspectivas, podemos definir a la naturaleza mediante como el ambiente no culturalizado que intermedia a la sociedad con su entorno.\par
\vspace{\baselineskip}
\newpage
% Referencias
\setcounter{secnumdepth}{0} %permite enumerar las secciones QUITAR SI QUIERES OG APA7
\renewcommand\refname{\textbf{Referencias}}
\bibliography{referencias}

\end{document}