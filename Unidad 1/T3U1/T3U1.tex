\documentclass[letterpaper, 12pt]{article}
\usepackage[letterpaper, top=2.5cm, bottom=2.5cm, left=3cm, right=3cm]{geometry} %margenes
\usepackage[utf8]{inputenc} %manejo de caracteres especiales
\usepackage[spanish]{babel} %manejo de encabezados de inglés a español
\usepackage{fancyhdr} %formato de los encabezados de página
\usepackage{ragged2e} %alineado real justficado
\usepackage{graphicx} %manejo de imagenes
\usepackage{amsmath} %manejo de notación matemática
\usepackage{mathtools} %manejo de notación matemática
\usepackage{blindtext} %texto de relleno
\usepackage{cancel} %permite la simbolización de cancelación de terminos
\usepackage{enumitem}[shortlabels] %listas con letras
\usepackage{amssymb} %manejo de simbología matematica
\usepackage{float}

\pagestyle{fancy}
\fancyhf{}
\rfoot{}

\begin{document}
\thispagestyle{fancy}
\lhead{\textbf{Tarea 3, U1}}
\rhead{\textbf{3 de marzo de 2021}}
\section*{Ecuaciones Diferenciales Separables}
\subsection*{Resolver lo siguiente:}
\subsubsection*{\emph{Problema 1:}}
Resuelve la siguiente ecuación diferencial:
\[\tan x\sin y\, dx + \cos^3 x\cot y\, dy = 0\]
\textbf{• Solución:} 
\begin{equation*}
    \begin{aligned}
        \tan x\sin y\, dx &= -\cos^3 x\cot y\, dy \\[5pt]
        \frac{\tan x}{\cos^3 x}\, dx &= -\frac{\cot y}{\sin y}\, dy\\[5pt]
        \underbrace{\frac{-\sin x}{\cos^4 x}}_{\tan x=\frac{\sin x}{\cos x}}\, dx &= \underbrace{\frac{\cos y}{\sin^2 y}}_{\cot y = \frac{\cos y}{\sin y}}\, dy\\[5pt]
        \int \frac{-\sin x}{\cos^4 x}\, dx &= \int \frac{\cos y}{\sin^2 y}\, dy\\[10pt]
        \underbrace{\int \frac{du_1}{u^4_1}}_{u_1=\cos x\, \rightarrow \, du_1=-\sin x\, dx} &= \underbrace{\int \frac{du_2}{u^2_2}}_{u_2=\sin y \, \rightarrow \, du_2=\cos y \, dy}\\[5pt]
        \int u_1^{-4}\, du_1 &= \int u_2^{-2}\, du_2\\[5pt]
        \frac{u_1^{-3}}{-3}+c &= \frac{u_2^{-1}}{-1}\\[5pt]
        \frac{1}{-3u_1^{3}}+c &= \frac{1}{-u_2}\\[5pt]
        \left(-3u_1^3+c\right)^{-1} &= -u_2^{-1}\\[5pt]
        \left(\left(-3u_1^3+c\right)^{-1}\right)^{-1} &= \left(-u_2^{-1}\right)^{-1}\\[5pt]
        -3u_1^{3} + c &= -u_2\\[5pt]
        c &= -u_2 + 3u_1^{3}\\[5pt]
        c &= 3u_1^{3} - u_2\\[5pt]
        c &= 3\cos^3 x - \sin y\,\, \textbf{(1.1)}\\[5pt]
    \end{aligned}
\end{equation*}
Donde \textbf{(1.1)} correponde a la respuesta final.

\newpage
\subsubsection*{\emph{Problema 2:}}
Resuelve la siguiente ecuación diferencial:
\[\frac{dy}{dx}=xe^{6x-5y}\]
\textbf{• Solución:}
\begin{equation*}
    \begin{aligned}
        \frac{dy}{dx} &= xe^{6x-5y}\\[5pt]
        dy &= xe^{6x-5y}\, dx\\[5pt]
        dy &= \frac{xe^{6x}}{e^{5y}}\, dx\\[5pt]
        e^{5y}\, dy &= xe^{6x}\, dx\\[5pt]
        \underbrace{\int e^{5y}\, dy}_{A} &= \underbrace{\int xe^{6x}\, dx}_{B}\\[5pt]
        A &= B\,\, \textbf{(1.2)}
    \end{aligned}
\end{equation*}
\justify
Por fines de simplicidad, procedemos a resolver \(A\) y \(B\) individualmente para posteriormente sustituirlas en la expresión original.
\\\newline
\textbf{Para \(A\):}
\begin{equation*}
    \begin{aligned}
        A:=&\: \int e^{5y}\, dy=\underbrace{\frac{1}{5}\int e^{u}\, du}_{u=5y\, \rightarrow \frac{du}{5}=dy}=\frac{1}{5}e^u=\underbrace{\frac{1}{5}e^{5y}+c}_{\textbf{(1.3)}}
    \end{aligned}
\end{equation*}
\textbf{Para \(B\):}
\begin{equation*}
    \begin{aligned}
        B:=&\: \int \underbrace{xe^{6x}}_{u\, dv}\, dx = \underbrace{x\cdot \frac{e^{6x}}{6}}_{uv}-\int \underbrace{\frac{e^{6x}}{6}\cdot 1}_{v\, du}\, dx
        =\frac{xe^{6x}}{6}-\frac{1}{6}\int e^{6x}\, dx =\\[5pt]
        =& \: \frac{xe^{6x}}{6}-\frac{1}{6}\cdot\underbrace{\frac{1}{6}\int e^u\, du}_{u=5x\, \rightarrow \, \frac{du}{5}=dx}
        =\frac{xe^{6x}}{6}-\frac{e^u}{36}+c=\underbrace{\frac{xe^{6x}}{6}-\frac{e^{6x}}{36}+c}_{\textbf{(1.4)}}
    \end{aligned}
\end{equation*}
Habiendo calculado \(A\) y \(B\) se procede a sustituir \textbf{(1.3)} y \textbf{(1.4)} en \textbf{(1.2)} para continuar con la resolución:
\\
\begin{equation*}
    \begin{aligned}
        A &= B\,\, \textbf{(1.2)}\\[5pt]
        \underbrace{\frac{1}{5}e^{5y}+c}_{\textbf{(1.3)}} &= \underbrace{\frac{xe^{6x}}{6}-\frac{e^{6x}}{36}+c}_{\textbf{(1.4)}}\\[5pt]
        \frac{e^{5y}}{5}&=\frac{xe^{6x}}{6}-\frac{e^{6x}}{36}+c\\[5pt]
    \end{aligned}
\end{equation*}
\begin{equation*}
    \begin{aligned}
        5\left(\frac{e^{5y}}{5}\right) &= \left(\frac{xe^{6x}}{6}-\frac{e^{6x}}{36}+c\right)5\\[5pt]
        e^{5y} &= \frac{5}{6}\cdot xe^{6x}-\frac{5}{36}e^{6x}\\[5pt]
        36\left(e^{5y}\right) &= \left(\frac{5}{6}\cdot xe^{6x}-\frac{5}{36}e^{6x}\right)36\\[5pt]
        36e^{5y} &= \frac{180}{6}\cdot xe^{6x}-\frac{180}{36}e^{6x}+c\\[5pt]
        36e^{5y} &= 30xe^{6x}-5e^{6x}+c\\[5pt]
        36e^{5y} - 30xe^{6x}+5e^{6x} &= c\\[5pt]
        30xe^{6x} - 5e^{6x} - 36e^{5y} &= c\,\, \textbf{(1.5)}\\[5pt]
    \end{aligned}
\end{equation*}
Donde \textbf{(1.5)} correponde a la respuesta final.
\end{document}