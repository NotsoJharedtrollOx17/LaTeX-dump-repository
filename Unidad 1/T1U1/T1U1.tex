\documentclass[letterpaper, 12pt]{article}
\usepackage[letterpaper, top=2.5cm, bottom=2.5cm, left=3cm, right=3cm]{geometry} %margenes
\usepackage[utf8]{inputenc} %manejo de caracteres especiales
\usepackage[spanish]{babel} %manejo de encabezados de inglés a español
\usepackage{fancyhdr} %formato de los encabezados de página
\usepackage{ragged2e} %alineado real justficado
\usepackage{graphicx} %manejo de imagenes
\usepackage{amsmath} %manejo de notación matemática
\usepackage{mathtools} %manejo de notación matemática
\usepackage{blindtext} %texto de relleno
\usepackage{cancel} %permite la simbolización de cancelación de terminos
\usepackage{enumitem}[shortlabels] %listas con letras
\usepackage{amssymb} %manejo de simbolog►1a matematica
\usepackage{float}

\pagestyle{fancy}
\fancyhf{}
\rfoot{\thepage}

\begin{document}
\setcounter{page}{1}
\thispagestyle{fancy}
\lhead{\textbf{Tarea 1, U1}}
\rhead{\textbf{25 de febrero de 2021}}
\section*{Escribe ecuaciones diferenciales}
\subsection*{Resolver lo siguiente:}
\subsubsection*{Problema 1:}
La tasa de cambio de calentamiento o enfriamiento es proporcional a la diferencia entre la temperatura ambiente
\(T_a\) y la temperatura actual \(T\) de la bebida. 
\\\newline
\textbf{¿Cuál ecuación describe esta relación?}
\[\frac{dT}{dt}=k(T_a-T)\]
\subsubsection*{Problema 2:}
Un químico se diluye en un tanque al inyectar agua puta en el tanque y extraer de él la solución existente, de manera que el volúmen en cualquier
momento \(t\) es \(20+2t\).
\\\newline
La cantidad \(z\) de químico en el tanque disminuye a una tasa proporcional a \(z\), e inversamente proporcional al volúmen de la solución en el tanque.
\\\newline
\textbf{¿Cuál ecuación describe esta relación?}
\[\frac{dz}{dt}=-\frac{kz}{20+2t}\]
\subsubsection*{Problema 3:}
Cada me el saldo \(B\) del préstamo de Harper crece un \(0\dot22\%\) y decrece \(\$250.00\).
\\\newline
\textbf{¿Cuál ecuación describe esta relación?}
\[\frac{dB}{dt}=0.0022B-250\]
\subsubsection*{Problema 4:}
La tasa de cambio del estímulo percibido, \(p\), con respecto a la intensida medida, \(s\), del estímulo es inversamente proporcional a la intensidad del estímulo.
\\\newline
\textbf{¿Cuál ecuación describe esta relación?}
\[\frac{dp}{ds}=\frac{k}{s}\]
\end{document}