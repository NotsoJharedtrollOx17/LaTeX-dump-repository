\documentclass[stu, 12pt, letterpaper, donotrepeattitle, floatsintext, natbib]{apa7}
\usepackage[utf8]{inputenc}
%\usepackage{fontspec} %paquete para usar la fuente Arial 12
\usepackage{comment}
\usepackage{marvosym}
\usepackage{graphicx}
\usepackage{float}
\usepackage[normalem]{ulem}
\usepackage[spanish]{babel} 
\usepackage{lastpage} %para le formato que quiere la profe QUITAR SI QUIERES OG APA7
\usepackage{ragged2e} %para le formato que quiere la profe QUITAR SI QUIERES OG APA7
\usepackage{indentfirst} %para le formato que quiere la profe QUITAR SI QUIERES OG APA7

\setcounter{secnumdepth}{0} %permite enumerar las secciones QUITAR SI QUIERES OG APA7

%comando para ajustar la fuente Arial en todo el documento
%\setmainfont{Arial} %COMPILAR DOC CON XeLateX DOS VECES

\DeclareCaptionLabelSeparator*{spaced}{\\[2ex]}
\captionsetup[table]{textfont=it,format=plain,justification=justified,
  singlelinecheck=false,labelsep=spaced,skip=1pt}

\selectlanguage{spanish}

\useunder{\uline}{\ul}{}
\newcommand{\myparagraph}[1]{\paragraph{#1}\mbox{}\\}

%\rfoot{Página \thepage \hspace{1pt} de \pageref{LastPage}}%QUITAR SI QUIERES OG APA7 
\rhead{} %QUITAR SI QUIERES OG APA7
\setcounter{secnumdepth}{3} %permite enumerar las secciones QUITAR SI QUIERES OG APA7
\setlength{\parindent}{1.27cm} %sangria forzada QUITAR SI QUIERES OG APA7

\renewcommand\labelitemi{$\bullet$}

\newcommand*\chem[1]{\ensuremath{\mathrm{#1}}}

\begin{document}
    %PORTADA
    \begin{titlepage}
        \begin{figure}[ht]
            \centering
            \includegraphics[width=15cm]{logosITT.png}
        \end{figure}
        \centering
        {\Large Tecnológico Nacional de México\\Instituto Tecnológico de Tijuana\par}
        \vspace{1cm}
        {\Large SCD-1011SC6C Ingeniería de Software\par}
        \vspace{1cm}
        {\Large Unidad 1\par}
        \vspace{2cm}
        {\Large\bfseries Ejercicio 5\par}
        \vspace{2cm}
        {\large Dra. Martha Elena Pulido\par}
        \vfill
            {\large Abraham Jhared Flores Azcona, 19211640\par}
        \vfill
        {\large 22 de febrero de 2021}
    \end{titlepage}

% Índices
\pagenumbering{arabic}
    % Contenido
\renewcommand\contentsname{Contenido}
\tableofcontents

% Cuerpo 
    %NOTA: PARA CITAR ESTILO "Merts (2003)" usar \cite{<nombre_cita_bib>}
    %                        "(Metz, 1978)" usar \citep{<nombre_cita_bib>}
\newpage
\section{Estudio de factibilidad}
\subsection{Definición}
Acorde \begin{justifying}
    a %citar a los de economipedia
    es un estudio que hace una empresa para determinar la posibilidad de poder desarrola un negocio o un proyecto que espera implementar \citep{quiroa-2021}. En otras palabras,
    les permite conoce si un negocio o proyecto se puede hacer o no sep uede hacer, cuáles son las condiciones ideales para realizarlo y cómo podría solucionar las
    dificultades que se puedan presentar.\par
\end{justifying}
Aparte, \begin{justifying}
    dichos estudios de factiblidad se pueden dividir en los siguientes ámbitos \citep{unknown-author-2019}:
\begin{itemize}
    \item \textbf{Factibilidad operativa:} competencias laborales del personal para realizar sus debidas funciones.
    \item \textbf{Factibilidad técnica:} posible respuesta favorable de la infraestructura para poder realizar el proyecto.
    \item \textbf{Factibilidad económica:} relación costo benefício del negocio o proyecto.
    \item \textbf{Factibilidad comercial:} potencial de clientela.
    \item \textbf{Factibilidad política y legal:} si el proyecto no atenta o inclumple alguna ley o norma de los distintos niveles administrativos.
    \item \textbf{Factibilidad de tiempo:} si el tiempo planificado coincide con el tiempo real de desarrollo.
\end{itemize}\par
\end{justifying}
\vspace{\baselineskip}
\subsection{Técnicas para realizar el estudio de factibilidad}
A \begin{justifying}
    grandes rasgos, las técnicas para realizar dichos estudios se pueden sintetizár en los siguientes puntos:
    \begin{enumerate}
        \item Delinear la idea o acción planificada. Tener claro que se quiere hacer.
        \item Examinar el espacio de mercado y la viavilidad comercial de la acción. Se pued lograr mediante recolección de investigaciones de otras empresas sobre nuestro rubro comercial.
        \item Detectar fortalezas y debilidades dentro de la idea.
        \item Determinar riesgos insuperables para la acción. Se recomienda recibir asesorías al respecto para considerar otras vulnerabilidades.
    \end{enumerate}\par
\end{justifying}
\vspace{\baselineskip}
\section{Conclusión}
Conocer \begin{justifying}
    sobre las técnicas y cnsideraciones pertinentes a los estudios de factibilidad permiten al interesado obtener una vista más definida de sus pros y contras
    para tomar cartas en el asunto y así empezar con la ejecución de sus planes y/ó la correción de los mismos.\par
\end{justifying}
\vspace{\baselineskip}
%debe estar dentro del mismo folder donde se encuentra el archivo .tex para citar las referencias deseadas

\newpage
% Referencias
\setcounter{secnumdepth}{0} %permite enumerar las secciones QUITAR SI QUIERES OG APA7
\renewcommand\refname{\textbf{Referencias}}
\bibliography{referencias}

\end{document}