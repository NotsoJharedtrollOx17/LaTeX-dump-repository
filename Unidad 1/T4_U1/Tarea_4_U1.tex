\documentclass[letterpaper, 12pt]{article}
\usepackage[letterpaper, top=2.5cm, bottom=2.5cm, left=3cm, right=3cm]{geometry} %margenes
\usepackage[utf8]{inputenc} %manejo de caracteres especiales
\usepackage[spanish]{babel} %manejo de encabezados de inglés a español
\usepackage{fancyhdr} %formato de los encabezados de página
\usepackage{ragged2e} %alineado real justficado
\usepackage{graphicx} %manejo de imagenes
\usepackage{amsmath} %manejo de notación matemática
\usepackage{mathtools} %manejo de notación matemática
\usepackage{blindtext} %texto de relleno
\usepackage{cancel} %permite la simbolización de cancelación de terminos
\usepackage{enumitem}[shortlabels] %listas con letras
\usepackage{amssymb} %manejo de simbolog►1a matematica

\pagestyle{fancy}
\fancyhf{}
\rfoot{\thepage}

\begin{document}

\setcounter{page}{1}
\thispagestyle{fancy}
\lhead{\textbf{Tarea 4, U1}}
\rhead{\textbf{22/09/2020}}
\section{Vectores en el espacio}
\subsection*{Encuentra cuales de los sig. vectores son ortogonales (perpendiculares)}
\justify
\begin{itemize}
    \item \(\vec{v_1}=\,<\!2,0,1\!>\)
    \item \(\vec{v_2}=\,<\!3,2,-1\!>\)
    \item \(\vec{v_3}=\,<\!2,-1,-1\!>\)
    \item \(\vec{v_4}=\,<\!1,-4,6\!>\)
    \item \(\vec{v_5}=\,<\!1,-1,1\!>\)
    \item \(\vec{v_6}=\,<\!-4,3,8\!>\)
\end{itemize}
\subsection*{Cálculos:}
\justify
Para saber cuantas operaciones debemos realizar, procederemos a aplicar el coeficiente binomial debido a que los cálculos requerirán el uso de dos coeficientes.
\[\binom{n}{k} =\binom{6}{2} =15\]
Por lo tanto, serán quince operaciones a realizar para obtener la ortogonalidad de los vectores. Mostrado de otra manera:
\begin{center}
    \begin{tabular}{ c c c c c c }
    Fila \(1\):& \(\vec{v_1}\cdot\vec{v_2}\)& \(\vec{v_1}\cdot\vec{v_3}\)& \(\vec{v_1}\cdot\vec{v_4}\)& \(\vec{v_1}\cdot\vec{v_5}\)& \(\vec{v_1}\cdot\vec{v_6}\)\\
    Fila \(2\):& \(\vec{v_2}\cdot\vec{v_3}\)&\(\vec{v_2}\cdot\vec{v_4}\)&\(\vec{v_2}\cdot\vec{v_5}\)&\(\vec{v_2}\cdot\vec{v_6}\)&  \\
    Fila \(3\):& \(\vec{v_3}\cdot\vec{v_4}\)&\(\vec{v_3}\cdot\vec{v_5}\)&\(\vec{v_4}\cdot\vec{v_6}\)& & \\
    Fila \(4\):& \(\vec{v_4}\cdot\vec{v_5}\)&\(\vec{v_4}\cdot\vec{v_6}\)& & & \\
    Fila \(5\):& \(\vec{v_5}\cdot\vec{v_6}\)& & & &
    \end{tabular}
\end{center}
Con esto, es mas sencillo mantener orden de las operaciones a realizar. Donde:
\[\vec{u}\perp \vec{v} \leftrightarrow \left(\theta=\frac{\pi}{2} \because \vec{u}\cdot\vec{v}=0\right)\]
\subsubsection*{Fila 1:}
\justify
%vector i con vector 2
\(\vec{v_1} \text{ y }\vec{v_2}\):\\ \newline
\(\vec{v_1}\cdot\vec{v_2}=v_{1_1}v_{2_1}+v_{1_2}v_{2_2}+v_{1_3}v_{2_3}=2(3)+0(2)+1(-1)=6-1=5\therefore\vec{v_1}\not\perp \vec{v_2}\)\\ \newline
%vector 1 con vector 3
\(\vec{v_1} \text{ y }\vec{v_3}\):\\ \newline
\(\vec{v_1}\cdot\vec{v_3}=v_{1_1}v_{3_1}+v_{1_2}v_{3_2}+v_{1_3}v_{3_3}=2(2)+0(-1)+1(-1)=4-1=3\therefore\vec{v_1}\not\perp \vec{v_3}\)\\ \newline
%vector 1 con vector 4
\(\vec{v_1} \text{ y }\vec{v_4}\):\\ \newline
\(\vec{v_1}\cdot\vec{v_4}=v_{1_1}v_{4_1}+v_{1_2}v_{4_2}+v_{1_3}v_{4_3}=2(1)+0(-4)+1(6)=2+6=8\therefore\vec{v_1}\not\perp \vec{v_4}\)\\ \newline
%vector 1 con vector 5
\(\vec{v_1} \text{ y }\vec{v_5}\):\\ \newline
\(\vec{v_1}\cdot\vec{v_5}=v_{1_1}v_{5_1}+v_{1_2}v_{5_2}+v_{1_3}v_{5_3}=2(1)+0(-1)+1(1)=2+1=3\therefore\vec{v_1}\not\perp \vec{v_5}\)\\ \newline
%vector 1 con vector 6
\(\vec{v_1} \text{ y }\vec{v_6}\):\\ \newline
\(\vec{v_1}\cdot\vec{v_6}=v_{1_1}v_{6_1}+v_{1_2}v_{6_2}+v_{1_3}v_{6_3}=2(-4)+0(3)+1(8)=-8+8=0\therefore\vec{v_1}\perp \vec{v_6}\)
\subsubsection*{Fila 2:}
\justify
%vector 2 con vector 3
\(\vec{v_2} \text{ y }\vec{v_3}\):\\ \newline
\(\vec{v_2}\cdot\vec{v_3}=v_{2_1}v_{3_1}+v_{2_2}v_{3_2}+v_{2_3}v_{3_3}=3(2)+2(-1)+(-1)(-1)=9-2+1=8\therefore\vec{v_2}\not\perp \vec{v_3}\)\\ \newline
%vector 2 con vector 4
\(\vec{v_2} \text{ y }\vec{v_4}\):\\ \newline
\(\vec{v_2}\cdot\vec{v_4}=v_{2_1}v_{4_1}+v_{2_2}v_{4_2}+v_{2_3}v_{4_3}=3(1)+2(-4)+(-1)(6)=3-8-6=-11\therefore\vec{v_2}\not\perp \vec{v_4}\)\\ \newline
%vector 2 con vector 5
\(\vec{v_2} \text{ y }\vec{v_5}\):\\ \newline
\(\vec{v_2}\cdot\vec{v_5}=v_{2_1}v_{5_1}+v_{2_2}v_{5_2}+v_{2_3}v_{5_3}=3(1)+2(-1)+(-1)(1)=3-2-1=0\therefore\vec{v_2}\perp \vec{v_5}\)\\ \newline
%vector 2 con vector 6
\(\vec{v_2} \text{ y }\vec{v_6}\):\\ \newline
\(\vec{v_2}\cdot\vec{v_6}=v_{2_1}v_{6_1}+v_{2_2}v_{6_2}+v_{2_3}v_{6_3}=3(-4)+2(3)+(-1)(8)=-12+6-8=-14\therefore\vec{v_2}\not\perp \vec{v_6}\)
\subsubsection*{Fila 3:}
\justify
%vector 3 con vector 4
\(\vec{v_3} \text{ y }\vec{v_4}\):\\ \newline
\(\vec{v_3}\cdot\vec{v_4}=v_{3_1}v_{4_1}+v_{3_2}v_{4_2}+v_{3_3}v_{4_3}=2(1)+(-1)(-4)+(-1)(6)=2+4-6=0\therefore\vec{v_3}\perp \vec{v_4}\)\\ \newline
%vector 3 con vector 5
\(\vec{v_3} \text{ y }\vec{v_5}\):\\ \newline
\(\vec{v_3}\cdot\vec{v_5}=v_{3_1}v_{5_1}+v_{3_2}v_{5_2}+v_{3_3}v_{5_3}=2(1)+(-1)(-1)+(-1)(1)=2+1-1=2\therefore\vec{v_3}\not\perp \vec{v_5}\)\\ \newline
%vector 3 con vector 6
\(\vec{v_3} \text{ y }\vec{v_6}\):\\ \newline
\(\vec{v_3}\cdot\vec{v_6}=v_{3_1}v_{6_1}+v_{3_2}v_{6_2}+v_{3_3}v_{6_3}=2(-4)+(-1)(3)+(-1)(8)=-8-3-8=-19\therefore\vec{v_3}\not\perp \vec{v_6}\)
\subsubsection*{Fila 4:}
\justify
%vector 4 con vector 5
\(\vec{v_4} \text{ y }\vec{v_5}\):\\ \newline
\(\vec{v_4}\cdot\vec{v_5}=v_{4_1}v_{5_1}+v_{4_2}v_{5_2}+v_{4_3}v_{5_3}=1(1)+(-4)(-1)+6(1)=1+4+6=11\therefore\vec{v_4}\not\perp \vec{v_5}\)\\ \newline
%vector 4 con vector 6
\(\vec{v_4} \text{ y }\vec{v_6}\):\\ \newline
\(\vec{v_4}\cdot\vec{v_6}=v_{4_1}v_{6_1}+v_{4_2}v_{6_2}+v_{4_3}v_{6_3}=1(-4)+(-4)(3)+6(8)=-4-12+48=32\therefore\vec{v_4}\not\perp \vec{v_6}\)
\subsubsection*{Fila 5:}
\justify
%vector 5 con vector 6
\(\vec{v_5} \text{ y }\vec{v_6}\):\\ \newline
\(\vec{v_5}\cdot\vec{v_6}=v_{5_1}v_{6_1}+v_{5_2}v_{6_2}+v_{5_3}v_{6_3}=1(-4)+(-1)(3)+1(8)=-4-3+8=-1\therefore\vec{v_5}\not\perp \vec{v_6}\)\\ \newline
Recopilando los vectores ortogonales tenemos que:
\begin{itemize}
    \item \(\vec{v_1}\perp \vec{v_6}\).
    \item \(\vec{v_2}\perp \vec{v_5}\).
    \item \(\vec{v_3}\perp \vec{v_4}\).
\end{itemize}
Concluyendo la interrogante.
\end{document}