\documentclass[letterpaper, 12pt]{article}
\usepackage[letterpaper, top=2.5cm, bottom=2.5cm, left=3cm, right=3cm]{geometry} %margenes
\usepackage[utf8]{inputenc} %manejo de caracteres especiales
\usepackage[spanish]{babel} %manejo de encabezados de inglés a español
\usepackage{fancyhdr} %formato de los encabezados de página
\usepackage{ragged2e} %alineado real justficado
\usepackage{graphicx} %manejo de imagenes
\usepackage{amsmath} %manejo de notación matemática
\usepackage{mathtools} %manejo de notación matemática
\usepackage{blindtext} %texto de relleno
\usepackage{cancel} %permite la simbolización de cancelación de terminos
\usepackage{enumitem}[shortlabels] %listas con letras
\usepackage{amssymb} %manejo de simbología matematica
\usepackage{float}

\pagestyle{fancy}
\fancyhf{}
\rfoot{\thepage}

\begin{document}
\setcounter{page}{1}
\thispagestyle{fancy}
\lhead{\textbf{Tarea 2, U1}}
\rhead{\textbf{26 de febrero de 2021}}
\section*{Verifica soluciones de ecuaciones diferenciales}
\subsection*{Resolver lo siguiente:}
\subsubsection*{\emph{Problema 1:}}
Dada la siguiente ecuación diferencial:
\[y^\prime=xy^2\]
¿La función \(y=\frac{2}{9+x^2}\) es una solución de la ecuación anterior?
\\\newline
\textbf{• Solución:} 
\begin{equation*}
    \begin{aligned}
        \frac{dy}{dx}&=\frac{d}{dx}\frac{2}{9+x^2}=\underbrace{2\frac{d}{dx}(9+x^2)^{-1}}_{u=9+x^2\,\rightarrow\, du=2x}=2\left(\frac{d}{du}u^{-1}\cdot du\right)=2(-u^{-2}\cdot du)=\\
        -\frac{2du}{u^2}&=-\frac{4x}{(9+x^2)^2}
    \end{aligned}
\end{equation*}
\,\, Comprobación:
\begin{equation*}
    \begin{aligned}
        -\frac{4x}{(9+x^2)^2}&=x\left(\frac{2}{9+x^2}\right)^2 \\
        -\frac{4x}{(9+x^2)^2}&=x\cdot\frac{4}{(9+x^2)^2} \\
        -\frac{4x}{(9+x^2)^2}&=\frac{4x}{(9+x^2)^2} \\
    \end{aligned}
\end{equation*}
\\
Por lo anterior \textbf{\(y\) no es una solución a la ecuación diferencial.}
\subsubsection*{\emph{Problema 2:}}
Dada la siguiente ecuación diferencial:
\[y^\prime=2x+3y-1\]
¿La función \(y=-\frac{2}{3}x+\frac{1}{9}\) es una solución de la ecuación anterior?
\\\newline
\textbf{• Solución:}
\begin{equation*}
    \begin{aligned}
        \frac{dy}{dx}=\frac{d}{dx}\left(-\frac{2}{3}x+\frac{1}{9}\right)=-\frac{2}{3}
    \end{aligned}
\end{equation*}
\,\, Comprobación:
\begin{equation*}
    \begin{aligned}
        -\frac{2}{3}&=2x+3\left(-\frac{2}{3}x+\frac{1}{9}\right)-1\\
        -\frac{2}{3}&=2x-2x+\frac{1}{3}-1\\
        -\frac{2}{3}&=-\frac{2}{3}
    \end{aligned}
\end{equation*}
\\
Por lo anterior \textbf{\(y\) si es una solución a la ecuación diferencial.}
\subsubsection*{\emph{Problema 3:}}
Dada la siguiente ecuación diferencial:
\[y^\prime=5y\]
¿La función \(y=3e^{x^5}\) es una solución de la ecuación anterior?
\\\newline
\textbf{• Solución:}
\begin{equation*}
    \begin{aligned}
        \frac{dy}{dx}&=\underbrace{\frac{d}{dx}3e^{x^5}}_{u=x^5\,\rightarrow\, du=5x^4}
        =3\left(\frac{d}{du}e^u\cdot du\right)=3\left(du\cdot e^u\right)=3\left(5e^{x^4}\right)=15e^{x^4}
    \end{aligned}
\end{equation*}
\,\, Comprobación:
\begin{equation*}
    \begin{aligned}
        15e^{x^4}&=5\left(3e^{x^5}\right)\\
        15e^{x^4}&=15e^{x^5}
    \end{aligned}
\end{equation*}
\\
Por lo anterior \textbf{\(y\) no es una solución a la ecuación diferencial.}
\subsubsection*{\emph{Problema 4:}}
Dada la siguiente ecuación diferencial:
\[\frac{dy}{dx}=\frac{5y}{x}\]
¿La función \(y=-2x^5\) es una solución de la ecuación anterior?
\\\newline
\textbf{• Solución:}
\begin{equation*}
    \begin{aligned}
        \frac{dy}{dx}=\frac{d}{dx}-2x^5=-2\frac{d}{dx}x^5=-2\left(5x^4\right)=-10x^4
    \end{aligned}
\end{equation*}
\,\, Comprobación:
\begin{equation*}
    \begin{aligned}
        -10x^4&=\frac{5\left(-2x^5\right)}{x}\\
        -10x^4&=\frac{-10x^5}{x}\\
        -10x^4&=-10x^4
    \end{aligned}
\end{equation*}
\\
Por lo anterior \textbf{\(y\) si es una solución a la ecuación diferencial.}
\end{document}