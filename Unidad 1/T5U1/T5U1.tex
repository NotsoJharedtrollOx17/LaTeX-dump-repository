\documentclass[letterpaper, 12pt]{article}
\usepackage[letterpaper, top=2.5cm, bottom=2.5cm, left=3cm, right=3cm]{geometry} %margenes
\usepackage[utf8]{inputenc} %manejo de caracteres especiales
\usepackage[spanish]{babel} %manejo de encabezados de inglés a español
\usepackage{fancyhdr} %formato de los encabezados de página
\usepackage{ragged2e} %alineado real justficado
\usepackage{graphicx} %manejo de imagenes
\usepackage{amsmath} %manejo de notación matemática
\usepackage{mathtools} %manejo de notación matemática
\usepackage{blindtext} %texto de relleno
\usepackage{cancel} %permite la simbolización de cancelación de terminos
\usepackage{enumitem}[shortlabels] %listas con letras
\usepackage{amssymb} %manejo de simbología matematica
\usepackage{float}

\pagestyle{fancy}
\fancyhf{}
\rfoot{}

\begin{document}
\thispagestyle{fancy}
\lhead{\textbf{Tarea 5, U1}}
\rhead{\textbf{5 de marzo de 2021}}
\section*{Ecuaciones Diferenciales Homogéneas de grado n}
\subsection*{Resolver las siguientes Ecuaciones Diferenciales}
\subsubsection*{\emph{Problema 1:}}
\justify
\[\frac{dy}{dx}=\frac{y}{x}+\csc \frac{y}{x}\]
\textbf{ • Solución:}
\begin{equation*}
    \begin{aligned}
        \frac{dy}{dx}&=\frac{y}{x}+\csc \frac{y}{x}\\[5pt]
        \frac{dy}{dx}&=u + \csc u\\[5pt]
        u+\csc u&=u+x\, \frac{du}{dx}\\[5pt]
        \csc u&=x\, \frac{du}{dx}\\[5pt]
        \frac{dx}{x}&= \frac{du}{\csc u}\\[5pt]
        \int \frac{dx}{x}&=\int \frac{du}{\csc u}\\[5pt]
        \int \frac{dx}{x}&=\int \frac{\frac{du}{1}}{\frac{1}{\sin u}}\\[5pt]
        \int \frac{dx}{x}&=\int \sin u\, du\\[5pt]
        \ln |x| + c&=-\cos u\\[5pt]
                  c&=-\cos \frac{y}{x} - \ln |x|\\[5pt]
                  c&=\cos \frac{y}{x} + \ln |x|\: \textbf{(1.1)}
    \end{aligned}
\end{equation*}
Por lo que \textbf{(1.1)} es la respuesta a nuestra ecuación diferencial.
\newpage
\subsubsection*{\emph{Problema 2:}}
\justify
\[x\left(\ln \frac{x}{y} + 1\right)\, dy - y\, dx=0\]
\textbf{ • Solución:}
    \begin{equation*}
        \begin{aligned}
            x\left(\ln \frac{x}{y} + 1\right)\, dy - y\, dx&=0\\[5pt]
            x\left(\ln \frac{x}{y} + 1\right)\, dy&=y\, dx\\[5pt]
            \frac{x\left(\ln \frac{x}{y} + 1\right)}{y}&=\frac{dx}{dy}\\[5pt]
            \frac{y\cdot \frac{x}{y}\left(\ln \frac{x}{y} + 1\right)}{y\, (1)}&=\\[5pt]
            v\left(\ln v + 1\right)&=\\[5pt]
            v\left(\ln v + 1\right)&=v+y\, \frac{dv}{dy}\\[5pt]
            v\ln v + v &= v+y\, \frac{dv}{dy}\\[5pt]
            v\ln v&=y\, \frac{dv}{dy}\\[5pt]
            \frac{dy}{y}&=\frac{dv}{v\ln v}\\[5pt]
            \int\frac{dy}{y}&=\int\frac{dv}{v\ln v}\\[5pt]
            \int\frac{dy}{y}&=\underbrace{\int\frac{d\beta}{\beta}}_{\beta=\ln v\,\rightarrow\, d\beta=\frac{dv}{v}}\\[5pt]
            \ln|y|+c&=\ln|\beta|\\[5pt]
            \ln|y|+c&=\ln|\ln|v||\\[5pt]
        \end{aligned}
    \end{equation*}
    \begin{equation*}
        \begin{aligned}
            c&=\ln|\ln|v||-\ln|y|\\[5pt]
            e^c&=e^{\ln|\ln|v||-\ln|y|}\\[5pt]
            c&=\frac{e^{\ln|\ln|v||}}{e^{\ln|y|}}\\[5pt]
            c&=\frac{\ln|v|}{y}\\[5pt]
            e^c&=e^{\frac{\ln|v|}{y}}\\[5pt]
            c&=\left(e^{\ln|v|}\right)^{\frac{1}{y}}\\[5pt]
            c&=\left(v\right)^{\frac{1}{y}}\\[5pt]
            c&=\sqrt[y]{\frac{x}{y}}\: \textbf{(1.2)}
        \end{aligned}
    \end{equation*}
    Por lo que \textbf{(1.2)} es la respuesta a nuestra ecuación diferencial.
\end{document}