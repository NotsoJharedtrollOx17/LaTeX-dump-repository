\documentclass[letterpaper, 12pt]{article}
\usepackage[letterpaper, top=2.5cm, bottom=2.5cm, left=3cm, right=3cm]{geometry} %margenes
\usepackage[utf8]{inputenc} %manejo de caracteres especiales
\usepackage[spanish]{babel} %manejo de encabezados de inglés a español
\usepackage{fancyhdr} %formato de los encabezados de página
\usepackage{ragged2e} %alineado real justficado
\usepackage{graphicx} %manejo de imagenes
\usepackage{amsmath} %manejo de notación matemática
\usepackage{mathtools} %manejo de notación matemática
\usepackage{blindtext} %texto de relleno
\usepackage{cancel} %permite la simbolización de cancelación de terminos
\usepackage{enumitem}[shortlabels] %listas con letras
\usepackage{amssymb} %manejo de simbología matematica
\usepackage{float}

\pagestyle{fancy}
\fancyhf{}
\rfoot{}

\begin{document}
\thispagestyle{fancy}
\lhead{\textbf{Tarea 6, U1}}
\rhead{\textbf{10 de marzo de 2021}}
\section*{Ecuaciones Diferenciales Exáctas}
\subsection*{Resolver la siguiente Ecuación Diferencial Exacta usando \(\frac{\partial f}{\partial x}=M(x,y)\):}
\justify
\[\left(2x+\frac{y}{x}\right)\, dx+\left(4y+\ln x\right)\, dy=0\]
{\large \textbf{ • Solución:}
\begin{equation*}
    \begin{aligned}
        \frac{\partial f}{\partial x}&=M(x,y)\\[5pt]
        \frac{\partial f}{\partial x}&=2x+\frac{y}{x}\\[5pt]
        \partial f&=\left(2x+\frac{y}{x}\right)\, dx\\[5pt]
        \int \partial f&=\int \left(2x+\frac{y}{x}\right)\, \partial x\\[5pt]
        f&=x^2+y\ln x+c(y)\\[5pt]
        \frac{\partial f}{\partial y}&=\frac{\partial}{\partial y} \left(x^2+y\ln x+c(y)\right)\\[5pt]
        &=\ln x+c^{\prime}\\[5pt]
        \underbrace{\ln x+c^{\prime}(y)}_{\frac{\partial M}{\partial y}}&=\underbrace{4y+\ln x}_{\frac{\partial N}{\partial x}}\\[5pt]
        c^{\prime}(y)&=4y\\[5pt]
        \int c^{\prime}(y)&=\int 4y\\[5pt]
        c(y)&=2y^2+c\therefore\\[5pt]
        \therefore f:= x^2+y\ln x+c(y)&=x^2+y\ln x+2y^2+c\\[5pt]
        c&=2y^2+y\ln x+x^2\: \textbf{(1.1)}\\[10pt]
    \end{aligned}
\end{equation*}
Por lo que \textbf{(1.1)} es la respuesta a nuestra ecuación diferencial.}
\end{document}