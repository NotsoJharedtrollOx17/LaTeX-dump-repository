% Preámbulo
\documentclass[stu, 12pt, letterpaper, donotrepeattitle, floatsintext, natbib]{apa7}
\usepackage[utf8]{inputenc}
\usepackage{comment}
\usepackage{marvosym}
\usepackage{graphicx}
\usepackage{float}
\usepackage[normalem]{ulem}
\usepackage[spanish]{babel} 
\usepackage{lastpage} %para le formato que quiere la profe QUITAR SI QUIERES OG APA7
\usepackage{ragged2e} %para le formato que quiere la profe QUITAR SI QUIERES OG APA7
\usepackage{indentfirst} %para le formato que quiere la profe QUITAR SI QUIERES OG APA7

\selectlanguage{spanish}
\useunder{\uline}{\ul}{}
\newcommand{\myparagraph}[1]{\paragraph{#1}\mbox{}\\}

\rfoot{Página \thepage \hspace{1pt} de \pageref{LastPage}}%QUITAR SI QUIERES OG APA7 
\rhead{} %QUITAR SI QUIERES OG APA7
\setcounter{secnumdepth}{3} %permite enumerar las secciones QUITAR SI QUIERES OG APA7
\setlength{\parindent}{1.27cm} %sangria forzada QUITAR SI QUIERES OG APA7

% Portada
\thispagestyle{empty}
\title{\Large Los aguacates de Portugal: ¿Oro verde o riesgo para el medioambiente?}
\author{Abraham Jhared Flores Azcona} % (autores separados, consultar al docente)
% Manera oficial de colocar los autores:
%\author{Autor(a) I, Autor(a) II, Autor(a) III, Autor(a) X}
\affiliation{Instituto Tecnológico de Tijuana}
\course{ACD-0908SC5C Desarrollo Sustentable}
\professor{M.C. Trinidad Castro Villa}
\duedate{7 de septiembre de 2021}

\begin{document}
\maketitle


% Índices
\pagenumbering{arabic}
    % Contenido
\renewcommand\contentsname{Contenido}
\tableofcontents

% Cuerpo 
    %NOTA: PARA CITAR ESTILO "Merts (2003)" usar \cite{<nombre_cita_bib>}
    %                        "(Metz, 1978)" usar \citep{<nombre_cita_bib>}
\newpage
\section{Introducción}
Uno \begin{justifying}
de los alimentos más populares de la última década es la del aguacate. Este alimento se caracteriza por ser una planta típica de las zonas tropicales, como lo es México
y el resto de Sudamérica, pero principalmente que requiere mucha agua para subsistir, sin embargo la agricultura intensiva ha propiciado su cultivo en  zonas
geográficas no nativas del alimento, como lo es la nación de Portugal lo que ha creado una inestabilidad de irrigación \citep{dw-2021}.\par
\vspace{\baselineskip}
En este resúmen se destacan los puntos relevantes, tales como la tendencia de dicho estilo agricultor así como un meta-análisis de conjenturas que van mas allá del Desarrollo 
Sustentable para generar una discusión más enriquezida del tema dentro de la materia.\par
\vspace{\baselineskip}
\end{justifying}

\section{Resúmen general}
Como \begin{justifying}
se había mencionado en la Introducción, el video a resumir se centra principalmente en las problematicas de la agricultura intensiva de aguacate en la nación de Portugal, donde el sector agricultor
ha invertido bastante en la crianza de dicho alimento. Como el aguacate es un alimento que requiere primordialmente de una irrigación constante, los patrones de las parcelas de producción intensiva
han optado por usar las reservas acuíferas disponibles así como otras soluciones competentes a la domótica (una de estas soluciones es la de irrigación controlada por computadora).\par
\vspace{\baselineskip}
Aun así, las quejas de las personas aledañas y del público enterado destacan el uso de la irrigación para la agricultura ``industrial'' como el principal factor del deterioro de la fertileza de las zonas rurales. En la parte gubernamental,
las instituciones portuguesas a cargo del resguardo del medio ambiente claman que reconocen la situación, sin embargo su poder actuar está limitado a dar los permisos regulatorios competentes para la irrigación.
Retomando la postura de los patrones de esta agricultura intensiva, estos mantienen firmemente que en primer lugar generar fuentes de trabajo para la nación, crecen la economía y por ende, mantienen bajos los precios
del aguacate a pesar de que dentro del documental se muestra que usan vacios legales dentro de la legislación europea pertinente en cuestión de inmigración y empleo para reclutar a los inmigrantes.
Los científicos que aparecen realizan la toma de muestras en la zona rural donde se grabó el documental para comprobar si el deterioro de la zona está relacionado por las parcelas de irrigación intensiva, lo que terminan
refutando y mostrando a las personas afectadas por los patrones de parcelas.\par
\vspace{\baselineskip}
\end{justifying}
\subsection{De la agricultura intensiva y la distribución del agua}  
Como \begin{justifying}
    se menciona dentro del video, el crecimiento de este estilo agricultor recae principalmente en las tendencias de comida que surgen a causa de las modas de internet. Portugal no es el único país en aplicarlo, E.U.A es un claro ejemplo
    y especificamente en el estado de California donde también se cultiva el aguacate. \par 
    \vspace{\baselineskip}
    California se caracteriza por ser una zona geográfica relativamente desertica y, por ende muy dificil de abastecerce de agua; su orígen
    proviene de otros puntos del país como el Río Colorado dentro del Gran Cañon \citep{california-department-of-water-resources-no-date} que nos permite inferir que la distribución del agua hace a la agricultura intensiva un estilo difícil en aspectos logísticos,
    no asequible e insostenible para California. Se puede decir que la situación dentro de Portugal pasa por cuestiones similares.\par
    \vspace{\baselineskip}
\end{justifying}
\subsection{De los aspectos económicos}
Esto \begin{justifying}
    se menciona brevemente en el documental. El argumento de los patrones de parcelas para la agricultura intensiva es que este ``boom'' de la popularidad del aguacate les ha dado la oportunidad de aprovechar dicha demanda y poder satisfacerla a tal grado
    de escalar la producción hasta un nivel global que en consecuencia genera el necesitar de mano de obra para poder satisfacer con eficiencia la demanda; generado también mas empleos, un ajuste reductivo a los precios de dicho bien y generando ganacias financieras
    a las partes productoras.\par
    \vspace{\baselineskip}
    Aunque lo mencionado es totalmente positivo para los involucrados, el coste ambiental para mantener dicha economía es muy alto. El simple hecho de usar tanta agua para una planta que no es nativa de la zona en la cual se esté plantando, genera desbalances
    dentro de los ecosistemas \citep{vega-2020} ya que como son sistemas dinámicos complejos, estos buscaran compensarse hasta llegar a un equilibrio \citep{maayan-2017} que no necesariamente sea el escenario mas favorable para los involucrados.\par
    \vspace{\baselineskip}
\end{justifying}
\subsection{De los aspectos morales}
Un \begin{justifying}
   tema que no se argumenta explicitamente que nos compete en aspectos morales es de las tendencias capitalistas y consumistas que al fín y al cabo, generan tendencias de comida como la del aguacate. Es de suma importancia remarcar que el consumismo
   se discute en la sección de comentarios del video adjunto. En consenso, se coincide que el consumo y producción no generan problemas en las regiones del planeta en donde el alimento en cuestión se pueda desarrollar sin mucho problema, como es el caso de México.\par
   \vspace{\baselineskip}
   En lo que respecta de la producción y consumo excesivo del aguacate en otras regiones, la opinión de esta tribuna pública tiende a destacar el consumismo como el culpable de los aspectos mencionados dentro del documental, específicamente en las zonas culturalmente asociadas con el Occidente
   ya que terminan incentivando (en terminos económicos) los estilos de agricultura expuestos en el video.
   Esto es importante para la discusión del documental ya que la visión de ideales para la cultura occidental recae bastante en la exaltación del individuo y su trascendencia del dejar de ser ``uno más del montón'' que incentiva a las personas a enfocarse en el crecimiento, desarrollo y protección individuales, así como el 
   subir la escalera económica mientras que en las civilizaciones de Oriente historicamente se le ha dado un peso moral mayor al poder contribuir a tu entorno de un manera desinteresada \citep{unknown-author-2020}, incluyendo al cuidado del ambiente y procurar un desarrollo sustentable (en términos modernos).\par
\end{justifying}
\vspace{\baselineskip}
\section{Conclusión}
Como \begin{justifying}
   se ha estado presentando en los últimos años, las distintas tendencias de consumo permiten abrir puertas económicas nunca antes vistas, sin embargo cuando dichas puertas requieren de una alta manutención de recursos naturales, es de suma importancia el analizar los prospectos 
   para poder aplicar en todos los rubros posibles la desición que sea lo más bénefica a las partes involucradas.\par
\end{justifying}

\newpage
% Referencias
\setcounter{secnumdepth}{0} %permite enumerar las secciones QUITAR SI QUIERES OG APA7
\renewcommand\refname{\textbf{Referencias}}
\bibliography{referencias}

\end{document}