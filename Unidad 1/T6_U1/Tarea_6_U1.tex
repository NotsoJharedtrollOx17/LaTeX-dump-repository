\documentclass[letterpaper, 12pt]{article}
\usepackage[letterpaper, top=2.5cm, bottom=2.5cm, left=3cm, right=3cm]{geometry} %margenes
\usepackage[utf8]{inputenc} %manejo de caracteres especiales
\usepackage[spanish]{babel} %manejo de encabezados de inglés a español
\usepackage{fancyhdr} %formato de los encabezados de página
\usepackage{ragged2e} %alineado real justficado
\usepackage{graphicx} %manejo de imagenes
\usepackage{amsmath} %manejo de notación matemática
\usepackage{mathtools} %manejo de notación matemática
\usepackage{blindtext} %texto de relleno
\usepackage{cancel} %permite la simbolización de cancelación de terminos
\usepackage{enumitem}[shortlabels] %listas con letras
\usepackage{amssymb} %manejo de simbolog►1a matematica

\pagestyle{fancy}
\fancyhf{}
\rfoot{\thepage}

\begin{document}

\setcounter{page}{1}
\thispagestyle{fancy}
\lhead{\textbf{Tarea 6, U1}}
\rhead{\textbf{24/09/2020}}
\section{Vectores en el espacio}
\subsection*{Dados \(\vec{u}=\hat{i}-\hat{j}+2\hat{k}\) y \(\vec{v}=2\hat{i}+\hat{j}+\hat{k}\) obtener:}
\begin{itemize}
    \item \(\vec{u}\times \vec{v}\)
    \item \(\vec{v}\times \vec{u}\)
    \item \(A_{P\,\vec{u},\vec{v}}\)
    \item \(A_{T\,\vec{u},\vec{v}}\)
\end{itemize}
\subsection*{Cálculos:}
Para proceder a realizar los mismos se usará lo siguiente:\\
\[\vec{u}\times \vec{v}=\text{det}\begin{pmatrix} %producto cruz de u y v
    \hat{i} &\hat{j}& \hat{k}\\
    \vec{u}_x &\vec{u}_y&\vec{u}_z\\
    \vec{v}_x&\vec{v}_y&\vec{v}_z\\
\end{pmatrix}=\hat{i}\begin{pmatrix}
    \vec{u}_y&\vec{u}_z\\
    \vec{v}_y&\vec{v}_z\\
\end{pmatrix}-\hat{j}\begin{pmatrix}
    \vec{u}_x&\vec{u}_z\\
    \vec{v}_x&\vec{v}_z\\
\end{pmatrix}+\hat{k}\begin{pmatrix}
    \vec{u}_x&\vec{u}_y\\
    \vec{v}_x&\vec{v}_y\\
\end{pmatrix}\]
\[A_{P\,\vec{u},\vec{v}}=\lVert\vec{u}\times \vec{v}\rVert\]
\[A_{T\,\vec{u},\vec{v}}=\frac{A_{P\,\vec{u},\vec{v}}}{2}\]
\(\vec{u}\times \vec{v}\):\\ \newline
\(\text{det}\begin{pmatrix}
    \hat{i} &\hat{j}& \hat{k}\\
            1&-1&2\\
            2&1&1\\
\end{pmatrix}=\hat{i}\begin{pmatrix}
    -1&2\\
    1&1\\
\end{pmatrix}-\hat{j}\begin{pmatrix}
    1&2\\
    2&1\\
\end{pmatrix}+\hat{k}\begin{pmatrix}
    1&-1\\
    2&1
\end{pmatrix}=\hat{i}(-1(1)-1(2))-\)\\ \newline
\(\hat{j}(1(1)-2(2))+\hat{k}(1(1)-2(-1))=\hat{i}(-1-2)-\hat{j}(1-4)+\hat{k}(1+2)=-3\hat{i}+3\hat{j}+3\hat{k}\) \\ \newline
\(\vec{v}\times \vec{u}\):\\ \newline %producto cruz de v y u
\(\text{det}\begin{pmatrix}
    \hat{i} &\hat{j}& \hat{k}\\
           2&1&1\\ 
           1&-1&2\\ 
\end{pmatrix}=\hat{i}\begin{pmatrix}
   1&1\\ 
-1&2\\
\end{pmatrix}-\hat{j}\begin{pmatrix}
  2&1\\  
  1&2\\  
\end{pmatrix}+\hat{k}\begin{pmatrix}
  2&1\\  
  1&-1\\  
\end{pmatrix}=\hat{i}(1(2)--1(1))-\)\\ \newline
\(\hat{j}(2(2)-1(1))+\hat{k}(2(-1)-1(1))=\hat{i}(2+1)-\hat{j}(4-1)+\hat{k}(-2-1)=3\hat{i}-3\hat{j}-3\hat{k}\) \\ \newline
\(A_{P\,\vec{u},\vec{v}}\):\\ \newline
\(\lVert\vec{u}\times \vec{v}\rVert=\sqrt{(-3)^2+3^2+3^2}=\sqrt{3(9)}=\sqrt{27}\) \\ \newline
\(A_{T\,\vec{u},\vec{v}}\): \\ \newline
\(\frac{A_{P\,\vec{u},\vec{v}}}{2}=\frac{\sqrt{27}}{2}\) \\ \newline
Dando por concluido las interrogantes.
\end{document}