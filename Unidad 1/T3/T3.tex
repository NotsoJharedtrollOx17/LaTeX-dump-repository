% Preámbulo
\documentclass[stu, 12pt, letterpaper, donotrepeattitle, floatsintext, natbib]{apa7}
\usepackage[utf8]{inputenc}
\usepackage{comment}
\usepackage{marvosym}
\usepackage{graphicx}
\usepackage{float}
\usepackage[normalem]{ulem}
\usepackage[spanish]{babel} 
\usepackage{lastpage} %para le formato que quiere la profe QUITAR SI QUIERES OG APA7
\usepackage{ragged2e} %para le formato que quiere la profe QUITAR SI QUIERES OG APA7
\usepackage{indentfirst} %para le formato que quiere la profe QUITAR SI QUIERES OG APA7

\selectlanguage{spanish}
\useunder{\uline}{\ul}{}
\newcommand{\myparagraph}[1]{\paragraph{#1}\mbox{}\\}

\rfoot{Página \thepage \hspace{1pt} de \pageref{LastPage}}%QUITAR SI QUIERES OG APA7 
\rhead{} %QUITAR SI QUIERES OG APA7
\setcounter{secnumdepth}{3} %permite enumerar las secciones QUITAR SI QUIERES OG APA7
\setlength{\parindent}{1.27cm} %sangria forzada QUITAR SI QUIERES OG APA7

% Portada
\thispagestyle{empty}
\title{\Large Valores, misión, visión y reglamento del ITT}
\author{Abraham Jhared Flores Azcona} % (autores separados, consultar al docente)
% Manera oficial de colocar los autores:
%\author{Autor(a) I, Autor(a) II, Autor(a) III, Autor(a) X}
\affiliation{Instituto Tecnológico de Tijuana}
\course{ACD-0908SC5C Desarrollo Sustentable}
\professor{M.C. Trinidad Castro Villa}
\duedate{31 de agosto de 2021}

\begin{document}
\maketitle


% Índices
\pagenumbering{arabic}
    % Contenido
\renewcommand\contentsname{Contenido}
\tableofcontents

% Cuerpo 
    %NOTA: PARA CITAR ESTILO "Merts (2003)" usar \cite{<nombre_cita_bib>}
    %                        "(Metz, 1978)" usar \citep{<nombre_cita_bib>}
\newpage
\section{Introducción}
En \begin{justifying}
    esta breve redacción, se desarrollan los aspectos relevantes que competen el reglamento escolar para alúmnos 
    del Instituto Tecnológico de Tijuana; estos incluyen la misión, visión, valores y reglas para el alumnado.\par
\vspace{\baselineskip}
\end{justifying}

\section{Valores}
Para \begin{justifying}
    cualquier organización es necesario identificar estos principios o valores en los cuales desea que se comprometa y responsabilize.
    Estos forman la fundación para un conjunto de compromisos éticos y acercamiento a la responsabilidad de dicha organización \citep{institute-of-business-ethics-no-date}.\par
    \vspace{\baselineskip}
    Teniendo esto en claro, entonces los valores descritos dentro del reglamento del ITT son listados con el afán de ser el profesionista ideal \citep{subdireccion-academica-itt-2020}. Es de remarque 
    que el valor de ``Pertenencia'' se haya listado como el primer valor. Como se redactó en la tarea anterior, una sociedad no existe si quienes la conforman no tienen
    ese sentido de pertenencia, impidiendo su integración plena ante sus allegados, y entorpecer su propia identidad por los conflictos de la búsqueda del ser y qué lo identifica.\par
    \vspace{\baselineskip}
    Agregando a lo anterior, un punto igual de importante (si no es que vitál) para los estudiantes de la institución es que dentro de su profesionalismo ellos mismos deben de mantenerse actualizados constantemente,
    que formalmente se conoce como el estar a la vanguardia. Como bién se sabe, la teconología avanza exponencialmente y estar a la vanguardia permite mantener ideas recientes en la mesa e incentivando la investigación
    para mantenerse competente en el ámbito laboral, y si se puede mencionar, para seguir continuando la búsqueda eterna por la sabiduría.\par
    \vspace{\baselineskip}
\end{justifying} 

\section{Misión}
En \begin{justifying}
    lo que se refiere la misión, se entiende que es la explicación concisa de la razón de existir de una organización. Describe su propósito y su intención general.
Este apoya a la visión y sirve para comunicar propósito y dirección a las partes interesadas del ente \citep{unknown-author-no-date}.\par
\vspace{\baselineskip}
En lo competente a la misión descrita dentro del reglamento, la misión del ITT es formar a profesionistas dignos de llamarse como tal \citep{subdireccion-academica-itt-2020}. Lo curioso es que se remarca el alto sentido de responsabilidad social,
que en el rubro de la materia de Desarrollo Sustentable es lo necesario para hacer del mundo, un lugar mejor para si mismo, sus colegas, conciudadanos y los que vendrán después de él.\par
\vspace{\baselineskip}
\end{justifying}

\section{Visión}
Este \begin{justifying}
    es algo vago de definír en términos concretos. A grandes rasgos es una declaración que mira adelante y crea una imágen mental del estado ideal de lo que dicha organización
desea lograr. Es inspiracional y aspiracional y debe retar a quienes es dirigido \citep{corporate-finance-institute-2020}.\par
\vspace{\baselineskip}
La visión del instituto básicamente quiere decirnos que el ITT desea ser la punta de lanza en los ámbitos tecnológico/científico en el noroeste del país, y posiblemente del mundo \citep{subdireccion-academica-itt-2020}.\par
\vspace{\baselineskip}
\end{justifying}

\section{Reglamento}
Son \begin{justifying}
las normas que desarrollan a otras de un rango superior, en otras palabras, de la ley. Permiten regular un sector en concreto y delimitan las normas que las personas involucradas deben seguir
con el afán de organizar un ámbito social o empresarial. También debe ser específico e imparcial y nunca debe estar arriba de las legislaciones judiciales vigentes ni de la Consistución aplicables \citep{trujillo-2021}.\par
\vspace{\baselineskip}\end{justifying} 
\subsection{Requisistos de Ingreso}
En 
\begin{justifying}
este apartado simplemente nos dice que para poder aplicar a una licenciatura en el ITT debes tener certificado de nivel bachillerato; presentar y aprobar el exámen de admisión, entre otros.\par    
\end{justifying}
\vspace{\baselineskip}
\subsection{De La Inscripción}
Se\begin{justifying}
estipula que se dara el número de control, y debe de cumplir con el papeleo correspondiente; cumplído esto, el sistema lo reconoce como alumno inscrito.\par
\vspace{\baselineskip}
Aparte de lo anterior, se estableces los minimos y máximos de creditos para cada semestre, y las penalizaciones a los alumnos irregulares por su rendimiento académico (deber materias).\par
\vspace{\baselineskip}
\end{justifying} 
\subsection{De Los Derechos y Obligaciones}
En \begin{justifying}
    general, que la enseña del estuadiante sea con una convivencia armónico y pacífica, atendiendole en lo que la escuela y sus trabajadores le puedan ofrecer. Algo muy curioso es que es derecho del estudiante es recibir
condecoraciones si su rendimiento academico es ejemplar.\par
\vspace{\baselineskip}
Entre sus obligaciones, principalmente se estipula que todo derecho que el educando ejerza (mientras sea alumno inscrito acorde a la sección anterior) debe de ser procurando el cómo hablar, y sin el objetivo
de hacer un daño a sí mismo como al resto de la comunidad escolar, dentro y fuera de la escuela.\par
\end{justifying}
\vspace{\baselineskip}
\subsection{De La Disciplina Escolar}
Básicamente, \begin{justifying}
    que se hace despues de que haya faltado a lo estipulado en el reglamento. Algo importante de remarcar es que el criterio de dichas snaciones se podrá imponer acorde al criterio del Director. Mientras más severa la falta (llegando hasta el punto de poder ser 
procesada ante la Ley), más severo es el castigo; siendo la sanción más severa la baja definitiva del plantel y del resto de sistema estudiantil del TNM.\par
\end{justifying}
\vspace{\baselineskip}
\subsection{Del Servicio Médico y Seguro Colectivo}
Finalmente, \begin{justifying}
    esta sección estipula que mientras sea estudiante acorde a lo estipulado en el reglamento, él mismo aplica a la aseguranza médica incorporada al IMSS.\par
\end{justifying}

\newpage
% Referencias
\setcounter{secnumdepth}{0} %permite enumerar las secciones QUITAR SI QUIERES OG APA7
\renewcommand\refname{\textbf{Referencias}}
\bibliography{referencias}

\end{document}