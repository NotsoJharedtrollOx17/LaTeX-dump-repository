\documentclass[letterpaper, 12pt]{article}
\usepackage[letterpaper, top=2.5cm, bottom=2.5cm, left=3cm, right=3cm]{geometry} %margenes
\usepackage[utf8]{inputenc} %manejo de caracteres especiales
\usepackage[spanish]{babel} %manejo de encabezados de inglés a español
\usepackage{fancyhdr} %formato de los encabezados de página
\usepackage{ragged2e} %alineado real justficado
\usepackage{graphicx} %manejo de imagenes
\usepackage{amsmath} %manejo de notación matemática
\usepackage{mathtools} %manejo de notación matemática
\usepackage{blindtext} %texto de relleno
\usepackage{cancel} %permite la simbolización de cancelación de terminos
\usepackage{enumitem}[shortlabels] %listas con letras
\usepackage{amssymb} %manejo de simbología matematica
\usepackage{float}

\pagestyle{fancy}
\fancyhf{}
\rfoot{}

\begin{document}
\thispagestyle{fancy}
\lhead{\textbf{Tarea 8, U1}}
\rhead{\textbf{17 de marzo de 2021}}
\section*{Ecuaciones Diferenciales Lineales}
\subsection*{Terminar de resolver la siguiente Ecuación Diferencial:}
\justify
\[x^3y=\int x\ln x\, dx\]
{\large \textbf{ • Solución:}
\begin{equation*}
    \begin{aligned}
        x^3y&=\underbrace{\int x\ln x\, dx}_{A}\\[5pt]
        A:= \int x\ln x\, dx&=\underbrace{\frac{x^2\ln x}{2}}_{uv}-\int \underbrace{\frac{x^2}{2}\cdot \frac{1}{x}\, dx}_{vdu}=\\[5pt]
        =\frac{x^2\ln x}{2}-\frac{1}{2}\int x\, dx&=\frac{x^2\ln x}{2}-\frac{x^2}{4}+c\therefore\\[5pt]
        \therefore x^3y&=\frac{x^2\ln x}{2}-\frac{x^2}{4}+c\\[5pt]
        y&=\frac{\ln x}{2x}-\frac{1}{4x}+\frac{c}{x^3}\: \textbf{(1.1)}
    \end{aligned}
\end{equation*}
Por lo que \textbf{(1.1)} es la respuesta a nuestra ecuación diferencial.}
\end{document}