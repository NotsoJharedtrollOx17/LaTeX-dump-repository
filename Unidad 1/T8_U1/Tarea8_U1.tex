\documentclass[letterpaper, 12pt]{article}
\usepackage[letterpaper, top=2.5cm, bottom=2.5cm, left=3cm, right=3cm]{geometry} %margenes
\usepackage[utf8]{inputenc} %manejo de caracteres especiales
\usepackage[spanish]{babel} %manejo de encabezados de inglés a español
\usepackage{fancyhdr} %formato de los encabezados de página
\usepackage{ragged2e} %alineado real justficado
\usepackage{graphicx} %manejo de imagenes
\usepackage{amsmath} %manejo de notación matemática
\usepackage{mathtools} %manejo de notación matemática
\usepackage{blindtext} %texto de relleno
\usepackage{cancel} %permite la simbolización de cancelación de terminos
\usepackage{enumitem}[shortlabels] %listas con letras
\usepackage{amssymb} %manejo de simbolog►1a matematica

\pagestyle{fancy}
\fancyhf{}
\rfoot{\thepage}

\begin{document}
    
\setcounter{page}{1}
\thispagestyle{fancy}
\lhead{\textbf{Tarea 8, U1}}
\rhead{\textbf{29/09/2020}}
\section{Vectores en el espacio}
\subsection*{Dadas las sig. ecuaciones vectoriales:}
\[L_1: <\!x,y,z\!>\,=\,<\!4+t,5+t,-1+2t\!>\]
\[L_2: <\!x,y,z\!>\,=\,<\!6+2s,11+4s,-3+s\!>\]
\subsection*{Obtener su intersección y de ser así, en que punto y el ángulo que forman}
\subsection*{Cálculos:}
Para obtener la intersección de dos rectas, se deben igualar las ecuaciones paramétricas correspondientes a cada uno de los ejes:
\[ \begin{matrix}
    x_{L_1}=x_{L_2}\\
    y_{L_1}=y_{L_2}\\
    z_{L_1}=z_{L_2}\\
\end{matrix}\rightarrow\begin{matrix}
    4+t=6+2s\\
    5+t=11+4s\\
    -1+2t=-3+s\\
\end{matrix}\]
Con ello procederemos a despejar los valores constantes al lado derecho de la igualdad y los valores con incógnitas al lado izquierdo (de preferencia):
\[\begin{matrix}
    t-2s=6-2\\
    t-4s=11-5\\
    2t-s=-3+1\\
\end{matrix}\rightarrow\begin{matrix}
   \text{1)}\,t-2s=4\\
    \text{2)}\,t-4s=6\\
    2t-s=-2\\
\end{matrix}\] 
Después, se procede a elegír dos de las tres ecuaciones (en este caso la 1) y 2))y en dichas ecuación se procede a encontrar los valores para las incógnitas.
\[\text{Para }t: \begin{matrix}
    t-2s=4\\
    t-4s=6
\end{matrix}\rightarrow \begin{matrix}
    t-2s=4\\
    -t+4s=-6\\
 \sum: 2s=-2   
\end{matrix}\therefore s=-1\]
\[s=-1: \begin{matrix}
    t-2(-1)=4\\
    t+2=4\\
    t=2
\end{matrix}\therefore t=2,\,s=-1\]
Procedemos a comprobar los valores obtenidos con la ecuación 1):
\[\begin{matrix}
    t-2s=4\\
    2-2(-1)=4\\
    2+2=4\\
    4=4
\end{matrix}\]
Por lo tanto, las soluciones son correctas y por ende, se intersectan. Para obtener el punto de intersección se procede a sustituir los valores encontrados de \(t\) y \(s\)
en cualquiera de las dos lineas (en este caso, en \(L_1\)):
\[L_1: <\!x,y,z\!>\,=\,<\!4+2,5+2,-1+2(2)\!>\]
\[L_1: <\!x,y,z\!>\,=\,<\!6,7,3>\]
Para obtener el ángulo se debe primero obtener el valor de \(\vec{v_1}\cdot\vec{v_2}\) para determinar que:
\[\vec{v_1}\perp \vec{v_2} \leftrightarrow \left(\theta=\frac{\pi}{2}\because \vec{u}\cdot\vec{v}=0\right)\]
Entonces si \(\vec{v_1}=\,<\!1,1,2\!>\) y \(\vec{v_2}=\,<\!2,4,1\!>\):
\[\vec{v_1}\cdot\vec{v_2}=1\times 2+1\times 4+2\times 1=2+4+2=8\]
En este caso, se procede a aplicar la propiedad de \(\vec{v_1}\cdot\vec{v_2}\) donde
\[\theta = \arccos \left(\frac{\vec{v_1}\cdot\vec{v_2}}{\lVert \vec{v_1}\rVert \lVert \vec{v_2}\rVert}\right)\]
\[\theta = \arccos \left(\frac{8}{\sqrt{6}\sqrt{21}}\right)\]
\[\theta = \arccos \left(\frac{8}{\sqrt{126}}\right)\]
\[\theta \approx 44.54\text{°}\]
Concluyendo lo encargado.
\end{document}