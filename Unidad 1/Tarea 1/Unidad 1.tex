\documentclass[stu, 12pt, letterpaper, donotrepeattitle, floatsintext, natbib]{apa7}
\usepackage[utf8]{inputenc}
%\usepackage{fontspec} %paquete para usar la fuente Arial 12
\usepackage{comment}
\usepackage{marvosym}
\usepackage{graphicx}
\usepackage{float}
\usepackage[normalem]{ulem}
\usepackage[spanish]{babel} 
\usepackage{lastpage} %para le formato que quiere la profe QUITAR SI QUIERES OG APA7
\usepackage{ragged2e} %para le formato que quiere la profe QUITAR SI QUIERES OG APA7
\usepackage{indentfirst} %para le formato que quiere la profe QUITAR SI QUIERES OG APA7

\setcounter{secnumdepth}{0} %permite enumerar las secciones QUITAR SI QUIERES OG APA7

%comando para ajustar la fuente Arial en todo el documento
%\setmainfont{Arial} %COMPILAR DOC CON XeLateX DOS VECES

\DeclareCaptionLabelSeparator*{spaced}{\\[2ex]}
\captionsetup[table]{textfont=it,format=plain,justification=justified,
  singlelinecheck=false,labelsep=spaced,skip=1pt}

\selectlanguage{spanish}

\useunder{\uline}{\ul}{}
\newcommand{\myparagraph}[1]{\paragraph{#1}\mbox{}\\}

%\rfoot{Página \thepage \hspace{1pt} de \pageref{LastPage}}%QUITAR SI QUIERES OG APA7 
\rhead{} %QUITAR SI QUIERES OG APA7
\setcounter{secnumdepth}{3} %permite enumerar las secciones QUITAR SI QUIERES OG APA7
\setlength{\parindent}{1.27cm} %sangria forzada QUITAR SI QUIERES OG APA7

\renewcommand\labelitemi{$\bullet$}

\newcommand*\chem[1]{\ensuremath{\mathrm{#1}}}

\begin{document}
    %PORTADA
    \begin{titlepage}
        \begin{figure}[ht]
            \centering
            \includegraphics[width=15cm]{logosITT.png}
        \end{figure}
        \centering
        {\Large Tecnológico Nacional de México\\Instituto Tecnológico de Tijuana\par}
        \vspace{1cm}
        {\Large SCD-1011SC6C Ingeniería de Software\par}
        \vspace{1cm}
        {\Large Unidad 1\par}
        \vspace{2cm}
        {\Large\bfseries Tarea 1\par}
        \vspace{2cm}
        {\large Dra. Martha Elena Pulido\par}
        \vfill
            {\large Abraham Jhared Flores Azcona, 19211640\par}
        \vfill
        {\large 10 de febrero de 2021}
    \end{titlepage}

% Índices
\pagenumbering{arabic}
    % Contenido
\renewcommand\contentsname{Contenido}
\tableofcontents

% Cuerpo 
    %NOTA: PARA CITAR ESTILO "Merts (2003)" usar \cite{<nombre_cita_bib>}
    %                        "(Metz, 1978)" usar \citep{<nombre_cita_bib>}
\newpage
\section{Introducción}
Las \begin{justifying}
    necesidades que se espera cumplír con un proyecto ingenieríl necesitan poderse plantear por medio de requerimientos. Sin embargo, el saber clasificarlos
    \end{justifying}
\vspace{\baselineskip}    
\section{Requisitos}
\subsection{Definición}
A \begin{justifying}
    grandes rasgos, los requisitos son la definición y la recolección de los servicios provistos por un sistema. Se acostumbra
    a referirse a los requisitos como un proceso debido a que con esto nos permite delimitar el proyecto. \citep{geeksforgeeks-2020}\par
  \end{justifying}
\vspace{\baselineskip}
\subsection{Tipos}
Se \begin{justifying}
    acostumbra a clasificar los tipos de requerimientos en dos tipos: funcionales y no funcionales \citep{altexsoft-2019, bigelow-2020}\par %cital al alexsoft y techtarget 
\end{justifying}
\subsubsection{Funcionales}
Se \begin{justifying}
    les refiere así a aquellos requisitos que competen a las capacidades/funciones que definen el comportamiento del sistema; específicamente el qué hace y que no hace.
    Vulgarmente hablando, estos requisitos competen al departamento de ingeniería debido a que dichos requisitos les permiten diseñar el sistema acorde a lo estipulado. \citep{altexsoft-2019, bigelow-2020}\par
\end{justifying}
Un \begin{justifying}
    ejemplo claro es el siguiente requerimiento: \emph{``El sistema necesita una autenticación de los datos del usuario para autorizarle la entrada al serivdor''}.
\end{justifying}
\vspace{\baselineskip}
\subsubsection{No funcionales}
Son \begin{justifying}
    aquellos requisitos que engloban la usabilidad del sistema. Esto también se conoce como UX (Experiencia de Usario) debido a que aquí
    remonta la comodidad y rendimiento para el usuario objetivo, que terminará dando su veredicto respecto al sistema. \citep{altexsoft-2019, bigelow-2020} \par
\end{justifying}
Un \begin{justifying}
    ejemplo es el siguiente requerimiento: \emph{``Las páginas del portar web deben cargar en un lapso promedio de 0.5 segundos''}.\par
\end{justifying}
\vspace{\baselineskip}
\section{Conclusión}
Como \begin{justifying}
    se aprecia, los tipos de requerimientos ofrecen distintos enfoques que se complementan para el objetivo final de cualquier proyecto, un producto de calidad.\par
\end{justifying}
\renewcommand\refname{\textbf{Referencias}}
\bibliography{referencias} %el archivo 'referencias.bib' debe estar dentro del mismo folder donde se encuentra el archivo .tex para citar las referencias deseadas

\end{document}