\documentclass[letterpaper, 12pt]{article}
\usepackage[letterpaper, top=2.5cm, bottom=2.5cm, left=3cm, right=3cm]{geometry} %margenes
\usepackage[backend=biber]{biblatex}\addbibresource{bibliografia.bib} %manejo de bibliografía (BORRAR SI NO ES NECESARIO)
\usepackage[utf8]{inputenc} %manejo de caracteres especiales
\usepackage[spanish]{babel} %manejo de encabezados de inglés a español
\usepackage{fancyhdr} %formato de los encabezados de página
\usepackage{ragged2e} %alineado real justficado
\usepackage{graphicx} %manejo de imagenes
\usepackage{amsmath} %manejo de notación matemática
\usepackage{mathtools} %manejo de notación matemática
\usepackage{blindtext} %texto de relleno
\usepackage{amssymb} %manejo de simbología
\usepackage{float} %centrado de imaene
\usepackage{hyperref} %manejo de enlaces e hipervínculos

\hypersetup{
  colorlinks   = true, %Colours links instead of ugly boxes
  urlcolor     = blue, %Colour for external hyperlinks
  linkcolor    = blue, %Colour of internal links
  citecolor   = red %Colour of citations
}

\pagestyle{fancy}
\fancyhf{}
\rfoot{\thepage}

\nocite{*}

\begin{document}
    
    %PORTADA
    \begin{titlepage}
        \begin{figure}[ht]
            \centering
            \includegraphics[width=15cm]{logosITT.png}
        \end{figure}
        \centering
        {\scshape\LARGE Tecnológico Nacional de México\\Instituto Tecnológico de Tijuana\par}
        \vspace{1cm}
        {\scshape\Large Simulación\par}
        \vspace{1cm}
        {\scshape\Large Unidad 1\par}
        \vspace{1.5cm}
        {\huge\bfseries Importancia de la Simulación\par}
        \vspace{2cm}
        {\Large\itshape C. Abraham Jhared Flores Azcona\\19211640\par}
        \vfill
        Profesor: Ing. Diego Saul Vasquez Rios\par
    
        \vfill

        {\large \today}
    \end{titlepage}

    %cuerpo
    \newpage
    \begin{justify}
        \setcounter{page}{1}
        \thispagestyle{fancy}
        \lhead{\textbf{Importancia de la Simulación}}
        \section*{Simulación}
        \justify
        Este concepto abarca muchos temas. Hablando específicamente en terminos de sistemas computacionales, nos referimos
        a la utilización de herramientas matemáticas, computacionales y deterministicas para imitar el comportamiento de un proceso deseado del mundo real.
        \\\newline
        En otras palabras, replicar el comportamiento del objetivo con la computadora.
        \section*{Sistema}
        \justify
        Es una colección de objetos los cuales deseamos estudiar. Generalmente sus parte trabajan entre sí con el proposito de funcionar como un solo ente.
        \\\newline
        Simple y sencillamente lo podemos considerar como la cosa en la cual estamos observando la función de todas sus partes.
        \section*{Modelo}
        \justify
        A grandes rasgos, es una representación miniatura de lagun sistema de interés. Es una construcción artificial creada por el humano para permitirle entender 
        dicho sistema de una manera más amena. Al crear uno se debe de considerar lo siguiente:
        \begin{itemize}
            \item se deben hacer presunciones simplificadas.
            \item se deben identificar las condiciones iniciales o de restricción.
            \item se debe entender el rango de aplicación del modelo.
        \end{itemize}
        En otras palabras, una abstracción de la realidad acomodada a lo que se necesite entender.
        \subsection*{Tipos de modelos}
        \justify
        Aqui encontramos los siguientes tipos:
        \begin{itemize}
            \item \textbf{Conceptuales:} permiten resaltar conexiones importantes de los sistemas y procesos del mundo real.
            \item \textbf{De Demostración:} son modelos físicos que pueden ser facilmente observados y manipulados que incluyen caraterísticas similares a aquellas encontradas en el mundo real.
            \item \textbf{Matemáticos y Estadísticos:} revuelven en resolver las ecuaciones de un sistema o caracterizarlo basado en su parámetros estadísticos.
            \item \textbf{De enseñanza con visualizaciones:} cualquier cosa que ayude a visualizar como funciona un sistema en particular.
        \end{itemize}
        \section*{Procesos}
        \justify
        A grandes rasgos, es una serie de acciones u operaciones que nos conducen a un fín. Dependiendo del ámbito en el cual dicho proceso sea aplicado, este va agenerar un resultado distinto.
        \\\newline
        En otras palabras, es una serie de pasos que generan un resultado.
        \section*{Conclusión}
        \justify
        Como se ha expuesto en estra breve recopilación de conceptos, la importancia de la simulación recae simple y sencillamente en conoces su poder en los distintos escenarios donde se requiera aplicar. Por
        ello es necesario considerar los conceptos expuestos, para tener una mejor compresión y apreciación de las capacidades de la simulación como una disciplina matemática-científica. 

    \end{justify}

    %bibliografía
    \newpage
        \thispagestyle{empty}
        \printbibliography
\end{document}