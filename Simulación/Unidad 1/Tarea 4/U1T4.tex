\documentclass[letterpaper, 12pt]{article}
\usepackage[letterpaper, top=2.5cm, bottom=2.5cm, left=3cm, right=3cm]{geometry} %margenes
\usepackage[backend=biber]{biblatex}\addbibresource{bibliografia.bib} %manejo de bibliografía (BORRAR SI NO ES NECESARIO)
\usepackage[utf8]{inputenc} %manejo de caracteres especiales
\usepackage[spanish]{babel} %manejo de encabezados de inglés a español
\usepackage{fancyhdr} %formato de los encabezados de página
\usepackage{ragged2e} %alineado real justficado
\usepackage{graphicx} %manejo de imagenes
\usepackage{amsmath} %manejo de notación matemática
\usepackage{mathtools} %manejo de notación matemática
\usepackage{blindtext} %texto de relleno
\usepackage{amssymb} %manejo de simbología
\usepackage{float} %centrado de imaene
\usepackage{hyperref} %manejo de enlaces e hipervínculos

\hypersetup{
  colorlinks   = true, %Colours links instead of ugly boxes
  urlcolor     = blue, %Colour for external hyperlinks
  linkcolor    = blue, %Colour of internal links
  citecolor   = red %Colour of citations
}

\pagestyle{fancy}
\fancyhf{}
\rfoot{\thepage}

\nocite{*}

\begin{document}
    
    %PORTADA
    \begin{titlepage}
        \begin{figure}[ht]
            \centering
            \includegraphics[width=15cm]{logosITT.png}
        \end{figure}
        \centering
        {\scshape\LARGE Tecnológico Nacional de México\\Instituto Tecnológico de Tijuana\par}
        \vspace{1cm}
        {\scshape\Large Simulación\par}
        \vspace{1cm}
        {\scshape\Large Unidad 1\par}
        \vspace{1.5cm}
        {\huge\bfseries Ventajas y desventajas de un proceso de simulación\par}
        \vspace{2cm}
        {\Large\itshape C. Abraham Jhared Flores Azcona\\19211640\par}
        \vfill
        Profesor: Ing. Diego Saul Vasquez Rios\par
    
        \vfill

        {\large 7 de marzo de 2021}
    \end{titlepage}

    %cuerpo
    \newpage
    \begin{justify}
        \setcounter{page}{1}
        \thispagestyle{fancy}
        \lhead{\textbf{Ventajas y desventajas de un proceso de simulación}}
        \section*{Ventajas}
        \justify
        \begin{itemize}
            \item Puede evitar el peligro y la pérdida de vida.
            \item Las condiciones pueden ser variadas y sus resultados pueden ser investigados.
            \item Situaciones críticas pueden ser investigadas sin riesgos.
            \item Optimíza costos.
            \item Se pueden acelerar para que su comportamiento pueda ser estudiado facilmente sobre un largo periodo de tiempo.
            \item Se pueden alentar para estudiar comportamientos de manera más cercana.
            \item Se puede estudiar el comportamiento de un sistema sin construirlo.
            \item Sus resultados son generalmente precisos, comparados al modelo analítico.
            \item Nos ayuda a encontrar fenómenos y comportamientos inesperados en el sistema.
            \item Es fácil realizar análisis de tipo "¿Qué tal si (...)?".
            \item Permite entrenar al usuario de la simulación en la comprensión del sistema y de las variables que interfieren en él.
            \item Permite operar en sistmas cuyo ensayo real conduciría a su destrucción, reproduciendolo cuantas veces se desee.
            \item Es seguro, robusto y presenta gran sensibilidad frente a los cambios.
            \item Es fiable (con validez jurídica en muchos países).
            \item No es necesario interrumpir las operaciones de la compañia.
            \item Permite analizar el efecto sobre el rendimiento global de un sistema, de pequeños cambios realizados en una o varias de sus componentes.
        \end{itemize}
        \section*{Desventajas}
        \justify
        \begin{itemize}
            \item Puede llegar a ser caro el medir como una cosa afecta a otra, tomar las mediciones iniciales y el crear modelo mismo (como tuneles de viento aerodinámico).
            \item Para simular algo, se necesita una comprensión extensiva y una conciencia de todos los factores involucrados. Sin esto, una simulación no puede ser creada.
            \item Puede difícil interpretar los resultados de la simulación.
            \item La calidad de la simulación es inferior a la calidad del modelo.
            \item Si un modelo no representa de manera adecuada un sistema, las conclusiones inferidas no sirven.
            \item Su desarrollo requiere mucho tiempo.
            \item Si los modelos de simulación sobre todo si son estocásticos, no son lo más recomendables para optimizar un sistema.
            \item Sus resultados solo son numéricos.
            \item Se requiere una gran cantidad de iteraciones computacionales para encontrar soluciones lo que aumenta los costos de simulación.
        \end{itemize}
        \section*{Procesos de identificación para variables de entrada, variables de salida y servicios}
        \justify
        En esta sección se explican a brevedad las maneras en la cuales podemos identificar a los sujetos mencionados.
        \\\newline
        Para dichas entradas, se les considera como todos los datos que hay que ingresar para la resolución del problema. 
        \\\newline
        El método de diseño de algoritmos en etapas, yendo de los conceptos
        generales a los de detalle, se conoce como el método descendente (\emph{top-down}).
        \\\newline
        En un algoritmo se debe de considerar lo siguiente:
        \begin{itemize}
            \item La información dada.
            \item Las operaciones o cálculos necesarios para encontrar la solución del problema.
            \item Respuestas dadas por el algoritmo o resultados finales de los procesos realizados.
        \end{itemize}
        Para las salidas es preciso analizar:
        \begin{itemize}
            \item Los datos o resulados que se esperan.
            \item La etrada que se suministra.
            \item El proceso requerido para obener los resultados esperados.
            \item Areas de trabajo, fórmulas y otros recursos. 
        \end{itemize}
        Para los procesos:
        \begin{itemize}
            \item Se puede hacer parseado.
            \item Tablas.
            \item Arboles de desición.
            \item Diagramas de flujo de datos por partición de eventos.
        \end{itemize}
    \end{justify}

    %bibliografía
        \thispagestyle{empty}
        \printbibliography
\end{document}