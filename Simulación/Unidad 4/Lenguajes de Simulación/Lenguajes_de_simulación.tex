\documentclass[letterpaper, 12pt]{article}
\usepackage[letterpaper, top=2.5cm, bottom=2.5cm, left=3cm, right=3cm]{geometry} %margenes
\usepackage[utf8]{inputenc} %manejo de caracteres especiales
\usepackage[spanish]{babel} %manejo de encabezados de inglés a español
\usepackage{fancyhdr} %formato de los encabezados de página
\usepackage{ragged2e} %alineado real justficado
\usepackage{graphicx} %manejo de imagenes
\usepackage{amsmath} %manejo de notación matemática
\usepackage{mathtools} %manejo de notación matemática
\usepackage{blindtext} %texto de relleno
\usepackage{amssymb} %manejo de simbología
\usepackage{float} %centrado de imaene
\usepackage{hyperref} %manejo de enlaces e hipervínculos 

\pagestyle{fancy}
\fancyhf{}
\rfoot{\thepage}

\begin{document}
    
    %PORTADA
    \begin{titlepage}
        \begin{figure}[ht]
            \centering
            \includegraphics[width=15cm]{logosITT.png}
        \end{figure}
        \centering
        {\scshape\LARGE Tecnológico Nacional de México\\Instituto Tecnológico de Tijuana\par}
        \vspace{1cm}
        {\scshape\Large Simulación\par}
        \vspace{1cm}
        {\scshape\Large Unidad 4\par}
        \vspace{1.5cm}
        {\huge\bfseries Lenguajes de simulación\par}
        \vspace{2cm}
        {\Large\itshape C. Abraham Jhared Flores Azcona\\19211640\par}
        \vfill
        Profesor: Ing. Diego Saul Vasquez Rios\par
    
        \vfill

        {\large 26 de mayo de 2021}
    \end{titlepage}

    %cuerpo
    \newpage
    \begin{justify}
        \setcounter{page}{1}
        \thispagestyle{empty}
        \lhead{\textbf{Lenguajes de simulación}}
        \subsection*{SOLIDWORKS}
        \justify
        Ofrecido por Dassault Systemes. un sistema tipo CAD para la educación y la manufactura para apoyar el diseño 2D y 3D,
        diseño digital, simulaciones y desarrollo de productos con herramientas colaborativas.
        \\\newline
        Una de sus ventajas es que las animaciones de ensamblado nos ayudan a entender nuestro diseño y nos permite conocer las restricciones del diseño y que
        tiene menús, barras de herramientas y detalles muy intuitivos.
        \\\newline
        Una de sus desventajas es que sus tiempos de renderizado consumen mucho tiempo y que construcciones grandes y complejas tardan mucho en cargar.
        \subsection*{Navisworks}
        \justify
        Ofrecido por Autodesk. Es un software de revisión de construcción y de simulación para mejorar la coordinación del BIM (el modelado de información de construcción),
        permitiendo al usuario la combinación de diseño y los datos de construcción en un solo modelo, e identificar y resolver problemas de choque e interferencia.
        \subsection*{Autodesk Fusion 360}
        \justify
        Herramienta estilo CAD basada en la nube que integra el diseño, modelado y capacidades de simulación. Es gratis.
        \subsection*{MATLAB}
        \justify
        Plataforma de programación y cómputo numérico que permite el analásis iterativo y el diseño de procesos con un lenguaje de programación que expresa matemáticas de matrices y arreglos
        de manera directa. Incluye un editor para crear scripts que combinan código, salidas y texto formateado en un bloc de notas ejecutable.
        \subsection*{Arena}
        \justify
        Permite realizar simulaciones de eventos discretos que permiten describir con un conjunto de eventos únicos y específicos en el tiempo. Esto es porque implementa modelos flexibles y basados en actividades
        que pueden ser efectivamente usados para simular casi cualquier proceso.
    \end{justify}
\end{document}