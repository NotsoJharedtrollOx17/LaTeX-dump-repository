\documentclass[letterpaper, 12pt]{article}
\usepackage[letterpaper, top=2.5cm, bottom=2.5cm, left=3cm, right=3cm]{geometry} %margenes
\usepackage[utf8]{inputenc} %manejo de caracteres especiales
\usepackage[spanish]{babel} %manejo de encabezados de inglés a español
\usepackage{fancyhdr} %formato de los encabezados de página
\usepackage{ragged2e} %alineado real justficado
\usepackage{graphicx} %manejo de imagenes
\usepackage{amsmath} %manejo de notación matemática
\usepackage{mathtools} %manejo de notación matemática
\usepackage{blindtext} %texto de relleno
\usepackage{amssymb} %manejo de simbología
\usepackage{float} %centrado de imaene
\usepackage{hyperref} %manejo de enlaces e hipervínculos 

\pagestyle{fancy}
\fancyhf{}
\rfoot{\thepage}

\begin{document}
    
    %PORTADA
    \begin{titlepage}
        \begin{figure}[ht]
            \centering
            \includegraphics[width=15cm]{logosITT.png}
        \end{figure}
        \centering
        {\scshape\LARGE Tecnológico Nacional de México\\Instituto Tecnológico de Tijuana\par}
        \vspace{1cm}
        {\scshape\Large Simulación\par}
        \vspace{1cm}
        {\scshape\Large Proyecto de investigación\par}
        \vspace{1.5cm}
        {\huge\bfseries Objetivos e Hipótesis\par}
        \vspace{2cm}
        {\Large\itshape C. Abraham Jhared Flores Azcona\\19211640\par}
        \vfill
        Profesor: Ing. Diego Saul Vasquez Rios\par
    
        \vfill

        {\large 1ro. de junio de 2021}
    \end{titlepage}

    %cuerpo
    \newpage
    \begin{justify}
        \setcounter{page}{1}
        \thispagestyle{fancy}
        \lhead{\textbf{Objetivos e Hipótesis}}
        \section*{Introducción}
        \justify
        En esta breve redacción, se expanden los objetivos e hipótesis del proyecto de Simulación a presentar a finales del semestre. Estrictamente hablando, se introducen
        los términos financieros que clarifican la comprensión de los planteamientos, los objetivos financieros a llegar con la simulación y finalmente las hipótesis corolarias
        a lo descrito antes.
        \subsection*{Conceptos clave}
        \justify
        Se describe brevemente aquellas palabras que se usarán a lo largo de este escrito. Los conceptos son enfocados en finanzas, por lo que una buena base de las disciplinas financieras
        facilita la comprensión del texto restante.
        \begin{itemize}
            \item \textbf{Acciones:} el instrumento de inversión elegido. Para simpleza, se eligen cinco acciones.
            \item \textbf{Capital:} para los propositos de la redacción, es el dinero el cual se dispone para invertir.
            \item \textbf{Ganancia:} es la cantidad de dinero que se gana dada la inversión. Generalmente se expresa como porcentaje.
            \item \textbf{MXN:} es la abreviación técnica del peso mexicano.
            \item \textbf{Paridad peso-dolar:} se refiere a la tasa de cambio de un peso mexicano covertido a un dólar estadounidense.
            \item \textbf{Tasa de inflación:} el porcentaje por el cual la cantidad de capital pierde poder adquisitivo (su valor baja a travéz del tiempo).
            \item \textbf{Ticker:} es la abreviación de la acción en la bolsa de valores de la compañia deseada.
            \item \textbf{USD:} abreviación técnica del dólar estadounidense.
            \item \textbf{Volatilidad:} el porcentaje del cambio de precio de una acción. Se interpreta que mientras más alto sea el porcentaje, el cambio es más drástico.
        \end{itemize}
        \section*{Planteamiento}        
        \justify
        Se va a desarrollar una simulación de inversión en cinco acciones que son populares en el subforo de Reddit llamado \emph{wallstreetbets} para observar si los rendimientos en las acciones
        populares son realmente una buena tésis de inversión y por ende, una buena opción como una inversión seria.
        \section*{Objetivos}
        \justify
        El primer objetivo es probar si una tésis de inversión en base al consenso y popularidad de ciertas acciones dentro de un foro de discusión es viable para poder ser aplicada como una estrategia de inversión. Esto 
        es debido a que los distintos proverbios de inversión en acciones específicas requiere de investigaciones completas y documentadas para evitar sentir que una inversión es basada en pura suerte, ó en este caso
        ``siguiendo a las masas''.
        \\\newline
        El segundo objetivo es observar el comportamiento de las acciones y su volatilidad para mediar el rendimiento futuro de la inversión dado el capital invertido y sus posibles ganancias. Esto es relevante ya que
        a lo largo del tiempo, cambios drásticos del precio pueden beneficiar o maleficiar el prospecto de las ganancias y, por ende, afectar el plazo deseado de inversión y decidir si terminar o continuar la inversión.
        \\\newline
        El tercer objetivo que se deriva del segundo es observar las acciones volatiles y definir si es conveniente administrar de manera frecuente las posiciones. Se reduce a si estar checando las posiciones es más conveniente que esperar
        al plazo de inversión establecido por el inversor ya que, si se administra de manera frecuente, se tiende a tener mayores problemas en el bienestar psicológico del inversionista.
        \\\newline
        El cuarto objetivo es determinar si los rendimientos de la inversión en las cinco acciones específicadas supera los rendimientos del índice de mercado S\&P 500. Esto es porque dicho índice es un buen medidor de rendimiento del mercado
        y un rendimiento estable con una tendencia creciente lo que lo hace en la gran mayoria de los casos de inversión personal la opción más simple, eficiente y congruente en comparación con elegir ciertas acciones y esperar rendimientos más
        altos que del índice.
        \\\newline
        Finalmente, el quinto objetivo es apreciar la ayuda del cómputo como una herramienta útil en la toma de desiciones de carácter financiero. Simple y sencillamente probar si ``el conocimiento es poder''. 
        \section*{Hipótesis}
        \justify
        Se tiene cuatro hipótesis basadas en lo recabado y como corolarios de los objetivos planteados. La primer hipótesis es que teniendo el consenso del foro, se puede mediar la posibilidad del éxito ó fracaso. Corolario del sentimiento del mercado ya que la bolsa de valores es una expresión\
        especulativa de valor que las personas activas en este consideran que tiene.
        \\\newline
        La segunda hipótesis es que mientras mas volatididad tenga una acción, se puede tener más riesgo y mayor ganancia. Recae en la noción intuitiva de la relación riesgo-ganancia que se puede experimentar dentro de esta simulación.
        \\\newline
        La tercera hipótesis que se debe dejar al menos por un año la inversión para obtener mayores ganancias ante la inversión. Se infiere de tener paciencia para poder crecer la ganancia y tener un poco más de madurez respecto al tema y evitar hacer acciones por impulso no-racional.
        \\\newline
        La cuarta y última hipótesis es si mientras más capital invertido máyor la posibilidad de pérdida ó de ganancia.
    \end{justify}

\end{document}