% Preámbulo
\documentclass[stu, 12pt, letterpaper, donotrepeattitle, floatsintext, natbib]{apa7}
\usepackage[utf8]{inputenc}
\usepackage{comment}
\usepackage{marvosym}
\usepackage{graphicx}
\usepackage{float}
\usepackage[normalem]{ulem}
\usepackage[spanish]{babel} 
\usepackage{indentfirst} %para le formato que quiere la profe QUITAR SI QUIERES OG APA7
\usepackage{ragged2e} %para le formato que quiere la profe QUITAR SI QUIERES OG APA7
\usepackage{indentfirst} %para le formato que quiere la profe QUITAR SI QUIERES OG APA7


\selectlanguage{spanish}
\useunder{\uline}{\ul}{}
\newcommand{\myparagraph}[1]{\paragraph{#1}\mbox{}\\}

% Portada
%\thispagestyle{empty}
\title{\Large Tarea 1 Unidad 3: Modelos de Graficación 3D}
\author{Abraham Jhared Flores Azcona} % (autores separados, consultar al docente)
% Manera oficial de colocar los autores:
%\author{Autor(a) I, Autor(a) II, Autor(a) III, Autor(a) X}
\affiliation{Instituto Tecnológico de Tijuana}
\course{SCC-1010SC5C: Graficación}
\professor{Dra. Martha Elena Pulido}
\duedate{28 de septiembre de 2021}

\begin{document}
    % Índices
    \pagenumbering{arabic}
    \maketitle

    % Contenido
    \renewcommand\contentsname{Contenido}
    \tableofcontents

    % Cuerpo 
    %NOTA: PARA CITAR ESTILO "Merts (2003)" usar \cite{<nombre_cita_bib>}
    %    
    \newpage
    \section*{Introducción}
    Como \begin{justifying}
    parte relevante a la Graficación, un aspecto importante de aprender es aquél de la graficación tridimensional, la que nos permite la creación
    de videojuegos como nuestras peliculas animadas favoritas. El poder aprender las herramientas que se emplean para ello es de suma relevancia para nuestra preparación profesional.\par  
    \end{justifying}
    \vspace{\baselineskip}
    \section{Graficación 3D}
    \subsection{Concepto}
    También \begin{justifying}
        conocido como el diseño tridimensional. Es el proceso de usar programas de modelado por computadora para crear un objeto el cuál contiene tres valores clave
        asignados a él para entender que existe dentro del espacio. Un factor muy útil del diseño 3D es que
        se puede usar para dar un mayo énfasis y variedad visual a los elementos de un diseño específico para UX \citep{silveira-2021}.
        \par
    \end{justifying}
    \vspace{\baselineskip}
    \subsection{Modelos}
    Como \begin{justifying}
        cualquier estilo de diseño, el mercado nos ofrece distintas herramientas de modelado. Algunas de estas se explican a continuación.
    \end{justifying}
    \subsubsection{3D Transform}
    Proporcionado \begin{justifying}
        por Adobe XD como un medio para convertir íconos y elementos en objetos de apariencia 3D.\par
    \end{justifying}
    \vspace{\baselineskip}
    \subsubsection{Blender}
    Es \begin{justifying}
        una herramienta open-source que es gratuita. Soporta la gama completa de 3D: modelado, animación, simulación, renderizado, composición y
        seguimiento de movimiento, e inclusive edición de video y creación de juegos \citep{unknown-author-no-date}. %citar al blender
        \par
    \end{justifying}
    \vspace{\baselineskip}
    \subsubsection{Autodesk Maya}
    Es \begin{justifying}
        usado por los estudios Pixar. Este contiene Bitfrost que permite crear simulaciones fisicamente certeras en un solo entorno de programación visual. También
        contiene Arlond el cual es el sistema de renderizado de la herramienta la cual permite cambiar entre renderizado CPU y GPU, y con una interfáz
        amigable con controles intuitivos \citep{unknown-author-2021}. %citar al de autodesk
        \par
    \end{justifying}
    \vspace{\baselineskip}
    \subsubsection{ZBrush}
    Su \begin{justifying}
        subscripción es muy asequible. Esta herramienta es de escultura 3D que se difrencia del rsto por hacer mímica de las técnicas de escultura tradicionales por medios
        digitales. Sus elementos del programa permiten una mayor eficiencia comparado a otros programas \citep{petty-2018}. %citar a petty
        \par
    \end{justifying}
    \vspace{\baselineskip}
    \subsubsection{TinkerCAD}
    Otra \begin{justifying}
        herramienta propiedad de Autodesk. Esta es basada en cómputo en la nube y es gratuita; con un enfoque para completos principiantes. Proveee un concepto de construcción por bloques muy intuitivo
        que permite desarrollar modelos para un conjunto de figuras básicas \citep{p-2021}. %citar al de 3d natives
        \par
    \end{justifying}
    \vspace{\baselineskip}
    \section{Conclusión}
    Los \begin{justifying}
        múltiples usos y herramientas de la graficación tridimensional nos permiten apreciar de mejor manera la labor que se debe hacer para crear
        nuestros medios preferidos de entretenimiento.\par
    \end{justifying}
    
    \newpage   
    % Referencias
    \renewcommand\refname{\textbf{Referencias}}
    \bibliography{referencias}
    
\end{document}