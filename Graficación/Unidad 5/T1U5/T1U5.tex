% Preámbulo
\documentclass[stu, 12pt, letterpaper, donotrepeattitle, floatsintext, natbib]{apa7}
\usepackage[utf8]{inputenc}
\usepackage{comment}
\usepackage{marvosym}
\usepackage{graphicx}
\usepackage{float}
\usepackage{amsmath}
\usepackage[normalem]{ulem}
\usepackage[spanish]{babel} 
\usepackage{indentfirst} %para le formato que quiere la profe QUITAR SI QUIERES OG APA7
\usepackage{ragged2e} %para le formato que quiere la profe QUITAR SI QUIERES OG APA7
\usepackage{indentfirst} %para le formato que quiere la profe QUITAR SI QUIERES OG APA7

\selectlanguage{spanish}
\useunder{\uline}{\ul}{}
\newcommand{\myparagraph}[1]{\paragraph{#1}\mbox{}\\}

% Portada
%\thispagestyle{empty}
\title{\Large Tarea 1 Unidad 5: Graficación por Computadora}
\author{Abraham Jhared Flores Azcona} % (autores separados, consultar al docente)
% Manera oficial de colocar los autores:
%\author{Autor(a) I, Autor(a) II, Autor(a) III, Autor(a) X}
\affiliation{Instituto Tecnológico de Tijuana}
\course{SCC-1010SC5C: Graficación}
\professor{Dra. Martha Elena Pulido}
\duedate{28 de octubre de 2021}

\begin{document}
    % Índices
    \pagenumbering{arabic}
    \maketitle

    % Cuerpo 
    %NOTA: PARA CITAR ESTILO "Merts (2003)" usar \cite{<nombre_cita_bib>}
    %    
    \newpage
    \section{Antecedentes}
    Inicialmente \begin{justify}
      solo era una herramienta exclusiva para científicos e ingenieros que trabajaban
      en el gobierno y centros de investigación corporativos. Su desarrollo previo como se menciona
      tomó lugar en centros académicos y en centros de investigación que lograron crear los gráficos
      para transmisión televisiva y luego en películas de los 70's y 80's.\citep{unknown-author-2020}\par 
    \end{justify}
    \vspace{\baselineskip}
    \section{Evolución}
    Como \begin{justifying}
      tal, los primeros indicios primitivos comienzan con el proyecto Whirlwind y el sistema computacional SAGE;
      el primero fue en crear un simulador de vuelo y el segundo en proveer un sistema de defensa aerea para la USAF.
      Eventualmente estos pryectos se provieron al mercado civil en forma de una expo de proyectos de IBM. Eventualmente,
      compañias como IBM no tenian computadoras con algún tipo de despliegue gráfico hasta que se presenta la investigación
      \emph{Sketchpad} dpmde explica la interacción de diseño con un monitor gráfico y una pluma táctil. 
      Eventualmente se desarrolló el cómo dibujar lineas en un dispositivo de rasterizado, el cuál se extiende a los círculos.
      El renderizado se descubre en los 70's y se demuestra la animación tridimensional por cuadros.
      Hasta la actualidad, los avances recaen en las comercialización de las tarjetas gráficas para el consumidor promedio que le permiten
      tener procesamiento gráfico muy asequible; en este mercado, el principal competidor es NVIDIA. \citep{unknown-author-no-date}\par   
    \end{justifying}
    \vspace{\baselineskip}
    \section{Area de Aplicación}
    A \begin{justifying}
      continuación se listan algunas aplicaciones: \par
      \begin{itemize}
        \item Gráficos de transmisión televisiva.
        \item Efectos computarizados para películas.
        \item Visualización de funciones matemáticas como gráficas.
        \item Simulaciones de carácter mecánico.
        \item Imágenes dadas por rayos X.
        \item etc.
      \end{itemize}\par
    \end{justifying}
    
    \vspace{\baselineskip}
    \newpage   
    % Referencias
    \renewcommand\refname{\textbf{Referencias}}
    \bibliography{referencias}
    
\end{document}