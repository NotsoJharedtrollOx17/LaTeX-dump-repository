\documentclass[letterpaper, 12pt]{article}
\usepackage[letterpaper, top=2.5cm, bottom=2.5cm, left=3cm, right=3cm]{geometry} %margenes
\usepackage[utf8]{inputenc} %manejo de caracteres especiales
\usepackage[spanish]{babel} %manejo de encabezados de inglés a español
\usepackage{fancyhdr} %formato de los encabezados de página
\usepackage{ragged2e} %alineado real justficado
\usepackage{graphicx} %manejo de imagenes
\usepackage{amsmath} %manejo de notación matemática
\usepackage{mathtools} %manejo de notación matemática
\usepackage{blindtext} %texto de relleno
\usepackage[backend=biber]{biblatex}\addbibresource{Bibliografia.bib} %manejo de bibliografía (BORRAR SI NO ES NECESARIO)

\pagestyle{empty}
\fancyhf{}
\rfoot{\thepage}

\nocite{*}

\begin{document}
    
    %PORTADA
    \begin{titlepage}
        \begin{figure}[ht]
            \centering
            \includegraphics[width=15cm]{logosITT.png}
        \end{figure}
        \centering
        {\scshape\LARGE Tecnológico Nacional de México\\Instituto Tecnológico de Tijuana\par}
        \vspace{1cm}
        {\scshape\Large Graficación\par}
        \vspace{1cm}
        {\scshape\Large Unidad 1\par}
        \vspace{1.5cm}
        {\huge\bfseries Actividad 1\par}
        \vspace{2cm}
        {\Large\itshape C. Abraham Jhared Flores Azcona\\\#: 19211640\par}
        \vfill
        Profesor: \par
        Dra. Martha Elena Pulido
        
        \vfill

        {\large 25 de agosto de 2021}
    \end{titlepage}

    \newpage
    \begin{justify}
        \thispagestyle{empty}
        \justify
        \section*{Antecedentes}
        \justify
        \subsection*{1941}
        \begin{itemize}
            \item Maquinas especiales tipo maquina de escribir eran usadas para perforar hojas gruesas de papel. Estas posteriormente podian ser leidas por computadoras basadas en óptica.
        \end{itemize}
        \subsection*{1950}
        \begin{itemize}
            \item Ben Laposky crea las primeras imágenes gráficas, un osciloscópio, generado por una máquina análoga electrónica. La imagen se producia al manipular haces electrónicos y grabandolos en fimle de alta velocidad.
        \end{itemize}
        \subsection*{1951}
        \begin{itemize}
            \item La computadora \emph{Whirlwind} del MIT fue la primer computadora con un despliegue de video de datos en tiempo real.
        \end{itemize}
        \subsection*{1955}
        \begin{itemize}
            \item Se introduce la \emph{light pen}.
        \end{itemize}
        \subsection*{1960}
        \begin{itemize}
            \item Se realiza el primer documento serio en Análisis de Elementos Finitos y posteriormente se publica. Esto nos permite probar productos virtualmente y produciendo resultados que son igual de certeros que las pruebas físicas.
        \end{itemize}
        \subsection*{1961}
        \begin{itemize}
            \item El primer videojuego, \emph{SpaceWar}, se ejecuta usando un osciloscopio como despliegue.
            \item Ivan utherland escribe el primer programa de dibujo computarizado - \emph{SketchPad} - el cual incluye elementos como menús pop-up.
        \end{itemize}
        \subsection*{1963}
        \begin{itemize}
            \item Doug Engelbart inventa el ratón de coputadora.
        \end{itemize}
        \subsection*{1965}
        \begin{itemize}
            \item Jack Bresenham inventa el algorítmo de trazado lineal ``ideál''.
            \item Se publica el software \emph{NASTRAN FEA.}
        \end{itemize}
        \subsection*{1972}
        \begin{itemize}
            \item Nolan Kay Bushnell crea \emph{Pong}, el videojuego arcade.
            \item Empiezan a aparecer los despliegues Raster.
            \item Se introduce el escáner CT.
        \end{itemize}
        \subsection*{1975}
        \begin{itemize}
            \item La diserción \emph{``Aplicaciones de diseño ayudadas por computadora de la forma aproximada B-Spline''} desarrolla la representación matemática de curvas arbitrarias acomodables para la computación.
        \end{itemize}
        \subsection*{1977}
        \begin{itemize}
            \item La \emph{Apple II} es la primer computadora personal con gráficos.
            \item Se publica Star Wars; sus únicos efectos por computadora fueron basados en vectores, y después filmados.
            \item Se publica \emph{CADAM}, el primer paquete de CAD 2D comercial.
        \end{itemize}
        \subsection*{1978}
        \begin{itemize}
            \item H. Voelcker et al desarrolla el primer estándar real para geometría solida contructiva.
            \item Charles Lang en la Unversidad de Cambridge desarrolla la primer engine de modelado con representación restringida.
        \end{itemize}
        \subsection*{1982}
        \begin{itemize}
            \item La \emph{Commodore 64} personal usó graficos raste para que televisiones regulares pudiesen ser dispositivos de despliegue.
            \item \emph{TRON} es la primer película que usa extensivamente graficos por computadora.
        \end{itemize}
        \subsection*{1985}
        \begin{itemize}
            \item Pixar publica Luxo, Jr.
            \item La teconología Voxel se encuentra en la gram mayoria de software de imágenes médicas.
        \end{itemize}
        \subsection*{1995}
        \begin{itemize}
            \item Tras ser publicada, Toy Story se convierte en la primer película hecha completamente por CGI.
            \item Se publica \emph{Solidworks.}
            \item Se publica \emph{MS Internet Explorer}
        \end{itemize}

        \section*{Síntesis}
        \justify
        Como se puede apreciar, la historia/evolución de la disciplina de la graficación computarizada (junto con el resto de las disciplinas afines a las ciencias computacionales) ha tenido un avance exponencial
        que hasta el momento de la redacción lo más seguro es que se está desarrollando la historia. A pesar de llegar a compilar ciertos años, la gran mayoria de los avances recientes han aparecido de una manera mucho más rápida con el advento del Internet
        y de la globalización por medio de la misma. 
        \\\newline
        Otros avances prometedores que surgieron no más de una década atrás han sido la popularidad de equipos computacionales capaces de soportar videojuegos de calidad
        gráfica muy elevada, el desarrollo de los ambientes de realidad virtual y aumentada, la llegada del telefono inteligente y con este la procuración de la UX y el UI como puntos relevantes para un software así como los aquellos a la 
        visión de computadora, entre otros.
    \end{justify}

    \newpage
        \thispagestyle{empty}
        \printbibliography  
\end{document}