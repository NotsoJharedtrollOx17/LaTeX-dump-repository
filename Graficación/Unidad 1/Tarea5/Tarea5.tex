% Preámbulo
\documentclass[stu, 12pt, letterpaper, donotrepeattitle, floatsintext, natbib]{apa7}
\usepackage[utf8]{inputenc}
\usepackage{comment}
\usepackage{marvosym}
\usepackage{graphicx}
\usepackage{float}
\usepackage[normalem]{ulem}
\usepackage[spanish]{babel} 
\usepackage{indentfirst} %para le formato que quiere la profe QUITAR SI QUIERES OG APA7
\usepackage{ragged2e} %para le formato que quiere la profe QUITAR SI QUIERES OG APA7
\usepackage{indentfirst} %para le formato que quiere la profe QUITAR SI QUIERES OG APA7


\selectlanguage{spanish}
\useunder{\uline}{\ul}{}
\newcommand{\myparagraph}[1]{\paragraph{#1}\mbox{}\\}

% Portada
%\thispagestyle{empty}
\title{\Large Tarea 5: Programas de Procesamiento de Imágenes}
\author{Abraham J. Flores Azcona} % (autores separados, consultar al docente)
% Manera oficial de colocar los autores:
%\author{Autor(a) I, Autor(a) II, Autor(a) III, Autor(a) X}
\affiliation{Instituto Tecnológico de Tijuana}
\course{SCC-1010SC5C: Graficación}
\professor{Dra. Martha Elena Pulido}
\duedate{10 de septiembre de 2021}

\begin{document}
    % Índices
    \pagenumbering{arabic}
    \maketitle

    % Contenido
    \renewcommand\contentsname{Contenido}
    \tableofcontents

    % Cuerpo 
    %NOTA: PARA CITAR ESTILO "Merts (2003)" usar \cite{<nombre_cita_bib>}
    %    
    \newpage
    \section*{Introducción}
    Como \begin{justifying}
    parte relevante a la Graficación, los programas de procesamiento de imágenes son herramienta indispensable para el estudio y aplicación de la disciplina.
    Por lo que es relevante revisar dichos conceptos para su futura referencia en la materia así como programas relevantes para cumplir con lo deseado (desde
    el punto de vista de Graficación).\par  
    \end{justifying}
    \section{Programas de procesamiento de imágenes}
    A \begin{justifying}
        grandes rasgos, son programas que están diseñados para manipular imágenes digitales. En particular, capturan la imágen si no se habia hecho,
    la convierte a forma digital y les realiza manipulaciones acorde a lo deseado \citep{gloog-no-date}.
    \par
    \end{justifying}
    \subsection{Adobe Photoshop}
    Sin \begin{justifying}
    lugar a dudas, el software más popular y reconocido sobre el tema en cuestión. Adobe Photoshop, abreviado coloquialmente como Photoshop, es un software provisto
    por la compañia Adobeque se usa extensivamente para edición de imágen por rasterizado, diseño gráfico y arte digital. Este da uso de capas para permitir
    profundidad y flexibilidad en el diseño y proceso de edición, así como el proveer de herramientas potentes de edición, que al ser compuestas, son capaces de casi todo.\par
    Aparte de lo mencionado antes, estas sobrecapas soportan transparencia y también pueden actuar como máscaras o filtros que pueden alterar imágenes sobrepuestas en las capas debajo
    de sí mismas. Las sombras y otros efectos como la composición alfa pueden ser aplicados. También es posible aplicar distintos modelos de color a las capas.\par
    Un detalle sumamente importante de este programa es el de que su extensión para archivos por defecto es .PSD el cual tiene un máximo de 30,000 pixeles de altura y anchura con un límite
    de 2 GB para su almacenamiento. Otro similar es el .PSB que es la versión para archivos sumamente grandes; la altura y anchura se extiende a 300,000 píxeles y su 
    tamaño de almacenamiento aumenta alrededor de 4 EB \citep{techopedia-2017}. \par
    \end{justifying}    
    \subsection{Microsoft MS Paint}
    Conocido \begin{justifying}
    también como Paint. Este es un programa simple de procesado de imágen que permite a los usuarios el crear arte gráfico básico en una computadora. Este ha sido incluido en cada
    versión de Microsoft Windows desde su concepción. Este programa provee funcionalidades como la de pintar a color o en blanco y negro, aparte de herramientas para la cambiar el trazo
    y crear figuras geométricas.\par
    En la versión más reciente de Paint se pueden crear, abrir, desplegar y editar archivos .bmp, .dib, .jpg, .gif, .tiff, .png y .ico. Sin el programa, seria necesario el instalar un
    visualizador externo para abrir las imágenes en el computador \citep{computer-hope-2021}.\par     
    \end{justifying}
    \subsection{Geogebra}
    Acorde \begin{justifying}
        a la descripción oficial de la página web de Geogebra, este programa es un compendio gratuito de herramientas matemáticas para graficación, geometría, visualización 3D y demás. Mas allá
    es un software matemático dinámico para todos los niveles educativos que junta la geometría, algebra, hojas de cálculo, graficación, estadística y cálculo en un solo paquete sencillo de usar \citep{geogebra-no-dateA}.\par
    Aparte de lo mencionado, se pueden crear herramientas de aprendizaje y adjuntarlas como páginas web y está disponible en muchos idiomas, aparte de que el software es Open-source \citep{geogebra-no-dateB}.\par 
    \end{justifying} 
    \subsection{Desmos}
    Similar \begin{justifying}
    a Geogebra en la cuestión de graficación. Este es una herramienta gratuita de graficación y enseñanza disponible en la red. Aparte de esbozar ecuaciones, las actividades de clase están disponibles
    para ayudar a los alumnos a aprender sobre una variedad de conceptos matemáticos \citep{unknown-author-2021}.\par 
    \end{justifying}
    \subsection{Snapseed}
    Un editor de fotografias propiedad de Google. Su interfáz es simple y clara. A pesar de que ofrece un retoque automático, dicha aplicación también ofrece ajustes de brillo, contraste, recortado y alineamiento.
    Sobre efectos especiales, contiene filtros Drama, Grunge, Vintage, enfoque central, etc \citep{pcmag-2013}.\par
    \section{Conclusión}
    Los \begin{justifying}
        distintos programas de procesado de imágen mostrados permiten ilustrar de una mejor manera la contribución de la Graficación como una herramienta vital para la vida cotidiana computarizada permitiendo
        al usuario una libertad que lo pued emparejar con profesionales.\par
    \end{justifying}
    
    \newpage   
    % Referencias
    \renewcommand\refname{\textbf{Referencias}}
    \bibliography{referencias}
    
\end{document}