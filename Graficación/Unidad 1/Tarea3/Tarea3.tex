\documentclass[letterpaper, 12pt]{article}
\usepackage[letterpaper, top=2.5cm, bottom=2.5cm, left=3cm, right=3cm]{geometry} %margenes
\usepackage[utf8]{inputenc} %manejo de caracteres especiales
\usepackage[spanish]{babel} %manejo de encabezados de inglés a español
\usepackage{fancyhdr} %formato de los encabezados de página
\usepackage{ragged2e} %alineado real justficado
\usepackage{graphicx} %manejo de imagenes
\usepackage{amsmath} %manejo de notación matemática
\usepackage{mathtools} %manejo de notación matemática
\usepackage{blindtext} %texto de relleno
\usepackage{float}
\usepackage[backend=biber]{biblatex}\addbibresource{referencias.bib} %manejo de bibliografía (BORRAR SI NO ES NECESARIO)

\pagestyle{empty}
\fancyhf{}
\rfoot{\thepage}

\nocite{*}

\begin{document}
    
    %PORTADA
    \begin{titlepage}
        \begin{figure}[ht]
            \centering
            \includegraphics[width=15cm]{logosITT.png}
        \end{figure}
        \centering
        {\scshape\LARGE Tecnológico Nacional de México\\Instituto Tecnológico de Tijuana\par}
        \vspace{1cm}
        {\scshape\Large Graficación\par}
        \vspace{1cm}
        {\scshape\Large Unidad 1\par}
        \vspace{1.5cm}
        {\huge\bfseries Tarea 3\par}
        \vspace{2cm}
        {\Large\itshape C. Abraham Jhared Flores Azcona\\\#: 19211640\par}
        \vfill
        Profesor: \par
        Dra. Martha Elena Pulido
        
        \vfill

        {\large 8 de septiembre de 2021}
    \end{titlepage}

    \newpage
    \section*{Introducción}
    En los aspectos que competen a la Graficación computarizada, uno de los principales puntos de gestión es de los formatos gráficos para
    salvaguardar nuestros resultados creados por las herramientas gráficas. En esta breve redacción se recopilan los formatos más conocidos 
    para futura referencia en las actividades de la materia.\par
    \vspace{\baselineskip}
    \section*{Formatos gráficos de almacenamiento}
    A grandes rasgos, todos los formatos gráficos de alamacenamiento son medios estandarizados para organizar y almacenar imágenes digitales.
    Un formato puede almacenar datos en un formato no-compreso o compreso. Estas imágenes están compuestas de datos digitales en uno de estos formatos
    para que estos datos puedan ser rasterizados para su uso en un monitor de computadora o impresora.\par
    \vspace{\baselineskip}
    \subsection*{TIFF}
    Formato de Archivo de Imagenes con Etiquetas (\emph{Tagged Image File Format en inglés}). Concebido para almacenar imágenes de mapa de bits. Desarrollado en
    1987 por Aldus (ahora propiedad de Adobe). Sus caracteristicas son:
    \begin{itemize}
        \item El más antiguo.
        \item Almacena mapas de bits muy grandes (de más de 4GB comprimidos).
        \item No pierde calidad.
        \item También permite almacenar imágenes en blanco y negro, en colores verdaderos (hasta 32 bits por pixel).
        \item Puede indexar imágenes itilizando una paleta.
        \item Permite que se utilicen varios espacios de color: RGB, CMYKM, CIE L*a*b, YUV/YcrCb. 
    \end{itemize}\par
    \vspace{\baselineskip}
    \subsection*{RAW}
    Quiere decir ``crudo'' en inglés. Coloquialmente se usa para describir algo sin algun tipo de censura. El formato de imágen funciona similar a su interpretación
    como palabra coloquial; es un formato que no se comprime y se procesa de forma mínima. Su razón de uso es para guardar imágenes con la mayor fidelidad posible a la
    resolución en megapíxeles de una cámara, por ende la necesidad de que dicho formato tenga un tamaño grande de almacenamiento. Sus caracteristicas sonÑ
    \begin{itemize}
        \item Como se menciona, almacena imágenes con una compresion casi nula.
        \item Por lo anterior, este formato es de tamaño grande.
        \item Permite edición posterior sin pérdida de calidad de imágen.
        \item Ofrece hasta 4.3 mil millones de tonos de colores.
        \item Permite ofrecer más brillo y menor ruido de imágen.
        \item Balance de blancos.
        \item Ideal para impresión.
    \end{itemize}\par
    \vspace{\baselineskip}
    \subsection*{JPEG}
    El formato más usado para la compresión de imágenes digitales. El acrónimo del formato viene del \emph{Joint Photographics Experts Group} quienes crearon el estándar en 1992.
    Su base de compresión es con la Transformada del Coseno Discreta. Sus características incluyen:
    \begin{itemize}
        \item Su compresión genera pérdidas de calidad de imágen.
        \item Por lo anterior, su profundidad de color es menor (de 8 a 24 bits).
        \item Se pueden visualizar dentro de navegadores Web.
        \item No admite transparencia.
        \item Descarta selectivamente datos de la imágen, por lo que un ajuste de mayor calidad descarta menos datos.
        \item Degrada detalles nítidos de una imagen, como aquellas que contiene texto o imágenes vectoriales.
    \end{itemize}\par
    \vspace{\baselineskip}
    \subsection*{GIF}
    El \emph{Graphic Interchange Format}. Fue desarrollado por Compuserve y las variantes más utilizadas son las de GIF 87a y GIF 89a, desarrolladas en 1987 y 1989 respectivamente.
    Cada una tiene caracteristicas que las distinguen, como aquella que soporta compresión LZW, entrelazdo, paleta de 256 colores y la posibilidad de crear imágenes animadas almacenando varias
    imágenes en un mismo archivo propias del GIF 87a. El GIF 89a añade la posibilidad de definir un color transparente en la paleta y permite precisar el tiempo de las animaciones. Otras caracteristicas son:
    \begin{itemize}
        \item Pueden contener de 2 a 256 colores.
        \item El tamaño de las imágenes con esta paleta es pequeño.
        \item Permite hacer animaciones con secuencia de imágenes.
        \item Su compresión no tiene pérdidas.
    \end{itemize}\par
    \vspace{\baselineskip}
    \subsection*{PNG}
    Conocido como el \emph{Portable Graphics Format}. Usado frecuentemente como un formato de rasterizado no-comprimido alrededor del internet. Esto fue creado para reemplazar el formato GIF. Sus caracteristicas son:
    \begin{itemize}
        \item No solamente se uede hacer un color transparente, pero el grado de transparencia, llamado opacidad puede ser controlado.
        \item Soporta interlazado de imágenes y este se desenvuelve más rápido que en los GIF,
        \item La correción Gamma permite ajustar el brillo de color de una imágen  requerida por manufacturadores de monitores específicos.
        \item Las imagenes pueden ser guardadas usando color verdadero, así como en la paleta y escala de grises.
    \end{itemize}\par
    \vspace{\baselineskip}
    \subsection*{PSD}
    Es el formato para Adobe Photoshop. Como el nombre lo indica, este formato es usado principalmente en Adobe Photoshop como el formato predeterminado para almacenaje. Generalmente
    este formato soporta múltiples imágenes, objetos, filtros, texto, así como el usar capas, caminos y formas vectoriales, y transparencia. En general:
    \begin{itemize}
        \item Formato propietario de Adobe Photoshop.
        \item Soporta muchos de los aspectos esperados del editor Adobe Photoshop.
    \end{itemize}\par
    \vspace{\baselineskip}
    \section*{Conclusión}
    Tener una gran variedad de formatos a nuestra disposición nos permite elaborar una idea de los formatos más populares y saber cuando aplicarlos para optimizar nuestro flujo de trabajo, y posiblemente los resultados del mismo.\par
    \vspace{\baselineskip}
    
    \newpage
        \thispagestyle{empty}
        \printbibliography  
\end{document}