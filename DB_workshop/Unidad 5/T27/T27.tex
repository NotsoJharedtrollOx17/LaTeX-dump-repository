% Preámbulo
\documentclass[stu, 12pt, letterpaper, donotrepeattitle, floatsintext, natbib]{apa7}
\usepackage[utf8]{inputenc}
\usepackage{comment}
\usepackage{marvosym}
\usepackage{graphicx}
\usepackage{float}
\usepackage{listings}
\usepackage{color}
\usepackage{fancyvrb} % for "\Verb" macro
\usepackage{amsmath}
%Includes "References" in the table of contents
\usepackage[nottoc]{tocbibind}
\usepackage[normalem]{ulem}
\usepackage[spanish]{babel} 
\usepackage{indentfirst} %para le formato que quiere la profe QUITAR SI QUIERES OG APA7
\usepackage{ragged2e} %para le formato que quiere la profe QUITAR SI QUIERES OG APA7
\usepackage{indentfirst} %para le formato que quiere la profe QUITAR SI QUIERES OG APA7

\renewcommand\labelitemi{$\bullet$}

%Customize a bit the look
\definecolor{mygreen}{rgb}{0,0.6,0}
\definecolor{mygray}{rgb}{0.5,0.5,0.5}
\definecolor{mymauve}{rgb}{0.58,0,0.82}
\definecolor{darkgray}{rgb}{.4,.4,.4}
\definecolor{purple}{rgb}{0.65, 0.12, 0.82}
\lstset{ %
backgroundcolor=\color{white}, % choose the background color; you must add \usepackage{color} or \usepackage{xcolor}
basicstyle=\footnotesize\ttfamily, % the size of the fonts that are used for the code
breakatwhitespace=false, % sets if automatic breaks should only happen at whitespace
breaklines=true, % sets automatic line breaking
captionpos=b, % sets the caption-position to bottom
commentstyle=\color{mygreen}, % comment style
deletekeywords={...}, % if you want to delete keywords from the given language
escapeinside={\%*}{*)}, % if you want to add LaTeX within your code
extendedchars=true, % lets you use non-ASCII characters; for 8-bits encodings only, does not work with UTF-8
frame=single, % adds a frame around the code
keepspaces=true, % keeps spaces in text, useful for keeping indentation of code (possibly needs columns=flexible)
keywordstyle=\color{blue}, % keyword style
% language=Octave, % the language of the code
morekeywords={*,...}, % if you want to add more keywords to the set
numbers=left, % where to put the line-numbers; possible values are (none, left, right)
numbersep=9pt, % how far the line-numbers are from the code
numberstyle=\footnotesize, % the style that is used for the line-numbers
rulecolor=\color{black}, % if not set, the frame-color may be changed on line-breaks within not-black text (e.g. comments (green here))
showspaces=false, % show spaces everywhere adding particular underscores; it overrides 'showstringspaces'
showstringspaces=false, % underline spaces within strings only
showtabs=false, % show tabs within strings adding particular underscores
stepnumber=1, % the step between two line-numbers. If it's 1, each line will be numbered
stringstyle=\color{mymauve}, % string literal style
tabsize=2, % sets default tabsize to 2 spaces
title=\lstname % show the filename of files included with \lstinputlisting; also try caption instead of title
}
%END of listing package%


% Portada
%\thispagestyle{empty}
\title{\Large Tarea 27}
\author{Abraham J. Flores A.\: 19211640} % (autores separados, consultar al docente)
% Manera oficial de colocar los autores:
%\author{Autor(a) I, Autor(a) II, Autor(a) III, Autor(a) X}
\affiliation{Tecnológico Nacional de México | Instituto Tecnológico de Tijuana}
\course{SCC-1010SC5C Graficación}
\professor{M.C. Margarita Ramirez Ramirez}
\duedate{1 de diciembre de 2021}

\begin{document}
    % Índices
    \pagenumbering{arabic}
    \begin{figure}[ht]
      \centering
      \includegraphics[width=16cm]{logosITT.png}
    \end{figure}
    \maketitle

    %indice
    \tableofcontents 

    % Cuerpo 
    %NOTA: PARA CITAR ESTILO "Merts (2003)" usar \cite{<nombre_cita_bib>}
    %    
    \newpage
    \section*{Triggers}
    \addcontentsline{toc}{section}{Triggers}
    \subsection*{Concepto}
    \addcontentsline{toc}{subsection}{Concepto}
    A \begin{justifying}
      grandes rasgos, un Trigger es un objeto de bases de datos que se dispara cuando un
    evento ocurra dentro de una base de datos. Por poner un ejemplo, un trigger puede configurarse en
    en un registro INSERT de una tabla de la base de datos \citep{middha-no-date} %citar al hindú
     . Sus tipos son:\par
    \end{justifying}
    \begin{itemize}
      \item DDL Trigger: se ejecutan en respuesta a eventos de comandos del lenguaje de definición de datos.
      \item DML Trigger: se ejecutan en respuesta a eventos de comandos del lenguaje de manipulación de datos.
    \end{itemize}\par
    En ámbas, su sintáxis básica es la siguiente:
    \vspace{\baselineskip}
    \begin{lstlisting}[language=SQL]
      CREATE TRIGGER <nombre_trigger> ON <objeto_db>
      AS
        BEGIN
          <sentencias DDL/DML>
        END\end{lstlisting}
    \subsection*{Ejemplos de Aplicación}
    \addcontentsline{toc}{subsection}{Ejemplos de Aplicación}
    \subsubsection*{Registro de cambios dentro de una tabla}
    \addcontentsline{toc}{subsubsection}{Registro de cambios dentro de una tabla}
    Como \begin{justifying}
      se vió en clase, un muy buen ejemplo de estos gatillos es el de poder registrar
    los cambios hechos dentro de una tabla al momento que se hace una inserción, una eliminación o alguna modificación.\par
    \end{justifying}
    \vspace{\baselineskip}
    \begin{lstlisting}[language=SQL]
      --creacion de los gatillos
    --PARA 'INSERT'
    CREATE TRIGGER altaAlumno ON Alumno
    FOR INSERT
    AS
        begin
            PRINT('Alumno agregado')
            INSERT INTO bitacora VALUES
            (
                'Insercion',
                CONVERT(date, getdate()),
                CURRENT_USER
            )
        end
    GO
        --PARA 'DELETE'
    CREATE TRIGGER bajaAlumno ON Alumno
    FOR DELETE
    AS
        begin
            PRINT('Alumno eliminado')
            INSERT INTO bitacora VALUES
            (
                'Eliminacion',
                CONVERT(date, getdate()),
                CURRENT_USER
            )
        end
    GO
            --PARA 'UPDATE'
    CREATE TRIGGER modAlumno ON Alumno
    FOR UPDATE
    AS
        begin
            PRINT('Alumno modificado')
            INSERT INTO bitacora VALUES
            (
                'Modificacion',
                CONVERT(date, getdate()),
                CURRENT_USER
            )
        end
    GO\end{lstlisting}
    \subsubsection*{Notificación de cambios a la BD por medio de correos electrónicos}
    \addcontentsline{toc}{subsubsection}{Notificación de cambios a la BD por medio de correos electrónicos}
    Podemos \begin{justifying}
      usar los TRIGGER's como una herramienta que envía correos electrónicos al correo ingresado para
    notificar sobre cambios al respecto u otros detalles relevantes \citep{unknown-author-2021}.\par
    \end{justifying}
    \vspace{\baselineskip}
    \begin{lstlisting}[language=SQL]
CREATE TRIGGER reminder2  
ON Sales.Customer  
AFTER INSERT, UPDATE, DELETE   
AS  
   EXEC msdb.dbo.sp_send_dbmail  
        @profile_name = 'AdventureWorks2012 Administrator',  
        @recipients = 'danw@Adventure-Works.com',  
        @body = 'Don''t forget to print a report for the sales force.',  
        @subject = 'Reminder';  
GO\end{lstlisting}
    \subsubsection*{Mensajes de aviso al momento de modificar la BD}
    \addcontentsline{toc}{subsubsection}{Mensajes de aviso al momento de modificar la BD}
    Otra \begin{justifying}
      utilidad de estos es mostrar mensajes personalizados en la ocnsola de SQL Server para lo que se necesite.\par
    \end{justifying}
    \vspace{\baselineskip}
    \begin{lstlisting}[language=SQL]
      CREATE TRIGGER reminder1  
      ON Sales.Customer  
      AFTER INSERT, UPDATE   
      AS RAISERROR ('Notify Customer Relations', 16, 10);  
      GO
    \end{lstlisting}
    
    \newpage   
    % Referencias
    \renewcommand\refname{\textbf{Referencias}}
    \bibliography{referencias}
\end{document}