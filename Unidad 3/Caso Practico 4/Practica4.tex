\documentclass[stu, 12pt, letterpaper, donotrepeattitle, floatsintext, natbib]{apa7}
\usepackage[utf8]{inputenc}
%\usepackage{fontspec} %paquete para usar la fuente Arial 12
\usepackage{comment}
\usepackage{marvosym}
\usepackage{graphicx}
\usepackage{float}
\usepackage[normalem]{ulem}
\usepackage[spanish]{babel} 
\usepackage{lastpage} %para le formato que quiere la profe QUITAR SI QUIERES OG APA7
\usepackage{ragged2e} %para le formato que quiere la profe QUITAR SI QUIERES OG APA7
\usepackage{indentfirst} %para le formato que quiere la profe QUITAR SI QUIERES OG APA7

\setcounter{secnumdepth}{0} %permite enumerar las secciones QUITAR SI QUIERES OG APA7

%comando para ajustar la fuente Arial en todo el documento
%\setmainfont{Arial} %COMPILAR DOC CON XeLateX DOS VECES

% \DeclareCaptionLabelSeparator*{spaced}{\\[2ex]}
% \captionsetup[table]{textfont=it,format=plain,justification=justified,
%   singlelinecheck=false,labelsep=spaced,skip=1pt}

\selectlanguage{spanish}

\useunder{\uline}{\ul}{}
\newcommand{\myparagraph}[1]{\paragraph{#1}\mbox{}\\}

%\rfoot{Página \thepage \hspace{1pt} de \pageref{LastPage}}%QUITAR SI QUIERES OG APA7 
\rhead{} %QUITAR SI QUIERES OG APA7
\setcounter{secnumdepth}{3} %permite enumerar las secciones QUITAR SI QUIERES OG APA7
\setlength{\parindent}{1.27cm} %sangria forzada QUITAR SI QUIERES OG APA7

\renewcommand\labelitemi{$\bullet$}

\newcommand*\chem[1]{\ensuremath{\mathrm{#1}}}

\begin{document}
    %PORTADA
    \begin{titlepage}
        \begin{figure}[ht]
            \centering
            \includegraphics[width=15cm]{logosITT.png}
        \end{figure}
        \centering
        {\Large Tecnológico Nacional de México\\Instituto Tecnológico de Tijuana\par}
        \vspace{1cm}
        {\Large SCD-1011SC6C Ingeniería de Software\par}
        \vspace{1cm}
        {\Large Unidad 3\par}
        \vspace{2cm}
        {\Large\bfseries Caso práctico 4\par}
        \vspace{2cm}
        {\large Dra. Martha Elena Pulido\par}
        \vfill
            {\large Abraham Jhared Flores Azcona, 19211640\par}
        \vfill
        {\large 6 de abril de 2022}
    \end{titlepage}

% Índices
%\pagenumbering{arabic}
    % Contenido
%\renewcommand\contentsname{Contenido}
%\tableofcontents

% Cuerpo 
    %NOTA: PARA CITAR ESTILO "Merts (2003)" usar \cite{<nombre_cita_bib>}
    %                        "(Metz, 1978)" usar \citep{<nombre_cita_bib>}
\newpage
\section*{Preguntas del caso}
\subsection*{Identifique los distintos conceptos de calidad presentes en el caso}
 \begin{justifying}
A grandes rasgos, el hotel establecia distintos estandares respecto al servicio que se consideraba optimo para los clientes;
al pasar el tiempo, se procuró establecer un orden en los procesos asegurandose de mantener
buenas relaciones con los proveedores que, aunado a lograr su certificación ISO, les permiten concentrarse
en la satisfacción del cliente de manera más minusciosa.\par
 \end{justifying}
\vspace{\baselineskip}
\subsection*{Identifique los objetivos perseguidos en relación con cada concepto de calidad}
\begin{justifying}
El objetivo mas prevalente es el de procurar el servicio de calidad al cliente, y la pulcridad de las instalaciones. Luego
procuraron extender sus miras ante el cómo se relacionaban con los proveedores para ahorrar costos en sus insumos.\par
\end{justifying}
\vspace{\baselineskip}
\subsection*{¿Con qué objetivo se ha desarrollado el Sistema de Aseguramiento de la Calidad en el Hotel Aljar? 
¿Qué implicaciones organizativas ha tenido el desarrollo de dicho sistema?}
\begin{justifying}
Es muy claro que el objetivo es de brindar un servicio de calidad y refinar dicho aspecto; esto implica
que la empresa crease el Departamento de Calidad para supervisar el factor diferenciador del negocio.\par
\end{justifying}
\vspace{\baselineskip}
\subsection*{¿Qué acciones relacionadas con la gestión de la calidad se han llevado a cabo en el Hotel Aljar desde 1995?
 ¿Por qué se han desarrollado dichas acciones si las mismas suponen unos costes adicionales para el hotel? Clasifique
 estas acciones según la tipología de los costes de la calidad.}
 \begin{justifying}
Las acciones principales son la certificación ISO respecto a estandares de calidad, mantener aseados las instalaciones
y enfocarse en la atención al cliente, mejorar las relaciones comerciales con los proveedores para tener mayores insumos a menor precio
y sobre todo, la creación del Departamento de Calidad para supervisar lo antes realizado.\par
COnsidero que se han realizado debido a que su atención les permite diferenciarse con el resto de los hoteles, ya que como cliente
espero que en un hotel se me atienda a lo que se me solicite, por lo que la clientela esta dispuesta a pagar más por ser servido al tope.\par
 \end{justifying}
\end{document}