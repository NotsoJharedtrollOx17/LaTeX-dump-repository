% Preámbulo
\documentclass[stu, 12pt, letterpaper, donotrepeattitle, floatsintext, natbib]{apa7}
\usepackage[utf8]{inputenc}
\usepackage{comment}
\usepackage{marvosym}
\usepackage{graphicx}
\usepackage{float}
\usepackage[normalem]{ulem}
\usepackage[spanish]{babel} 
\usepackage{lastpage} %para le formato que quiere la profe QUITAR SI QUIERES OG APA7
\usepackage{ragged2e} %para le formato que quiere la profe QUITAR SI QUIERES OG APA7
\usepackage{indentfirst} %para le formato que quiere la profe QUITAR SI QUIERES OG APA7

\selectlanguage{spanish}
\useunder{\uline}{\ul}{}
\newcommand{\myparagraph}[1]{\paragraph{#1}\mbox{}\\}

\rfoot{Página \thepage \hspace{1pt} de \pageref{LastPage}}%QUITAR SI QUIERES OG APA7 
\rhead{} %QUITAR SI QUIERES OG APA7
\setcounter{secnumdepth}{3} %permite enumerar las secciones QUITAR SI QUIERES OG APA7
\setlength{\parindent}{1.27cm} %sangria forzada QUITAR SI QUIERES OG APA7

% Portada
\thispagestyle{empty}
\title{\Large Cultura y la Diversidad Cultural}
\author{Abraham Jhared Flores Azcona} % (autores separados, consultar al docente)
% Manera oficial de colocar los autores:
%\author{Autor(a) I, Autor(a) II, Autor(a) III, Autor(a) X}
\affiliation{Instituto Tecnológico de Tijuana}
\course{ACD-0908SC5C Desarrollo Sustentable}
\professor{M.C. Trinidad Castro Villa}
\duedate{6 de octubre de 2021}

\renewcommand\labelitemi{$\bullet$}

\newcommand*\chem[1]{\ensuremath{\mathrm{#1}}}

\begin{document}
\maketitle


% Índices
\pagenumbering{arabic}
    % Contenido
\renewcommand\contentsname{Contenido}
\tableofcontents

% Cuerpo 
    %NOTA: PARA CITAR ESTILO "Merts (2003)" usar \cite{<nombre_cita_bib>}
    %                        "(Metz, 1978)" usar \citep{<nombre_cita_bib>}
\newpage
\section*{Introducción}
\addcontentsline{toc}{section}{Introducción}
Como \begin{justifying}
    se mencionó en la asignación anterior, la cultura es ese pegamento que nos permite coercer una sociedad. Sin embargo el que és exactamente
    así como la importancia de la diversidad de estos no se ha tocado hasta ahora. En esta breve redacción se define la cultura y la diversidad cultural
    y los valores que rigen al Desarrollo Sustentable para profundizar nuestra enseñanza.\par
\end{justifying}
\vspace{\baselineskip}
\section{Cultura}
Para \begin{justifying}
    \cite{zimmermann-2017}
    la cultura son las características y el conocimiento de un grupo particular de personas, que incluye al lenguaje, religión, comida, hábitos sociales y las bellas artes.
    Tambien expone que otros autores incluyen que los patrones compartidos de los comportamientos e interacciones, construcciones cognitivas y entendimientos aprendidos por la socialización son
    parte de la cultura.\par
\end{justifying}
Para \begin{justifying}
    \cite{tracy-evans-santa-ana-college-no-date} %citar al de lumen
    la cultura son los patrones de comportamiento adquiridos y comportamientos y creencias de un grupo social, étnico o de edad particular. Expande que los humanos usan la cultura para
    adaptar y transformar el mundo en el que viven.\par
\end{justifying}
Para \begin{justifying}
     \cite{unknown-author-no-dateA} %citar al de la UNAM
    la cultura es uno de los conceptos más definidos de las ciencias sociales. Expone que se puede definir como el conjunto integral constituido por los utensilios y bienes de los consumidores,
    por el cuerpo de normas que rige los diversos grupos sociales, por las ideas y artesanías, creencias y costumbres.\par
\end{justifying}
\vspace{\baselineskip}
\section{Diversidad Cultural}
Es \begin{justifying}
    de suma iportancia destacar que la diversidad cultural se le atribuye al cambio constante de las culturas. No importa la culutra la cual la gente forma parte, una cosa cierta
    es la de que va a cambiar. A pesar de que el cambio es inevitable, el pasado también debe ser respetado y preservado \citep{zimmermann-2017} . %citar al zimmerman
\end{justifying}
Para \begin{justifying}
    \cite{belfield-2012} %citar al de purdue
    la diversidad cultural también puede ser conocida como \emph{multiculturalismo}; este es un sistema de creencias y comportamientos que reconocen y respetan la presencia de todos los grupos diversos en
    una organización o sociedad, reconoce y valora sus diferencias socio-culturales, e incentiva y permite su distribución continua con
    un contexto cultural inclusivo el cual empodera a todos dentro de la organización o sociedad.\par
\end{justifying}
Para \begin{justifying}
    \cite{alpert-2021} %citar al de diversityresources
    la diversidad cultural se refiere a los siete esenciales de la cultura de competencia laboral: los valores, normas y tradiciones que afectan la manera la cual un miembro
    típicamente percibe, piensa, comporta y juzga. También menciona que puede afectar la percepción de tiempo, que puede impactar la agenda y fechas de entrega de labores.\par
\end{justifying}
Para \begin{justifying}
    \cite{romero-2021} %citar a BBVA
    la diversidad cultural es la manifestación de la originalidad y la pluralidad de las identidades que caracterizan a los grupos y las sociedades que componen a la humanidad, que permiten ser
    una fuente de intercambios, innovación y creatividad.\par
\end{justifying}
\vspace{\baselineskip}
\section{Valores y principios que rigen el concepto del Desarrollo Sustentable}
Con \begin{justifying}
    visto hasta ahora, se pueden discernir los valores que rigen a la materia. Acorde a %citar al nursus
    estos incluyen:
    \begin{itemize}
        \item \emph{Suficiencia.}
        \item \emph{Equidad y justicia.}
        \item \emph{Inclusión social.}
        \item \emph{Participación y empoderamiento.}
        \item \emph{Eficiencia ecológica.}
        \item \emph{Biodiversidad y espacios eerdes.}
        \item \emph{Necesidades y Derechos Humanos.}
        \item \emph{Etica en las inversiones y un comercio justo.}
        \item \emph{Consumo sostenible.}
        \item \emph{Derechos y necesidades de los animales.}
        \item \emph{Democracia y participación.}
        \item \emph{Conservación y eficiencia de los recursos.}
        \item \emph{Comunidad y reciprocidad.}
        \item \emph{Satisfacción de las necesidades.}
        \item \emph{Durabilidad y flexibilidad.}
        \item \emph{Sistema de salud y bienestar.}
        \item \emph{Un buen futuro.}
    \end{itemize}\par
\end{justifying}
\section*{Conclusión}
\addcontentsline{toc}{section}{Conclusión}
El \begin{justifying}
    comprender el impacto de la cultura y su diversidad, así como los valores de la materia, nos permite considerar de mejor manera la razón por la cuál
se imparte la materia, la cual es tener un futuro mejor.\par
\end{justifying}

\newpage
% Referencias
\setcounter{secnumdepth}{0} %permite enumerar las secciones QUITAR SI QUIERES OG APA7
\renewcommand\refname{\textbf{Referencias}}
\bibliography{referencias} %el archivo 'referencias.bib' debe estar dentro del mismo folder donde se encuentra el archivo .tex para citar las referencias deseadas

\end{document}