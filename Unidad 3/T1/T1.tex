% Preámbulo
\documentclass[stu, 12pt, letterpaper, donotrepeattitle, floatsintext, natbib]{apa7}
\usepackage[utf8]{inputenc}
\usepackage{comment}
\usepackage{marvosym}
\usepackage{graphicx}
\usepackage{float}
\usepackage[normalem]{ulem}
\usepackage[spanish]{babel} 
\usepackage{lastpage} %para le formato que quiere la profe QUITAR SI QUIERES OG APA7
\usepackage{ragged2e} %para le formato que quiere la profe QUITAR SI QUIERES OG APA7
\usepackage{indentfirst} %para le formato que quiere la profe QUITAR SI QUIERES OG APA7

\selectlanguage{spanish}
\useunder{\uline}{\ul}{}
\newcommand{\myparagraph}[1]{\paragraph{#1}\mbox{}\\}

\rfoot{Página \thepage \hspace{1pt} de \pageref{LastPage}}%QUITAR SI QUIERES OG APA7 
\rhead{} %QUITAR SI QUIERES OG APA7
\setcounter{secnumdepth}{3} %permite enumerar las secciones QUITAR SI QUIERES OG APA7
\setlength{\parindent}{1.27cm} %sangria forzada QUITAR SI QUIERES OG APA7

% Portada
\thispagestyle{empty}
\title{\Large Sociedad y Organización Social Relacionada con la Comunidad}
\author{Abraham Jhared Flores Azcona} % (autores separados, consultar al docente)
% Manera oficial de colocar los autores:
%\author{Autor(a) I, Autor(a) II, Autor(a) III, Autor(a) X}
\affiliation{Instituto Tecnológico de Tijuana}
\course{ACD-0908SC5C Desarrollo Sustentable}
\professor{M.C. Trinidad Castro Villa}
\duedate{5 de octubre de 2021}

\renewcommand\labelitemi{$\bullet$}

\newcommand*\chem[1]{\ensuremath{\mathrm{#1}}}

\begin{document}
\maketitle


% Índices
\pagenumbering{arabic}
    % Contenido
\renewcommand\contentsname{Contenido}
\tableofcontents

% Cuerpo 
    %NOTA: PARA CITAR ESTILO "Merts (2003)" usar \cite{<nombre_cita_bib>}
    %                        "(Metz, 1978)" usar \citep{<nombre_cita_bib>}
\newpage
\section*{Introducción}
\addcontentsline{toc}{section}{Introducción}
En \begin{justifying}
    nuestros estudios de Desarrollo Sustentable, el factor humano afecta para bién y para mal al ecosistema, sin embargo su estudio como individuo
    no es de mucha utilidad para lo competente a la materia ya que el estudiarlo en conjunto nos permite un mayor panorama, lo cual se expone con los conceptos
    de sociedad y de organización social.\par
\end{justifying}
\vspace{\baselineskip}
\section{Sociedad}
\subsection{Concepto}
Acorde \begin{justifying} 
    a \cite{lumen-learning-no-date}
    la sociedad describe a un grupo de personas que viven en un área demográfica definida,
    los cuales interactuan unos con otros y que comparten una cultura común.\par
    Para \cite{copp-1992} %citar a COPP
    la sociedad se asume que es mas o menos una asociación auto-suficiente de personas las cuales
    en sus relaciones de unos a otros reconocen ciertas reglas de códigos de conducta
    como una unión en donde ellos toman parte en actuar acorde a dichas reglas.\par
    Para \cite{ellwood-1907} %citar al Ellwood
    la sociedad se define sin algún criterio consistente entre los autores; para unos sociólogos
    veteranos, la sociedad es un cuerpo de personas políticamente organizadas en un gobierno independiente. 
    Para otros, la sociedad es un grupo de personas las cuales tienen una civilización común.\par
\end{justifying}        
\vspace{\baselineskip}
\subsection{Características}
Para\begin{justifying}
    \cite{husain-no-date} %citar al de lucknow
    la sociedad tiene cinco características:
    \begin{itemize}
        \item \emph{Es abstracta:} Sus relaciones son invisibles.
        \item \emph{Existe la similitud y la diferencia:} Esto asegura la existencia y continuidad de la sociedad.
        \item \emph{Existe cooperación y conflicto:} La sociedad se basa en cooperación, pero por diferencias internas, existe conflicto entre los miembros.
        \item \emph{Es un proceso y no un producto:} La sociedad existe como una secuencia de tiempo. Se convierte a, no es un ser. 
    \end{itemize}\par
\end{justifying}
\vspace{\baselineskip}
\subsection{Ventajas y Desventajas}
Tomando \begin{justifying}
    a la sociedad como un grupo, las ventajas y desventajas se hacen evidentes \citep{torres-2021}. %cite psicologia
    \par
\end{justifying}
\vspace{\baselineskip}
\subsubsection{Ventajas}
\begin{justifying}
    \begin{itemize}
        \item Suponen un factor de protección.
        \item Nos proporciona modelos a seguir.
        \item Refuerzan nuestra autoestima.
        \item Combaten la soledad.
        \item Nos proporciona información.        
    \end{itemize}\par
\end{justifying}
\vspace{\baselineskip}
\subsubsection{Desventajas}
\begin{justifying}
    \begin{itemize}
        \item Aumentan el riesgo de contagios por enfermedades.
        \item Se exacérba la presión y el coercimiento social.
        \item Existe penalización por disidencia.
        \item Aparecen liderazgos \emph{de facto}.
        \item Legitimiza la hostilidad hacia quienes no forman parte del grupo.
    \end{itemize}
\end{justifying}
\section{Organización Social}
\subsection{Concepto}
Acorde \begin{justifying}
    a \cite{unknown-author-no-date} %citar a los community
    una organización social se refiere a la red de relaciones en un grupo y como están interconectados.
    Dicha red ayuda a los miembros del grupo estar conectados unos a otros en orden para mantener un sentido
    de comunidad dentro de este.\par
\end{justifying}
Para \begin{justifying}
    otros autores, la definición difiere en unos puntos \citep{unknown-author-2020A}. %citar al de studylecturenotes
    Se puede definir como la interdependencia de las partes las cuales son esenciales para las entidades colectivas, grupos, 
    comunidades y sociedades. Otra definición es la que plantea que una organización social es una articulación de partes distintas las cuales 
    realizan distintas funciones, es un dispotivo activo del grupo que permite realizar algo.\par
\end{justifying}
\vspace{\baselineskip}
\subsection{Características}
Para \begin{justifying}
    \cite{unknown-author-2020B} %citar a los de caracteristicas
    se contienen los siguientes puntos:
    \begin{itemize}
        \item \emph{Origen:} Estas organizaciones han existido con nosotros desde siempre.
        \item \emph{Fin específico:} Se estructuran siempre en torno a un fin último; supervivencia, administración de recursos, etc.
        \item \emph{Estructura y jerarquización:} Existe la repartición de labores y complejidades dentro de este.
        \item \emph{Patrón sinérgico e interdependencia:} Los elementos abandonan otras funcionas y las relegan acorde a lo necesario.
        \item \emph{Instituciones:} Se encargan de diversos cometidos. El más complejo es El Estado.
        \item \emph{Marco político contemporáneo:} siempre existen conjuntos de agrupaciones de participación política popular.
        \item \emph{Resistencia al desorden:} estas organizaciones sobreviven en base a resistir al crecimiento del grado de desorden en su seno.
    \end{itemize}\par
\end{justifying}
\vspace{\baselineskip}
\subsection{Ventajas y Desventajas}
Existen \begin{justifying}
    similitudes con las sociedades \citep{archit-no-date}. %citar al de archit
\end{justifying}
\subsubsection{Ventajas}
\vspace{\baselineskip}
\begin{justifying}
    \begin{itemize}
        \item \emph{Unión:} La unión hace la fuerza.
        \item \emph{Menos posibilidades de acoso:} El formar parte de un grupo mantiene menores chances de sufrir acoso por no pertenecer a uno.
        \item \emph{Confianza:} Los miembros tienen una confianza de que no están solos.
        \item \emph{Seguridad:} No es fácil el poder hacer daño a un miembro de la organización.
        \item \emph{Apoyo:} Pueden proveer apoyo financiero y emocional en tiempos de crisis.
        \item \emph{Coordinación:} Es mejor coordinarse para tener un mejor resultado de nuestra labor.
    \end{itemize}
\end{justifying}
\subsubsection{Desventajas}
\begin{justifying}
    \begin{itemize}
        \item \emph{Bullying}: Generalmente los grupos poderosos abusan de su posición para dar una desventaja a los inferiores.
        \item \emph{Conflictos entre ellos:} Los grupos que apoyan una causa pueden conflictuarse por tener visiones relativamente distintas.
        \item \emph{Roces internos:} Pueden existir roces dentro de un grupo por cuestiones de liderazgo, etc.
        \item \emph{Huelgas:} generado por los puntos anteriores.
        \item \emph{Pérdida del público:} Por conflictos, la audiencia puede perder interes por no involucrarse con lo asociado al grupo en cuestión.
    \end{itemize}\par
\end{justifying}
\vspace{\baselineskip}
\section*{Conclusión}
\addcontentsline{toc}{section}{Conclusión}
El \begin{justifying}
\end{justifying}
poder definir de manera concreta los conceptos expuestos, así como sus similitudes y diferencias de definición como de ventajas y desventas facilita razonar
el impacto socio-cultural para nuestro análisis de los temas de la materia, y por ende, considerarlo para soluciones congruentes a nuestras culturas.\par
\newpage
% Referencias
\setcounter{secnumdepth}{0} %permite enumerar las secciones QUITAR SI QUIERES OG APA7
\renewcommand\refname{\textbf{Referencias}}
\bibliography{referencias} %el archivo 'referencias.bib' debe estar dentro del mismo folder donde se encuentra el archivo .tex para citar las referencias deseadas

\end{document}