% Preámbulo
\documentclass[stu, 12pt, letterpaper, donotrepeattitle, floatsintext, natbib]{apa7}
\usepackage[utf8]{inputenc}
\usepackage{comment}
\usepackage{marvosym}
\usepackage{graphicx}
\usepackage{float}
\usepackage[normalem]{ulem}
\usepackage[spanish]{babel} 
\usepackage{lastpage} %para le formato que quiere la profe QUITAR SI QUIERES OG APA7
\usepackage{ragged2e} %para le formato que quiere la profe QUITAR SI QUIERES OG APA7
\usepackage{indentfirst} %para le formato que quiere la profe QUITAR SI QUIERES OG APA7

\selectlanguage{spanish}
\useunder{\uline}{\ul}{}
\newcommand{\myparagraph}[1]{\paragraph{#1}\mbox{}\\}

\rfoot{Página \thepage \hspace{1pt} de \pageref{LastPage}}%QUITAR SI QUIERES OG APA7 
\rhead{} %QUITAR SI QUIERES OG APA7
\setcounter{secnumdepth}{3} %permite enumerar las secciones QUITAR SI QUIERES OG APA7
\setlength{\parindent}{1.27cm} %sangria forzada QUITAR SI QUIERES OG APA7

% Portada
\thispagestyle{empty}
\title{\Large Estudio de Poblaciones}
\author{Abraham Jhared Flores Azcona} % (autores separados, consultar al docente)
% Manera oficial de colocar los autores:
%\author{Autor(a) I, Autor(a) II, Autor(a) III, Autor(a) X}
\affiliation{Instituto Tecnológico de Tijuana}
\course{ACD-0908SC5C Desarrollo Sustentable}
\professor{M.C. Trinidad Castro Villa}
\duedate{11 de octubre de 2021}

\renewcommand\labelitemi{$\bullet$}

\newcommand*\chem[1]{\ensuremath{\mathrm{#1}}}

\begin{document}
\maketitle


% Índices
\pagenumbering{arabic}
    % Contenido
\renewcommand\contentsname{Contenido}
\tableofcontents

% Cuerpo 
    %NOTA: PARA CITAR ESTILO "Merts (2003)" usar \cite{<nombre_cita_bib>}
    %                        "(Metz, 1978)" usar \citep{<nombre_cita_bib>}
\newpage
\section*{Introducción}
\addcontentsline{toc}{section}{Introducción}
Es \begin{justifying}
    un hecho que el estudio del Desarrollo Sustentable abarca distintos temas tanto del aspecto natural como del aspecto cientifico; los temas
    antropológicos no son la excepción. Uno de estos temas de antropología es aquel de los estudios y dinámicas poblacionales. En esta breve redacción se
    explican tanto la demografía como la dinámica poblacional para comprenderlos de mejor manera para nuestros estudios competentes a la materia.\par
\end{justifying}
\vspace{\baselineskip}
\section{Estudio de Poblaciones}
Para \begin{justifying}
    la \cite{unknown-author-no-dateA} %citar a la comecso
    el estudio de las poblaciones se conoce como la \emph{Demografía}, la cual la define como una ciencia
    que tiene como finalidad el estudio de la población humana y que se ocupa de su dimensión, estructura, evolución
    y caracteres generales considerados fundamentalmente desde un punto de vista cuantitativo.\par
\end{justifying}
Para \begin{justifying}
    la \cite{unknown-author-no-dateB} %citar a la Stockholm
    es el estudio de las poblaciones; su tamaño, composición y distribución atravez del espacio y el proceso por el cual
    las poblaciones cambian.\par
\end{justifying}
En \begin{justifying}
    ámbas tenemos presentes cuatro puntos característicos:
    \begin{itemize}
        \item \emph{Dimensión:} hace referencia al tamaño de la población.
        \item \emph{Estructura:} la población se estudia según distintos caracteres que la dividen en subpoblaciones de interés.
        \item \emph{Evolución:} en tamaño, evolución temporal, etc.
        \item \emph{Caracteres generales:} estado de salud, coeficiente intelectual, código genético, etc.
    \end{itemize}\par
\end{justifying}
\vspace{\baselineskip}
\section{Dinámica Poblacional}
Para \begin{justifying}
    la \cite{unknown-author-no-dateC} %semarnat
    explica con el contexto de México, los cambios expresados en la población. Un claro ejemplo de ello es la reducción de la tasa de crecimiento
    poblacional a 1.2\% hasta el 2000. Específicamente, se proyectan que las dinámicas poblacionales se encuentren en la estructura de la pirámide
    de edades, los estados del norte contienen menor cantidad de población en contraste con el sur. Una tendencia destacable es que la población tiende 
    a las urbes.\par
\end{justifying}
El \begin{justifying}
    reporte de \cite{conapo-2013} refuerza estas hipótesis. La explosión demográfica creció un 1.13 porciento. La fecundidad va en descenso, mientras que la
    esperanza de vida va en aumento así como la bonanza de juventud en comparación con la vejéz.\par 
\end{justifying}
\section*{Conclusión}
\addcontentsline{toc}{section}{Conclusión}
El \begin{justifying}
    estudio de la demografía y la dinámica de las poblaciones permite ilustrar las expectativas de crecimiento de distintos
    rubros humanos que nos permiten considerar la parte demográfica como factor igual de primordial a considerar en las integraciones del
    Desarrollo Sustentable.\par
\end{justifying}
\newpage
% Referencias
\setcounter{secnumdepth}{0} %permite enumerar las secciones QUITAR SI QUIERES OG APA7
\renewcommand\refname{\textbf{Referencias}}
\bibliography{referencias} %el archivo 'referencias.bib' debe estar dentro del mismo folder donde se encuentra el archivo .tex para citar las referencias deseadas

\end{document}