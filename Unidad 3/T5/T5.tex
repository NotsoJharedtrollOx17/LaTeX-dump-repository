% Preámbulo
\documentclass[stu, 12pt, letterpaper, donotrepeattitle, floatsintext, natbib]{apa7}
\usepackage[utf8]{inputenc}
\usepackage{comment}
\usepackage{marvosym}
\usepackage{graphicx}
\usepackage{float}
\usepackage[normalem]{ulem}
\usepackage[spanish]{babel} 
\usepackage{lastpage} %para le formato que quiere la profe QUITAR SI QUIERES OG APA7
\usepackage{ragged2e} %para le formato que quiere la profe QUITAR SI QUIERES OG APA7
\usepackage{indentfirst} %para le formato que quiere la profe QUITAR SI QUIERES OG APA7

\selectlanguage{spanish}
\useunder{\uline}{\ul}{}
\newcommand{\myparagraph}[1]{\paragraph{#1}\mbox{}\\}

\rfoot{Página \thepage \hspace{1pt} de \pageref{LastPage}}%QUITAR SI QUIERES OG APA7 
\rhead{} %QUITAR SI QUIERES OG APA7
\setcounter{secnumdepth}{3} %permite enumerar las secciones QUITAR SI QUIERES OG APA7
\setlength{\parindent}{1.27cm} %sangria forzada QUITAR SI QUIERES OG APA7

% Portada
\thispagestyle{empty}
\title{\Large Diferentes fenómenos naturales y los desastres}
\author{Abraham Jhared Flores Azcona} % (autores separados, consultar al docente)
% Manera oficial de colocar los autores:
%\author{Autor(a) I, Autor(a) II, Autor(a) III, Autor(a) X}
\affiliation{Instituto Tecnológico de Tijuana}
\course{ACD-0908SC5C Desarrollo Sustentable}
\professor{M.C. Trinidad Castro Villa}
\duedate{19 de octubre de 2021}

\renewcommand\labelitemi{$\bullet$}

\newcommand*\chem[1]{\ensuremath{\mathrm{#1}}}

\begin{document}
\maketitle


% Índices
\pagenumbering{arabic}
    % Contenido
\renewcommand\contentsname{Contenido}
\tableofcontents

% Cuerpo 
    %NOTA: PARA CITAR ESTILO "Merts (2003)" usar \cite{<nombre_cita_bib>}
    %                        "(Metz, 1978)" usar \citep{<nombre_cita_bib>}
\newpage
\section*{Introducción}
\addcontentsline{toc}{section}{Introducción}
Uno \begin{justifying}
   de los motivadores del estudio del Desarrollo SUstentable es aquel de los fenómenos naturales así como los respectivos desastres.
   En esta breve redacción se explican ambos conceptos así como su importancia para esta materia como futura referencia.\par
\end{justifying}
\vspace{\baselineskip}
\section{Fenomenos Naturales}
\subsection{Concepto}
Un \begin{justifying}
    fenómeno natural es aquel que se puede detectar por los sentidos y no es creado por el hombre. Mas allá,
    no es una manifestación que se expresa a travéz de la intuición o el razonamiento \citep{unknown-author-no-dateA}. \par %citar a los de la UE
\end{justifying}
\subsection{Importancia}
Como \begin{justifying}
    lo mencionan \citep{brown-no-date}%citar sdcof
    un aspecto muy relevante de dichos fenomenos es el poder instruir a las ciencias en base a ellos.
    No necesariamente los fenomenos deben de ser impresionantes ya que al observarlos con el enfoque científico,
    puede incitar a la curiosidad y al cuestionamiento de los estudiantes y el público general.\par
\end{justifying}
Esto \begin{justifying}
    se relaciona con la materia y el tema en cuestión debido a que para el incitar una importancia, es necesario mostrar
    dichos fenómenos para apreciar lo que debemos cuidar y por ende, instruir para concientizar a la población.\par
\end{justifying}
\vspace{\baselineskip}
\subsection{Ejemplos}
\begin{justifying}
    Por fines de brevedad, se muestran dos ejemplos de fenómenos naturales: \emph{relámpagos volcanicos y las bahías bioluminiscentes.}\par     
\end{justifying}
\vspace{\baselineskip}
\subsubsection{Relámpagos Volcanicos}
Nadie \begin{justifying}
    sabe exactamente el porque dicho fenomeno ocurre, pero científicos teorízan que el responssble es la separación de cargas.
    Al momento que el magma, ceniza y roca colisionan con el aire frío, las partículas opuestamente cargadas se separan de cada una por lo que
    el trueno se forma para balancear la distribución de cargas \citep{unknown-author-2018}. %citar al articulo ese
\end{justifying}
\vspace{\baselineskip}
\subsubsection{Bahías bioluminiscentes}
Dicha \begin{justifying}
    luz es producida por los microorganismos llamados fitopláncton, que iluman las playas alrededor del mundo. Dicha vista de cuento de hadas
    sucede cuando un pigmento llamado luciferina reacciona con el oxigeno, mientras que una enzima llamada liferasa agiliza el proceso \citep{unknown-author-2018}.\par %citar al de la revista
\end{justifying}
\vspace{\baselineskip}
\section{Desastres Naturales}
\subsection{Concepto}
Acorde \begin{justifying}
    a \cite{unknown-author-2021} %citar a los del cbp
    los desastres naturales incluye a todos los tipos de clima severo, los cuales tienen el potencial de poseer un riesgo
    significativo a la salúd y seguridad humana, propiedad, infraestructura crítica y la seguridad del territorio. Estos pueden
    ocurrir por temporada y sin aviso, subjetando a la nación a periodos frecuentes de inseguridad, disrupción y pérdida económica.\par
\end{justifying}
\vspace{\baselineskip}
\subsection{Importancia}
Lo \begin{justifying}
    más relevante a nivel global es que países industrializados han reducido los impactos
    de los distintos desastres naturales debido a la aplicación de medidas mitigantes como zonas 
    de exclusión, mejoramiento de infraestrucutra acorde a los desastres naturales más frecuentes
    de cada zona e instalación de sistemas predictivos, de monitoreo, de alarma y de evacuación.\par
\end{justifying}
Esto \begin{justifying}
    también incluye a las medidas mitigantes de estragos económicos en cuestiones de desarrollo. Los
    desastres naturales generan una gran demanda de capital para la reparación/reemplazo de los daños;
    esto es donde la población debe involucrarse más para ello ya que esta fase representa el aspecto más manejable
    de los estragos de dichos desastres \citep{unknown-author-no-dateB}.\par %citar al de la OEA
\end{justifying}
Irónicamente, \begin{justifying}
    acorde a \cite{RME529} %cite al de revista
    los esfuerzos por la mitigación de los daños van aumentando debido a que la conducta humana
    está alterando el medio ambiente y los ciclos ecológicos, como bién se ha visto en la materia.\par %citar a la borbon
\end{justifying}
\vspace{\baselineskip}
\subsection{Ejemplos}
Por \begin{justifying}
    fines de brevedad, se explican dos ejemplos: \emph{tornados y los incendios forestales.}\par
\end{justifying}
\subsubsection{Tornados y Torméntas Severas}
Los \begin{justifying}
    tornados son los descontroles de las tormentas potentes que aparecen como nubes rotativas con forma de embudo.
    Se extienden desde una tormenta hasta el suelo con vientos violentos que promedian las 30 millas por hora. Estas
    tormentas destructivas pueden atacar con poco o nulo aviso \citep{unknown-author-no-dateC}.\par %citar al de samhsa
\end{justifying}
\vspace{\baselineskip}
\subsubsection{Incendios forestales}
Son \begin{justifying}
    usualmente causados por truenos o accidentes y generalmente pasan desapercibidos. Se pueden esparcer rápidamente
    y son especialmente destructivos si ocurren en los anteriormente mencionados bosques, areas rurales,
    sitios montañosos remotos, y otros lugares con madera donde la gente vive \cite{unknown-author-no-dateD}.\par %citar a los de samhsa
\end{justifying}
\vspace{\baselineskip}
\section*{Conclusión}
Revisar \begin{justifying}
    los distintos fenomenos naturales que podemos preservar así como los desastres naturales que podemos mitigar y respetar por su potencia
    permite darnos la motivación de seguir estudiando el Desarrollo Sustentable. \par 
\end{justifying}
\newpage
% Referencias
\setcounter{secnumdepth}{0} %permite enumerar las secciones QUITAR SI QUIERES OG APA7
\renewcommand\refname{\textbf{Referencias}}
\bibliography{referencias} %el archivo 'referencias.bib' debe estar dentro del mismo folder donde se encuentra el archivo .tex para citar las referencias deseadas

\end{document}