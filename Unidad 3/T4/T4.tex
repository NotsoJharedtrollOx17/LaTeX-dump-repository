% Preámbulo
\documentclass[stu, 12pt, letterpaper, donotrepeattitle, floatsintext, natbib]{apa7}
\usepackage[utf8]{inputenc}
\usepackage{comment}
\usepackage{marvosym}
\usepackage{graphicx}
\usepackage{float}
\usepackage[normalem]{ulem}
\usepackage[spanish]{babel} 
\usepackage{lastpage} %para le formato que quiere la profe QUITAR SI QUIERES OG APA7
\usepackage{ragged2e} %para le formato que quiere la profe QUITAR SI QUIERES OG APA7
\usepackage{indentfirst} %para le formato que quiere la profe QUITAR SI QUIERES OG APA7

\selectlanguage{spanish}
\useunder{\uline}{\ul}{}
\newcommand{\myparagraph}[1]{\paragraph{#1}\mbox{}\\}

\rfoot{Página \thepage \hspace{1pt} de \pageref{LastPage}}%QUITAR SI QUIERES OG APA7 
\rhead{} %QUITAR SI QUIERES OG APA7
\setcounter{secnumdepth}{3} %permite enumerar las secciones QUITAR SI QUIERES OG APA7
\setlength{\parindent}{1.27cm} %sangria forzada QUITAR SI QUIERES OG APA7

% Portada
\thispagestyle{empty}
\title{\Large Desarrollo Urbano y Rural}
\author{Abraham Jhared Flores Azcona} % (autores separados, consultar al docente)
% Manera oficial de colocar los autores:
%\author{Autor(a) I, Autor(a) II, Autor(a) III, Autor(a) X}
\affiliation{Instituto Tecnológico de Tijuana}
\course{ACD-0908SC5C Desarrollo Sustentable}
\professor{M.C. Trinidad Castro Villa}
\duedate{12 de octubre de 2021}

\renewcommand\labelitemi{$\bullet$}

\newcommand*\chem[1]{\ensuremath{\mathrm{#1}}}

\begin{document}
\maketitle


% Índices
\pagenumbering{arabic}
    % Contenido
\renewcommand\contentsname{Contenido}
\tableofcontents

% Cuerpo 
    %NOTA: PARA CITAR ESTILO "Merts (2003)" usar \cite{<nombre_cita_bib>}
    %                        "(Metz, 1978)" usar \citep{<nombre_cita_bib>}
\newpage
\section*{Introducción}
\addcontentsline{toc}{section}{Introducción}
Continuando \begin{justifying}
    con los temas demográficos, es de vital importancia el entender como las urbes y el campo son diferentes para comprender el desarrollo humano
    a una mejor escala. Por ello en esta breve redacción se explican tanto el desarrollo urbano como el rural, así como sus indicadores para tener mejores
    estadísticas acorde a lo necesitado para la materia.\par
\end{justifying}
\vspace{\baselineskip}
\section{Desarrollo Urbano}
\subsection{Concepto}
Para \begin{justifying}
    \cite{corvo-2020}
    el Desarrollo Urbano es el crecimiento de los servicios básicos y su calidad en las ciudades para la población
    de bajos recursos. Esto es porque las áreas urbanas son sitios de innovación, porque pueden disfrutar los beneficios
    de la proximidad por la densidad poblacional y de negocios. Esto también las hace acreedoras de ciertas cargas
    como la contaminación o la falta de vivienda.\par
\end{justifying}
Este \begin{justifying}
    también incluye el cubrir infraestructura para la educación, salúd, justicia, desechos sólidos, mercados, pavimentos y protección de herencias
    culturales. Generalmente dichas construcciones forman parte de programas de sector específicos, incluyendo las medidas de capacidad de construcción.\par
\end{justifying}
Para \begin{justifying}
    \cite{unknown-author-no-dateA} %citar el banco mundial
    las ciudades juegan un papel importante en el combate al cambio climático, debido a que su exposición al clima y a los riesgos de desastre aumentan mientras
    dichas urbes crecen. Es de relevancia destacar que el 90\% de la expansión urbana en paises en vias de desarrollo ocurre en zonas de alto
    riesgo y construidas en asentamientos irregulares y no planeados.\par
\end{justifying}
\vspace{\baselineskip}
\subsection{Indicadores}
Los \begin{justifying}
    indicadores mostrados son aquellos que el Banco Mundial utiliza para medir el desarrollo urbano \citep{mavric-2015}. %citar a mavric y bobek
    Estos son:
    \begin{itemize}
        \item \emph{Proporción de la población urbana con acceso a servicios de salúd mejorados.}
        \item \emph{Proporción de la población urbana con acceso a recursos de agua.}
        \item \emph{Número de vehículos motorizadoss por cada 1000 habitantes.}
        \item \emph{Número de automóviles de pasajeros por cada 1000 habitantes.}
        \item \emph{Emisiones de partículas inhalables menores o iguales a \(10_{\mu m}\) (en microgramos por \(m^3\)).}
        \item \emph{Proporción de la pobreza.}
        \item \emph{Precios del combustible.}
        \item \emph{Consumo de combustible per capita.}
        \item \emph{Porcentaje de consumo de combustible de la población urbana.}
    \end{itemize}\par
\end{justifying}
\vspace{\baselineskip}
\section{Desarrollo Rural}
\subsection{Concepto}
Para \begin{justifying}
    \cite{corvo-2020} %citar otra vez al de lifeder
    el Desarrollo Urbano es similar a su contraparte urbana, ya que prioriza iniciativas y acciones para mejorar la calidad
    de vida de las colectividades rurales. Estas comunidades generalmente coinciden en tener una densidad poblacional muy baja.
    También incluye el fortalezer en todos los aspectos posibles a los agricultores así como aumentar el rendimiento sin mucho esfuerzo
    de sus cultivos/ganado, procurando la conservación del ambiente y que los sistemas de producción y técnicas empleadas respeten al medio
    natural y al legado histórico de la zona.\par
\end{justifying}
Es \begin{justifying}
    de vital importancia coordinar las iniciativas en materia de desarrollo rural para que puedan contribuir a estilos de vida sustentables
    con esfuerzos a nivel global, regional, nacional y local conforme a lo apropiado; esto también incluye el potencial y la lejanía en dichas areas.
    Similarmente, en el aspecto económico es importante mantener un sector agricultor sano y dinámico para generar enlaces fuertes hacia otros sectores
    económicos \citep{unknown-author-no-dateB}.\par %citar al de la ONU
\end{justifying}
\vspace{\baselineskip}
\subsection{Indicadores}
Para \begin{justifying}
    estos, usaremos algunos de los descritos por \cite{statistics-austria-2006}, %citar a los austriacos
    específicamente los acordes a la demografía, migración, economía y capital humano.\par
\end{justifying}
\vspace{\baselineskip}
\subsubsection{Demografía y Migración}
\begin{justifying}
    \begin{itemize}
        \item \emph{Dinámica poblacional acorde al cambio total de la población.}
        \item \emph{Migración neta según la migración específica por edad acorde al género.}
        \item \emph{Cambió natural de población por la proporción nacimientos/decesos y su cambio porcentual.}
    \end{itemize}\par
\end{justifying}
\vspace{\baselineskip}
\subsubsection{Economía y Capital Humano}
\begin{justifying}
    \begin{itemize}
        \item \emph{Formas de empleo acorde al porcentaje de personas empleadas.}
        \item \emph{Importancia de los distintos sectores acorde al empleo en sectores rurales.}
        \item \emph{Importancia del sector público seguún el porcentaje de empleo en el sector público por género.}
    \end{itemize}\par
\end{justifying}
\vspace{\baselineskip}
\subsubsection{Estructura Económica y su Rendimiento (Agricultura y Forestación)}
\begin{justifying}
    \begin{itemize}
        \item \emph{Acceso a caminos automóviles y vias locomotoras acorde a EUROSTAT.}
        \item \emph{Oferta de las escuelas según el número de escuelas por cada 100000 habitantes.}
        \item \emph{Proximidad a las escuelas primarias calculado por EUROSTAT.}
    \end{itemize}\par
\end{justifying}
\vspace{\baselineskip}
\subsubsection{Bienestar social}
\begin{justifying}
    \begin{itemize}
        \item \emph{Riqueza relativa de la población acorde al PIB per capita.}
        \item \emph{Pobreza acorde a la tasa de riesgo de pobreza.}
        \item \emph{Calidad de vida según la acomodación y condiciones de vivienda.}
    \end{itemize}\par
\end{justifying}
\vspace{\baselineskip}
\section*{Conclusión}
\addcontentsline{toc}{section}{Conclusión}
El \begin{justifying}
    poder entender las diferencias y similitudes del desarrollo urbano y rural, así como los indicadores que se pueden aplicar
    permiten tener estadísticas fidedignas y refutables ó contrastantes para evitar un estancamiento estadístico y poder analizar dichos desarrollos
    en distintas perspectivas para generar soluciones acordes a sus necesidades así como aquellas del Desarrollo Sustentable.\par
\end{justifying}

\newpage
% Referencias
\setcounter{secnumdepth}{0} %permite enumerar las secciones QUITAR SI QUIERES OG APA7
\renewcommand\refname{\textbf{Referencias}}
\bibliography{referencias} %el archivo 'referencias.bib' debe estar dentro del mismo folder donde se encuentra el archivo .tex para citar las referencias deseadas

\end{document}