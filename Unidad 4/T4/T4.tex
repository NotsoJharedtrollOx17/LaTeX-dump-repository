% Preámbulo
\documentclass[stu, 12pt, letterpaper, donotrepeattitle, floatsintext, natbib]{apa7}
\usepackage[utf8]{inputenc}
\usepackage{comment}
\usepackage{marvosym}
\usepackage{graphicx}
\usepackage{float}
\usepackage[normalem]{ulem}
\usepackage[spanish]{babel} 
\usepackage{lastpage} %para le formato que quiere la profe QUITAR SI QUIERES OG APA7
\usepackage{ragged2e} %para le formato que quiere la profe QUITAR SI QUIERES OG APA7
\usepackage{indentfirst} %para le formato que quiere la profe QUITAR SI QUIERES OG APA7

\selectlanguage{spanish}
\useunder{\uline}{\ul}{}
\newcommand{\myparagraph}[1]{\paragraph{#1}\mbox{}\\}

\rfoot{Página \thepage \hspace{1pt} de \pageref{LastPage}}%QUITAR SI QUIERES OG APA7 
\rhead{} %QUITAR SI QUIERES OG APA7
\setcounter{secnumdepth}{3} %permite enumerar las secciones QUITAR SI QUIERES OG APA7
\setlength{\parindent}{1.27cm} %sangria forzada QUITAR SI QUIERES OG APA7

% Portada
\thispagestyle{empty}
\title{\Large Producto Interno }
\author{Abraham Jhared Flores Azcona} % (autores separados, consultar al docente)
% Manera oficial de colocar los autores:
%\author{Autor(a) I, Autor(a) II, Autor(a) III, Autor(a) X}
\affiliation{Instituto Tecnológico de Tijuana}
\course{ACD-0908SC5C Desarrollo Sustentable}
\professor{M.C. Trinidad Castro Villa}
\duedate{8 de noviembre de 2021}

\renewcommand\labelitemi{$\bullet$}

\newcommand*\chem[1]{\ensuremath{\mathrm{#1}}}

\begin{document}
\maketitle


% Índices
\pagenumbering{arabic}
    % Contenido
\renewcommand\contentsname{Contenido}
\tableofcontents
\renewcommand{\listfigurename}{Ecuaciones}
\listoffigures

% Cuerpo 
    %NOTA: PARA CITAR ESTILO "Merts (2003)" usar \cite{<nombre_cita_bib>}
    %                        "(Metz, 1978)" usar \citep{<nombre_cita_bib>}
\newpage
\section{Producto Interno Bruto}
\subsection{Concepto}
Acorde \begin{justifying}
    a \cite{arias-2021A}%citar al de economipedia
    es un indicador económico que refleja el valor monetario de todos los bienes y servicios finales producidos por
    un territorio en un determinado periodo de tiempo; específicamente se usa para medir la riqueza generada de un país.\par
\end{justifying}
Para \begin{justifying}
  \cite{banco-de-mexico-no-date}%citar al banco de méxico
  es simplemente una forma de medir el crecimiento económico de un país. Agrega que el PIB solo contabiliza los bienes y servicios y se 
  considera como ``interno'' ya que solo se consideran los biene y servicios producidos dentro de un país.\par
\end{justifying}
Para \begin{justifying}
  \cite{lorente-2020}%citar a la de bbva
  es el valor de mercado de todos los bienes y servicios finales producidos en el interior de un país durante un periodo de tiempo determinado. Agrega
  que dicho indicador refleja la evolución económica de país.\par
\end{justifying}
\vspace{\baselineskip}
\subsection{Distribución en México}
Acorde a \begin{justifying}
  lo recabado por \cite{rombiola-2020} %citar al de la economia
  las estadísticas del PIB indican que el PIB se distribuye en grandes porcentajes en las actividades primarias; esto implíca que nuestra economía
   y su cecimiento depende bastante de la recolección de materias primas para el desarrollo del país; las implicaciones respecto a la sustentabilidad
   de nuestra economía se hacen aparentes por el simple hecho de que el PIB de nuestro país tiene un peso enorme con el sector primario, lo que incentiva a seguir
  teniendo una economía de explotación natural.\par
\end{justifying}

\newpage
% Referencias
\setcounter{secnumdepth}{0} %permite enumerar las secciones QUITAR SI QUIERES OG APA7
\renewcommand\refname{\textbf{Referencias}}
\bibliography{referencias} %el archivo 'referencias.bib' debe estar dentro del mismo folder donde se encuentra el archivo .tex para citar las referencias deseadas

\end{document}