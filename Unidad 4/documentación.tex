% Preámbulo
\documentclass[stu, 12pt, letterpaper, donotrepeattitle, floatsintext, natbib]{apa7}
\usepackage[utf8]{inputenc}
\usepackage{comment}
\usepackage{marvosym}
\usepackage{graphicx}
\usepackage{float}
\usepackage{color}
\usepackage{fancyvrb} % for "\Verb" macro
\usepackage{amsmath}
%Includes "References" in the table of contents
\usepackage[nottoc]{tocbibind}
\usepackage[normalem]{ulem}
\usepackage[spanish]{babel} 
\usepackage{indentfirst} %para le formato que quiere la profe QUITAR SI QUIERES OG APA7
\usepackage{ragged2e} %para le formato que quiere la profe QUITAR SI QUIERES OG APA7
\usepackage{indentfirst} %para le formato que quiere la profe QUITAR SI QUIERES OG APA7

\renewcommand\labelitemi{$\bullet$}

% Portada
%\thispagestyle{empty}
\title{\Large Instalación de Máquiina Virtual y Sistemas Operativos}
\author{Abraham J. Flores A.\: 19211640} % (autores separados, consultar al docente)
% Manera oficial de colocar los autores:
%\author{Autor(a) I, Autor(a) II, Autor(a) III, Autor(a) X}
\affiliation{Tecnológico Nacional de México | Instituto Tecnológico de Tijuana}
\course{SCC-1010SC5C Graficación}
\professor{Ing. Erasmo Estrada Peña}
\duedate{13 de diciembre de 2021}

\begin{document}
    % Índices
    \pagenumbering{arabic}
    \begin{figure}[ht]
      \centering
      \includegraphics[width=16cm]{logosITT.png}
    \end{figure}
    \maketitle

    %indice
    \tableofcontents

    % Cuerpo 
    %NOTA: PARA CITAR ESTILO "Merts (2003)" usar \cite{<nombre_cita_bib>}
    %    
    \newpage
    \section*{Introducción}
    \addcontentsline{toc}{section}{Introducción}
    \subsection*{¿Qué es una Máquina Virtual?}
    \addcontentsline{toc}{subsection}{¿Qué es una Máquina Virtual?}
    Una \begin{justifying}
      máquina virtual es un recurso computacional que usa software en vez de una computadora física para
    ejecutar programas y desplegar aplicaciones. Una o muchas máquinas ``huesped'' corren dentro de una máquina
    ``anfitriona'' en donde cada máquina virtual esta ejecutando su propio sistema operativo y sus respectivas funciones
    separadamente de otras máquinas virtuales, aúnque éstas estén dentro del mismo anfitrión. Lo anterior significa que
    una máquina virtual con MacOS puede ejecutarse dentro de una computadora física \citep{unknown-author-no-dateA}.\par
    \end{justifying}
    \vspace{\baselineskip}
    \subsection*{¿Qué es un Sistema Operativo?}
    \addcontentsline{toc}{subsection}{¿Qué es un Sistema Operativo?}
    Un \begin{justifying}
      sistema operativo es un programa que actua como una interfáz entre los componentes físicos de la computador y el usuario. Cualquier
    sistema computacional debe tener al menos un sistema operativo para que las aplicaciones puedan usar los recursos del computador sin muchos problemas \citep{williams-no-date}.\par
    \end{justifying}
    \vspace{\baselineskip}
    \section*{Instalación de una Máquina Virtual}
    \addcontentsline{toc}{section}{Instalación de una Máquina Virtual}
    \subsection*{Software de Máquina Virtual Instalado}
    \addcontentsline{toc}{subsection}{Software de Máquina Virtual Instalado}
    \subsection*{Proceso de Instalación}
    \addcontentsline{toc}{subsection}{Proceso de Instalación}
    \subsubsection*{Paso 1}
    \addcontentsline{toc}{subsubsection}{Paso 1}
    Se \begin{justifying}
      necesita el software de la máquina virtual. Dentro de un buscador en linea se busca ``Maquinas Virtuales'' y se descarga el más asequible, que 
    en este caso es Oracle VM VirtualBox, el cual es gratuito. Su tiempo de descarga en mucho menor comparado a los sistemas operativos seleccionados.\par
    \end{justifying}
    \vspace{\baselineskip}
    \subsubsection*{Paso 2}
    \addcontentsline{toc}{subsubsection}{Paso 2}
    Habiendo \begin{justifying}
      descargado VirtualBox, procedemoos a ejecutar el programa, dentro del programa buscamos la opción de ``Nuevo'' para la creación
    de nuestra maquina virtual a la cual queremos instalar el Dual Boot.\par
    \end{justifying}
    En \begin{justifying}
      el menu que aparece, se procede a llenar/seleccionar lo que se necesite. En este caso, el tipo de sistema operativo y version son Linux de 64 bits.\par
    \end{justifying}
    \section*{Instalación de un Sistema Operativo}
    \addcontentsline{toc}{section}{Instalación de Un Sistema Operativo}
    \subsection*{Sistemas Operativos Instalados}
    \addcontentsline{toc}{subsection}{Sistemas Operativos Instalados}
    Los sistemas operativos instalados en la máquina virtual fueron Linux Mint Xfce y Ubuntu.\par
    \subsubsection*{Linux Mint Xfce}
    \addcontentsline{toc}{subsubsection}{Linux Mint Xfce}
    Es \begin{justifying}
      una distribución de Linux Mint ``más estable y más ligera'' \citep{unknown-author-no-dateB}. Esto es debido a que, en comparación de otras distribuciones de 
    Linux Mint, esta posee menos caracteristicas que Cinammon o MATE, lo cuál lo hace más estable y por ende más ligero en uso derecursos.\par
    La principal razón por la cuál se decidió instalar dicho Sistema Operativo fue porque este sistema operativo permitia instalar sin muchos problemas
    el menú GNU Boot Loader que permite elegír cualesquiera sistemas operativos que se encuentren dentro de un dispositivo de almacenamiento al momento de la instalación,
    dejando listo los posibles sistemas operativos a usar en el arranque de la máquina virtual.\par 
    \end{justifying}
    \vspace{\baselineskip}
    \subsubsection*{Ubuntu}
    \addcontentsline{toc}{subsubsection}{Ubuntu}
    Es \begin{justifying}
      una distribución muy popular de Linux basada en Debian \citep{unknown-author-2018}. Es facil de manejar, se actualiza frecuentemente, se actualiza frecuentemente,
    la instalación del sistema operativo es muy simple así como la instalación de los programas. La razón por la cuál se eligió
    este Sistema Operativo fue porque en un tutoríal de Youtube se mostró el como instalar dos sistemas operativos dentro de una máquina virtual y el propósito
    del haber instalado Ubuntu dentro de la misma máquina virtual fue para establecer las particiones del almacenamiento de la máquina virtual y así preparar
    los detalles de cada partición para una instalación fructífera tanto de Ubuntu como el segundo sistema operativo elegido.\par
    \end{justifying}
    \vspace{\baselineskip}
    \subsection*{Proceso de Instalación}
    \addcontentsline{toc}{subsection}{Proceso de Instalación}
    A continuación se muestra el listado de pasos seguidos en el proceso de instalación. Cada uno se explica con mayor detalle a lo largo de esta sección.\par
    \begin{enumerate}
      \item Descargar los archivos .iso de los sistemas operativos elegidos.
      \item Seleccionar las opciones de una máquina virtual previamente hecha; elegír ``Almacenamiento'' y elegir la unidad de disco para cargar el .iso.
      \item Iniciar la máquina virtual y de ahí seguir las instrucciones de instalación.
      \item Hasta llegar a ``Lugar de instalación'', seleccionar ``Personalizado'' y crear las particiones correspondientes.
      \item Instalar el .iso dentro de una partición.
      \item Al terminar la primer instalación, apagar la máquina virtual, abrir su ocnfiguración y volver a cargar en disco la segunda imágen .iso.
      \item Iniciar la máquina virtual y seguir las isntrucciones de instalación del segundo sistema operativo.
      \item En el lugar de instalación seleccionar la segunda partición realizada con anterioridad.
      \item Al terminar la instalación, volver a a apagar la máquina virtual, volverla a prender y confirmar si aparece el menú de GNU Boot Loader con el nombre de los SO instalados.
    \end{enumerate}
    \vspace{\baselineskip}
    \subsubsection*{Paso 1}
    \addcontentsline{toc}{subsubsection}{Paso 1}
    Como \begin{justifying}
      parte primordial, es totalmente necesario descargar los archivos .iso de los sistemas operativos elegidos. Esto simplemente requiere de buscar
    en cualquier buscador en linea el nombre del sistema operativo deseado y seleccionar la versión más reciente. Esto se debe de realizar con cierta anterioridad
    debido a que estos archivos tienden a tardar bastante tiempo en descargar.\par
    \end{justifying}
    \subsubsection*{Paso 4}
    \addcontentsline{toc}{subsubsection}{Paso 4}
    Seleccionando \begin{justifying}
      la máquina virtual previamente configurada, seleccionamos la opción de ``Configuración''. Nos va aparecer un menú con distintas opciones, elegimos la opción de ``Almacenamiento''.\par
    Seleccionamos dentro de la opción ``IDE'' el disco vacio. A la derecha nos aparece la opción de atributos donde selecionaremos el botón de disco que va a desplegar las opciones de .iso disponibles y procedemos
    a elegír el distro de Ubuntu.\par
    \end{justifying}
    \subsubsection*{Paso 5}
    \addcontentsline{toc}{subsubsection}{Paso 5}
    Ahora cerramos la ventana de configuración y arrancamos la máquina virtual con el botón Iniciar.\par
    Posteriormente va a cargar el distro de Ubuntu para instalarlo.\par
    Estando \begin{justifying}
      en las opciones de instalaciones, seguimos las instrucciones de instalación hasta llegar al punto de la localización de la instalación del Sistema Operativo. En esta parte elegimos la opción de
    ``Especificar Particiones Manualmente'' para preparar el almacenamiento de la Máquina Virtual para ámbos Sistemas Operativos.\par
    \end{justifying}
    Dentro \begin{justifying}
      del administrador de las particiones, procedemos a crear una nueva tabla de particiones donde crearemos la primer partición, de preferencia que tenga un espacio de más o menos la mitád del espacio total del 
    almacenamiento de la máquina virtual. Se debe cerciorar que el lugar de montaje de la primer partición debe ser en ``/'' para evitar problemas de arranque.\par
    \end{justifying}
    Procedemos \begin{justifying}
      a realizar otra partición con un tamaño similar a la anterior el cual la vamos a montar en ``/home''. De ahí solamente seguimos las indicaciones restantes para instalar correctamente el sistema operativo.\par
    \end{justifying}
    \subsubsection*{Paso 6}
    \addcontentsline{toc}{subsubsection}{Paso 6}
    \subsubsection*{Paso 7}
    \addcontentsline{toc}{subsubsection}{Paso 7}
    \subsubsection*{Paso 8}
    \addcontentsline{toc}{subsubsection}{Paso 8}
    \subsubsection*{Paso 9}
    \addcontentsline{toc}{subsubsection}{Paso 9}
    \subsubsection*{Paso 10}
    \addcontentsline{toc}{subsubsection}{Paso 10}
    \subsubsection*{Paso 11}
    \addcontentsline{toc}{subsubsection}{Paso 11}
    
    \newpage   
    % Referencias
    \renewcommand\refname{\textbf{Referencias}}
    \bibliography{referencias}
\end{document}