% Preámbulo
\documentclass[stu, 12pt, letterpaper, donotrepeattitle, floatsintext, natbib]{apa7}
\usepackage[utf8]{inputenc}
\usepackage{comment}
\usepackage{marvosym}
\usepackage{graphicx}
\usepackage{float}
\usepackage[normalem]{ulem}
\usepackage[spanish]{babel} 
\usepackage{lastpage} %para le formato que quiere la profe QUITAR SI QUIERES OG APA7
\usepackage{ragged2e} %para le formato que quiere la profe QUITAR SI QUIERES OG APA7
\usepackage{indentfirst} %para le formato que quiere la profe QUITAR SI QUIERES OG APA7

\selectlanguage{spanish}
\useunder{\uline}{\ul}{}
\newcommand{\myparagraph}[1]{\paragraph{#1}\mbox{}\\}

\rfoot{Página \thepage \hspace{1pt} de \pageref{LastPage}}%QUITAR SI QUIERES OG APA7 
\rhead{} %QUITAR SI QUIERES OG APA7
\setcounter{secnumdepth}{3} %permite enumerar las secciones QUITAR SI QUIERES OG APA7
\setlength{\parindent}{1.27cm} %sangria forzada QUITAR SI QUIERES OG APA7

% Portada
\thispagestyle{empty}
\title{\Large The Story of Stuff }
\author{Abraham Jhared Flores Azcona} % (autores separados, consultar al docente)
% Manera oficial de colocar los autores:
%\author{Autor(a) I, Autor(a) II, Autor(a) III, Autor(a) X}
\affiliation{Instituto Tecnológico de Tijuana}
\course{ACD-0908SC5C Desarrollo Sustentable}
\professor{M.C. Trinidad Castro Villa}
\duedate{9 de noviembre de 2021}

\renewcommand\labelitemi{$\bullet$}

\newcommand*\chem[1]{\ensuremath{\mathrm{#1}}}

\begin{document}
\maketitle


% Índices
\pagenumbering{arabic}
    % Contenido
\renewcommand\contentsname{Contenido}
\tableofcontents
\renewcommand{\listfigurename}{Ecuaciones}
\listoffigures

% Cuerpo 
    %NOTA: PARA CITAR ESTILO "Merts (2003)" usar \cite{<nombre_cita_bib>}
    %                        "(Metz, 1978)" usar \citep{<nombre_cita_bib>}
\newpage
\section{Obsolescencia Percibida y Planificada}
\subsection{Concepto de obsolescencia}
Acorde a \begin{justifying}
  \cite{hallgrave-2019} %citar a investopedia
  la obsolescencia se debe de tratar como un rieso; éste es el cual un proceso, producto o tecnología usado o producido por una 
  compañia para obtener ganancias se convierta obsoleto, y po ende no pueda competir en el mercado.
  Lo anterior es debido a que está íntimamente asociado con la innovación; sin una tecnología dinámica, la obsolecencia sería
inexistente. Esta surge como el resultado de la ocurrencia de cambios tecnológicos de un tipo de contexto particular. \citep{10.2307/1826804} %citar al moonitz
\par
\end{justifying}
Una \begin{justifying}
  definición interesante es la que proponen \cite{article} %citar a los de researchgate
  donde definen a la obsolescencia como elgrado en el cual los profesionales carecen del conocimientoy habilidades más recientes para mantenerse
  en un rendimiento efectivo en sus roles actuales o futuros.\par
\end{justifying}
\vspace{\baselineskip}
\subsubsection{Obsolescencia Percibida}
Es aquella \begin{justifying}
  donde el progreso tecnológico hace parecer a un producto nuevo mucho mejor que su versión anterior, o lo convierte en un objeto obsoleto
  en la conciencia del consumidor debido a los efectos de la mercadotecnia del producto, forzandolos a seguir la tendencia y renovar su producto a pesar
  de no ser necesario. \citep{unknown-author-no-dateA} %citar a los de IGI
  \par
\end{justifying}
Retomando \begin{justifying}
  lo descrito por \cite{article} %citar otravez a los de researchgate
la obsolescencia percibida acata el grado el cual los empleados percatan a sus habilidades como anticuadas.\par
\end{justifying}
\vspace{\baselineskip}
\subsubsection{Obsolescencia Planificada}
Acorde \begin{justifying}
  a \cite{consumers-international-no-date}%citar a los de consumerinternational
  se refiere a cuando los fabricantes diseñan de manera deliberada productos que fallan prematuramente o se hacen anticuados, la gran mayoría de las veces
  se hace para vender otro producto o mejorarlo. Esto deriva a que también restrínjan la habilidad del consumidor para reparar los productos adquiridos.\par
\end{justifying}
Para \begin{justifying}
  \cite{unknown-author-no-dateB}%citar a sustainability4all
  este refiere al acortamiento deliberado de la utilidad de un producto por el fabricante en orden para incrementar el consumo.\par
\end{justifying}
\vspace{\baselineskip}
\subsection{Ejemplos de Obsolescencia}
En \begin{justifying}
  estos casos tenemos distintos ejemplos, por mencionar algunos:
  \begin{itemize}
    \item Ecosistemas de dispositivos inteligentes requiren las últimas actualizaciones para funcionar plenamente.
    \item Los celulares ya no retienen la carga eléctrica a travéz de los años.
    \item Marcas de ropa que se promocionan cosntantemente en internet y a precios relativamente altos.
    \item Los anuncios de Apple hacen sentír la necesidad de comprar el producto para pertenecer al estátus.
    \item etc.
  \end{itemize}\par
\end{justifying}
\vspace{\baselineskip}
\section{Resumen del video adjunto}
El \begin{justifying}
  video \emph{The Story of Stuff} nos explica que el sistema de producción no es lineal, es muy complejo ya que incluye a las culturas, personas y demás.
  Las corporaciones generalmente tienen mayor poder que los gobiernos debido al sistema capitalista que nos rige, aparte de que los gobiernos de primer mundo terminan
  dominando al tercer mundo por su cantidad de capital e inversión extranjera, explotando los recursos de estos para beneficio propio. Otro problema es que se utilizan químicos
  tóxicos que afectan el desarrollo humano; específicamente el de los infantes. Debido a la globalización, los verdaderos costos de producción recaen en el medio ambiente
  debido a los consumidores y la tendencia del querer estar a la vanguardia, lo que nos lleva a la obsolescencia planeada y la obsolescencia percibida; la primera se resume
  en ``ser diseñado para tirarse'' mientras que la segunda nos convence que nuestros bienes están muy anticuados y por ende es necesario cambiarlos todo el tiempo. \citep{unknown-author-2009}\par
\end{justifying}
\vspace{\baselineskip}
\subsection{Relación con la Obsolescencia Percibida y Planificada}
Como bien \begin{justifying}
  lo mencionan, la relación es explícita debido a que explican ámbos conceptos con distintos ejemplos. La obsolescencia planeada es cuando se crea un producto para que
  pueda ser desechado mientras que la obsolescencia percibida es la que obtenemos por la excesiva cantidad de promoción de productos en los medios de comunicación.\par
\end{justifying}
\newpage
% Referencias
\setcounter{secnumdepth}{0} %permite enumerar las secciones QUITAR SI QUIERES OG APA7
\renewcommand\refname{\textbf{Referencias}}
\bibliography{referencias} %el archivo 'referencias.bib' debe estar dentro del mismo folder donde se encuentra el archivo .tex para citar las referencias deseadas

\end{document}