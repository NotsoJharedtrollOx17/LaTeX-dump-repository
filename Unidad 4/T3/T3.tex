% Preámbulo
\documentclass[stu, 12pt, letterpaper, donotrepeattitle, floatsintext, natbib]{apa7}
\usepackage[utf8]{inputenc}
\usepackage{comment}
\usepackage{marvosym}
\usepackage{graphicx}
\usepackage{float}
\usepackage[normalem]{ulem}
\usepackage[spanish]{babel} 
\usepackage{lastpage} %para le formato que quiere la profe QUITAR SI QUIERES OG APA7
\usepackage{ragged2e} %para le formato que quiere la profe QUITAR SI QUIERES OG APA7
\usepackage{indentfirst} %para le formato que quiere la profe QUITAR SI QUIERES OG APA7
\usepackage{multirow,booktabs,setspace,caption} %formato de figuras APA
\DeclareCaptionLabelSeparator*{spaced}{\\[2ex]}
\captionsetup[figure]{textfont=it,format=plain,justification=justified,
  singlelinecheck=false,labelsep=spaced,skip=0pt}

\selectlanguage{spanish}
\useunder{\uline}{\ul}{}
\newcommand{\myparagraph}[1]{\paragraph{#1}\mbox{}\\}

\rfoot{Página \thepage \hspace{1pt} de \pageref{LastPage}}%QUITAR SI QUIERES OG APA7 
\rhead{} %QUITAR SI QUIERES OG APA7
\setcounter{secnumdepth}{3} %permite enumerar las secciones QUITAR SI QUIERES OG APA7
\setlength{\parindent}{1.27cm} %sangria forzada QUITAR SI QUIERES OG APA7

% Portada
\thispagestyle{empty}
\title{\Large Globalización, Economía y Acuerdos}
\author{Abraham Jhared Flores Azcona} % (autores separados, consultar al docente)
% Manera oficial de colocar los autores:
%\author{Autor(a) I, Autor(a) II, Autor(a) III, Autor(a) X}
\affiliation{Instituto Tecnológico de Tijuana}
\course{ACD-0908SC5C Desarrollo Sustentable}
\professor{M.C. Trinidad Castro Villa}
\duedate{1ro. de noviembre de 2021}

\renewcommand\labelitemi{$\bullet$}

\newcommand*\chem[1]{\ensuremath{\mathrm{#1}}}

\begin{document}
\maketitle


% Índices
\pagenumbering{arabic}
    % Contenido
\renewcommand\contentsname{Contenido}
\tableofcontents
\renewcommand{\listfigurename}{Figuras}
\listoffigures

% Cuerpo 
    %NOTA: PARA CITAR ESTILO "Merts (2003)" usar \cite{<nombre_cita_bib>}
    %                        "(Metz, 1978)" usar \citep{<nombre_cita_bib>}
\newpage
\section{Globalización}
\subsection{Concepto}
Acorde \begin{justifying}
  a \cite{fernando-2020}%citar al de investopedia
  la globalización es la propagación de productos, tecnología, información y trabajos a travéz
  de las fronteras nacionales y de las culturas. En términos estríctamente económicos, describe una 
  interdependencia de naciones alrededor del planeta, fomentado por el libre comercio.\par
\end{justifying}
Para \begin{justifying}
  \cite{kolb-2018}%citar wita
  describe la creciente interdependencia de las economias, culturas y poblaciones del mundo, provistas
  por el comercio inter-frontera de bienes y servicios, tecnología y flujos de inversión, gente e información.
  Esto es debido a quelos países han construido relaciones económicas para facilitar dichos movimientos.\par
\end{justifying}
\vspace{\baselineskip}
\subsection{Características}
Las \begin{justifying}
  principales son las siguientes: \citep{unknown-author-no-date} %citar a SQ Study
  \begin{itemize}
    \item Mayor comercio de bienes y servicios entre las naciones y las regiones.
    \item Aboliciona las estructuras y límites viejos del Estado.
    \item Cambia las concepciones, pensamientos y creencias de los seres humanos.
    \item Factor relevante de inversión en compañias.
    \item Libre comercio dentre las naciones; ausencia del control gubernamental excesivo sobre el comercio.
    \item Aumento de las transacciones comerciales económicas y del intercambio comercial de las culturas.
    \item Desarollo de marcas globales que sirven a los mercados de naciones con mayor y menor ingreso.
    \item etc.
  \end{itemize}\par
\end{justifying}
\vspace{\baselineskip}
\section{Interrelaciones Entre la Economía Global y Nacional}
Una de \begin{justifying}
  las principales negativas de la globalización recae en la depreciación de las economías y comunidades locales, y por ende
  su poco incentivo de aspirar a ayudar a la comunidad local. Como lo menciona \cite{kanter-2014} %citar a la de harvard
  el entorno competitivo que incluye a los inversionistas extranjeros en los mercados locales, termina orientando a los
  negocios locales a expander su presencia, uniendose a los corporativos para aprovechar el alcance internacional.
\end{justifying}
Para \begin{justifying}
  evitar el conflicto de intereses al nivel global y local, los negocios deben sabe cómo responder
  a las necesidades de las comunidades que habitan al mismo tiempo que se globalizan, y las comunidades deben determinar
  la mejor manera de conectar a los cosmopolítas y a los locales, así como crear una cultura cívica que atraiga y retenga
  a las compañias en vias de establecimiento.\par
\end{justifying}
\vspace{\baselineskip}
\section{Acuerdos en Entidades Económicas}
Son \begin{justifying}
  puntos de concordancia entre distintas entidades para un objetivo en común, en este caso, 
  del Banco Mundial y sus Articulos de Concordancia. En estos se explican las bases para que dicha
  entidad funcione acorde a los siguientes propositos \citep{international-bank-for-reconstruction-and-development-2021}: %citar al banco mundial
  \begin{itemize}
    \item Asistir en la reconstrucción y desarrollo de los territorios de los miembros.
    \item Promover la inversión extranjera por medios de garantías ó participaciones en préstamos u otros isntrumentos financieros.
    \item Promover el crecimiento balanceado a largo plazo del comercio internacional y el equilibrio de los pagos.
    \item Acomodar los prestamos por prioridades.
    \item Conducir sus operaciones con pleno conocimiento del efecto de la inversión internacional dentro de las condiciones de los negocios de los territorios.
  \end{itemize}\par
\end{justifying}
\vspace{\baselineskip}


\newpage
% Referencias
\setcounter{secnumdepth}{0} %permite enumerar las secciones QUITAR SI QUIERES OG APA7
\renewcommand\refname{\textbf{Referencias}}
\bibliography{referencias} %el archivo 'referencias.bib' debe estar dentro del mismo folder donde se encuentra el archivo .tex para citar las referencias deseadas

\end{document}