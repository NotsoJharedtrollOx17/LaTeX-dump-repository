% Preámbulo
\documentclass[stu, 12pt, letterpaper, donotrepeattitle, floatsintext, natbib]{apa7}
\usepackage[utf8]{inputenc}
\usepackage{comment}
\usepackage{marvosym}
\usepackage{graphicx}
\usepackage{float}
\usepackage[normalem]{ulem}
\usepackage[spanish]{babel} 
\usepackage{lastpage} %para le formato que quiere la profe QUITAR SI QUIERES OG APA7
\usepackage{ragged2e} %para le formato que quiere la profe QUITAR SI QUIERES OG APA7
\usepackage{indentfirst} %para le formato que quiere la profe QUITAR SI QUIERES OG APA7
\usepackage{multirow,booktabs,setspace,caption} %formato de figuras APA
\DeclareCaptionLabelSeparator*{spaced}{\\[2ex]}
\captionsetup[figure]{textfont=it,format=plain,justification=justified,
  singlelinecheck=false,labelsep=spaced,skip=0pt}

\selectlanguage{spanish}
\useunder{\uline}{\ul}{}
\newcommand{\myparagraph}[1]{\paragraph{#1}\mbox{}\\}

\rfoot{Página \thepage \hspace{1pt} de \pageref{LastPage}}%QUITAR SI QUIERES OG APA7 
\rhead{} %QUITAR SI QUIERES OG APA7
\setcounter{secnumdepth}{3} %permite enumerar las secciones QUITAR SI QUIERES OG APA7
\setlength{\parindent}{1.27cm} %sangria forzada QUITAR SI QUIERES OG APA7

% Portada
\thispagestyle{empty}
\title{\Large Sistemas de Producción}
\author{Abraham Jhared Flores Azcona} % (autores separados, consultar al docente)
% Manera oficial de colocar los autores:
%\author{Autor(a) I, Autor(a) II, Autor(a) III, Autor(a) X}
\affiliation{Instituto Tecnológico de Tijuana}
\course{ACD-0908SC5C Desarrollo Sustentable}
\professor{M.C. Trinidad Castro Villa}
\duedate{27 de octubre de 2021}

\renewcommand\labelitemi{$\bullet$}

\newcommand*\chem[1]{\ensuremath{\mathrm{#1}}}

\begin{document}
\maketitle


% Índices
\pagenumbering{arabic}
    % Contenido
\renewcommand\contentsname{Contenido}
\tableofcontents
\renewcommand{\listfigurename}{Figuras}
\listoffigures

% Cuerpo 
    %NOTA: PARA CITAR ESTILO "Merts (2003)" usar \cite{<nombre_cita_bib>}
    %                        "(Metz, 1978)" usar \citep{<nombre_cita_bib>}
\newpage
\section{Sistemas de Producción}
\subsection{Concepto}
Para \begin{justifying}  %citar economipedia
    \cite{quiroa-2020} el sistema de producción, es el modo cómo se utilizan y se combinan los factores productivos para llevar
    a cabo su transformación y posteriormente convertirlos en bienes y servicios. Por decirlo de una manera, son las
    actividades que se encargan de hacer eficiente el proceso de producción por medio de entradas de recursos y salidas de productos.\par
\end{justifying}
Acorde a \begin{justifying}
    \cite{unknown-author-no-date} es un sub sistema de la empresa donde su función principal consiste en la transformación de materiales
    en productos que sean aptos para su consumo y que satisfagan las necesidades de la demanda. Agrega que dicho sistema
    se encarga de combinar los factores de producción con el fín de que el resultado obtenido sea el mejor posible para la empresa.\par
\end{justifying}
\vspace{\baselineskip}
\subsection{Tipos}
En general, \begin{justifying}
    tenemos cuatro tipos de sistemas de producción: \emph{continua, intermitente, modular y por proyectos.} \citep{unknown-author-1995}\par %citar al del librote
\end{justifying}
\vspace{\baselineskip}
\subsubsection{Continua}
Las \begin{justifying}
    instalaciones se adaptan a ciertos itinerarios y flujos de operación, que siguen una escala no afectada por interrupciones. Todas las operaciones
    se organizan para lograr una situación ideal, en la que estas mismas operaciones, se combinan con el transporte de tal manera que los materiales
    son procesados mientras se mueven.\par
\end{justifying}
\vspace{\baselineskip}
\subsubsection{Intermitente}
Se produce \begin{justifying}
    por lotes. Esta es inevitable cuando la demanda de un producto no es lo bastante grande para utilizar el tiempo de fabricación continuo. 
    La economía de manufactura favorece a este tipo de producción, por lo que las empresas fabrican gran variedad de productos.\par
\end{justifying}
\subsubsection{Modular}
Básicamente \begin{justifying}
    se refiere a producir pedazo a pedazo el producto, con el afán de máximizar la variedad de partes como refacciones.\par
\end{justifying}
\vspace{\baselineskip}
\subsubsection{Por proyectos}
Es cuando \begin{justifying}
    se considera el potencial de un producto por medio de una idea que procura tener objetivos establecidos para apegarse a una ruta de mejora. Se considera que
    este tipo de producción es dificil debido a que se requiere de supervisión constante para apegarse a los objetivos.\par
\end{justifying}
\vspace{\baselineskip}
\subsubsection{Ejemplo}
Un claro \begin{justifying}
    de los tipos de producción, es el del sistema agrícola por lotes ya que los agricultores producen los alimentos de temporada
    acorde a lo más popular y lo más asequible par el mercado.\par
\end{justifying}
\subsection{Factores}
A grandes \begin{justifying}
    rasgos tenemos cuatro factores: \emph{tierra, trabajo, capital y capacidad empresarial}. \citep{roldan-2021}\par %citar otra vez a ecnomipedia
\end{justifying}
\vspace{\baselineskip}
\subsubsection{Tierra}
Comprende \begin{justifying}
    a todos los recursos naturales que pueden ser utilizados en el proceso productivo, así como las fuentes de energía como agua,
    gas natural, carbón, etc.\par
\end{justifying}
\subsubsection{Trabajo}
Es \begin{justifying}
    el tiempo que las personas dedican a la producción.\par
\end{justifying}
\vspace{\baselineskip}
\subsubsection{Capital}
Son los\begin{justifying}
    bienes durables que son utilizados para fabricar otros bienes o servicios.\par
\end{justifying}
\vspace{\baselineskip}
\subsubsection{Capacidad Empresarial}
Nos referimos \begin{justifying}
    al conjunto de conocimientos y técnicas que, aplicados de forma lógica y ordenada, permiten a las personas solucionar problemas, modificar su entorno
    y adaptarse al medio ambiente. Este último es el más reciente en incluirse a los modelos económicos; también se le denomina
    como \emph{tecnología.}
\end{justifying}
\vspace{\baselineskip}
\subsubsection{Ejemplo}
Tomando el \begin{justifying}
    factor de \emph{Tierra}, un claro ejemplo es justamente la tierra cultivable para las cosechas agragrias. Sin esa tierra sería muy dificil, si no es que imposible
    el generar algún tipo de producción física.\par
\end{justifying}

\newpage
% Referencias
\setcounter{secnumdepth}{0} %permite enumerar las secciones QUITAR SI QUIERES OG APA7
\renewcommand\refname{\textbf{Referencias}}
\bibliography{referencias} %el archivo 'referencias.bib' debe estar dentro del mismo folder donde se encuentra el archivo .tex para citar las referencias deseadas

\end{document}