\documentclass[letterpaper, 12pt]{article}
\usepackage[letterpaper, top=2.5cm, bottom=2.5cm, left=3cm, right=3cm]{geometry} %margenes
\usepackage[utf8]{inputenc} %manejo de caracteres especiales
\usepackage[spanish]{babel} %manejo de encabezados de inglés a español
\usepackage{fancyhdr} %formato de los encabezados de página
\usepackage{ragged2e} %alineado real justficado
\usepackage{graphicx} %manejo de imagenes
\usepackage{amsmath} %manejo de notación matemática
\usepackage{mathtools} %manejo de notación matemática
\usepackage{blindtext} %texto de relleno
\usepackage[titles]{tocloft} %manejo de elementos para el índice
\usepackage{float} %manejo de centrado para figuras

\pagestyle{fancy}
\fancyhf{}
\rfoot{\thepage}

\begin{document}
    
    %PORTADA
    \begin{titlepage}
        \begin{figure}[ht]
            \centering
            \includegraphics[width=15cm]{logosITT.png}
        \end{figure}
        \centering
        {\scshape\LARGE Tecnológico Nacional de México\\Instituto Tecnológico de Tijuana\par}
        \vspace{1cm}
        {\scshape\Large Estructura de Datos\par}
        \vspace{1cm}
        {\scshape\Large Unidad 4\par}
        \vspace{1.5cm}
        {\huge\bfseries Ensayo acerca de Arboles\par}
        \vspace{2cm}
        {\Large\itshape C. Abraham Jhared Flores Azcona\\19211640\par}
        \vfill
        Profesora: \par
        M.C. Claudia Negrete Sanchez
        
        \vfill

        {\large 6 de noviembre del 2020}
    \end{titlepage}

        \newpage
        \begin{justify}
            \setcounter{page}{1}
            \thispagestyle{fancy}
            \lhead{\textbf{Ensayo acerca de Arboles}}
            \section{Introducción}
            En este breve ensayo se resalta lo más relevante respecto a los videos de introducción al tema de la Unidad 4; así como detalles considerados importantes para el estudio de la misma. 
            \section{Conceptos básicos}
            A grandes rasgos, un árbol es una estructura no lineal dinámica porque no necesariamente tiene un solo sucesor y un solo predecesor; este puede cambiar durante la ejecución del programa, también es una estructura jerárquica aplicada sobre nodos donde el primero es
            la raíz y los que se desparraman de él tienen parentésco, como un árbol genealógico. Explicado de manera recursiva, dicho árbol puede ser vacio, o un nodo raíz con un subárbol izquierdo y derecho.
            \\\newline Se pueden establecer las relaciones de un arbol en:
            \begin{itemize}
                \item En relación con otros nodos: \emph{Padre \(\rightarrow \begin{matrix}
                    \text{Hijo.}\\
                    \text{Hermano.}
                \end{matrix}\)}
                \item En relación a la posición: \emph{Raíz \(\rightarrow\) Ramas \(\rightarrow\) Hojas.}
                \item En relación a su tamaño: \emph{Grado de un nodo (subárboles de un nodo), longitud del camino (Cuantos enlaces desde la raíz a un nodo dado), profundidad (número máximo de nodos) y por peso (cantidad de nodos terminales).}
            \end{itemize}
            Por último, las operaciones que podemos realizar son las siguientes:
            \begin{itemize}
                \item \emph{Insertar}.
                \item \emph{Buscar}.
                \item \emph{Eliminar}.
                \item \emph{Recorrer}.
            \end{itemize}
            Cabe destacar que el árbol empleado para estructuras de datos es el \emph{árbol binario}, el cuál su caracteristica principal es que \emph{sus nodos solo pueden contener un máximo de dos elementos}. 
            \section{Inserción y Búsqueda en árboles binarios}
            \subsection{Inserción}
            El acomodo resultante de la Inserción en un árbol binario depende del tipo de dato el cual se maneje.
            \subsubsection{Para datos numéricos}
            De primera instancia, el primer dato de dicha lista va a ser nuestro nodo raíz. Los subsecuentes se insertar en base a la izquierda o la derecha dependiendo de si dichos valores son menores o mayores entre ellos mismos y así de manera recursiva hasta agotar.
            \subsubsection{Para datos string}
            De manera similar para datos numericos, nuestra jeraquía va a ser el orden alfabético y la cantidad de caractéres de las cadenas dadas. El primer dato dado será nuestra raíz. Los subsecuentes se van a insertar a la izquierda o derecha dependiendo de lo mencionado anteriormente; así de manera recursiva hasta agotar. 
            \section{Recorrido y Eliminación en árboles binarios}
            \subsection{Recorrido}
            A grandes rasgos ``como consultar los datos''. Los dos tipos de recorrido son los siguientes: \emph{por profundidad y por amplitud.} Los recorridos (y su orden de recorrido) son los siguientes:
            \subsubsection{Por longitud}
            \begin{itemize}
                \item \emph{RID (Preorden):} Visitar Raíz \(\rightarrow\) Recorrer subárbol izquierdo en RID \(\rightarrow\) Recorrer subárbol derecho en RID.
                \item \emph{IRD (Enorden):} Recorrer subárbol izquierdo en IRD \(\rightarrow\) Visitar Raíz \(\rightarrow\) Recorrer subárbol derecho en IRD.
                \item \emph{IDR (Postorden):} Recorrer subárbol izquierdo en IDR \(\rightarrow\) Recorrer subárbol derecho en IRD\(\rightarrow\) Visitar Raíz.
            \end{itemize}
            \subsubsection{Por amplitud}
            \begin{itemize}
                \item Visitar raíz \(\rightarrow\) Recorrer el árbol por niveles \(\rightarrow\) Terminar recorrido en el último nivel.
            \end{itemize}
            \subsection{Eliminación}
            Existen 4 casos al respecto:
            \begin{itemize}
                \item \emph{Una hoja:} Se elimina un nodo que no tiene hijo. El árbol no cambia drásticamente de forma.
                \item \emph{Un nodo padre con hijo único:} Se elimina solamente el nodo deseado. Su hijo toma la posición del padre en la estructura.
                \item \emph{Un nodo con dos híjos:} Se elimina el nodo deseado y una hoja cuyo valor sea el más proximo al del nodo elimidado toma su lugar.
                \item \emph{Un nodo no existente:} El programa busca mas allá del padre que (en teoría) contiene al nodo inexistente.
            \end{itemize}

            \section{Conclusión}
            En lo general y en lo particular, me parece una de las estructuras (hasta la fecha) mas interesantes de manejar debido a mi previa exposición de los temas en la asignatura de Matemáticas Discretas y que los temas relacionados a la Teoría de Grafos muestran relaciones a primera vista inexistentes dentro de un conjunto de datos.
            Esto me sirve como una buena remenbranza para la unidad.
        \end{justify}

\end{document}