\documentclass[letterpaper, 12pt]{article}
\usepackage[letterpaper, top=2.5cm, bottom=2.5cm, left=3cm, right=3cm]{geometry} %margenes
\usepackage[utf8]{inputenc} %manejo de caracteres especiales
\usepackage[spanish]{babel} %manejo de encabezados de inglés a español
\usepackage{fancyhdr} %formato de los encabezados de página
\usepackage{ragged2e} %alineado real justficado
\usepackage{graphicx} %manejo de imagenes
\usepackage{amsmath} %manejo de notación matemática
\usepackage{mathtools} %manejo de notación matemática
\usepackage{blindtext} %texto de relleno
\usepackage[titles]{tocloft} %manejo de elementos para el índice
\usepackage{float} %manejo de centrado para figuras
\usepackage{amssymb} %manejo de simbolog►1a matematica

\pagestyle{fancy}
\fancyhf{}
\rfoot{\thepage}

\begin{document}
    
    %PORTADA
    \begin{titlepage}
        \begin{figure}[ht]
            \centering
            \includegraphics[width=15cm]{logosITT.png}
        \end{figure}
        \centering
        {\scshape\LARGE Tecnológico Nacional de México\\Instituto Tecnológico de Tijuana\par}
        \vspace{1cm}
        {\scshape\Large Estructura de Datos\par}
        \vspace{1cm}
        {\scshape\Large Unidad 5\par}
        \vspace{1.5cm}
        {\huge\bfseries Ejercicios QuickSort\par}
        \vspace{2cm}
        {\Large\itshape C. Abraham Jhared Flores Azcona\\19211640\par}
        \vfill
        Profesora: \par
        M.C. Claudia Negrete Sanchez
        
        \vfill

        {\large 26 de noviembre del 2020}
    \end{titlepage}

    \newpage
    \thispagestyle{fancy}
    \setcounter{page}{1}
    \lhead{\textbf{Ejercicios de Sorteo Rapido}}
    \section{Problema}
    Resuelve en ejercicio como se muestra en el video adjunto y plasmalo en tu cuaderno o en un documento, entregalo en tiempo y forma para que sea evaluado. Los datos del arreglo son:
    \[\text{A}[7]=\{23,\,6,\,45,\,12,\,7,\,34,\,9\}\]
    No olvides poner cual es tu pivote. 
    \section{Resolución}
    Para el Quick Sort, ocupamos conocer los valores de izquierda (\(I\)), derecha (\(D\)) y pivote \(P\). \(I\) y \(D\) se van a poner en los extremos del arreglo, y el pivote se considera el valor del extremo izquierdo. Los valores de falsedad y verdad
    se representan con 0 y 1 respectivamente.
    \\• \(I=\_,\, D=9,\, P=23\)
    \[\{\_,6,45,12,7,34,9\}\, D<P:1\therefore\]
    • \(I=6,\, D=\_,\, P=23\)
    \[\therefore \{9,6,45,12,7,34,\_\},\, I>P:0 \therefore\]
    • \(I=45,\, D=\_,\,P=23\)
    \[\therefore \{9,6,45,12,7,34,\_\},\, I>P:1 \therefore\]
    • \(I=\_,\, D=34,\, P=23\)
    \[\therefore \{9,6,\_,12,7,34,45\},\, D<P:0 \therefore\]
    • \(I=\_,\, D=7,\, P=23\)
    \[\therefore \{9,6,\_,12,7,34,45\},\, D<P:1 \therefore\]
    • \(I=12,\, D=\_, P=23\)
    \[\therefore \{9,6,7,12,\_,34,45\},\, I>P:0 \therefore\]
    • \(I=_,\, D=\_, P=23\)
    \[\therefore \{9,6,7,12,\_,34,45\},\, I>P:\text{inconcluso} \therefore\]
    \[\therefore \text{A}[7]=\{9,6,7,12,23,34,45\}\] 
    Como el arreglo aún no se se ha terminado de ordenar, ahora se partirá en dos el arreglo realizar el sorteo. Como el resultado \(\frac{7}{2}\) es 3.5, se procedera a tomar los primeros tres números del arreglo para hacer la comparación de la primer partición.
    \\• \(I=\_,\, D=7,\, P=9\)
    \[\{\_,6,7,12,23,34,45\},\, D<P:1 \therefore\]
    • \(I=6,\, D=\_,\, P=9\)
    \[\therefore \{7,6,\_,12,23,34,45\} I>P:0 \therefore\]
    • \(I=_,\, D=\_,\, P=9\)
    \[\therefore \{7,6,\_,12,23,34,45\} I>P:0 \therefore\]
    \[\therefore \text{A}[7]=\{7,6,9,12,23,34,45\}\]
    Como la otra mitad del arreglo esta acomodada, procedemos a reducir más el rango de evaluación, como el resultado de \(\frac{3}{2}\) es 1.5, procederemos a redondear a dos para tener un márgen mayor para evaluar los datos.
    \\• \(I=\_,\, D=6,\, P=7\)
    \[\{\_,6,9,12,23,34,45\},\, D<P:1 \therefore\]
    • \(I=\_,\, D=\_,\, P=7\)
    \[\therefore\{6,\_,9,12,23,34,45\},\, I>P:\text{incloncluso} \therefore\]
    \[\therefore \text{A}[7]=\{6,7,9,12,23,34,45\}\]
    Dando por concluido el ordenamiento.
\end{document}