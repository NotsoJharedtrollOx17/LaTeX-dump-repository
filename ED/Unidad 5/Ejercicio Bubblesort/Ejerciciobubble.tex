\documentclass[letterpaper, 12pt]{article}
\usepackage[letterpaper, top=2.5cm, bottom=2.5cm, left=3cm, right=3cm]{geometry} %margenes
\usepackage[utf8]{inputenc} %manejo de caracteres especiales
\usepackage[spanish]{babel} %manejo de encabezados de inglés a español
\usepackage{fancyhdr} %formato de los encabezados de página
\usepackage{ragged2e} %alineado real justficado
\usepackage{graphicx} %manejo de imagenes
\usepackage{amsmath} %manejo de notación matemática
\usepackage{mathtools} %manejo de notación matemática
\usepackage{blindtext} %texto de relleno
\usepackage[titles]{tocloft} %manejo de elementos para el índice
\usepackage{float} %manejo de centrado para figuras
\usepackage{amssymb} %manejo de simbolog►1a matematica

\pagestyle{fancy}
\fancyhf{}
\rfoot{\thepage}

\begin{document}
    
    %PORTADA
    \begin{titlepage}
        \begin{figure}[ht]
            \centering
            \includegraphics[width=15cm]{logosITT.png}
        \end{figure}
        \centering
        {\scshape\LARGE Tecnológico Nacional de México\\Instituto Tecnológico de Tijuana\par}
        \vspace{1cm}
        {\scshape\Large Estructura de Datos\par}
        \vspace{1cm}
        {\scshape\Large Unidad 4\par}
        \vspace{1.5cm}
        {\huge\bfseries Ejercicios de Burbuja\par}
        \vspace{2cm}
        {\Large\itshape C. Abraham Jhared Flores Azcona\\19211640\par}
        \vfill
        Profesora: \par
        M.C. Claudia Negrete Sanchez
        
        \vfill

        {\large 21 de noviembre del 2020}
    \end{titlepage}

    \newpage
    \thispagestyle{fancy}
    \setcounter{page}{1}
    \lhead{\textbf{Ejercicios de Burbuja}}
    \section{Problema}
    Con el video adjunto, realiza el ejercicio reolviendolo de manera gráfica como se muestra en el video, plásmalo en tu cuaderno o en un documento y subelo en tiempo y forma,
    entregalo a tiempo para ser tomado en cuenta. Los datos del arreglo son:
    \[\text{Numeros}[8]=\{56, 4, 89, 12, 7, 3, 89, 876\}\]
    \section{Resolución}
    Para aclarar, el subrayado indica que valores estamos evaluando para la comparación características del \emph{Bubble Sort}.
    \[\text{Numeros}[8]=\{\underline{56, 4}, 89, 12, 7, 3, 89, 876\}\]
    • \(56>4: \text{true} \therefore\)
    \[\therefore \text{Numeros}[8]=\{4, \underline{56, 89}, 12, 7, 3, 89, 876\}\]
    • \(56>89: \text{false} \therefore\)
    \[\therefore \text{Numeros}[8]=\{4, 56, \underline{89, 12}, 7, 3, 89, 876\}\]
    • \(89>12: \text{true} \therefore \)
    \[\therefore \text{Numeros}[8]=\{4, 56, 12, \underline{89, 7}, 3, 89, 876\}\]
    • \(89>7: \text{true} \therefore\)
    \[\therefore \text{Numeros}[8]=\{4, 56, 12, 7, \underline{89, 3}, 89, 876\}\]
    • \(89>3: \text{true} \therefore\)
    \[\therefore \text{Numeros}[8]=\{4, 56, 12, 7, 3, \underline{89, 89}, 876\}\]
    • \(89>89: \text{false} \therefore\)
    \[\therefore \text{Numeros}[8]=\{4, 56, 12, 7, 3, 89, \underline{89, 876}\}\]
    • \(89>876: \text{false} \therefore\)
    \[\therefore \text{Numeros}[8]=\{\underline{4, 56}, 12, 7, 3, 89, 89, 876\}\]
    • \(4>56: \text{false} \therefore\)
    \[\therefore \text{Numeros}[8]=\{4, \underline{56, 12}, 7, 3, 89, 89, 876\}\]
    • \(56>12: \text{true} \therefore\)
    \[\therefore \text{Numeros}[8]=\{4, 12, \underline{56, 7}, 3, 89, 89, 876\}\]
    • \(56>7: \text{true} \therefore\)
    \[\therefore \text{Numeros}[8]=\{4, 12, 7, \underline{56, 3}, 89, 89, 876\}\]
    • \(56>3: \text{true} \therefore\)
    \[\therefore \text{Numeros}[8]=\{4, 12, 7, 3, \underline{56, 89}, 89, 876\}\]
    • \(56>89: \text{false} \therefore\)
    \[\therefore \text{Numeros}[8]=\{4, \underline{12, 7}, 3, 56, 89, 89, 876\}\]
    Para ahorrar la escritura, se procedera a regresarse a los valores restantes de acomodar en base a lo evaluado anteriormente.
    \\\newline• \(12>7: \text{true} \therefore\)
    \[\therefore \text{Numeros}[8]=\{4, 7, \underline{12, 3}, 56, 89, 89, 876\}\]
    • \(12>3: \text{true} \therefore\)
    \[\therefore \text{Numeros}[8]=\{4, 7, 3, \underline{12, 56}, 89, 89, 876\}\]
    • \(12>56: \text{false} \therefore\)
    \[\therefore \text{Numeros}[8]=\{4, \underline{7, 3}, 12, 56, 89, 89, 876\}\]
    • \(7>3: \text{true} \therefore\)
    \[\therefore \text{Numeros}[8]=\{4, 3, \underline{7, 12}, 56, 89, 89, 876\}\]
    • \(7>12: \text{false} \therefore\)
    \[\therefore \text{Numeros}[8]=\{\underline{4, 3}, 7, 12, 56, 89, 89, 876\}\]
    • \(4>3: \text{true} \therefore\)
    \[\therefore \text{Numeros}[8]=\{3, 4, 7, 12, 56, 89, 89, 876\}\]
    Dando por concluido dicho ordenamiento.
\end{document}