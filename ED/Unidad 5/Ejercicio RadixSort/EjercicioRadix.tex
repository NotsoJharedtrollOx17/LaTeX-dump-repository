\documentclass[letterpaper, 12pt]{article}
\usepackage[letterpaper, top=2.5cm, bottom=2.5cm, left=3cm, right=3cm]{geometry} %margenes
\usepackage[utf8]{inputenc} %manejo de caracteres especiales
\usepackage[spanish]{babel} %manejo de encabezados de inglés a español
\usepackage{fancyhdr} %formato de los encabezados de página
\usepackage{ragged2e} %alineado real justficado
\usepackage{graphicx} %manejo de imagenes
\usepackage{amsmath} %manejo de notación matemática
\usepackage{mathtools} %manejo de notación matemática
\usepackage{blindtext} %texto de relleno
\usepackage[titles]{tocloft} %manejo de elementos para el índice
\usepackage{float} %manejo de centrado para figuras
\usepackage{amssymb} %manejo de simbolog►1a matematica

\pagestyle{fancy}
\fancyhf{}
\rfoot{\thepage}

\begin{document}
    
    %PORTADA
    \begin{titlepage}
        \begin{figure}[ht]
            \centering
            \includegraphics[width=15cm]{logosITT.png}
        \end{figure}
        \centering
        {\scshape\LARGE Tecnológico Nacional de México\\Instituto Tecnológico de Tijuana\par}
        \vspace{1cm}
        {\scshape\Large Estructura de Datos\par}
        \vspace{1cm}
        {\scshape\Large Unidad 5\par}
        \vspace{1.5cm}
        {\huge\bfseries Ejercicios Radix Sort\par}
        \vspace{2cm}
        {\Large\itshape C. Abraham Jhared Flores Azcona\\19211640\par}
        \vfill
        Profesora: \par
        M.C. Claudia Negrete Sanchez
        
        \vfill

        {\large 26 de noviembre del 2020}
    \end{titlepage}

    \newpage
    \thispagestyle{fancy}
    \setcounter{page}{1}
    \lhead{\textbf{Ejercicios de Raíz}}
    \section{Problema}
    En base al video adjunto gnera en una hoja de tu cuaderno o en un documento el ejercicio usando este tipo de ordenamiento, los datos del arreglo a ser ordenados son:
    \[\text{A}[8]=\{67,3,789,123,4,12,34,78\}\]
    Entregalo en tiempo y forma para que sea tomado en cuenta. Para claridad, el dígito que se evalua está subrayado.
    \section{Resolución}
    \justify
    • 1er. dígito:
    \[\text{A}[8]=\{6\underline{7},\underline{3},78\underline{9},12\underline{3},\underline{4},1\underline{2},3\underline{4},7\underline{8}\}\]
    \[\begin{matrix}
        \text{Dígito}&\text{Número}\\
        0&\\
        1&\\
        2&12\\
        3&123\\
        4&4,\,34\\
        5&\\
        6&6\\
        7&67\\
        8&78\\
        9&789
    \end{matrix}\]
    \[\text{Arreglo auxiliar}[8]=\{12,123,4,34,6,67,78,789\};\,\text{A}=\text{Arreglo auxiliar};\]
    • 2do. dígito:\\
    Los números que no tienen un 2do. dígito se consideran que tienen un dígito de 0 en dicha posición.
    \[\text{A}[8]=\{\underline{1}2,1\underline{2}3,\underline{0}4,\underline{3}4,\underline{0}6,\underline{6}7,\underline{7}8,7\underline{8}9\}\]
    \[\begin{matrix}
        \text{Dígito}&\text{Número}\\
        0&4,\,6\\
        1&12\\
        2&123\\
        3&34\\
        4&\\
        5&\\
        6&67\\
        7&78\\
        8&789\\
        9&
    \end{matrix}\]
    \[\text{Arreglo auxiliar}[8]=\{4,6,12,123,34,67,78,789\};\,\text{A}=\text{Arreglo auxiliar};\]
    • 3er. dígito:\\
    De manera similar al paso anterior, se trata a los números que no tienen un 3er. dígito como si tuvieran un 0 en dicha posición.
    \[\text{A}[8]=\{\underline{0}04,\underline{0}06,\underline{0}12,\underline{1}23,\underline{0}34,\underline{0}67,\underline{0}78,\underline{7}89\}\]
    \[\begin{matrix}
        \text{Dígito}&\text{Número}\\
        0&4,\,6,\,12,\,34,\,67,\,78\\
        1&123\\
        2&\\
        3&\\
        4&\\
        5&\\
        6&\\
        7&789\\
        8&\\
        9&
    \end{matrix}\]
    \[\text{Arreglo auxiliar}[8]=\{4,6,12,34,67,78,123,789\};\,\text{A}=\text{Arreglo auxiliar};\]
    Concluyendo el ordenamiento de manera ascendente.
\end{document}