\documentclass[letterpaper, 12pt]{article}
\usepackage[letterpaper, top=2.5cm, bottom=2.5cm, left=3cm, right=3cm]{geometry} %margenes
\usepackage[utf8]{inputenc} %manejo de caracteres especiales
\usepackage[spanish]{babel} %manejo de encabezados de inglés a español
\usepackage{fancyhdr} %formato de los encabezados de página
\usepackage{ragged2e} %alineado real justficado
\usepackage{graphicx} %manejo de imagenes
\usepackage{amsmath} %manejo de notación matemática
\usepackage{mathtools} %manejo de notación matemática
\usepackage{blindtext} %texto de relleno
\usepackage[titles]{tocloft} %manejo de elementos para el índice
\usepackage{float} %manejo de centrado para figuras
\usepackage{amssymb} %manejo de simbolog►1a matematica

\pagestyle{fancy}
\fancyhf{}
\rfoot{\thepage}

\begin{document}
    
    %PORTADA
    \begin{titlepage}
        \begin{figure}[ht]
            \centering
            \includegraphics[width=15cm]{logosITT.png}
        \end{figure}
        \centering
        {\scshape\LARGE Tecnológico Nacional de México\\Instituto Tecnológico de Tijuana\par}
        \vspace{1cm}
        {\scshape\Large Estructura de Datos\par}
        \vspace{1cm}
        {\scshape\Large Unidad 4\par}
        \vspace{1.5cm}
        {\huge\bfseries Ejercicios de Burbuja\par}
        \vspace{2cm}
        {\Large\itshape C. Abraham Jhared Flores Azcona\\19211640\par}
        \vfill
        Profesora: \par
        M.C. Claudia Negrete Sanchez
        
        \vfill

        {\large 23 de noviembre del 2020}
    \end{titlepage}

    \newpage
    \thispagestyle{fancy}
    \setcounter{page}{1}
    \lhead{\textbf{Ejercicios de Shell}}
    \section{Problema}
    Con el video adjunto, realiza el ordenamiento, y plasmalo en un documento o en tu cuarderno, y entregalo en tiempo y forma. Los datos serán ordenados de manera ascendente y los datos son:
    \[\text{E}[8]=\{78,\,34,\,6,\,12,\,87,\,45,\,7,\,3\}\]
    \section{Resolución}
    Para abreviar escritura, los valores 0 y 1 se usan para indicar verdad o falsedad de las desigualdades.
    \\\newline• Longitud de intervalos: \(\frac{8}{2}=4\): \newline
    * \(78>87:0 \therefore\)
    \[\therefore \text{E}[8]=\{87,\,34,\,6,\,12,\,78,\,45,\,7,\,3\}\]
    * \(34>45:0 \therefore\)
    \[\therefore \text{E}[8]=\{87,\,45,\,6,\,12,\,78,\,34,\,7,\,3\}\]
    * \(6>7:0\therefore\)
    \[\therefore \text{E}[8]=\{87,\,45,\,7,\,12,\,78,\,34,\,6,\,3\}\]
    * \(12>3:1;\,87>78:1;\,45>34:1;\,7>6:1;\)
    \\\newline
    • Longitud de intervalos: \(\frac{4}{2}=2\): \newline
    * \(87>7:1;\, 45>12:1;\, 7>78:0\therefore\)
    \[\therefore \text{E}[8]=\{87,\,45,\,78,\,12,\,7,\,34,\,6,\,3\}\]
    * \(12>34:0\therefore\)
    \[\therefore \text{E}[8]=\{87,\,45,\,78,\,34,\,7,\,12,\,6,\,3\}\]
    * \(87>78:1;\,45>34:1;\,78>7:1;\,34>12:1;\,7>6:1;\,12>3:1;\)
    \\\newline
    • Longitud de intervalos: \(\frac{2}{2}=1\): \newline
    * \(84>45:1;\, 45>78:0 \therefore\)
    \[\therefore \text{E}[8]=\{84,\,78,\,45,\,34,\,7,\,12,\,6,\,3\}\]
    * \(45>34:1;\, 7>12:0 \therefore\)
    \[\therefore \text{E}[8]=\{84,\,78,\,45,\,34,\,12,\,7,\,6,\,3\}\]
    * \(7>6:1;\, 6>3:1;\, 87>78:2;\, 78>45:1;\, 45>34:1;\, 34>12:1\, 12>7:1;\)
    \\\newline
    Dando por concluido el ejercicio.
\end{document}