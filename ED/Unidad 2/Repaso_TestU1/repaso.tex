\documentclass[letterpaper, 12pt]{article}
\usepackage[letterpaper, top=2.5cm, bottom=2.5cm, left=3cm, right=3cm]{geometry} %margenes
\usepackage[utf8]{inputenc} %manejo de caracteres especiales
\usepackage[spanish]{babel} %manejo de encabezados de inglés a español
\usepackage{fancyhdr} %formato de los encabezados de página
\usepackage{ragged2e} %alineado real justficado
\usepackage{graphicx} %manejo de imagenes
\usepackage{amsmath} %manejo de notación matemática
\usepackage{mathtools} %manejo de notación matemática
\usepackage{blindtext} %texto de relleno
\usepackage{amssymb}

\pagestyle{fancy}
\fancyhf{}
\rfoot{\thepage}

\begin{document}
\thispagestyle{fancy}
\lhead{\textbf{Repaso para el exámen U1}}
\rhead{09/10/2020}
    \subsection*{Instrucciones}
    Escriba un program donde ingreses en el método principal (Main) el nombre y la edad de tres personas, usando un método deberas obtener la edad mayor,
    en el principal mostraras la persona cuya edad es mayor
    \subsection*{Codigo}
    \begin{verbatim}
static void Main(string[] args)
{
    //variables auxiliares
    int []age = new int[3]; int i;
    string []name = new string[3];
    Console.Title="Edad mayor de tres personas";
    for(i=0;i<name.Length;i++)
    {
        Console.Write("Ingrese el nombre {0}:",i+1);
        name[i]=Console.ReadLine();
        Console.Write("Ingrese la edad {0}:",i+1);
        age[i]=int.Parse(Console.ReadLine());
    }
    new Program().Despliegue(age,name);
} 
        
public void Despliegue(int[] edades,string[] nombres)
{
    Console.Clear();
    int valormax = edades.Max();
    int indexedad = Array.IndexOf(edades, valormax); //busca el
    // indice del valor mayor correspondiente al arreglo edades
    Console.WriteLine("{0} tiene la edad mayor ({1})",
    nombres[indexedad],valormax);
    Console.ReadKey();
}
    \end{verbatim}
\end{document}